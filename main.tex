\documentclass[12pt]{article}

\usepackage[margin=1in]{geometry}

% Español y codificación
\usepackage[spanish]{babel}
\usepackage[utf8]{inputenc}
\usepackage[T1]{fontenc}

% Matemáticas
\usepackage{amsmath}
\usepackage{amssymb}
\usepackage{mathtools} 
\usepackage{graphicx}  % <-- ya no dará error cuando esté instalado
\usepackage{float}   % para usar [H] en figuras y tablas
\usepackage{tabularx}
% Gráficos
%\usepackage{float}       % para [H]
%\usepackage{tikz}
%\usepackage{pgfplots}
%5\pgfplotsset{compat=1.18}



\title{El modelo monatario clásico: desarrollo matemático del capítulo 2 de \emph{Monetary Policy, Inflation, and the Business Cycle}}
\author{Gabriel Tehozol}
\date{Noviembre de 2025}
\begin{document}

\maketitle
%========================================
% 2.1 HOGARES: Problema de optimización
%========================================
\subsection*{Hogares: definición del problema de optimización dinámica}
\label{subsec:hogares}

En este apartado describimos qué hace el hogar en el modelo.
La idea básica es:

\begin{itemize}
  \item El hogar vive muchos periodos (t, t+1, t+2, \dots).
  \item En cada periodo decide cuánto consumir ($C_t$) y cuánto trabajar ($N_t$).
  \item También puede ahorrar o endeudarse con bonos ($B_t$).
  \item Quiere que, en promedio, su ``nivel de felicidad'' (utilidad) a lo largo del tiempo
        sea lo más alto posible, pero está limitado por el dinero que tiene
        y el que puede conseguir.
\end{itemize}

A esto le llamamos un \emph{problema de optimización dinámica}: el hogar
toma decisiones hoy pensando en sus consecuencias mañana y en el futuro.

%--------------------------------------------------
\subsubsection*{Ecuación (1): función objetivo (utilidad esperada descontada)}
\label{subsubsec:ecuacion1}

El objetivo del hogar se resume en la siguiente expresión:

\begin{equation}
E_0 \sum_{t=0}^{\infty} \beta^t U(C_t, N_t; Z_t).
\tag{1}
\end{equation}

Esta ecuación se puede leer \emph{de izquierda a derecha} como una frase:

\begin{center}
\emph{``El hogar quiere maximizar el valor esperado, visto desde hoy,
de la suma de su utilidad en cada periodo, ponderada por cuánto le
importa el futuro.''}
\end{center}

\paragraph{Paso a paso.}

\begin{enumerate}
  \item $U(C_t,N_t;Z_t)$ es la \textbf{utilidad} del hogar en el periodo $t$.
  Depende de:
  \begin{itemize}
    \item $C_t$: consumo del bien (comida, servicios, etc.).
    \item $N_t$: horas de trabajo (más trabajo suele dar más ingreso, pero menos ocio).
    \item $Z_t$: shock o desplazador de preferencias, que representa cambios
          en gustos, hábitos, etc.
  \end{itemize}

  \item La suma $\displaystyle \sum_{t=0}^{\infty}$ indica que el hogar
  se preocupa por \textbf{todos} los periodos: $t=0,1,2,\dots$.

  \item El factor $\beta^t$, donde $0<\beta<1$, \emph{descuenta} la importancia
  de la utilidad de cada periodo. Cuanto mayor es $t$, menor es el peso $\beta^t$.
  \begin{itemize}
    \item Si $\beta$ es cercano a 1, el hogar es paciente (le importa mucho el futuro).
    \item Si $\beta$ es pequeño, el hogar es impaciente (valora más el presente).
  \end{itemize}

  \item $E_0[\cdot]$ es el \textbf{operador de expectativas}. Simplemente
  significa: ``el valor promedio que el hogar espera hoy ($t=0$) que tendrá
  esa suma, tomando en cuenta que el futuro es incierto''.
\end{enumerate}

\paragraph{Supuestos básicos sobre la utilidad.}

Para que el problema tenga sentido económico, se imponen algunos supuestos
estándar sobre $U(C_t,N_t;Z_t)$:

\begin{itemize}
  \item La utilidad aumenta con el consumo:
  \[
    U_{c,t} \equiv \frac{\partial U}{\partial C_t} > 0,
  \]
  y la utilidad marginal del consumo es decreciente:
  \[
    U_{cc,t} \equiv \frac{\partial^2 U}{\partial C_t^2} \le 0.
  \]
  Es decir, consumir más siempre gusta, pero cada unidad extra aporta un
  poco menos que la anterior.

  \item El trabajo genera desutilidad:
  \[
    U_{n,t} \equiv \frac{\partial U}{\partial N_t} \le 0,
  \]
  de modo que $-U_{n,t} > 0$ es la \emph{desutilidad marginal del trabajo}
  (trabajar más cansa).

  \item El shock $Z_t$ desplaza las preferencias. Supondremos que un aumento
  en $Z_t$ eleva la utilidad marginal del consumo:
  \[
    U_{cz,t} \equiv \frac{\partial^2 U}{\partial C_t\,\partial Z_t} > 0.
  \]
\end{itemize}

\noindent
La ecuación (1) dice:
el hogar quiere elegir sus secuencias de $C_t$ y $N_t$ para que, en promedio,
la suma de sus felicidades presentes y futuras sea lo más alta posible.

%--------------------------------------------------
\subsubsection*{Ecuación (2): restricción presupuestaria de flujo}
\label{subsubsec:ecuacion2}

El hogar no puede elegir cualquier combinación de consumo y trabajo:
cada periodo está limitado por sus ingresos y por lo que puede ahorrar
o endeudarse. Esta idea se resume en la \textbf{restricción de flujo}:

\begin{equation}
P_t C_t + Q_t B_t \le B_{t-1} + W_t N_t + D_t.
\tag{2}
\end{equation}

Podemos leer esta ecuación como:

\begin{center}
\emph{``Uso de recursos en el periodo $t$ $\le$ Fuentes de recursos en el periodo $t$.''}
\end{center}

\paragraph{Lado izquierdo: usos (en qué se gasta).}

\begin{itemize}
  \item $P_t C_t$: gasto en consumo.  
        $P_t$ es el precio nominal del bien; $C_t$ es la cantidad consumida.
  \item $Q_t B_t$: gasto en compra de bonos.  
        $Q_t$ es el precio hoy de un bono que paga 1 unidad de dinero
        en $t+1$; $B_t$ es el número de bonos que compra el hogar.
\end{itemize}

\paragraph{Lado derecho: fuentes (de dónde vienen los recursos).}

\begin{itemize}
  \item $B_{t-1}$: bonos que se compraron en $t-1$ y pagan hoy (ingreso financiero).
  \item $W_t N_t$: ingreso laboral (salario nominal $W_t$ por horas trabajadas $N_t$).
  \item $D_t$: dividendos que el hogar recibe como dueño de las empresas.
\end{itemize}

En equilibrio competitivo es natural que la desigualdad se cumpla
con igualdad (el hogar no deja dinero sin usar), es decir:
\[
  P_t C_t + Q_t B_t = B_{t-1} + W_t N_t + D_t.
\]

\noindent
Cada periodo, lo que el hogar gasta en consumo y bonos no puede superar 
lo que entra por salario, dividendos y bonos que vencen.

%--------------------------------------------------
\subsubsection*{Ecuación (3): restricción de solvencia (no-Ponzi)}
\label{subsubsec:ecuacion3}

Hasta ahora, la restricción de flujo controla qué pasa \emph{dentro} de cada
periodo. Pero como el horizonte es infinito, en principio el hogar podría
intentar endeudarse cada vez más, sin pagar nunca. Para evitar ese tipo
de planes no realistas, se impone una \textbf{restricción de solvencia}
o \textbf{restricción de no-Ponzi}:

\begin{equation}
\lim_{T\to\infty} E_t \left\{ \varXi_{t,T} \frac{B_T}{P_T} \right\} \ge 0.
\tag{3}
\end{equation}

\paragraph{¿Qué significa esta condición?}

\begin{itemize}
  \item $B_T/P_T$ es la \emph{riqueza real en bonos} en el periodo $T$.
  \item $\varXi_{t,T}$ es un \textbf{factor de descuento estocástico},
  definido como:
  \[
    \varXi_{t,T} \equiv \beta^{T-t} \frac{U_{c,T}}{U_{c,t}}.
  \]
  Este factor combina:
  \begin{itemize}
    \item el descuento puro del tiempo ($\beta^{T-t}$),
    \item y el cambio en la utilidad marginal del consumo
          ($U_{c,T}/U_{c,t}$).
  \end{itemize}
\end{itemize}

La expresión $E_t\{\varXi_{t,T} B_T/P_T\}$ se puede interpretar como el
\emph{valor presente esperado} de la riqueza real en bonos en el horizonte $T$,
medido desde el punto de vista del periodo $t$.

La condición
\[
  \lim_{T\to\infty} E_t \left\{ \varXi_{t,T} \frac{B_T}{P_T} \right\} \ge 0
\]
dice que, en el límite, ese valor presente esperado no puede ser negativo.
Intuitivamente:

\begin{center}
\emph{``El hogar no puede sostener un esquema en el que su deuda crezca
tanto que, incluso descontada, termine siendo impagable.''}
\end{center}

\noindent
La ecuación (3) evita el endeudamiento explosivo.
Es una condición técnica, pero muy importante, para que el problema infinito
esté bien definido.

%--------------------------------------------------
\subsubsection*{Ecuación (4): condición de optimalidad intratemporal}
\label{subsubsec:ecuacion4}

Ahora sí, pasamos a una de las \textbf{condiciones de primer orden}
más importantes: la que relaciona consumo, trabajo y salarios \emph{dentro}
de un mismo periodo. Esta condición se escribe como:

\begin{equation}
-\frac{U_{n,t}}{U_{c,t}} = \frac{W_t}{P_t}.
\tag{4}
\end{equation}

Esta ecuación se llama:
\begin{itemize}
  \item \textbf{Condición de optimalidad intratemporal}, porque compara
        decisiones dentro del periodo $t$.
  \item \textbf{Condición de oferta de trabajo}, porque nos dice cómo el hogar
        decide cuántas horas trabajar dado el salario real.
\end{itemize}

\paragraph{Interpretación de los términos.}

\begin{itemize}
  \item $U_{c,t}$: utilidad marginal del consumo (cuánto aumenta la utilidad
        si el hogar consume un poco más).
  \item $U_{n,t}$: utilidad marginal del trabajo (cuánto cambia la utilidad
        si el hogar trabaja un poco más). Suele ser negativa: trabajar cansa.
  \item $-U_{n,t}/U_{c,t}$: \textbf{tasa marginal de sustitución} (TMS)
        entre ocio y consumo.  
        Dice cuántas unidades extra de consumo necesita el hogar para
        aceptar trabajar una unidad extra.
  \item $W_t/P_t$: \textbf{salario real}, es decir, cuántas unidades de bien
        de consumo recibe el hogar por cada unidad de trabajo.
\end{itemize}

La ecuación (4) dice:

\begin{center}
\emph{``El hogar elige sus horas de trabajo de modo que su TMS subjetiva
entre ocio y consumo sea igual al salario real que ofrece el mercado.''}
\end{center}

\paragraph{Derivación paso a paso (argumento variacional).}

La idea es imaginar un pequeño cambio en las decisiones del hogar en el
periodo $t$, y pedir que ese cambio \emph{no mejore} la situación si ya
estamos en el óptimo.

\begin{enumerate}
  \item \textbf{Paso 1: pequeña desviación en consumo y trabajo.}  
  Partimos de un plan óptimo $(C_t,N_t)$ y consideramos una pequeña desviación
  $(dC_t,dN_t)$ en el periodo $t$, manteniendo todo lo demás igual.
  
  El cambio en la utilidad instantánea es:
  \[
    dU_t = U_{c,t}\, dC_t + U_{n,t}\, dN_t.
  \]

  Si el plan es óptimo, cualquier desviación que respete el presupuesto
  no puede mejorar la utilidad. En el margen, esto implica que el cambio
  en utilidad asociado a una desviación factible debe ser cero:
  \begin{equation}
    U_{c,t}\, dC_t + U_{n,t}\, dN_t = 0.
    \label{eq:variacion_utilidad}
  \end{equation}

  \item \textbf{Paso 2: pequeña desviación en la restricción de flujo.}  
  Ahora miramos cómo se ve esa misma desviación en la restricción presupuestaria
  del periodo $t$:
  \[
    P_t C_t + Q_t B_t = B_{t-1} + W_t N_t + D_t.
  \]
  Si mantenemos fijo $B_t$ (no tocamos el ahorro) y sólo cambiamos $C_t$ y $N_t$,
  la variación de esta igualdad es:
  \begin{equation}
    P_t\, dC_t = W_t\, dN_t.
    \label{eq:variacion_presupuesto}
  \end{equation}
  Es decir, el gasto adicional en consumo tiene que ser financiado por un
  ingreso adicional de trabajo.

  \item \textbf{Paso 3: expresar $dC_t$ en función de $dN_t$.}  
  De \eqref{eq:variacion_presupuesto} despejamos:
  \begin{equation}
    dC_t = \frac{W_t}{P_t} dN_t.
    \label{eq:dC_en_funcion_dN}
  \end{equation}

  \item \textbf{Paso 4: sustituir en la variación de utilidad.}  
  Sustituimos \eqref{eq:dC_en_funcion_dN} en \eqref{eq:variacion_utilidad}:
  \[
    U_{c,t} \left( \frac{W_t}{P_t} dN_t \right) + U_{n,t} dN_t = 0.
  \]
  Factorizamos $dN_t$:
  \[
    \left[
      U_{c,t} \frac{W_t}{P_t} + U_{n,t}
    \right] dN_t = 0.
  \]

  Como $dN_t$ representa una desviación arbitraria (pequeña pero no necesariamente
  cero), para que el producto sea cero, el término entre corchetes debe ser:
  \[
    U_{c,t} \frac{W_t}{P_t} + U_{n,t} = 0.
  \]

  \item \textbf{Paso 5: aislar la TMS.}  
  Reordenando:
  \[
    U_{c,t} \frac{W_t}{P_t} = -U_{n,t},
  \]
  y dividiendo entre $U_{c,t} > 0$:
  \[
    -\frac{U_{n,t}}{U_{c,t}} = \frac{W_t}{P_t},
  \]
  que es exactamente la ecuación (4).
\end{enumerate}

\paragraph{Entonces tenemos...}

\begin{itemize}
  \item La variación en la utilidad nos dice cómo cambian las preferencias
        del hogar ante pequeños ajustes en $C_t$ y $N_t$.
  \item La variación en el presupuesto nos dice qué combinaciones de
        $dC_t$ y $dN_t$ son factibles (se pueden pagar).
  \item Al combinar ambas y exigir que no haya manera de mejorar la utilidad
        con una desviación factible, obtenemos la condición de primer orden.
  \item El resultado final iguala:
        \begin{itemize}
          \item el \emph{costo subjetivo} de trabajar más (perder ocio),
          \item con el \emph{beneficio objetivo} de mercado (el salario real).
        \end{itemize}
\end{itemize}

%--------------------------------------------------
\subsubsection*{Ecuación (5): condición de optimalidad intertemporal (Ecuación de Euler)}
\label{subsubsec:ecuacion5}

Además de decidir cuánto trabajar dentro de cada periodo (condición intratemporal),
el hogar debe decidir \emph{cuánto consumir hoy y cuánto dejar para consumir mañana}.
Esta decisión se resume en la \textbf{condición de optimalidad intertemporal}
o \textbf{Ecuación de Euler}:

\begin{equation}
Q_t = \beta E_t \left\{ \frac{U_{c,t+1}}{U_{c,t}} \frac{P_t}{P_{t+1}} \right\}.
\tag{5}
\end{equation}

\paragraph{Lectura intuitiva.}

La ecuación (5) se puede leer como:

\begin{center}
\emph{``El precio del bono hoy ($Q_t$) debe ser igual al valor presente esperado
de la tasa a la que el hogar está dispuesto a intercambiar consumo de hoy por
consumo de mañana, medido en unidades de bien de consumo y multiplicado por
el factor de descuento $\beta$.''}
\end{center}

\noindent
Donde:
\begin{itemize}
  \item $Q_t$ es el \textbf{precio del bono} nominal libre de riesgo.
  \item $\displaystyle \frac{U_{c,t+1}}{U_{c,t}}$ mide cómo cambia la utilidad marginal
        del consumo entre $t$ y $t+1$.
  \item $\displaystyle \frac{P_t}{P_{t+1}}$ ajusta por la inflación esperada
        (pasa de unidades nominales a reales y viceversa).
  \item $\beta$ descuenta el futuro (impaciencia).
\end{itemize}

\paragraph{Desglose de los términos de (5).}

\begin{itemize}
  \item \textbf{Precio del bono:} $Q_t$ es el precio en $t$ de un bono que paga
  1 unidad de dinero en $t+1$. El rendimiento nominal bruto del bono es
  $R_t^n = 1/Q_t$.

  \item \textbf{Utilidad marginal del consumo:} 
  $U_{c,t} \equiv \partial U(C_t,N_t;Z_t)/\partial C_t$ es la utilidad marginal
  en el periodo $t$. La razón $\displaystyle U_{c,t+1}/U_{c,t}$ captura cómo el
  hogar valora una unidad extra de consumo mañana en comparación con hoy.

  \item \textbf{Relación de precios:} $\displaystyle P_t/P_{t+1}$ es el inverso
  de la inflación bruta esperada entre $t$ y $t+1$. Si los precios suben mucho,
  una misma cantidad de dinero rinde menos en términos de consumo futuro.

  \item \textbf{Operador de expectativas:} $E_t\{\cdot\}$ promedia sobre todos
  los escenarios posibles del futuro, dados los shocks, usando la información
  que el hogar tiene en $t$.
\end{itemize}

\paragraph{Derivación paso a paso}

La ecuación (5) se obtiene al analizar una pequeña reubicación de consumo
entre los periodos $t$ y $t+1$. La idea es:

\begin{center}
\emph{``¿Qué pasa si reduzco mi consumo hoy y uso ese ahorro para
aumentar mi consumo mañana, sin violar el presupuesto? En el óptimo, ese
cambio no debe mejorar mi utilidad esperada.''}
\end{center}

\begin{enumerate}
  \item \textbf{Paso 1: variación en la utilidad intertemporal.}  

  Consideremos un plan óptimo y una pequeña desviación $(dC_t, dC_{t+1})$
  que sólo afecta al consumo en $t$ y $t+1$. El cambio en la utilidad total
  (medida en $t$ y descontando el futuro) es aproximadamente:
  \begin{equation}
    U_{c,t}\, dC_t + \beta E_t\{U_{c,t+1}\, dC_{t+1}\} = 0.
    \tag{A}
  \end{equation}
  Explicación de cada término:
  \begin{itemize}
    \item $U_{c,t} dC_t$: cambio en la utilidad del \emph{periodo actual}.
    \item $\beta E_t\{U_{c,t+1} dC_{t+1}\}$: cambio esperado en la utilidad
          del periodo siguiente, descontado a valor presente con $\beta$.
  \end{itemize}
  En un óptimo, cualquier desviación factible en la que sólo movemos consumo
  entre $t$ y $t+1$ no debe aumentar la utilidad, por lo que el cambio marginal
  debe ser cero.

  \item \textbf{Paso 2: variación en el presupuesto intertemporal.}  

  Ahora vemos cómo se traduce esa misma desviación en la restricción
  presupuestaria. La idea es:

  \begin{itemize}
    \item Si reducimos el consumo hoy en una cantidad $dC_t<0$, liberamos
          recursos por un monto nominal $P_t dC_t$.
    \item Esos recursos se usan para comprar más bonos: el número adicional
          de bonos es
          \[
            dB_t = -\,\frac{P_t}{Q_t}\, dC_t,
          \]
          porque cada bono cuesta $Q_t$ unidades de dinero.
    \item Mañana, esos bonos pagan 1 unidad de dinero cada uno, de modo que
          el ingreso adicional nominal en $t+1$ es $(1)\cdot dB_t$.
  \end{itemize}

  El cambio en la restricción de flujo en $t+1$ implica que ese ingreso
  adicional se destina a mayor consumo $dC_{t+1}$:
  \[
    P_{t+1} dC_{t+1} = dB_t
    = -\,\frac{P_t}{Q_t} dC_t.
  \]
  Es decir:
  \begin{equation}
    P_{t+1} dC_{t+1} = -\,\frac{P_t}{Q_t} dC_t.
    \tag{B}
  \end{equation}

  \item \textbf{Paso 3: expresar $dC_{t+1}$ en función de $dC_t$.}  

  De (B) despejamos:
  \begin{equation}
    dC_{t+1} = -\,\frac{P_t}{Q_t P_{t+1}}\, dC_t.
    \tag{C}
  \end{equation}

  \item \textbf{Paso 4: sustituir en la variación de utilidad.}  

  Sustituimos (C) en (A):
  \[
    U_{c,t} dC_t 
    + \beta E_t\left\{
      U_{c,t+1}
      \left(-\,\frac{P_t}{Q_t P_{t+1}} dC_t\right)
    \right\} = 0.
  \]
  Factorizamos $dC_t$:
  \[
    \left[
      U_{c,t}
      - \beta E_t\left\{
        U_{c,t+1}\frac{P_t}{Q_t P_{t+1}}
      \right\}
    \right] dC_t = 0.
  \]
  Como $dC_t$ representa una desviación arbitraria (pequeña, pero distinta de cero),
  el término entre corchetes debe ser nulo:
  \begin{equation}
    U_{c,t} 
    = \beta E_t\left\{
      U_{c,t+1}\frac{P_t}{Q_t P_{t+1}}
    \right\}.
    \label{eq:euler_bruta}
  \end{equation}

  \item \textbf{Paso 5: aislar $Q_t$ y obtener la Ecuación de Euler.}  

  De \eqref{eq:euler_bruta}, multiplicamos ambos lados por
  $Q_t/U_{c,t}$:
  \[
    Q_t = \beta E_t\left\{
      \frac{U_{c,t+1}}{U_{c,t}}
      \frac{P_t}{P_{t+1}}
    \right\},
  \]
  que es exactamente la ecuación (5).
\end{enumerate}

\paragraph{Entonces tenemos...}

\begin{itemize}
  \item El hogar compara el ``costo'' de sacrificar consumo hoy con el
        ``beneficio'' de tener más consumo mañana.
  \item El costo se mide por la utilidad marginal actual $U_{c,t}$.
  \item El beneficio se mide por la utilidad marginal futura $U_{c,t+1}$,
        ajustada por:
        \begin{itemize}
          \item el factor de descuento $\beta$,
          \item la inflación esperada $P_t/P_{t+1}$,
          \item y el precio del bono $Q_t$.
        \end{itemize}
  \item En el óptimo, no hay reubicación de consumo (entre hoy y mañana)
        que pueda mejorar la utilidad: eso es lo que recoge la Ecuación de Euler.
\end{itemize}

%--------------------------------------------------
\subsubsection*{Ecuación (6): condición de transversalidad}
\label{subsubsec:ecuacion6}

La última pieza de las condiciones de optimalidad del hogar es la llamada
\textbf{condición de transversalidad}. Esta condición está estrechamente
ligada a la restricción de solvencia (3), pero ahora se impone como
\emph{igualdad} en el óptimo:

\begin{equation}
\lim_{T\to\infty} E_t \left\{ \varXi_{t,T} \frac{B_T}{P_T} \right\} = 0.
\tag{6}
\end{equation}

Recordemos que:
\[
  \varXi_{t,T} \equiv \beta^{T-t} \frac{U_{c,T}}{U_{c,t}}
\]
es el \textbf{factor de descuento estocástico} que mide cuánto vale, en términos
de utilidad marginal en $t$, una unidad de consumo (o riqueza real) en $T$.

\paragraph{Relación con la restricción de solvencia.}

La restricción de solvencia (3) decía:
\[
  \lim_{T\to\infty} E_t \left\{ \varXi_{t,T} \frac{B_T}{P_T} \right\} \ge 0.
  \tag{3}
\]

Es decir:
\begin{itemize}
  \item En el límite, el \emph{valor presente esperado} de la riqueza real
        en bonos no puede ser negativo.
  \item Esto impide que el hogar aplique esquemas de endeudamiento tipo Ponzi.
\end{itemize}

La \textbf{condición de transversalidad} (6) dice algo más fuerte:
\[
  \lim_{T\to\infty} E_t \left\{ \varXi_{t,T} \frac{B_T}{P_T} \right\} = 0.
\]

\paragraph{¿Por qué debe valer la igualdad en el óptimo?.}

Supongamos, para ver el argumento, que en el óptimo se cumpliera:
\[
  \lim_{T\to\infty} E_t \left\{ \varXi_{t,T} \frac{B_T}{P_T} \right\} = L > 0.
\]

\begin{itemize}
  \item Entonces, incluso después de descontar por $\beta^{T-t}$ y por
        el cociente $U_{c,T}/U_{c,t}$, el hogar estaría \emph{dejando}
        una cantidad positiva de riqueza real en el infinito.
  \item Dado que la utilidad marginal del consumo es positiva
        ($U_{c,t} > 0$ para todo $t$), el hogar podría mejorar su plan:
        reducir ligeramente $B_T$ (o, en la práctica, consumir un poco más
        en algún periodo finito) sin violar la restricción de solvencia,
        usando parte de ese ``excedente'' $L$.
  \item Al hacer esto, aumentaría su consumo en algún periodo sin disminuir
        el consumo en otros lo suficiente como para compensar, por lo que
        la utilidad total aumentaría.
\end{itemize}

Pero esto contradice la suposición de que el plan original era óptimo.
Por tanto, en un óptimo:
\[
  \lim_{T\to\infty} E_t \left\{ \varXi_{t,T} \frac{B_T}{P_T} \right\} = 0,
\]
que es precisamente la ecuación (6).

\paragraph{En términos económicos...}

\begin{itemize}
  \item La ecuación (3) asegura que la deuda no crece de manera explosiva.
  \item La ecuación (6) asegura que el hogar \emph{no deja recursos sin usar}
        en el límite: toda la riqueza potencial que pueda aumentar la utilidad
        se habrá utilizado en algún momento.
  \item En muchas aplicaciones, con un hogar representativo y oferta neta
        de deuda cero, se cumple en equilibrio que $B_t = 0$ para todo $t$,
        lo que hace que (6) sea automáticamente cierta. Aun así, es importante
        tenerla explícitamente como parte de las condiciones de optimalidad.
\end{itemize}

\bigskip

\noindent
Con las ecuaciones (4), (5) y (6) tenemos ahora el conjunto completo de
\textbf{condiciones de optimalidad del hogar}:
\begin{itemize}
  \item (4): condición intratemporal (oferta de trabajo).
  \item (5): condición intertemporal (Ecuación de Euler para consumo/ahorro).
  \item (6): condición de transversalidad (uso eficiente de la riqueza en el tiempo).
\end{itemize}

En el siguiente paso, estas condiciones se combinarán con el comportamiento
de las empresas y las identidades de equilibrio para construir el modelo
macroeconómico completo.

%--------------------------------------------------
\subsubsection*{Ecuación (7): especificación funcional de la utilidad
y oferta de trabajo competitiva}
\label{subsubsec:ecuacion7}

En esta sección, seguimos el capítulo y adoptamos una \textbf{forma funcional
particular} para la utilidad periódica:

\begin{equation}
U(C_t, N_t; Z_t) =
\left(
\frac{C_t^{1-\sigma} - 1}{1-\sigma}
- \frac{N_t^{1+\phi}}{1+\phi}
\right) Z_t,
\qquad \sigma \neq 1,
\tag{11}
\end{equation}
donde $\sigma \ge 0$ y $\phi \ge 0$ son parámetros.

\paragraph{Parámetros:}

Para mayor claridad, presentamos los parámetros que aparecen en la ecuación (11):

\begin{itemize}
  \item $\sigma$ (\emph{sigma}): parámetro de curvatura de la utilidad del consumo.
  En esta clase de modelos suele interpretarse como:
  \begin{itemize}
    \item medida de \textbf{aversión relativa al riesgo} del hogar, y
    \item (en muchas especificaciones) inverso de la \textbf{elasticidad intertemporal
          de sustitución del consumo}.
  \end{itemize}
  Valores más altos de $\sigma$ implican que la utilidad marginal del consumo
  cae más rápido cuando aumenta $C_t$.

  \item $\phi$ (\emph{phi}): parámetro de curvatura de la desutilidad del trabajo.
  Controla qué tan rápido aumenta la desutilidad marginal de trabajar más horas.
  Suele interpretarse como el \textbf{inverso de la elasticidad de la oferta
  de trabajo}: valores altos de $\phi$ indican que la oferta de trabajo es menos
  sensible a cambios en el salario real.
\end{itemize}

Nuestro objetivo ahora es:  
(1) calcular $U_{c,t}$ y $U_{n,t}$ a partir de (11), y  
(2) sustituirlos en (4) para obtener la versión \emph{específica} de la
oferta de trabajo, la \textbf{Ecuación (7)}.

%-----------------------
\paragraph{Paso 1: utilidad marginal del consumo $U_{c,t}$.}

Partimos de:
\[
  U(C_t, N_t; Z_t) =
  \left(
    \frac{C_t^{1-\sigma} - 1}{1-\sigma}
    - \frac{N_t^{1+\phi}}{1+\phi}
  \right) Z_t.
\]

Tomamos la derivada parcial respecto a $C_t$:
\begin{equation}
  U_{c,t} \equiv \frac{\partial U_t}{\partial C_t}
  = \frac{\partial}{\partial C_t}
    \left[
      \left(
        \frac{C_t^{1-\sigma} - 1}{1-\sigma}
        - \frac{N_t^{1+\phi}}{1+\phi}
      \right) Z_t
    \right].
\end{equation}

Como $Z_t$ está multiplicando todo, podemos sacarlo como constante:
\[
  U_{c,t}
  = Z_t \, \frac{\partial}{\partial C_t}
    \left(
      \frac{C_t^{1-\sigma} - 1}{1-\sigma}
      - \frac{N_t^{1+\phi}}{1+\phi}
    \right).
\]

El segundo término no depende de $C_t$, así que su derivada es cero.
Nos queda:
\[
  U_{c,t}
  = Z_t \, \frac{\partial}{\partial C_t}
    \left(
      \frac{C_t^{1-\sigma} - 1}{1-\sigma}
    \right).
\]

Ahora derivamos usando la regla de la potencia:
\[
  \frac{\partial}{\partial C_t}
  \left(
    \frac{C_t^{1-\sigma} - 1}{1-\sigma}
  \right)
  = \frac{1}{1-\sigma} \cdot (1-\sigma) C_t^{(1-\sigma)-1}
  = C_t^{-\sigma}.
\]

Por lo tanto,
\begin{equation}
  U_{c,t} = C_t^{-\sigma} Z_t.
\end{equation}

\noindent
\textbf{Tenemos...}
\begin{itemize}
  \item Si $\sigma > 0$, entonces $U_{c,t} > 0$ y decrece cuando $C_t$ aumenta
        (porque $C_t^{-\sigma}$ disminuye con $C_t$).  
        Esto refleja la idea de \emph{utilidad marginal decreciente del consumo}.
  \item El factor $Z_t$ amplifica o reduce la utilidad marginal del consumo
        según el estado de las preferencias.
\end{itemize}

%-----------------------
\paragraph{Paso 2: utilidad marginal del trabajo $U_{n,t}$.}

Volvemos a la función de utilidad:
\[
  U(C_t, N_t; Z_t) =
  \left(
    \frac{C_t^{1-\sigma} - 1}{1-\sigma}
    - \frac{N_t^{1+\phi}}{1+\phi}
  \right) Z_t.
\]

Tomamos ahora la derivada parcial respecto a $N_t$:
\begin{equation}
  U_{n,t} \equiv \frac{\partial U_t}{\partial N_t}
  = \frac{\partial}{\partial N_t}
    \left[
      \left(
        \frac{C_t^{1-\sigma} - 1}{1-\sigma}
        - \frac{N_t^{1+\phi}}{1+\phi}
      \right) Z_t
    \right].
\end{equation}

De nuevo, sacamos $Z_t$ como constante:
\[
  U_{n,t}
  = Z_t \, \frac{\partial}{\partial N_t}
    \left(
      \frac{C_t^{1-\sigma} - 1}{1-\sigma}
      - \frac{N_t^{1+\phi}}{1+\phi}
    \right).
\]

El primer término no depende de $N_t$, por lo que su derivada es cero.
Nos queda:
\[
  U_{n,t}
  = Z_t \, \frac{\partial}{\partial N_t}
    \left(
      - \frac{N_t^{1+\phi}}{1+\phi}
    \right).
\]

Aplicando la regla de la potencia:
\[
  \frac{\partial}{\partial N_t}
  \left(
    - \frac{N_t^{1+\phi}}{1+\phi}
  \right)
  = - \frac{1}{1+\phi} (1+\phi) N_t^{(1+\phi)-1}
  = - N_t^{\phi}.
\]

Por lo tanto,
\begin{equation}
  U_{n,t} = - N_t^{\phi} Z_t.
\end{equation}

\noindent
\textbf{Tenmos...}
\begin{itemize}
  \item Para $N_t > 0$ y $Z_t > 0$, se cumple $U_{n,t} < 0$: trabajar más
        \emph{reduce} la utilidad (desutilidad del esfuerzo).
  \item El término $-U_{n,t} = N_t^{\phi} Z_t$ es la \emph{desutilidad marginal
        del trabajo}, creciente en $N_t$ si $\phi > 0$.
\end{itemize}

%-----------------------
\paragraph{Paso 3: sustitución en la condición intratemporal (4).}

La condición general de optimalidad intratemporal es:
\[
  -\frac{U_{n,t}}{U_{c,t}} = \frac{W_t}{P_t}.
\]

Sustituimos las expresiones obtenidas:
\[
  -\frac{U_{n,t}}{U_{c,t}}
  = -\frac{-N_t^{\phi} Z_t}{C_t^{-\sigma} Z_t}
  = \frac{N_t^{\phi} Z_t}{C_t^{-\sigma} Z_t}.
\]

Observamos que $Z_t$ aparece como factor multiplicando tanto en el numerador
como en el denominador, por lo que se cancela:
\[
  -\frac{U_{n,t}}{U_{c,t}}
  = \frac{N_t^{\phi}}{C_t^{-\sigma}}.
\]

Usando que $1/C_t^{-\sigma} = C_t^{\sigma}$, obtenemos:
\[
  \frac{N_t^{\phi}}{C_t^{-\sigma}}
  = N_t^{\phi} C_t^{\sigma}.
\]

Por lo tanto,
\begin{equation}
  -\frac{U_{n,t}}{U_{c,t}}
  = C_t^{\sigma} N_t^{\phi}.
\end{equation}

Igualando esto al salario real, según (4), llegamos a:
\begin{equation}
  \frac{W_t}{P_t} = C_t^{\sigma} N_t^{\phi},
\tag{7}
\end{equation}
que es la \textbf{Ecuación (7)}.

%-----------------------
\paragraph{Interpretación económica de la Ecuación (7).}

La ecuación
\[
  \frac{W_t}{P_t} = C_t^{\sigma} N_t^{\phi}
\]
puede leerse como:

\begin{center}
\emph{``El salario real debe igualar el costo subjetivo marginal de trabajar
una unidad adicional, medido en unidades de consumo, dado el nivel de consumo
$C_t$ y trabajo $N_t$.''}
\end{center}

Más concretamente:

\begin{itemize}
  \item Si $N_t$ aumenta (el hogar trabaja más), el término $N_t^{\phi}$
        aumenta (para $\phi > 0$), por lo que el lado derecho crece.  
        \textbf{Es decir:} para convencer al hogar de trabajar más,
        el salario real $W_t/P_t$ debe ser mayor.

  \item Si $C_t$ aumenta, el término $C_t^{\sigma}$ también aumenta
        (para $\sigma > 0$). Esto refleja que, cuando el hogar ya consume mucho,
        la utilidad marginal del consumo es baja; por tanto, el hogar necesita
        un salario real más alto para renunciar a ocio (seguir trabajando)
        y mantener el equilibrio óptimo. Esto está relacionado con el
        \emph{efecto ingreso} sobre la oferta de trabajo.

  \item El producto $C_t^{\sigma} N_t^{\phi}$ es, en este contexto,
        la \textbf{tasa marginal de sustitución} entre ocio y consumo bajo
        la forma funcional elegida.
\end{itemize}

%-----------------------
\paragraph{Separabilidad y el papel de $Z_t$.}

Un detalle importante del resultado es que $Z_t$ \emph{desaparece}
de la condición intratemporal:

\begin{itemize}
  \item El choque de preferencias $Z_t$ multiplica toda la utilidad, por lo que
        aparece tanto en $U_{c,t}$ como en $U_{n,t}$, y se cancela en el cociente
        $-U_{n,t}/U_{c,t}$.
  \item Esto implica que la \textbf{oferta de trabajo intratemporal} (ecuación 7)
        no se ve afectada directamente por $Z_t$.
  \item En cambio, $Z_t$ sí influye en las decisiones \emph{intertemporales} a través
        de la Ecuación de Euler (5), donde entra la razón $U_{c,t+1}/U_{c,t}$.
\end{itemize}

La Ecuación (7) es la versión concreta, para la utilidad (11),
de la condición de optimalidad intratemporal del hogar. Representa la
\textbf{curva de oferta de trabajo competitiva}: para cada nivel de consumo
$C_t$, indica qué combinación de salario real $W_t/P_t$ y horas trabajadas
$N_t$ es consistente con el óptimo del hogar.
%--------------------------------------------------
\subsubsection*{Ecuación (8): Ecuación de Euler intertemporal específica}
\label{subsubsec:ecuacion8}

En la sección anterior obtuvimos la \emph{Ecuación de Euler general} para el
consumo y el ahorro del hogar:
\begin{equation}
Q_t = \beta E_t \left\{
  \frac{U_{c,t+1}}{U_{c,t}} \frac{P_t}{P_{t+1}}
\right\},
\tag{5}
\end{equation}
donde $U_{c,t}$ es la utilidad marginal del consumo en el periodo $t$.

Ahora queremos ver \emph{cómo se ve} esta condición cuando usamos la
función de utilidad específica:
\begin{equation}
U(C_t, N_t; Z_t) =
\left(
\frac{C_t^{1-\sigma} - 1}{1-\sigma}
- \frac{N_t^{1+\phi}}{1+\phi}
\right) Z_t,
\qquad \sigma \neq 1,
\tag{11}
\end{equation}
que introdujimos al derivar la ecuación (7).

%-----------------------
\paragraph{Parámetros relevantes}

En esta ecuación de Euler aparecen de nuevo dos parámetros importantes:

\begin{itemize}
  \item $\sigma$ (\emph{sigma}): parámetro de curvatura de la utilidad del consumo.
  Como ya se discutió, se interpreta habitualmente como una medida de
  \textbf{aversión relativa al riesgo} y está estrechamente relacionada con la
  \textbf{elasticidad intertemporal de sustitución del consumo}.

  \item $\beta$ (\emph{beta}): \textbf{factor de descuento intertemporal} del hogar,
  con $0 < \beta < 1$. Controla qué tanto valora el hogar la utilidad futura
  respecto a la presente; valores altos de $\beta$ indican mayor paciencia.
\end{itemize}

Nuestro objetivo es expresar la ecuación (5) en términos de tasas de
crecimiento de consumo y del factor de preferencias $Z_t$.

%-----------------------
\paragraph{Paso 1: utilidades marginales del consumo.}

De la derivación previa (sección de la ecuación 7), recordamos que:
\[
  U_{c,t} = C_t^{-\sigma} Z_t,
  \qquad
  U_{c,t+1} = C_{t+1}^{-\sigma} Z_{t+1}.
\]

Estas expresiones provienen de derivar (11) respecto a $C_t$ y $C_{t+1}$:

\[
  U_{c,t} \equiv \frac{\partial U(C_t,N_t;Z_t)}{\partial C_t}
  = C_t^{-\sigma} Z_t,
\]
\[
  U_{c,t+1} \equiv \frac{\partial U(C_{t+1},N_{t+1};Z_{t+1})}{\partial C_{t+1}}
  = C_{t+1}^{-\sigma} Z_{t+1}.
\]

%-----------------------
\paragraph{Paso 2: razón de utilidades marginales $U_{c,t+1}/U_{c,t}$.}

Formamos la razón que aparece en la Ecuación de Euler:
\begin{equation}
\frac{U_{c,t+1}}{U_{c,t}}
= \frac{C_{t+1}^{-\sigma} Z_{t+1}}{C_t^{-\sigma} Z_t}.
\end{equation}

Agrupamos términos de consumo y de preferencias por separado:
\[
  \frac{U_{c,t+1}}{U_{c,t}}
  = \left(
      \frac{C_{t+1}^{-\sigma}}{C_t^{-\sigma}}
    \right)
    \left(
      \frac{Z_{t+1}}{Z_t}
    \right).
\]

En el primer cociente aplicamos la propiedad de potencias:
\[
  \frac{C_{t+1}^{-\sigma}}{C_t^{-\sigma}}
  = \left(
      \frac{C_{t+1}}{C_t}
    \right)^{-\sigma}.
\]

Por lo tanto:
\begin{equation}
\frac{U_{c,t+1}}{U_{c,t}}
= \left(
    \frac{C_{t+1}}{C_t}
  \right)^{-\sigma}
  \left(
    \frac{Z_{t+1}}{Z_t}
  \right).
\label{eq:razon_Uc}
\end{equation}

\noindent
\textbf{Podemos leer...}
\begin{itemize}
  \item El término $\displaystyle (C_{t+1}/C_t)^{-\sigma}$ recoge cómo el
        crecimiento del consumo afecta la utilidad marginal relativa entre
        $t$ y $t+1$.
  \item El término $\displaystyle Z_{t+1}/Z_t$ refleja cómo cambian las
        preferencias (el ``peso'' que el hogar asigna a la utilidad del consumo)
        entre ambos periodos.
\end{itemize}

%-----------------------
\paragraph{Paso 3: sustitución en la Ecuación de Euler general.}

Tomamos la Ecuación de Euler:
\[
  Q_t = \beta E_t \left\{
    \frac{U_{c,t+1}}{U_{c,t}} \frac{P_t}{P_{t+1}}
  \right\},
\]
y sustituimos la expresión de \eqref{eq:razon_Uc}:

\[
  Q_t = \beta E_t \left\{
    \left(
      \frac{C_{t+1}}{C_t}
    \right)^{-\sigma}
    \left(
      \frac{Z_{t+1}}{Z_t}
    \right)
    \frac{P_t}{P_{t+1}}
  \right\}.
\]

Esto nos da la \textbf{Ecuación (8)}:

\begin{equation}
  Q_t = \beta E_t \left\{
    \left(
      \frac{C_{t+1}}{C_t}
    \right)^{-\sigma}
    \left(
      \frac{Z_{t+1}}{Z_t}
    \right)
    \left(
      \frac{P_t}{P_{t+1}}
    \right)
  \right\}.
\tag{8}
\end{equation}

%-----------------------
\paragraph{Interpretación de la Ecuación (8).}

La ecuación (8) es la versión \emph{específica} de la condición de Euler del
hogar bajo la utilidad (11). Podemos leerla así:

\begin{center}
\emph{``El precio del bono hoy ($Q_t$) es igual al valor esperado, descontado,
de lo que vale una unidad de consumo futuro en términos de utilidad marginal
de hoy, ajustado por inflación y por cambios en preferencias.''}
\end{center}

Más en detalle:

\begin{itemize}
  \item $Q_t$: precio actual de una promesa de 1 unidad de dinero en $t+1$.
        Como antes, el rendimiento nominal bruto del bono es $R_t^n = 1/Q_t$.

  \item $\beta$: factor de descuento (\emph{beta}). Captura la impaciencia
        pura del hogar: cuanto menor es $\beta$, más ``caro'' le resulta
        postergar consumo.

  \item $\displaystyle \left(\frac{C_{t+1}}{C_t}\right)^{-\sigma}$:
        este término recoge la \textbf{sustitución intertemporal del consumo}.
        \begin{itemize}
          \item Si se espera que $C_{t+1}$ sea bajo respecto a $C_t$, la
                utilidad marginal futura será alta, y el hogar valora mucho
                poder aumentar $C_{t+1}$: esto tiende a elevar el valor del
                bono (sube $Q_t$).
          \item El parámetro $\sigma$ (\emph{sigma}) gobierna qué tan sensible
                es esta valoración a cambios en la tasa de crecimiento del
                consumo.
        \end{itemize}

  \item $\displaystyle \frac{Z_{t+1}}{Z_t}$:
        refleja la \textbf{tasa de crecimiento del factor de preferencias}.
        \begin{itemize}
          \item Si $Z_{t+1}/Z_t$ es grande, el consumo futuro ``pesa más''
                en la función de utilidad; el hogar está dispuesto a pagar
                más hoy por una unidad de consumo mañana (aumenta $Q_t$).
          \item Dicho de otra forma, $Z_t$ actúa como un \emph{choque al
                factor de descuento efectivo}.
        \end{itemize}

  \item $\displaystyle \frac{P_t}{P_{t+1}}$:
        es el \textbf{inverso de la inflación bruta esperada}. Si se espera
        alta inflación, una unidad nominal de mañana vale menos en términos
        de consumo, lo que tiende a reducir $Q_t$ (para un mismo nivel de
        utilidad marginal futura).
\end{itemize}

\noindent
\textbf{Tenemos...}

La Ecuación (8) combina tres elementos fundamentales:

\begin{enumerate}
  \item \textbf{Tiempo e impaciencia} ($\beta$): preferencia por el presente.
  \item \textbf{Riesgo y crecimiento del consumo} 
        ($\left(C_{t+1}/C_t\right)^{-\sigma}$ y $\sigma$): cómo valora el hogar
        el consumo en estados futuros donde $C_{t+1}$ puede ser alto o bajo.
  \item \textbf{Choques de preferencias e inflación}
        ($Z_{t+1}/Z_t$ y $P_t/P_{t+1}$): cómo cambian la ``importancia''
        del consumo futuro y el poder adquisitivo de los pagos nominales.
\end{enumerate}

En conjunto, estos factores determinan el \emph{precio justo} del bono
$Q_t$ en equilibrio.
%--------------------------------------------------
\subsubsection*{Nota: Reescritura y descomposición de la ecuación (8)}

Partimos de la condición de optimalidad intertemporal específica:
\begin{equation}
  Q_t = \beta E_t \left\{
    \left(
      \frac{C_{t+1}}{C_t}
    \right)^{-\sigma}
    \left(
      \frac{Z_{t+1}}{Z_t}
    \right)
    \left(
      \frac{P_t}{P_{t+1}}
    \right)
  \right\}.
  \tag{8}
\end{equation}

Para fijar ideas, consideremos primero el caso sin incertidumbre (o el valor esperado
como ``escenario central''), de modo que el operador de expectativas puede omitirse.
Definimos las tasas brutas de crecimiento
\[
  g_c \;\equiv\; \frac{C_{t+1}}{C_t},
  \qquad
  g_z \;\equiv\; \frac{Z_{t+1}}{Z_t},
\]
y escribimos la razón de precios en función de la inflación esperada $\pi_{t+1}^e$:
\[
  \frac{P_t}{P_{t+1}}
  \;=\;
  \frac{1}{1+\pi_{t+1}^e}.
\]

Con estas definiciones, la ecuación (8) se puede reescribir como
\begin{equation}
  Q_t
  \;=\;
  \beta \,
  g_c^{-\sigma} \,
  g_z \,
  \frac{1}{1+\pi_{t+1}^e}.
  \label{eq:Qt_factores}
\end{equation}
Esta expresión hace explícito que el precio del bono nominal de un período $Q_t$ es el
producto de cuatro componentes:
\begin{enumerate}
  \item $\beta$: \emph{impaciencia pura} del hogar.
  \item $g_c^{-\sigma}$:
    componente de \emph{sustitución intertemporal del consumo}, que recoge cómo
    el crecimiento esperado del consumo afecta la utilidad marginal futura.
  \item $g_z$:
    componente asociado al \emph{factor de preferencias} $Z_t$, que actúa como un
    shock al factor de descuento efectivo.
  \item $\dfrac{1}{1+\pi_{t+1}^e}$:
    componente que recoge el \emph{castigo por inflación esperada}, al corregir
    la pérdida de poder adquisitivo del bono nominal.
\end{enumerate}

Tomando logaritmos naturales en \eqref{eq:Qt_factores}, obtenemos una descomposición
aditiva particularmente útil:
\begin{equation}
  \ln Q_t
  \;=\;
  \ln \beta
  \;-\;
  \sigma \ln g_c
  \;+\;
  \ln g_z
  \;-\;
  \ln(1+\pi_{t+1}^e).
  \label{eq:lnQt_descomposicion}
\end{equation}
La ecuación \eqref{eq:lnQt_descomposicion} muestra cómo cada
componente (crecimiento esperado del consumo, evolución del factor de preferencias
y inflación esperada) contribuye (con signo y magnitud) al nivel de
$Q_t$, y por tanto, de la tasa de interés nominal bruta
$1+i_t = 1/Q_t$.

\subsubsection*{Interpretación gráfica de la ecuación (8)}

La expresión
\begin{equation}
  Q_t
  \;=\;
  \beta \,
  g_c^{-\sigma} \,
  g_z \,
  \frac{1}{1+\pi_{t+1}^e},
  \qquad
  g_c \equiv \frac{C_{t+1}}{C_t},
  \quad
  g_z \equiv \frac{Z_{t+1}}{Z_t},
  \tag{8 revisada}
\end{equation}
muestra que el precio del bono nominal de un período $Q_t$ depende de tres
bloques económicos: (i) el crecimiento esperado del consumo $g_c$ (sustitución
intertemporal), (ii) la evolución del factor de preferencias $g_z$ y
(iii) la inflación esperada $\pi_{t+1}^e$ que entra vía $P_t/P_{t+1} = 1/(1+\pi_{t+1}^e)$.
Las figuras siguientes ilustran cómo cambia $Q_t$ cuando variamos cada uno de
estos componentes, manteniendo constantes los demás.

\begin{figure}[H]
  \centering
  \includegraphics[width=0.7\textwidth]{ecu8_Q_vs_gc}%
  \caption{Efecto del crecimiento esperado del consumo $g_c = C_{t+1}/C_t$ sobre $Q_t$.}
  \label{fig:Qt_gc}
\end{figure}

En la Figura~\ref{fig:Qt_gc} se representa $Q_t$ como función del crecimiento
esperado del consumo $g_c$. Dado que $Q_t \propto g_c^{-\sigma}$, la curva es
decreciente: cuanto mayor es el crecimiento esperado del consumo
($g_c \uparrow$), menor es $Q_t$. Económicamente, si el hogar espera ser más
``rico'' mañana (mayor consumo futuro), la utilidad marginal futura es más baja
y, por tanto, está dispuesto a pagar menos por una unidad de consumo futuro:
el bono se vende más barato ($Q_t$ cae) y la tasa de interés nominal
$1+i_t = 1/Q_t$ aumenta.

\footnote{El símbolo $\propto$ denota \emph{proporcionalidad}. 
El precio del bono $Q_t$ es \emph{proporcional} a $g_z$ elevado a la 
potencia menos sigma.}

\begin{figure}[H]
  \centering
  \includegraphics[width=0.7\textwidth]{ecu8_Q_vs_gz}%
  \caption{Efecto del factor de preferencias $g_z = Z_{t+1}/Z_t$ sobre $Q_t$.}
  \label{fig:Qt_gz}
\end{figure}

La Figura~\ref{fig:Qt_gz} muestra el efecto del factor de preferencias $g_z$.
Como $Q_t \propto g_z$, la relación es creciente: un aumento de $Z_{t+1}$ en
relación con $Z_t$ desplaza al alza $Q_t$. En términos económicos, un
incremento en $g_z$ puede interpretarse como un shock que eleva la
valoración relativa del consumo futuro; el hogar se vuelve más paciente y
está dispuesto a pagar más por el bono hoy. En consecuencia, el precio del
bono sube y la tasa de interés nominal cae.

\begin{figure}[H]
  \centering
  \includegraphics[width=0.7\textwidth]{ecu8_Q_vs_pi}%
  \caption{Efecto de la inflación esperada $\pi_{t+1}^e$ sobre $Q_t$.}
  \label{fig:Qt_pi}
\end{figure}

Por último, la Figura~\ref{fig:Qt_pi} representa la relación entre $Q_t$ y la
inflación esperada $\pi_{t+1}^e$, manteniendo constantes $g_c$ y $g_z$. Como
$Q_t \propto 1/(1+\pi_{t+1}^e)$, la curva es decreciente: una mayor inflación
esperada reduce el valor real del pago nominal del bono, de modo que los
hogares sólo están dispuestos a comprarlo a un precio menor. Esto se traduce
en un $Q_t$ más bajo y, por tanto, en una tasa de interés nominal más alta.

En conjunto, las Figuras~\ref{fig:Qt_gc}--\ref{fig:Qt_pi} ilustran que el precio
del bono $Q_t$ sintetiza tres tipos de consideraciones: (i) expectativas sobre
el perfil de consumo, (ii) choques de preferencias que modifican el peso
relativo del futuro y (iii) expectativas de inflación que erosionan el valor
real de los pagos nominales.
 
\subsection*{Ecuación (9): Oferta de trabajo log-lineal}

La ecuación (9) es la representación \emph{log-lineal} de la
\textbf{condición de optimalidad intratemporal} (ecuación (7)), que describe
la \emph{oferta de trabajo competitiva} del hogar. Se escribe como:
\begin{equation}
  w_t - p_t = \sigma c_t + \phi n_t.
  \tag{9}
\end{equation}

\noindent
En este contexto, adoptamos la convención estándar de que las letras
minúsculas denotan el \textbf{logaritmo natural} de la variable correspondiente
en niveles:
\[
  w_t \equiv \log W_t, \quad
  p_t \equiv \log P_t, \quad
  c_t \equiv \log C_t, \quad
  n_t \equiv \log N_t.
\]

\subsubsection*{I. Desglose riguroso de términos}

\begin{center}
\small
\begin{tabular}{lp{0.32\textwidth}p{0.4\textwidth}}
\hline
Símbolo & Definición & Significado\\
\hline
$w_t$ & $\log W_t$ & Logaritmo del salario nominal. \\
$p_t$ & $\log P_t$ & Logaritmo del nivel de precios del bien de consumo. \\
$w_t - p_t$ & $\log (W_t/P_t)$ & Logaritmo del salario real (precio del trabajo en unidades de bien). \\
$c_t$ & $\log C_t$ & Logaritmo del consumo; indicador del nivel de consumo/riqueza. \\
$n_t$ & $\log N_t$ & Logaritmo de horas de trabajo/empleo. \\
$\sigma$ & Curvatura de la utilidad del consumo & Mide el efecto ingreso del consumo sobre la oferta de trabajo. \\
$\phi$ & Parámetro de desutilidad del trabajo & Inverso de la elasticidad (Frisch) de la oferta de trabajo. \\
\hline
\end{tabular}
\end{center}


\noindent
Recordemos que:
\begin{itemize}
  \item $\sigma$ (sigma) es el parámetro que controla la curvatura de la
        utilidad en consumo; cuanto mayor es $\sigma$, más sensible es la
        utilidad marginal ante cambios en $C_t$.
  \item $\phi$ (phi) está asociado a la desutilidad del trabajo; cuanto mayor
        es $\phi$, más empinada es la desutilidad marginal de trabajar, y
        menor es la elasticidad (Frisch) de la oferta de trabajo.
\end{itemize}

\subsubsection*{II. Proceso de obtención (log-linealización)}

La ecuación (9) se obtiene aplicando logaritmos a la
\textbf{condición de optimalidad intratemporal específica} (ecuación (7)),
derivada previamente a partir de la función de utilidad (ecuación (11)):
\begin{equation}
  \frac{W_t}{P_t} = C_t^{\sigma} N_t^{\phi}.
  \tag{7}
\end{equation}

\paragraph{Paso 1: aplicar logaritmo natural a ambos lados.}

Tomamos $\log(\cdot)$ en ambos lados de (7):
\begin{equation}
  \log \left( \frac{W_t}{P_t} \right)
  \;=\;
  \log \left( C_t^{\sigma} N_t^{\phi} \right).
  \label{eq:log_saldo_inicial}
\end{equation}

\paragraph{Paso 2: usar propiedades básicas de logaritmos.}

Utilizamos tres identidades elementales (que es útil que el lector tenga
claras):
\begin{align*}
  \log\left(\frac{A}{B}\right) &= \log A - \log B, \\
  \log(AB) &= \log A + \log B, \\
  \log(A^x) &= x \log A.
\end{align*}

Aplicando estas propiedades en \eqref{eq:log_saldo_inicial}, obtenemos:
\begin{align*}
  \log \left( \frac{W_t}{P_t} \right)
  &= \log W_t - \log P_t, \\
  \log \left( C_t^{\sigma} N_t^{\phi} \right)
  &= \sigma \log C_t + \phi \log N_t.
\end{align*}
Por lo tanto:
\begin{equation}
  \log W_t - \log P_t
  \;=\;
  \sigma \log C_t + \phi \log N_t.
\end{equation}

\paragraph{Paso 3: sustituir notación log-lineal.}

Sustituyendo $w_t = \log W_t$, $p_t = \log P_t$,
$c_t = \log C_t$ y $n_t = \log N_t$, llegamos a:
\begin{equation}
  w_t - p_t = \sigma c_t + \phi n_t,
\end{equation}
que corresponde a la ecuación (9).

\subsubsection*{III. Interpretación didáctica y rigurosa}

La ecuación (9) puede leerse como la \emph{función de oferta de trabajo} del
hogar en términos logarítmicos:
\begin{equation*}
  \underbrace{w_t - p_t}_{\text{salario real (en log)}}
  \;=\;
  \underbrace{\sigma c_t}_{\text{efecto ingreso vía consumo}}
  \;+\;
  \underbrace{\phi n_t}_{\text{respuesta de la oferta de trabajo}}.
\end{equation*}

\begin{itemize}
  \item \textbf{Relación con el salario real.}
        Para un nivel dado de consumo $c_t$, un aumento del salario real
        $w_t - p_t$ debe estar asociado a un aumento de $n_t$. Esto refleja
        el \emph{efecto sustitución}: cuando el salario real sube, el ocio
        se encarece y el hogar está dispuesto a ofrecer más trabajo.

  \item \textbf{Efecto del consumo (efecto ingreso).}
        El término $\sigma c_t$ recoge que, cuanto más alto es el consumo
        (y, en cierto sentido, la ``riqueza'' del hogar), mayor debe ser el
        salario real para inducir la misma cantidad de trabajo. Intuitivamente:
        si el hogar ya consume mucho, necesita un incentivo mayor para sacrificar
        ocio.

  \item \textbf{Papel de $\phi$ y elasticidad de la oferta de trabajo.}
        Como $\phi$ es el inverso de la elasticidad de la oferta de
        trabajo, un valor elevado de $\phi$ implica que $n_t$ responde poco
        a cambios en el salario real: la oferta de trabajo es relativamente
        inelástica. En cambio, un $\phi$ pequeño implica una oferta de trabajo
        más sensible a los cambios en $w_t - p_t$.
\end{itemize}

\subsubsection*{IV. Invarianza a choques de preferencias}

Un punto clave es que la posición de la curva de oferta de trabajo en (9) es
\textbf{invariante} frente a los choques de preferencias $Z_t$.
Esto se debe a que el factor $Z_t$ se cancela en la razón
$-U_{n,t}/U_{c,t}$ al derivar la condición intratemporal (7). Por tanto:
\begin{itemize}
  \item $Z_t$ \emph{no} altera la decisión óptima de cuánto trabajar dado un
        salario real en el período $t$ (ecuación (9));
  \item en cambio, sí afecta las decisiones \emph{intertemporales} de consumo
        y ahorro a través de la ecuación de Euler (ecuación (8)).
\end{itemize}
Desde el punto de vista didáctico, esto permite separar claramente:
(i) la determinación del salario real y la oferta de trabajo dentro del período,
y (ii) las decisiones de ahorro/consumo entre períodos, que son las que sí
resultan sensibles a los choques de preferencias $Z_t$.

\subsubsection*{Interpretación gráfica de la ecuación (9)}

Recordemos la forma log-lineal de la condición de oferta de trabajo:
\begin{equation}
  w_t - p_t = \sigma c_t + \phi n_t.
  \tag{9}
\end{equation}
Para fijar ideas, conviene reescribirla como:
\begin{equation}
  w_t - p_t
  \;=\;
  \underbrace{\sigma c_t}_{\text{término constante}}
  \;+\;
  \underbrace{\phi n_t}_{\text{pendiente por unidad de empleo}}.
  \label{eq:ls_lineal}
\end{equation}
Si fijamos el nivel de consumo $c_t$, la ecuación \eqref{eq:ls_lineal} describe
una recta en el plano cuyo eje horizontal es el empleo $n_t$ y cuyo eje
vertical es el salario real $w_t - p_t$:
\begin{itemize}
  \item La \textbf{pendiente} de la curva es $\phi$.
  \item La \textbf{ordenada al origen} (intersección con el eje vertical) es
        $\sigma c_t$.
\end{itemize}

\begin{figure}[h!]
  \centering
  \includegraphics[width=0.7\textwidth]{ls_n_vs_w}%
  \caption{Oferta de trabajo competitiva: salario real vs.\ empleo,
           para un consumo dado $c_t$.}
  \label{fig:ls_n_vs_w}
\end{figure}

En la Figura~\ref{fig:ls_n_vs_w} se representa el salario real de equilibrio
$w_t - p_t$ en función del empleo $n_t$, manteniendo fijo el nivel de consumo
$c_t$. La curva es creciente: para que el hogar esté dispuesto a ofrecer más
trabajo (mayor $n_t$), el mercado debe ofrecer un salario real más alto. Esta
relación recoge el \emph{efecto sustitución}: al subir el salario real, el ocio
se vuelve más caro y el hogar decide trabajar más.

\subsubsection*{Desplazamientos por cambios en el consumo $c_t$}

Si comparamos dos niveles de consumo, $c_t^{\text{bajo}}$ y $c_t^{\text{alto}}$,
con $c_t^{\text{alto}} > c_t^{\text{bajo}}$, la ecuación (9) implica:
\[
  (w_t - p_t)^{\text{alto}}
  = \sigma c_t^{\text{alto}} + \phi n_t,
  \qquad
  (w_t - p_t)^{\text{bajo}}
  = \sigma c_t^{\text{bajo}} + \phi n_t.
\]

Como $\sigma > 0$, un mayor $c_t$ desplaza la curva hacia arriba: para cualquier
nivel dado de empleo $n_t$, el salario real compatible con la optimalidad
intratemporal es más alto. Gráficamente:

\begin{figure}[h!]
  \centering
  \includegraphics[width=0.7\textwidth]{ls_shift_c}%
  \caption{Desplazamiento de la oferta de trabajo ante un aumento en el consumo
           $c_t$.}
  \label{fig:ls_shift_c}
\end{figure}

En la Figura~\ref{fig:ls_shift_c} se muestra cómo una curva de oferta de trabajo
asociada a un bajo consumo $c_t^{\text{bajo}}$ (línea inferior) se desplaza
hacia arriba cuando el consumo aumenta a $c_t^{\text{alto}}$ (línea superior).
Económicamente:
\begin{itemize}
  \item Si el hogar ya consume más (es “más rico”), necesita un salario real
        más alto para estar dispuesto a trabajar la misma cantidad de horas.
  \item Esto es precisamente el \textbf{efecto ingreso}: una mayor riqueza
        tiende a reducir la oferta de trabajo a un salario real dado.
\end{itemize}

\subsubsection*{El papel de $\phi$: curvatura y elasticidad de la oferta de trabajo}

El parámetro $\phi$ controla la sensibilidad de la oferta de trabajo al salario
real. De (9), si mantenemos $c_t$ fijo, podemos interpretar:
\[
  w_t - p_t = \sigma c_t + \phi n_t
  \quad\Longrightarrow\quad
  \frac{\partial (w_t - p_t)}{\partial n_t} = \phi.
\]

\begin{figure}[h!]
  \centering
  \includegraphics[width=0.7\textwidth]{ls_phi_pendiente}%
  \caption{Curvas de oferta de trabajo con distinta pendiente $\phi$.}
  \label{fig:ls_phi}
\end{figure}

En la Figura~\ref{fig:ls_phi} se comparan dos curvas de oferta de trabajo que
pasan por un mismo punto de referencia, pero con diferente valor de $\phi$:
\begin{itemize}
  \item Una curva con \emph{baja} $\phi$ es más plana: pequeños cambios en el
        salario real generan cambios relativamente grandes en el empleo $n_t$.
        La oferta de trabajo es \textbf{más elástica}.
  \item Una curva con \emph{alta} $\phi$ es más empinada: incluso cambios
        grandes en el salario real producen cambios modestos en $n_t$.
        La oferta de trabajo es \textbf{más inelástica}.
\end{itemize}

Dado que $\phi$ es el inverso de la elasticidad de la oferta de
trabajo, estas comparaciones gráficas ayudan a visualizar cómo la estructura
de preferencias del hogar se traduce en distintas sensibilidades de la oferta
de trabajo ante variaciones en el salario real.

\subsubsection*{Ecuación (10): Ecuación de Euler log-linealizada}

Partimos de la ecuación de Euler específica:
\begin{equation}
  Q_t = \beta E_t \left\{
    \left(
      \frac{C_{t+1}}{C_t}
    \right)^{-\sigma}
    \left(
      \frac{Z_{t+1}}{Z_t}
    \right)
    \left(
      \frac{P_t}{P_{t+1}}
    \right)
  \right\},
  \label{eq:euler8}
\end{equation}
y queremos obtener su versión log-lineal, que en el texto se reporta como:
\begin{equation}
  c_t
  =
  E_t\{c_{t+1}\}
  - \frac{1}{\sigma}
    \bigl(
      i_t - E_t\{\pi_{t+1}\} - \rho
    \bigr)
  + \frac{1}{\sigma}(1 - \rho_z)\,z_t .
  \tag{10}\label{eq:euler10}
\end{equation}

\paragraph{Notación log-lineal.}
Recordamos que las letras minúsculas denotan logaritmos:
\[
  c_t \equiv \log C_t, \quad
  p_t \equiv \log P_t, \quad
  z_t \equiv \log Z_t.
\]
La inflación es
\[
  \pi_{t+1} \equiv p_{t+1} - p_t,
\]
y la tasa de interés nominal se define a partir del precio del bono
$Q_t$ como
\[
  I_t \equiv \frac{1}{Q_t}
  \quad\Rightarrow\quad
  i_t \equiv \log I_t = - \log Q_t .
\]
Además, definimos
\[
  \rho \equiv - \log \beta,
\]
de modo que $\rho$ es la “tasa de descuento” en términos logarítmicos.

\paragraph{Desglose de términos en \eqref{eq:euler10}.}

\begin{table}[H]
  \centering
  \small
  \begin{tabular}{l p{0.68\textwidth}}
    \hline
    Símbolo & Definición y papel económico \\
    \hline
    $c_t$ 
      & $\log C_t$. Consumo actual (en logaritmos). \\
    $E_t\{c_{t+1}\}$ 
      & Consumo futuro esperado. Resume las expectativas del hogar sobre su senda de consumo. \\
    $i_t$ 
      & Logaritmo de la tasa de interés nominal bruta: $i_t = -\log Q_t$. Es el costo de oportunidad de consumir hoy en lugar de ahorrar. \\
    $E_t\{\pi_{t+1}\}$ 
      & Inflación esperada entre $t$ y $t+1$, donde $\pi_{t+1} = p_{t+1} - p_t$. Mide la pérdida esperada de poder adquisitivo de los activos nominales. \\
    $\rho$ 
      & $\rho = -\log \beta$. Captura la impaciencia del hogar: cuanto mayor es $\rho$, más valora el consumo presente respecto al futuro. \\
    $z_t$ 
      & $\log Z_t$. Choque de preferencias (``shifter''), que modifica la utilidad marginal del consumo. \\
    $\sigma$ 
      & Curvatura de la utilidad del consumo. Controla la aversión al riesgo y la elasticidad de sustitución intertemporal ($1/\sigma$). \\
    $\rho_z$ 
      & Parámetro de persistencia del proceso AR(1) de $z_t$: $z_t = \rho_z z_{t-1} + \varepsilon_{z,t}$. \\
    \hline
  \end{tabular}
\end{table}

\subsubsection*{Derivación paso a paso}

\paragraph{1. Reescribir la ecuación de Euler en términos logarítmicos.}

Partimos de \eqref{eq:euler8} y usamos que
\[
  \frac{C_{t+1}}{C_t}
  = \exp(c_{t+1}-c_t),\quad
  \frac{Z_{t+1}}{Z_t}
  = \exp(z_{t+1}-z_t),\quad
  \frac{P_t}{P_{t+1}}
  = \exp(-\pi_{t+1}).
\]
Sustituyendo en \eqref{eq:euler8}:
\[
  Q_t
  =
  \beta E_t\left\{
    \exp\bigl(
      -\sigma(c_{t+1}-c_t)
      + (z_{t+1}-z_t)
      - \pi_{t+1}
    \bigr)
  \right\}.
\]

Definimos $i_t \equiv -\log Q_t$ y $\rho \equiv -\log \beta$. Entonces
$\log Q_t = -i_t$ y $\log \beta = -\rho$, y podemos escribir
\[
  \exp(-i_t)
  =
  \exp(-\rho)\,
  E_t\left\{
    \exp\bigl(
      -\sigma(c_{t+1}-c_t)
      + (z_{t+1}-z_t)
      - \pi_{t+1}
    \bigr)
  \right\}.
\]

Multiplicando ambos lados por $\exp(\rho)$:
\[
  \exp(\rho - i_t)
  =
  E_t\left\{
    \exp\bigl(
      -\sigma(c_{t+1}-c_t)
      + (z_{t+1}-z_t)
      - \pi_{t+1}
    \bigr)
  \right\}.
\]

\textit{(Nota didáctica: aquí simplemente hemos pasado de una igualdad en
términos de $Q_t$ y $\beta$ a una igualdad en términos de sus logaritmos
$i_t$ y $\rho$. Esto facilita la aproximación lineal posterior.)}

\paragraph{2. Aproximación de primer orden (log-linealización).}

El siguiente paso consiste en aproximar la expresión anterior alrededor de un
estado estacionario con inflación y crecimiento del consumo constantes. Para
pequeñas desviaciones en torno a ese estado estacionario, usamos que
\[
  \exp(x) \approx 1 + x 
  \quad\text{cuando $x$ es pequeño.}
\]

La igualdad anterior se puede reescribir de forma esquemática como:
\[
  1
  =
  E_t\left\{
    \exp\bigl(
      i_t - \rho
      -\sigma(c_{t+1}-c_t)
      - \pi_{t+1}
      + (z_{t+1}-z_t)
    \bigr)
  \right\},
\]
donde hemos reorganizado términos para que el lado izquierdo sea $1$.
Aplicando la aproximación de primer orden $\exp(x)\approx 1+x$ y tomando
desviaciones respecto al estado estacionario, obtenemos aproximadamente:
\begin{equation}
  0
  \;\approx\;
  E_t\left\{
    i_t - \rho
    - \sigma(c_{t+1}-c_t)
    - \pi_{t+1}
    + (z_{t+1}-z_t)
  \right\}.
  \label{eq:euler_bruta_lin}
\end{equation}

\textit{(Nota didáctica: la ecuación \eqref{eq:euler_bruta_lin} dice que,
en promedio, la combinación lineal de la tasa de interés, el crecimiento
del consumo, la inflación y el choque de preferencias debe ser cercana a cero
si el hogar está optimizando intertemporalmente.)}

\paragraph{3. Reorganizar en términos de consumo.}

Tomando expectativas condicionales a la información en $t$,
\eqref{eq:euler_bruta_lin} se puede escribir como:
\[
  0
  \approx
  i_t - \rho
  - \sigma\bigl(E_t\{c_{t+1}\} - c_t\bigr)
  - E_t\{\pi_{t+1}\}
  + E_t\{z_{t+1}\} - z_t .
\]

Despejamos el término de consumo:
\begin{align*}
  \sigma\bigl(E_t\{c_{t+1}\} - c_t\bigr)
  &\approx
  i_t - E_t\{\pi_{t+1}\} - \rho
  + E_t\{z_{t+1}\} - z_t, \\
  E_t\{c_{t+1}\} - c_t
  &\approx
  \frac{1}{\sigma}
  \left[
    i_t - E_t\{\pi_{t+1}\} - \rho
    + E_t\{z_{t+1}\} - z_t
  \right].
\end{align*}

\paragraph{4. Uso del proceso AR(1) para el choque de preferencias.}

Suponemos que el choque de preferencias $z_t$ sigue un proceso AR(1):
\[
  z_t = \rho_z z_{t-1} + \varepsilon_{z,t},
\]
de donde se obtiene
\[
  E_t\{z_{t+1}\} = \rho_z z_t
  \quad\Rightarrow\quad
  E_t\{z_{t+1}\} - z_t = (\rho_z - 1) z_t = - (1-\rho_z) z_t.
\]

Sustituyendo en la expresión anterior:
\[
  E_t\{c_{t+1}\} - c_t
  \approx
  \frac{1}{\sigma}
  \left[
    i_t - E_t\{\pi_{t+1}\} - \rho
    - (1-\rho_z) z_t
  \right].
\]

Cambiamos de lado $c_t$ y multiplicamos por $-1$:
\[
  c_t
  \approx
  E_t\{c_{t+1}\}
  - \frac{1}{\sigma}
    \bigl(i_t - E_t\{\pi_{t+1}\} - \rho\bigr)
  + \frac{1}{\sigma}(1-\rho_z)\,z_t,
\]
que es precisamente la ecuación (10),
\eqref{eq:euler10}.

\subsubsection*{Interpretación económica de la ecuación (10)}

Es útil reescribir \eqref{eq:euler10} usando la tasa de interés real esperada:
\[
  r_t \equiv i_t - E_t\{\pi_{t+1}\}.
\]
Entonces:
\begin{equation}
  c_t
  =
  E_t\{c_{t+1}\}
  - \frac{1}{\sigma}(r_t - \rho)
  + \frac{1}{\sigma}(1-\rho_z)\,z_t .
  \label{eq:euler10_real}
\end{equation}

\begin{itemize}
  \item El término $r_t - \rho$ mide cuánto se desvía la tasa de interés
        real esperada $r_t$ de la tasa de descuento del hogar $\rho$.
        Si $r_t$ aumenta por encima de $\rho$, el término
        $-\frac{1}{\sigma}(r_t - \rho)$ es negativo:
        el hogar reduce el consumo actual $c_t$ en relación con el consumo
        futuro esperado $E_t\{c_{t+1}\}$ para aprovechar el mayor rendimiento
        del ahorro. Esto es la \textbf{sustitución intertemporal del consumo}.
  \item El término $\frac{1}{\sigma}(1-\rho_z)\,z_t$ refleja el efecto del
        choque de preferencias. Si $z_t$ aumenta (el hogar valora más el
        consumo actual) y $0 \le \rho_z < 1$, el término es positivo: 
        para que la condición de Euler vuelva a cumplirse, el consumo actual
        $c_t$ debe aumentar, reduciendo así la utilidad marginal del consumo.
  \item El parámetro $\sigma$ aparece en el denominador: cuanto mayor es
        $\sigma$ (mayor aversión al riesgo y menor elasticidad de sustitución
        intertemporal), \emph{menor} es la respuesta de $c_t$ a un cambio dado
        en $(r_t - \rho)$ o en $z_t$. Es decir, las decisiones de consumo son
        menos sensibles a la tasa de interés real y a los choques de
        preferencias.
\end{itemize}

\textit{En resumen:} la ecuación (10) muestra cómo el consumo actual se ajusta
para equilibrar tres fuerzas: (i) las expectativas de consumo futuro,
(ii) la comparación entre la tasa de interés real esperada y la tasa de
descuento del hogar y (iii) los choques de preferencias que hacen más o menos
atractivo consumir hoy.

\subsubsection*{Ecuación (11): demanda de saldos reales log-linealizada}

En la versión del modelo donde la política monetaria se formula en términos
de la oferta de dinero, es necesario introducir una ecuación de demanda de
saldos reales. De forma log-lineal (ignorando una constante aditiva para no
cargar la notación), Galí postula:
\begin{equation}
  m_t - p_t = c_t - \eta i_t .
  \tag{11}\label{eq:money_demand}
\end{equation}

\paragraph{Notación y símbolos.}

Recordamos que las letras minúsculas denotan logaritmos naturales de las
variables correspondientes:
\[
  m_t \equiv \log M_t, \quad
  p_t \equiv \log P_t, \quad
  c_t \equiv \log C_t,
  \quad
  i_t \equiv \log I_t,
\]
donde $M_t$ es la cantidad de dinero nominal, $P_t$ el nivel de precios,
$C_t$ el consumo y $I_t$ la tasa de interés nominal bruta.

\begin{table}[H]
  \centering
  \small
  \begin{tabularx}{0.9\textwidth}{lX}
    \hline
    Símbolo & Definición y papel económico \\
    \hline
    $m_t$ 
      & $\log M_t$. Logaritmo de la cantidad nominal de dinero. \\
    $p_t$ 
      & $\log P_t$. Logaritmo del nivel de precios. \\
    $m_t - p_t$ 
      & $\log(M_t/P_t)$. Saldos reales demandados: poder adquisitivo que los
        agentes mantienen en forma de dinero. \\
    $c_t$ 
      & $\log C_t$. Logaritmo del consumo (escala de la actividad económica).
        Funciona como proxy de ingreso / volumen de transacciones. \\
    $i_t$ 
      & Logaritmo de la tasa de interés nominal bruta. Representa el
        \emph{costo de oportunidad} de mantener dinero en lugar de bonos. \\
    $\eta$ (eta)
      & \emph{Semielasticidad de la demanda de dinero respecto a la tasa de
        interés}, con $\eta \ge 0$. Mide cuán sensible es la demanda de
        saldos reales ante cambios en $i_t$. \\
    \hline
  \end{tabularx}
\end{table}

\subsubsection*{De dónde sale la forma funcional}

En esta sección del capítulo, la ecuación \eqref{eq:money_demand} se introduce
de manera \emph{reducida}: no se deriva explícitamente de un problema de
optimización, sino que se postula como una relación empíricamente razonable
entre saldos reales, nivel de actividad y tasa de interés.

Una forma estándar de partir es una función de demanda de dinero del tipo:
\[
  \frac{M_t}{P_t}
  = \Phi(C_t, i_t),
\]
donde:
\begin{itemize}
  \item $\Phi$ crece con el nivel de gasto/transacciones ($C_t$),
  \item $\Phi$ decrece con la tasa de interés nominal ($i_t$), que mide el
        rendimiento de los activos alternativos al dinero.
\end{itemize}

Si se asume una forma log-lineal simple:
\begin{equation}
  \log\left(\frac{M_t}{P_t}\right)
  = \alpha + \psi\,c_t - \eta\,i_t,
  \label{eq:money_demand_general}
\end{equation}
con $\psi > 0$ y $\eta \ge 0$, entonces:

- el término $\psi c_t$ recoge el \textbf{motivo transacción}:
  más consumo $\Rightarrow$ más dinero demandado;
- el término $-\eta i_t$ recoge el \textbf{motivo de costo de oportunidad}:
  si sube $i_t$, mantener dinero es más caro $\Rightarrow$ menos saldos reales.

En el texto se simplifica esta expresión normalizando la pendiente respecto a
$c_t$ a la unidad ($\psi = 1$) y absorbiendo la constante $\alpha$ en el
nivel estacionario de las variables. Con esa normalización, la ecuación
\eqref{eq:money_demand_general} se reduce exactamente a:
\[
  m_t - p_t = c_t - \eta i_t,
\]
que es la ecuación (11).

\subsubsection*{Interpretación económica didáctica}

La ecuación \eqref{eq:money_demand} puede leerse como:
\[
  \text{(saldos reales demandados)}
  =
  \text{(escala de la actividad)}
  -
  \text{(penalización por el costo de oportunidad del dinero)}.
\]

\begin{itemize}
  \item \textbf{Efecto del consumo ($c_t$).}  
        Si $c_t$ aumenta (el hogar está consumiendo más, la economía está
        más activa), la demanda de saldos reales aumenta uno a uno:
        \[
          \frac{\partial (m_t - p_t)}{\partial c_t} = 1.
        \]
        Didácticamente: más compras $\Rightarrow$ hace falta más dinero
        en la cartera para efectuar esas transacciones.

  \item \textbf{Efecto de la tasa de interés nominal ($i_t$).}  
        El término $-\eta i_t$ indica que, si sube $i_t$, los saldos reales
        demandados caen:
        \[
          \frac{\partial (m_t - p_t)}{\partial i_t} = -\eta < 0.
        \]
        Intuición: a mayor tasa de interés, más caro es “tener el dinero
        parado” en saldos líquidos en lugar de colocarlo en bonos que pagan
        interés. Entonces los hogares reducen su demanda de dinero.

  \item \textbf{El papel de $\eta$ (eta).}  
        Cuanto mayor es $\eta$, más sensible es la demanda de dinero al costo
        de oportunidad. Una $\eta$ pequeña describe una demanda de dinero
        poco sensible a $i_t$ (dinero casi “necesario” independientemente del
        interés); una $\eta$ grande describe una demanda que responde mucho
        a los cambios en la tasa de interés.
\end{itemize}

\subsubsection*{Comentario sobre microfundamentos}

Más adelante en el capítulo (cuando se introduce una función de utilidad que
incluye explícitamente los saldos reales, o una restricción de tipo
\emph{cash-in-advance}), puede mostrarse que una ecuación de la forma
\eqref{eq:money_demand} aparece como condición de primer orden del problema
del hogar. Es decir, en una versión más rica del modelo, la demanda de dinero
no es sólo una ecuación “ad hoc”, sino el resultado de maximizar utilidad
sujeto a restricciones tecnológicas y financieras.

Para los fines de la parte básica del capítulo, sin embargo, basta trabajar
con la versión reducida:
\[
  m_t - p_t = c_t - \eta i_t,
\]
que permite cerrar el modelo cuando la autoridad monetaria fija una senda
para la oferta de dinero en lugar de fijar directamente la tasa de interés.

\section*{2.2 Firmas: tecnología y producción}

En esta sección pasamos del problema de los hogares al comportamiento
de las \emph{firmas} (empresas). En el modelo, opera un número muy grande
de firmas idénticas que producen un único bien de consumo. Cada firma
toma como dados el precio de ese bien ($P_t$) y el salario nominal
($W_t$) y elige cuánto trabajar (empleo $N_t$) para maximizar sus
beneficios en cada período.

\subsubsection*{Función de producción (Ecuación 12)}

La tecnología disponible para la firma representativa se describe mediante
una función de producción Cobb--Douglas con un único factor variable
(trabajo):
\begin{equation}
  Y_t = A_t N_t^{1-\alpha},
  \tag{12}\label{eq:produccion}
\end{equation}
donde:

\begin{itemize}
  \item $Y_t$ es el nivel de producción (output) en el período $t$.
  \item $N_t$ es el empleo (horas trabajadas) en el período $t$.
  \item $A_t$ es el nivel de productividad total de los factores
        (\emph{Total Factor Productivity}, TFP) en $t$.
  \item $\alpha$ (alfa) es un parámetro con $0 < \alpha < 1$.
\end{itemize}

\paragraph{Lectura económica de \eqref{eq:produccion}.}

\begin{itemize}
  \item La exponente $1-\alpha$ es la \textbf{elasticidad del producto
        respecto al trabajo}.\footnote{En una versión más completa del modelo,
        puede pensarse que existe también un factor capital con exponente
        $\alpha$, de modo que la suma de exponentes es 1 y hay rendimientos
        constantes a escala. Aquí el capital se mantiene implícito y fijo.}
        Si el empleo aumenta un 1\%, la producción aumenta aproximadamente
        $(1-\alpha)\%$.
  \item $A_t$ multiplica a toda la función: un aumento de $A_t$ eleva la
        producción para cualquier nivel dado de empleo $N_t$. Es un
        \textbf{choque de productividad}: la tecnología permite producir
        más con la misma cantidad de trabajo.
\end{itemize}

\subsubsection*{Proceso estocástico de la productividad}

Como en el caso de las preferencias de los hogares, se supone que el nivel
de productividad $A_t$ es \emph{exógeno} para las firmas: ninguna empresa
lo controla individualmente. Trabajamos con su logaritmo:
\[
  a_t \equiv \log A_t,
\]
y se asume que sigue un proceso autorregresivo de orden 1 (AR(1)):
\begin{equation}
  a_t = \rho_a a_{t-1} + \varepsilon_{a,t},
  \label{eq:shock_productividad}
\end{equation}
donde:
\begin{itemize}
  \item $\rho_a$ es el parámetro de persistencia del shock tecnológico,
        con $|\rho_a| < 1$;
  \item $\varepsilon_{a,t}$ es una innovación (shock) de productividad en
        el período $t$, con media cero.
\end{itemize}

\paragraph{Intuición didáctica.}

\begin{itemize}
  \item Si $\varepsilon_{a,t} > 0$, entonces $a_t$ aumenta, y por tanto
        $A_t = e^{a_t}$ es más alto: con la misma cantidad de trabajo, la
        firma puede producir más $Y_t$.
  \item Si $\rho_a$ es grande (por ejemplo, $\rho_a \approx 0.9$), los
        efectos de un shock de productividad se \emph{propagan en el tiempo}:
        un aumento de productividad hoy implica un nivel de productividad
        relativamente alto también mañana y en los periodos siguientes.
\end{itemize}

\subsubsection*{Versión log-lineal de la tecnología}

Tomando logaritmos en la ecuación de producción \eqref{eq:produccion}:
\[
  \log Y_t = \log A_t + (1-\alpha)\,\log N_t.
\]

Usando la notación log-lineal estándar del modelo:
\[
  y_t \equiv \log Y_t,
  \quad
  n_t \equiv \log N_t,
  \quad
  a_t \equiv \log A_t,
\]
podemos escribir simplemente:
\begin{equation}
  y_t = a_t + (1-\alpha)\,n_t.
  \label{eq:produccion_log}
\end{equation}

Esta versión logarítmica es muy útil porque:
\begin{itemize}
  \item hace explícitas las elasticidades (el coeficiente $1-\alpha$ es
        directamente la elasticidad del producto respecto al empleo),
  \item y facilita las log-linealizaciones posteriores cuando combinemos
        esta ecuación con las condiciones de optimalidad de hogares y firmas.
\end{itemize}

\subsubsection*{Problema de maximización de beneficios de la firma}

Cada firma toma como dados el precio del bien $P_t$ y el salario nominal
$W_t$, y elige cuánto empleo $N_t$ demandar para maximizar sus beneficios
en el período $t$:
\begin{equation}
  \max_{N_t} \;\; \Pi_t 
  \equiv P_t Y_t - W_t N_t
  \quad\text{sujeto a}\quad
  Y_t = A_t N_t^{1-\alpha}.
  \label{eq:beneficios_firma}
\end{equation}

Sustituyendo la tecnología en la función de beneficios:
\begin{equation}
  \max_{N_t} \;\; 
  \Pi_t = P_t A_t N_t^{1-\alpha} - W_t N_t.
  \label{eq:beneficios_sustituida}
\end{equation}

\paragraph{Condición de primer orden (idea general).}

Para encontrar el empleo óptimo $N_t^\ast$, derivamos la función de
beneficios respecto a $N_t$ e igualamos a cero:
\[
  \frac{\partial \Pi_t}{\partial N_t}
  = P_t A_t (1-\alpha) N_t^{-\alpha} - W_t = 0.
\]

Reordenando:
\[
  P_t (1-\alpha) A_t N_t^{-\alpha} = W_t.
\]

Dividiendo ambos lados entre $P_t$ obtenemos:
\[
  (1-\alpha) A_t N_t^{-\alpha} = \frac{W_t}{P_t}.
\]

El lado izquierdo es el \textbf{producto marginal del trabajo} (en términos
reales), y el lado derecho es el \textbf{salario real}. En palabras:

\begin{center}
  \emph{En un equilibrio competitivo, la firma contrata trabajo hasta el
  punto en que el producto marginal del trabajo es igual al salario real.}
\end{center}

En la siguiente ecuación (Ecuación 14 del capítulo) escribiremos esta
condición de forma más compacta y, posteriormente, en forma log-lineal
para poder integrarla con el resto del sistema del modelo.


\subsubsection*{De la condición de primer orden en niveles (14) a su forma logarítmica (15)}

En la sección de firmas vimos que, al maximizar beneficios en competencia
perfecta, la condición de primer orden con respecto al empleo exige que
el salario real sea igual al producto marginal del trabajo. En niveles,
esto se escribe como:
\begin{equation}
  \frac{W_t}{P_t}
  = (1-\alpha) A_t N_t^{-\alpha},
  \tag{14}\label{eq:FOC_firma_niveles}
\end{equation}
donde:
\begin{itemize}
  \item $W_t$ es el salario nominal.
  \item $P_t$ es el nivel de precios.
  \item $\dfrac{W_t}{P_t}$ es el salario real.
  \item $A_t$ es el nivel de productividad (TFP).
  \item $N_t$ es el empleo.
  \item $\alpha$ (alfa) es el parámetro de participación del factor
        no observado (capital) en la función Cobb--Douglas.
\end{itemize}

La ecuación \eqref{eq:FOC_firma_niveles} dice:
\emph{la firma contrata trabajo hasta que el salario real iguala el producto
marginal del trabajo}, dado por $(1-\alpha) A_t N_t^{-\alpha}$.

\paragraph{Paso a paso: aplicación del logaritmo natural.}

El objetivo es obtener una versión \emph{log-lineal} de esta condición,
que será muy útil para combinarla con las ecuaciones de los hogares
y las curvas de oferta y demanda agregadas. Para ello tomamos logaritmos
a ambos lados de \eqref{eq:FOC_firma_niveles}.

\medskip

\noindent\textbf{Paso 1.} Aplicar $\log$ a ambos lados:
\begin{equation}
  \log\!\left( \frac{W_t}{P_t} \right)
  = \log\!\left( (1-\alpha) A_t N_t^{-\alpha} \right).
  \label{eq:log_FOC_paso1}
\end{equation}

\noindent\textbf{Paso 2.} Lado izquierdo (salario real).

Usamos la propiedad $\log(X/Y) = \log X - \log Y$:
\[
  \log\!\left( \frac{W_t}{P_t} \right)
  = \log W_t - \log P_t.
\]

\noindent\textbf{Paso 3.} Lado derecho (producto marginal del trabajo).

Aplicamos dos reglas básicas de logaritmos:
\begin{itemize}
  \item $\log(XY) = \log X + \log Y$,
  \item $\log(X^k) = k\,\log X$.
\end{itemize}
Primero, separamos el producto:
\[
  \log\!\left( (1-\alpha) A_t N_t^{-\alpha} \right)
  = \log(1-\alpha) + \log A_t + \log(N_t^{-\alpha}).
\]
Luego usamos $\log(N_t^{-\alpha}) = -\alpha \log N_t$:
\[
  \log(1-\alpha) + \log A_t - \alpha \log N_t.
\]

\noindent\textbf{Paso 4.} Sustituir notación logarítmica.

Definimos, como en el resto del capítulo, las letras minúsculas como
logaritmos naturales de las variables:
\[
  w_t \equiv \log W_t,\qquad
  p_t \equiv \log P_t,\qquad
  a_t \equiv \log A_t,\qquad
  n_t \equiv \log N_t.
\]

Con esta notación, el lado izquierdo se escribe como
\[
  \log W_t - \log P_t = w_t - p_t,
\]
y el lado derecho como
\[
  \log(1-\alpha) + a_t - \alpha n_t.
\]

Igualando ambos lados obtenemos:
\begin{equation}
  w_t - p_t = a_t - \alpha n_t + \log(1-\alpha).
  \tag{15}\label{eq:FOC_firma_log}
\end{equation}

\subsubsection*{Interpretación didáctica de la ecuación (15)}

La ecuación \eqref{eq:FOC_firma_log} es la \textbf{función de demanda de
trabajo de la firma en términos logarítmicos}:
\[
  w_t - p_t = a_t - \alpha n_t + \log(1-\alpha).
\]

Podemos leerla así:

\begin{itemize}
  \item \textbf{Relación inversa salario real--empleo.}  
        El coeficiente de $n_t$ es $-\alpha < 0$.  
        Para un nivel dado de productividad $a_t$, si el empleo $n_t$
        aumenta, el término $-\alpha n_t$ se hace más negativo, de modo que
        el salario real de equilibrio $w_t - p_t$ debe ser menor.
        \begin{quote}
          \emph{Cuanto más trabajo contrata la firma, menor es el producto
          marginal del último trabajador, y por tanto menor es el salario
          real que está dispuesta a pagar.}
        \end{quote}
  \item \textbf{Efecto de la productividad $a_t$.}  
        El salario real se mueve en la misma dirección que $a_t$:
        un aumento de productividad desplaza la curva de demanda
        de trabajo hacia arriba (la firma está dispuesta a pagar
        un salario real más alto para cada nivel de empleo).
  \item \textbf{El término constante $\log(1-\alpha)$.}  
        Afecta sólo la posición vertical de la curva (la ordenada
        al origen) en el plano $(n_t,\ w_t - p_t)$, pero no la pendiente.
        En muchas aplicaciones empíricas o teóricas se reabsorbe este
        término en una constante general.
\end{itemize}

\paragraph{Comparación con la oferta de trabajo.}

Es útil notar que:
\begin{itemize}
  \item La ecuación de \textbf{oferta de trabajo} del hogar (ecuación (9))
        tiene pendiente positiva en el espacio
        $(n_t,\ w_t - p_t)$.
  \item La ecuación de \textbf{demanda de trabajo} de la firma (ecuación (15))
        tiene pendiente negativa ($-\alpha$).
\end{itemize}

\noindent
El equilibrio en el mercado laboral se obtiene precisamente en el punto
en el que ambas rectas se cruzan: allí coincide el salario real que hace
óptima la decisión de trabajo

\subsubsection*{Ecuación (16): Condición de vaciado del mercado de bienes}

En el modelo clásico con un único bien de consumo y sin inversión,
gasto público ni comercio exterior, el equilibrio en el mercado de
bienes implica que toda la producción agregada se destina a consumo.
En niveles, esto se escribe simplemente como:
\[
  Y_t = C_t.
\]

Trabajando con la notación logarítmica utilizada en el resto del
capítulo, definimos:
\[
  y_t \equiv \log Y_t, \qquad
  c_t \equiv \log C_t.
\]

Al tomar logaritmos en ambos lados de la identidad $Y_t = C_t$ obtenemos:
\begin{equation}
  y_t = c_t.
  \tag{16}\label{eq:clearing_bienes}
\end{equation}

\paragraph{Significado de cada término.}

\begin{itemize}
  \item $y_t$ representa el \emph{logaritmo de la producción total}
        (oferta agregada de bienes).
  \item $c_t$ representa el \emph{logaritmo del consumo total}
        (demanda agregada de bienes).
  \item La igualdad $y_t = c_t$ expresa que, en cada período $t$,
        \emph{la cantidad de bienes producidos es exactamente igual
        a la cantidad de bienes demandados para consumo}.
\end{itemize}

\paragraph{Interpretación económica.}

La ecuación \eqref{eq:clearing_bienes} es la versión log-lineal de la
identidad:
\[
  Y_t^{\text{oferta}} = C_t^{\text{demanda}}.
\]

En una economía más general tendríamos:
\[
  Y_t = C_t + I_t + G_t + NX_t,
\]
donde $I_t$ es inversión, $G_t$ gasto público y $NX_t$ exportaciones
netas. En este modelo básico, se \emph{abstrae} de estos componentes:
$I_t = G_t = NX_t = 0$. Por eso, toda la producción se destina a
consumo, y la condición de equilibrio en el mercado de bienes se reduce
a $Y_t = C_t$.

\medskip

Desde el punto de vista de la solución del modelo, la ecuación
\eqref{eq:clearing_bienes} cumple un papel clave:

\begin{itemize}
  \item Permite \textbf{identificar producción y consumo}:
        sólo necesitamos determinar una de estas variables reales,
        porque la otra viene dada por la igualdad $y_t = c_t$.
  \item Al combinarla con:
        \begin{itemize}
          \item la oferta de trabajo del hogar (ecuación (9)),
          \item la demanda de trabajo de la firma (ecuación (15)),
          \item y la función de producción agregada (ecuación (17)),
        \end{itemize}
        se obtiene un sistema que determina de manera conjunta
        empleo, producción, salario real y tasa de interés real.
\end{itemize}

\noindent
Esta estructura es la que, en el modelo clásico con precios flexibles,
lleva al resultado de que las \emph{variables reales} (como $y_t$,
$n_t$ o el salario real) se determinan independientemente de la política
monetaria: el dinero sólo fija el nivel de precios, mientras que
la ecuación \eqref{eq:clearing_bienes} ayuda a cerrar el lado real
de la economía.

\subsubsection*{Ecuación (17): Relación de producción agregada log-lineal}

Partimos de la función de producción Cobb--Douglas de la firma
representativa:
\[
  Y_t = A_t N_t^{1-\alpha},
\]
donde:
\begin{itemize}
  \item $Y_t$ es el \emph{nivel de producción} (output) en el período $t$,
  \item $N_t$ es el \emph{empleo} o número de horas trabajadas,
  \item $A_t$ es el \emph{nivel de tecnología} o productividad total de los factores,
  \item $\alpha$ (alfa) $\in (0,1)$ es el parámetro asociado (implícitamente) al
        factor capital, por lo que $1-\alpha$ es la \emph{participación del trabajo}
        en el ingreso.
\end{itemize}

Trabajando en la notación log-lineal del modelo, definimos:
\[
  y_t \equiv \log Y_t, \qquad
  a_t \equiv \log A_t, \qquad
  n_t \equiv \log N_t.
\]

\paragraph{Derivación paso a paso.}

Aplicamos el logaritmo natural a ambos lados de la función de producción:
\begin{equation*}
  \log Y_t
    = \log\bigl( A_t N_t^{1-\alpha} \bigr).
\end{equation*}

Usamos ahora dos propiedades básicas de los logaritmos:
\begin{align*}
  \log(XY) &= \log X + \log Y, \\
  \log(X^k) &= k \log X.
\end{align*}

Aplicándolas obtenemos:
\begin{align*}
  \log Y_t
    &= \log A_t + \log\bigl( N_t^{1-\alpha} \bigr) \\
    &= \log A_t + (1-\alpha)\,\log N_t.
\end{align*}

Sustituyendo la notación log-lineal ($y_t, a_t, n_t$), llegamos a:
\begin{equation}
  y_t = a_t + (1-\alpha)\,n_t.
  \tag{17}\label{eq:produccion_loglineal}
\end{equation}

\paragraph{Interpretación económica.}

La ecuación \eqref{eq:produccion_loglineal} resume la tecnología de la
economía en términos logarítmicos:

\begin{itemize}
  \item $y_t$ es el \emph{logaritmo de la producción agregada}, es decir,
        la oferta total de bienes.
  \item $a_t$ es el \emph{logaritmo del nivel de tecnología} o choque de
        productividad. Un aumento en $a_t$ (por ejemplo, un avance tecnológico)
        desplaza hacia arriba la producción para cualquier nivel de empleo.
  \item $n_t$ es el \emph{logaritmo del empleo}. El coeficiente $(1-\alpha)$
        mide la sensibilidad (elasticidad) de la producción con respecto al
        trabajo: un incremento de $1\%$ en $N_t$ aumenta $Y_t$ en
        aproximadamente $(1-\alpha)\%$.
\end{itemize}

\noindent
En el contexto del equilibrio general del modelo clásico, la ecuación
\eqref{eq:produccion_loglineal} es una pieza clave porque conecta el
lado \emph{real} de la economía:
\begin{itemize}
  \item Por un lado, el empleo $n_t$ se determina a partir de la
        interacción entre la \emph{oferta de trabajo} del hogar
        (ecuación (9)) y la \emph{demanda de trabajo} de la firma
        (ecuación (15)).
  \item Por otro lado, la producción $y_t$ viene dada por la tecnología
        $a_t$ y por ese nivel de empleo a través de
        \eqref{eq:produccion_loglineal}.
\end{itemize}

\noindent
De este modo, la relación de producción log-lineal permite expresar
las variables reales de la economía (en particular $y_t$) en función
de choques tecnológicos ($a_t$) y de decisiones de empleo ($n_t$),
lo cual será fundamental para analizar cómo se determina el equilibrio
y, más adelante, para estudiar la neutralidad del dinero en este
entorno con precios totalmente flexibles.


\subsubsection*{Ecuación (18): Nivel de equilibrio del empleo}

A partir de las condiciones de optimalidad del hogar y de la firma, junto con
las condiciones de vaciado de mercado, podemos obtener una expresión cerrada
para el empleo de equilibrio. El resultado central es que el logaritmo del
empleo $n_t$ depende únicamente del logaritmo del nivel de tecnología $a_t$:
\begin{equation}
  n_t = \psi_{na}\, a_t + \psi_n.
  \tag{18}\label{eq:n_equilibrio}
\end{equation}

\noindent
En esta expresión:
\begin{itemize}
  \item $n_t \equiv \log N_t$ es el logaritmo del empleo (o del número de
        horas trabajadas).
  \item $a_t \equiv \log A_t$ es el logaritmo del nivel de tecnología
        (productividad total de los factores).
  \item $\psi_{na}$ es el coeficiente que mide la \emph{sensibilidad del empleo}
        ante cambios en la tecnología.
  \item $\psi_n$ es un término constante que recoge el efecto conjunto de los
        parámetros estructurales del modelo.
\end{itemize}

\paragraph{Parámetros estructurales.}

Recordamos el significado de los parámetros que intervienen:
\begin{itemize}
  \item $\sigma$ (sigma) $\ge 0$ es el parámetro de curvatura de la utilidad
        del consumo; mide la aversión al riesgo y el inverso de la elasticidad
        de sustitución intertemporal.
  \item $\phi$ (phi) $\ge 0$ es el parámetro asociado a la desutilidad del
        trabajo; es (aproximadamente) el inverso de la elasticidad Frisch de
        la oferta de trabajo.
  \item $\alpha$ (alfa) $\in (0,1)$ es el parámetro de participación del capital
        en la función de producción Cobb--Douglas; el trabajo tiene
        participación $1-\alpha$.
\end{itemize}

\subsubsection*{Derivación a partir del sistema real}

El sistema real que determina las variables \emph{reales} de la economía está
formado por:

\begin{itemize}
  \item Oferta de trabajo del hogar (ecuación (9)):
  \[
    w_t - p_t = \sigma\, c_t + \phi\, n_t.
  \]
  \item Demanda de trabajo de la firma (ecuación (15)):
  \[
    w_t - p_t = a_t - \alpha\, n_t + \log(1-\alpha).
  \]
  \item Condición de equilibrio del mercado de bienes (ecuación (16)):
  \[
    y_t = c_t.
  \]
  \item Relación de producción agregada (ecuación (17)):
  \[
    y_t = a_t + (1-\alpha)\, n_t.
  \]
\end{itemize}

\noindent
El salario real $w_t - p_t$ debe ser el mismo para hogares y firmas, de modo
que igualamos las ecuaciones (9) y (15):
\begin{equation}
  \sigma\, c_t + \phi\, n_t
    = a_t - \alpha\, n_t + \log(1-\alpha).
  \label{eq:mercado_laboral_eq}
\end{equation}

Por la condición de vaciado del mercado de bienes (16) y la tecnología (17),
podemos escribir el consumo como:
\[
  c_t = y_t = a_t + (1-\alpha)\, n_t.
\]

Sustituyendo esta expresión de $c_t$ en
\eqref{eq:mercado_laboral_eq} obtenemos:
\begin{align*}
  \sigma \bigl[ a_t + (1-\alpha)\, n_t \bigr] + \phi\, n_t
    &= a_t - \alpha\, n_t + \log(1-\alpha), \\[0.4em]
  \sigma a_t + \sigma(1-\alpha)\, n_t + \phi\, n_t
    &= a_t - \alpha\, n_t + \log(1-\alpha).
\end{align*}

Reagrupamos por términos en $n_t$ y en $a_t$:
\begin{align*}
  \bigl[\sigma(1-\alpha) + \phi + \alpha\bigr]\, n_t
    &= (1-\sigma)\, a_t + \log(1-\alpha).
\end{align*}

Resolviendo para $n_t$:
\begin{equation*}
  n_t
    = \frac{1-\sigma}{\sigma(1-\alpha) + \phi + \alpha}\, a_t
      + \frac{\log(1-\alpha)}{\sigma(1-\alpha) + \phi + \alpha}.
\end{equation*}

Definimos entonces:
\begin{align*}
  \psi_{na} &\equiv \frac{1-\sigma}{\sigma(1-\alpha) + \phi + \alpha}, \\
  \psi_n   &\equiv \frac{\log(1-\alpha)}{\sigma(1-\alpha) + \phi + \alpha},
\end{align*}
lo que nos lleva directamente a la forma compacta de la ecuación
\eqref{eq:n_equilibrio}:
\[
  n_t = \psi_{na}\, a_t + \psi_n.
\]

\subsubsection*{Interpretación económica de la Ecuación (18)}

La expresión \eqref{eq:n_equilibrio} contiene varios resultados clave del
modelo clásico:

\begin{enumerate}
  \item \textbf{El empleo depende sólo de la tecnología.}

  \noindent
  Obsérvese que en \eqref{eq:n_equilibrio} \emph{no} aparecen:
  \begin{itemize}
    \item la tasa de interés nominal $i_t$,
    \item la tasa de inflación esperada $E_t\{\pi_{t+1}\}$,
    \item ni el choque de preferencias $z_t$.
  \end{itemize}
  Esto refleja que, con precios totalmente flexibles y mercados competitivos,
  las variables \emph{reales} (como el empleo $n_t$) se determinan únicamente
  por la tecnología $a_t$ y los parámetros de preferencias y tecnología
  $(\sigma, \phi, \alpha)$: es la idea de \emph{neutralidad de la política
  monetaria} en el modelo clásico.

  \item \textbf{El papel de $\psi_{na}$: respuesta del empleo a choques
        tecnológicos.}

  \noindent
  El coeficiente
  \[
    \psi_{na} = \frac{1-\sigma}{\sigma(1-\alpha) + \phi + \alpha}
  \]
  mide la sensibilidad del empleo a cambios en la productividad $a_t$.

  \begin{itemize}
    \item Si $\sigma < 1$ (sigma menor que uno), entonces $1-\sigma > 0$ y
          $\psi_{na} > 0$. Un aumento de la productividad ($a_t$ más alto)
          incrementa el salario real y el efecto sustitución (trabajar más) 
          domina al efecto ingreso (trabajar menos por ser más rico), de modo
          que el empleo $n_t$ \emph{aumenta}.
    \item Si $\sigma > 1$, entonces $1-\sigma < 0$ y $\psi_{na} < 0$.
          En este caso, el efecto ingreso domina: un choque tecnológico positivo
          hace que los hogares estén dispuestos a trabajar menos, por lo que
          el empleo puede \emph{disminuir}.
    \item Si $\sigma = 1$ (utilidad logarítmica en consumo), se tiene
          $1-\sigma = 0$ y, por tanto, $\psi_{na} = 0$. En este caso, los
          efectos ingreso y sustitución se cancelan exactamente y el empleo
          de equilibrio no responde a la productividad: $n_t$ es constante.
  \end{itemize}

  \item \textbf{El término constante $\psi_n$.}

  \noindent
  La constante
  \[
    \psi_n = \frac{\log(1-\alpha)}{\sigma(1-\alpha) + \phi + \alpha}
  \]
  fija el \emph{nivel medio} de empleo de equilibrio, dado el conjunto de
  parámetros estructurales. Un cambio en la participación del trabajo
  $(1-\alpha)$, o en la rigidez de la oferta de trabajo (capturada por
  $\phi$), desplaza hacia arriba o hacia abajo toda la relación entre $n_t$
  y $a_t$.
\end{enumerate}

\noindent
La ecuación \eqref{eq:n_equilibrio} sintetiza el equilibrio del
mercado laboral en el modelo clásico: el empleo se ajusta únicamente a choques
de productividad, y la política monetaria no tiene efecto permanente sobre
esta variable real.


\subsubsection*{Ecuación (19): Nivel de equilibrio del producto}

Una vez determinado el empleo de equilibrio, podemos caracterizar el
\emph{producto} de equilibrio. La Ecuación (19) expresa el logaritmo del
\emph{output} $y_t$ como función lineal del único choque real del modelo:
el nivel de tecnología $a_t$:
\begin{equation}
  y_t = \psi_{ya}\, a_t + \psi_y.
  \tag{19}\label{eq:y_equilibrio}
\end{equation}

\noindent
Donde:
\begin{itemize}
  \item $y_t \equiv \log Y_t$ es el logaritmo del producto agregado.
  \item $a_t \equiv \log A_t$ es el logaritmo del nivel de tecnología
        (choque de productividad).
  \item $\psi_{ya}$ es el coeficiente (o \emph{multiplicador}) que mide
        la sensibilidad del producto ante cambios en la tecnología.
  \item $\psi_y$ es un término constante que fija el nivel medio de
        producto en equilibrio.
\end{itemize}

\paragraph{Parámetros estructurales.}

Recordamos el significado de los parámetros que intervienen en los
coeficientes:
\begin{itemize}
  \item $\sigma$ (sigma) $\ge 0$:
        parámetro de curvatura de la utilidad del consumo; mide la aversión
        al riesgo y el inverso de la elasticidad de sustitución intertemporal.
  \item $\phi$ (phi) $\ge 0$:
        parámetro asociado a la desutilidad del trabajo; es (aprox.) el
        inverso de la elasticidad Frisch de la oferta laboral.
  \item $\alpha$ (alfa) $\in (0,1)$:
        parámetro tecnológico de la función de producción Cobb--Douglas;
        el trabajo tiene participación $(1-\alpha)$.
  \item $\psi_{na}$ y $\psi_n$ son los coeficientes obtenidos en la ecuación
        de equilibrio del empleo:
        \[
          n_t = \psi_{na}\, a_t + \psi_n,
        \]
        con
        \[
          \psi_{na} = \frac{1-\sigma}{\sigma(1-\alpha)+\phi+\alpha},
          \qquad
          \psi_n   = \frac{\log(1-\alpha)}{\sigma(1-\alpha)+\phi+\alpha}.
        \]
\end{itemize}

\subsubsection*{Derivación a partir de la tecnología agregada}

Partimos de la relación de producción agregada log-lineal (ecuación (17)):
\begin{equation}
  y_t = a_t + (1-\alpha)\, n_t.
  \label{eq:produccion_log}
\end{equation}

\noindent
Y usamos el resultado para el empleo de equilibrio (ecuación (18)):
\[
  n_t = \psi_{na}\, a_t + \psi_n.
\]

\noindent
Sustituyendo $n_t$ en \eqref{eq:produccion_log}:
\begin{align*}
  y_t
    &= a_t + (1-\alpha)\bigl(\psi_{na}\, a_t + \psi_n\bigr) \\
    &= \bigl[1 + (1-\alpha)\psi_{na}\bigr]\, a_t
       + (1-\alpha)\, \psi_n.
\end{align*}

Definimos entonces:
\begin{align}
  \psi_{ya} &\equiv 1 + (1-\alpha)\psi_{na},
  \label{eq:psiya_def_general} \\[0.3em]
  \psi_y    &\equiv (1-\alpha)\, \psi_n.
  \label{eq:psiy_def_general}
\end{align}

Sustituyendo las expresiones explícitas de $\psi_{na}$ y $\psi_n$:
\[
  \psi_{na} = \frac{1-\sigma}{\sigma(1-\alpha)+\phi+\alpha},
  \qquad
  \psi_n   = \frac{\log(1-\alpha)}{\sigma(1-\alpha)+\phi+\alpha},
\]
obtenemos, tras simplificar,
\begin{equation}
  \psi_{ya}
    = 1 + (1-\alpha)\,\frac{1-\sigma}{\sigma(1-\alpha)+\phi+\alpha}
    = \frac{1+\phi}{\sigma(1-\alpha)+\phi+\alpha},
  \label{eq:psiya_explicita}
\end{equation}
y
\begin{equation}
  \psi_y
    = (1-\alpha)\, \psi_n
    = (1-\alpha)\,
      \frac{\log(1-\alpha)}{\sigma(1-\alpha)+\phi+\alpha}.
  \label{eq:psiy_explicita}
\end{equation}

\noindent
Sustituyendo \eqref{eq:psiya_explicita} y \eqref{eq:psiy_explicita} en
\eqref{eq:y_equilibrio} tenemos la forma explícita de la Ecuación (19):
\[
  y_t
    = \frac{1+\phi}{\sigma(1-\alpha)+\phi+\alpha}\, a_t
      + (1-\alpha)\,
        \frac{\log(1-\alpha)}{\sigma(1-\alpha)+\phi+\alpha}.
\]

\subsubsection*{Interpretación económica de la Ecuación (19)}

La ecuación \eqref{eq:y_equilibrio} resume cómo se determina el producto
real de equilibrio en el modelo clásico:

\begin{enumerate}
  \item \textbf{El producto depende sólo de la tecnología.}

  \noindent
  Al igual que en el caso del empleo, en \eqref{eq:y_equilibrio}
  \emph{no} aparecen:
  \begin{itemize}
    \item la tasa de interés nominal $i_t$,
    \item la inflación esperada $E_t\{\pi_{t+1}\}$,
    \item ni el choque de preferencias $z_t$.
  \end{itemize}
  En consecuencia, el producto de equilibrio $y_t$ es una variable
  \emph{real} que se determina únicamente por la productividad $a_t$ y
  por los parámetros estructurales $(\sigma,\phi,\alpha)$: es otra
  manifestación de la \emph{neutralidad de la política monetaria} en este
  entorno clásico con precios perfectamente flexibles.

  \item \textbf{Signo y magnitud de $\psi_{ya}$.}

  \noindent
  Del resultado explícito:
  \[
    \psi_{ya} = \frac{1+\phi}{\sigma(1-\alpha)+\phi+\alpha},
  \]
  vemos que, bajo los supuestos estándar $\sigma \ge 0$, $\phi \ge 0$ y
  $\alpha \in (0,1)$:
  \begin{itemize}
    \item El numerador $1+\phi$ es siempre estríctamente positivo.
    \item El denominador $\sigma(1-\alpha)+\phi+\alpha$ es también
          estríctamente positivo.
  \end{itemize}
  Por tanto,
  \[
    \psi_{ya} > 0.
  \]
  \emph{Conclusión}: un choque tecnológico positivo ($a_t$ más alto)
  \emph{siempre} incrementa el producto de equilibrio $y_t$.

  Además, la magnitud de $\psi_{ya}$ depende de los parámetros:
  \begin{itemize}
    \item Un mayor $\phi$ (oferta de trabajo más rígida en sentido Frisch)
          incrementa el numerador y el denominador, teniendo un efecto
          ambiguo sobre $\psi_{ya}$; en general, $\phi$ tiende a moderar
          el ajuste del empleo, con lo que parte del ajuste ante el choque
          se refleja en salarios reales.
    \item Un mayor $\sigma$ (sigma; mayor aversión al riesgo y menor
          elasticidad de sustitución intertemporal) aumenta el término
          $\sigma(1-\alpha)$ en el denominador, tendiendo a reducir
          $\psi_{ya}$: el consumo (y por tanto el trabajo vía equilibrio
          general) reacciona menos ante cambios en las tasas reales
          implícitas.
    \item El parámetro tecnológico $\alpha$ afecta tanto el peso del
          trabajo $(1-\alpha)$ en la producción como el denominador.
          Cuando $(1-\alpha)$ es grande (el trabajo tiene un peso elevado),
          cambios en el empleo inducidos por la tecnología tienen un impacto
          relativamente mayor en el producto.
  \end{itemize}

  \item \textbf{El término constante $\psi_y$.}

  \noindent
  La constante
  \[
    \psi_y = (1-\alpha)\, \psi_n
           = (1-\alpha)\,
             \frac{\log(1-\alpha)}{\sigma(1-\alpha)+\phi+\alpha}
  \]
  fija el \emph{nivel medio} de $y_t$ en relación con los parámetros
  estructurales. Cambios en la participación del trabajo $(1-\alpha)$ o
  en la rigidez de la oferta de trabajo (capturada por $\phi$) desplazan
  hacia arriba o hacia abajo la relación entre $y_t$ y $a_t$ sin cambiar
  su pendiente $\psi_{ya}$.
\end{enumerate}

\noindent
La ecuación \eqref{eq:y_equilibrio} completa la caracterización
del equilibrio real del modelo clásico: tanto el empleo como el producto
fluctúan exclusivamente en respuesta a los choques tecnológicos, mientras
que la política monetaria sólo puede afectar variables nominales (como
precios y nivel de dinero) pero no las cantidades reales de largo plazo.


\subsubsection*{Ecuación (20): Tasa de interés real de equilibrio}

La tasa de interés real se define como
\[
  r_t \equiv i_t - E_t\{\pi_{t+1}\},
\]
es decir, la tasa nominal descontada por la inflación esperada entre $t$ y $t+1$.
En el modelo clásico, $r_t$ se determina exclusivamente por factores reales.

La Ecuación (20) la expresa en dos formas equivalentes:
\begin{equation}
  r_t
    = \rho + (1-\rho_z)\, z_t + \sigma\, E_t\{\Delta y_{t+1}\}
    = \rho + (1-\rho_z)\, z_t
      - \sigma (1-\rho_a)\, \psi_{ya}\, a_t.
  \tag{20}\label{eq:r_equilibrio}
\end{equation}

\paragraph{Parámetros y variables.}

\begin{itemize}
  \item $r_t$:
        tasa de interés real de equilibrio.
  \item $i_t$:
        tasa de interés nominal.
  \item $\pi_{t+1} \equiv p_{t+1} - p_t$:
        inflación entre $t$ y $t+1$.
  \item $\rho$ (rho):
        tasa de descuento del hogar, definida como $\rho \equiv -\log\beta$;
        determina el nivel de $r_t$ en estado estacionario.
  \item $z_t \equiv \log Z_t$:
        choque de preferencias; sigue un proceso AR(1)
        $z_t = \rho_z z_{t-1} + \varepsilon_{z,t}$.
  \item $\rho_z$ (rho sub z), con $0 \le \rho_z < 1$:
        parámetro de persistencia del choque de preferencias.
  \item $a_t \equiv \log A_t$:
        choque de tecnología (nivel de productividad); sigue
        $a_t = \rho_a a_{t-1} + \varepsilon_{a,t}$.
  \item $\rho_a$ (rho sub a), con $0 \le \rho_a < 1$:
        parámetro de persistencia del choque tecnológico.
  \item $\sigma$ (sigma) $\ge 0$:
        parámetro de curvatura de la utilidad del consumo; mide la aversión
        al riesgo y el inverso de la elasticidad de sustitución intertemporal.
  \item $y_t \equiv \log Y_t$:
        logaritmo del producto; $\Delta y_{t+1} \equiv y_{t+1} - y_t$ es
        su tasa de crecimiento.
  \item $\psi_{ya}$ (psi sub ya):
        multiplicador del producto respecto a la tecnología, definido como
        \[
          \psi_{ya} \equiv \frac{1+\phi}{\sigma(1-\alpha)+\phi+\alpha},
        \]
        donde $\phi$ (phi) es el inverso aproximado de la elasticidad Frisch
        de la oferta de trabajo y $\alpha$ (alfa) es el parámetro tecnológico
        de la función de producción Cobb--Douglas.
\end{itemize}

\subsubsection*{Derivación a partir de la Ecuación de Euler}

Partimos de la Ecuación de Euler log-linealizada (ecuación (10)):
\begin{equation}
  c_t
    = E_t\{c_{t+1}\}
      - \frac{1}{\sigma}\bigl(i_t - E_t\{\pi_{t+1}\} - \rho\bigr)
      + \frac{1}{\sigma}(1-\rho_z)\, z_t.
  \label{eq:euler_log_lineal}
\end{equation}

\noindent
Recordamos dos relaciones clave:
\begin{itemize}
  \item Definición de la tasa de interés real:
        \[
          r_t \equiv i_t - E_t\{\pi_{t+1}\}.
        \]
  \item Condición de vaciado del mercado de bienes (ecuación (16)):
        \[
          y_t = c_t.
        \]
\end{itemize}

\paragraph{Paso 1: expresar $r_t$ en función del crecimiento esperado del consumo.}

Sustituimos $r_t$ en \eqref{eq:euler_log_lineal}:
\[
  c_t
    = E_t\{c_{t+1}\}
      - \frac{1}{\sigma}\bigl(r_t - \rho\bigr)
      + \frac{1}{\sigma}(1-\rho_z)\, z_t.
\]
Reordenamos para aislar $r_t$:
\begin{align*}
  -\frac{1}{\sigma}(r_t - \rho)
    &= c_t - E_t\{c_{t+1}\}
       - \frac{1}{\sigma}(1-\rho_z)\, z_t, \\[0.3em]
  r_t - \rho
    &= -\sigma\bigl(c_t - E_t\{c_{t+1}\}\bigr)
       + (1-\rho_z)\, z_t, \\[0.3em]
  r_t
    &= \rho + (1-\rho_z)\, z_t
       + \sigma\bigl(E_t\{c_{t+1}\} - c_t\bigr).
\end{align*}

Usando $c_t = y_t$ (ecuación (16)), obtenemos:
\begin{equation}
  r_t
    = \rho + (1-\rho_z)\, z_t
      + \sigma\, E_t\{\Delta y_{t+1}\},
  \label{eq:r_intermedio}
\end{equation}
donde $\Delta y_{t+1} \equiv y_{t+1}-y_t$.

Esta es la \emph{primera forma} de la Ecuación (20): la tasa real es la
tasa de descuento ajustada por los choques de preferencias y por el
crecimiento esperado del producto.

\paragraph{Paso 2: sustituir la trayectoria de equilibrio del producto.}

De la ecuación (19) sabemos que el producto de equilibrio es:
\[
  y_t = \psi_{ya}\, a_t + \psi_y,
\]
donde $\psi_y$ es una constante. Dado que $a_t$ sigue un proceso AR(1),
\[
  a_{t+1} = \rho_a a_t + \varepsilon_{a,t+1},
\]
tenemos, tomando expectativas condicionales a la información en $t$:
\[
  E_t\{a_{t+1}\} = \rho_a a_t.
\]

Por tanto,
\begin{align*}
  E_t\{y_{t+1}\}
    &= \psi_{ya}\, E_t\{a_{t+1}\} + \psi_y
     = \psi_{ya}\, \rho_a a_t + \psi_y, \\[0.3em]
  E_t\{\Delta y_{t+1}\}
    &= E_t\{y_{t+1} - y_t\} \\
    &= \bigl(\psi_{ya}\, \rho_a a_t + \psi_y\bigr)
       - \bigl(\psi_{ya}\, a_t + \psi_y\bigr) \\
    &= \psi_{ya}(\rho_a - 1)\, a_t
     = - (1-\rho_a)\, \psi_{ya}\, a_t.
\end{align*}

Sustituyendo esta expresión en \eqref{eq:r_intermedio} obtenemos la
\emph{segunda forma} de la Ecuación (20):
\begin{equation}
  r_t
    = \rho + (1-\rho_z)\, z_t
      - \sigma (1-\rho_a)\, \psi_{ya}\, a_t,
\end{equation}
que coincide con la línea final de \eqref{eq:r_equilibrio}.

\subsubsection*{Interpretación económica de la Ecuación (20)}

La expresión \eqref{eq:r_equilibrio} permite descomponer la tasa de
interés real de equilibrio en tres componentes:

\begin{enumerate}
  \item \textbf{Componente de estado estacionario:} $\rho$ (rho)

  \noindent
  En ausencia de choques ($a_t = 0$, $z_t = 0$), la ecuación se reduce a
  \[
    r_t = \rho,
  \]
  es decir, la tasa de interés real de largo plazo coincide con la tasa
  de descuento subjetiva del hogar. Esto es coherente con la interpretación
  de $\rho$ como el rendimiento real requerido para que el hogar esté
  dispuesto a trasladar consumo en el tiempo.

  \item \textbf{Choque de preferencias: $(1-\rho_z)\, z_t$}

  \noindent
  El término $(1-\rho_z)z_t$ muestra cómo un choque de preferencias
  $z_t$ (que desplaza la utilidad marginal del consumo) se traslada a la
  tasa real:
  \begin{itemize}
    \item Si $z_t$ aumenta (el hogar valora más el consumo presente
          respecto al futuro), la tasa de interés real de equilibrio
          \emph{aumenta}. Intuitivamente, el hogar exige un mayor
          rendimiento real para estar dispuesto a ahorrar.
    \item La magnitud del impacto depende de la persistencia $\rho_z$:
          si $\rho_z$ es cercana a uno, el factor $(1-\rho_z)$ es pequeño
          y el efecto contemporáneo sobre $r_t$ es más moderado.
  \end{itemize}

  \item \textbf{Choque tecnológico: $- \sigma (1-\rho_a)\, \psi_{ya}\, a_t$}

  \noindent
  El término asociado a $a_t$ tiene signo negativo:
  \[
    - \sigma (1-\rho_a)\, \psi_{ya} < 0,
  \]
  dado que $\sigma > 0$, $1-\rho_a > 0$ y $\psi_{ya} > 0$ en el modelo.
  Por tanto:
  \begin{itemize}
    \item Un choque tecnológico positivo ($a_t$ mayor) tiende a
          \emph{reducir} la tasa de interés real $r_t$.
    \item Intuitivamente, una mejora de la productividad eleva el
          producto (y el ingreso) esperado; esto hace que, para mantener
          el equilibrio intertemporal del consumo, el rendimiento real
          requerido sobre los activos pueda ser más bajo: los hogares
          no necesitan una tasa tan alta para inducir el mismo patrón de
          ahorro.
    \item La importancia cuantitativa de este efecto viene modulada por:
          \begin{itemize}
            \item $\sigma$ (sigma): cuanto mayor es la aversión al riesgo
                  y menor la elasticidad de sustitución intertemporal,
                  más fuerte es el ajuste de la tasa real ante cambios
                  en el crecimiento esperado.
            \item $(1-\rho_a)$: cuando la tecnología es muy persistente
                  ($\rho_a \approx 1$), el término $(1-\rho_a)$ es
                  pequeño y el impacto contemporáneo sobre $r_t$ se
                  atenúa.
            \item $\psi_{ya}$: captura cómo un choque en $a_t$ se
                  traduce en cambios en el producto $y_t$; a mayor
                  $\psi_{ya}$, mayor es el efecto de un mismo $a_t$
                  sobre el crecimiento esperado y, por ende, sobre $r_t$.
          \end{itemize}
  \end{itemize}
\end{enumerate}

\noindent
En conjunto, la Ecuación \eqref{eq:r_equilibrio} refuerza el mensaje
central del modelo clásico: la tasa de interés real es una variable
\emph{real} determinada por preferencias intertemporales y choques
productivos, mientras que la política monetaria sólo influye sobre
variables nominales como la inflación y el nivel de precios.

\subsubsection*{Ecuación (21): Salario real de equilibrio}

La última variable real por caracterizar en el modelo es el \emph{salario real},
definido como
\[
  \omega_t \equiv w_t - p_t,
\]
donde $w_t \equiv \log W_t$ es el logaritmo del salario nominal y
$p_t \equiv \log P_t$ el logaritmo del nivel de precios.

En equilibrio, el salario real viene dado por la demanda de trabajo de la firma
(Ecuación (15)):
\begin{equation}
  \omega_t
    = a_t - \alpha n_t + \log(1-\alpha).
  \label{eq:omega_demanda_trabajo}
\end{equation}

Al sustituir el nivel de empleo de equilibrio (Ecuación (18)):
\[
  n_t = \psi_{na} a_t + \psi_n,
\]
se obtiene una expresión reducida para $\omega_t$ que depende únicamente de la
tecnología:
\begin{equation}
  \omega_t
    = \psi_{\omega a} a_t + \psi_{\omega}.
  \tag{21}\label{eq:omega_equilibrio}
\end{equation}

\paragraph{Coeficientes de la Ecuación (21).}

Al sustituir $n_t = \psi_{na} a_t + \psi_n$ en
\eqref{eq:omega_demanda_trabajo} se obtiene:
\begin{align*}
  \omega_t
    &= a_t - \alpha(\psi_{na} a_t + \psi_n) + \log(1-\alpha) \\[0.3em]
    &= \bigl[1 - \alpha \psi_{na}\bigr] a_t
       + \bigl[\log(1-\alpha) - \alpha \psi_n\bigr].
\end{align*}
Por tanto, identificamos:
\[
  \psi_{\omega a} \equiv 1 - \alpha \psi_{na},
  \qquad
  \psi_{\omega} \equiv \log(1-\alpha) - \alpha \psi_n.
\]

Sustituyendo los valores de $\psi_{na}$ y $\psi_n$ obtenidos previamente en
la Ecuación (18),
\[
  \psi_{na} \equiv \frac{1-\sigma}{\sigma(1-\alpha)+\phi+\alpha},
  \qquad
  \psi_n \equiv \frac{\log(1-\alpha)}{\sigma(1-\alpha)+\phi+\alpha},
\]
y simplificando, el texto recoge los coeficientes en la forma:
\begin{align*}
  \psi_{\omega a}
    &\equiv \frac{\sigma + \phi}{\sigma(1-\alpha)+\phi+\alpha}, \\[0.3em]
  \psi_{\omega}
    &\equiv
      \frac{\bigl[\sigma(1-\alpha)+\phi\bigr]\log(1-\alpha)}
           {\sigma(1-\alpha)+\phi+\alpha}.
\end{align*}

\paragraph{Interpretación económica.}

La expresión reducida \eqref{eq:omega_equilibrio} muestra que el salario real
de equilibrio depende únicamente del choque tecnológico $a_t$ (y de parámetros):

\begin{itemize}
  \item \textbf{Neutralidad monetaria.}
        Ninguna variable monetaria (como $i_t$ o $m_t$) ni el choque de
        preferencias $z_t$ aparece en \eqref{eq:omega_equilibrio}. La política
        monetaria sólo puede afectar el salario nominal $w_t$ a través del
        nivel de precios $p_t$, pero no el salario real $\omega_t$.

  \item \textbf{Efecto de la productividad.}
        Dado que
        \[
          \psi_{\omega a}
            = \frac{\sigma + \phi}{\sigma(1-\alpha)+\phi+\alpha} > 0,
        \]
        un aumento en la tecnología $a_t$ (\emph{choque positivo de
        productividad}) incrementa siempre el salario real de equilibrio
        $\omega_t$. La productividad más alta eleva el producto marginal del
        trabajo y, en competencia perfecta, esto se traduce en un salario real
        más elevado.

  \item \textbf{Contraste con el empleo.}
        Mientras que el producto $y_t$ y el salario real $\omega_t$ aumentan
        de forma inequívoca ante un choque tecnológico positivo, la reacción
        del empleo $n_t$ es ambigua y depende del parámetro $\sigma$:
        \begin{itemize}
          \item si $\sigma < 1$, el efecto sustitución domina al efecto
                ingreso y el empleo tiende a aumentar;
          \item si $\sigma > 1$, el efecto ingreso puede dominar y el empleo
                tender a disminuir;
          \item si $\sigma = 1$, los efectos se compensan y el empleo es
                invariante a $a_t$.
        \end{itemize}
        En todos los casos, sin embargo, el salario real se mueve en la misma
        dirección que la productividad.
\end{itemize}

La Ecuación \eqref{eq:omega_equilibrio} completa la caracterización
de las variables reales del modelo clásico: producción, empleo, tasa de interés
real y salario real se determinan exclusivamente por choques reales (tecnología
y preferencias) y parámetros estructurales, confirmando la neutralidad de la
política monetaria sobre estas magnitudes.



\subsubsection*{Ecuación (22): La ecuación de Fisher}

La Ecuación (22) marca el inicio del análisis de la \emph{determinación del nivel
de precios y la política monetaria}. Es el puente entre las variables reales,
ya determinadas en las secciones anteriores, y las variables nominales del
modelo.

La ecuación se conoce como la \emph{ecuación de Fisher}:
\begin{equation}
  i_t = E_t\{\pi_{t+1}\} + r_t.
  \tag{22}\label{eq:fisher}
\end{equation}

\paragraph{Desglose de términos y contexto.}

La Ecuación \eqref{eq:fisher} relaciona tres objetos clave:

\begin{itemize}
  \item $i_t$:
        tasa de interés nominal en $t$, entendida como el logaritmo del
        rendimiento bruto del bono nominal, $I_t \equiv 1/Q_t$.
        En el modelo, es la variable que el banco central puede fijar
        directamente cuando conduce la política monetaria mediante una
        \emph{regla de tasa de interés}.

  \item $E_t\{\pi_{t+1}\}$:
        inflación esperada entre $t$ y $t+1$, donde
        $\pi_{t+1} \equiv p_{t+1} - p_t$ y $p_t \equiv \log P_t$.
        Es una variable nominal ligada al sendero del nivel de precios.

  \item $r_t$:
        tasa de interés real, definida por
        \[
          r_t \equiv i_t - E_t\{\pi_{t+1}\}.
        \]
        En el modelo clásico, $r_t$ está determinada exclusivamente por
        factores reales (tecnología y preferencias), como quedó recogido en la
        Ecuación (20).
\end{itemize}

\paragraph{Papel de la ecuación de Fisher en el modelo clásico.}

La Ecuación \eqref{eq:fisher} es crucial por varias razones:

\begin{itemize}
  \item \textbf{Ajuste uno a uno entre $i_t$ y la inflación esperada.}
        Dado un valor de la tasa real $r_t$, la ecuación de Fisher implica que
        cambios en la inflación esperada $E_t\{\pi_{t+1}\}$ deben reflejarse
        uno a uno en la tasa nominal $i_t$. En otras palabras, para un $r_t$
        dado, la política monetaria que fija $i_t$ determina implícitamente el
        sendero de la inflación esperada.

  \item \textbf{Restricción real sobre la política monetaria.}
        La Ecuación (20) mostró que $r_t$ es una variable real, función de los
        choques tecnológicos $a_t$ y de preferencias $z_t$, y de parámetros
        estructurales, pero independiente de la forma específica de la política
        monetaria. La ecuación de Fisher, al imponer
        \[
          i_t - E_t\{\pi_{t+1}\} = r_t,
        \]
        obliga a que la combinación de tasa nominal e inflación esperada sea
        consistente con ese valor de equilibrio de la tasa real.

  \item \textbf{Determinación de las variables nominales.}
        Dado el sendero de $r_t$ (fijado por la economía real) y una regla de
        política para $i_t$, la ecuación de Fisher se convierte en un elemento
        central para determinar el sendero de la inflación esperada y, en
        consecuencia, del nivel de precios $p_t$ y de otras variables nominales.
        A diferencia de las variables reales ($y_t$, $n_t$, $r_t$, $\omega_t$),
        los valores de equilibrio de las variables nominales no pueden
        determinarse sin especificar la política monetaria.
\end{itemize}

\paragraph{Implicación en estado estacionario.}

En un estado estacionario sin crecimiento y sin choques, la tasa de interés
real coincide con la tasa de descuento del hogar:
\[
  r = \rho.
\]
En un \emph{perfect foresight steady state}, donde la inflación es constante e
igual a $\pi$, la ecuación de Fisher se reduce a:
\[
  i = \rho + \pi.
\]
Esta relación resume la idea básica de Fisher: en el largo plazo, la tasa de
interés nominal se descompone en una parte real (la tasa de descuento) y una
parte puramente nominal (la inflación).


\subsubsection*{Ecuación (23): Proceso del choque de política monetaria}

La Ecuación (23), introducida en la Sección 2.4.1 (\emph{“Una trayectoria
exógena para la tasa de interés nominal”}), especifica la dinámica del
\emph{componente estocástico} de la política monetaria. No es una condición de
equilibrio ni de optimalidad, sino una hipótesis sobre cómo se comporta la
autoridad monetaria.

El choque de política monetaria $v_t$ sigue un proceso autorregresivo de orden
uno (AR(1)):
\begin{equation}
  v_t = \rho_v v_{t-1} + \epsilon_{v,t},
  \tag{23}\label{eq:vt_proceso}
\end{equation}
donde $|\rho_v|<1$ garantiza que el proceso sea estacionario.

\paragraph{Desglose de términos.}

\begin{itemize}
  \item $v_t$:
        componente estocástico de la política monetaria en $t$. Representa la
        desviación transitoria de la conducta “habitual” de la tasa de interés
        nominal.

  \item $\rho_v$:
        coeficiente de persistencia del choque monetario. Si $\rho_v$ es alto,
        un shock en $v_t$ tiene efectos prolongados sobre la trayectoria de la
        tasa nominal; si es bajo, el efecto se disipa rápidamente.

  \item $\epsilon_{v,t}$:
        innovación estocástica, con media cero y varianza constante.
        Representa la parte totalmente inesperada (no anticipada) del shock de
        política monetaria.
\end{itemize}

\paragraph{Vínculo con la regla de política monetaria.}

La Ecuación \eqref{eq:vt_proceso} se inserta en una regla sencilla para la tasa
de interés nominal:
\[
  i_t = \bar{i} + v_t,
\]
donde:

\begin{itemize}
  \item $\bar{i}$ es el nivel “normal” o de referencia de la tasa nominal que
        el banco central mantendría en ausencia de shocks.
  \item $v_t$ introduce desviaciones transitorias alrededor de ese nivel
        normal, interpretadas como “\emph{monetary policy shocks}”.
\end{itemize}

Desde un punto de vista didáctico, $v_t$ puede captar:
\begin{itemize}
  \item cambios inesperados en las preferencias del formulador de política;
  \item respuestas puntuales a eventos no anticipados;
  \item errores o imperfecciones en la implementación de la regla usual.
\end{itemize}

\paragraph{Rol analítico en el modelo clásico.}

Aunque en la versión clásica del modelo las variables reales son neutrales a la
política monetaria, la dinámica de $v_t$ resulta crucial para entender la
determinación de las variables nominales:

\begin{itemize}
  \item En combinación con la ecuación de Fisher,
        \[
          i_t = E_t\{\pi_{t+1}\} + r_t,
        \]
        la regla $i_t = \bar{i} + v_t$ implica que la inflación esperada viene
        dada por
        \[
          E_t\{\pi_{t+1}\} = \bar{i} + v_t - r_t,
        \]
        donde $r_t$ ya fue determinada por factores reales (Ecuación (20)).

  \item De este modo, el proceso \eqref{eq:vt_proceso} fija la dinámica de la
        inflación esperada y, por extensión, condiciona la trayectoria del nivel
        de precios $p_t$.

  \item En la estructura clásica, esto conduce a un resultado importante: aun
        cuando las variables reales estén bien determinadas, el \emph{nivel de
        precios} puede ser indeterminado o susceptible a fluctuaciones no
        fundamentales (shocks tipo \emph{sunspots}), mientras que la inflación
        esperada sí queda unívocamente determinada por la combinación de
        $r_t$ e $i_t$.
\end{itemize}

La Ecuación (23) formaliza cómo la política monetaria introduce
shocks estocásticos en la tasa nominal a través de un proceso AR(1), lo cual es
clave para estudiar la transmisión de la política monetaria sobre las
variables nominales, incluso en un entorno donde las variables reales son
neutrales a dicha política.

\subsubsection*{Ecuación (24): Ecuación de diferencias para la inflación}

La Ecuación (24) resulta de combinar la regla de política monetaria con la
ecuación de Fisher y es el punto de partida para analizar la determinación y la
unicidad del equilibrio de inflación y del nivel de precios.

La ecuación se escribe en términos de desviaciones respecto al estado
estacionario:
\begin{equation}
  \varphi_\pi \,\hat{\pi}_t = E_t\{\hat{\pi}_{t+1}\} + \hat{r}_t - v_t,
  \tag{24}\label{eq:ecuacion24}
\end{equation}
donde $\hat{\pi}_t \equiv \pi_t - \pi$ es la desviación de la inflación respecto
a su valor estacionario, y $\hat{r}_t \equiv r_t - \rho$ es la desviación de la
tasa de interés real respecto a la tasa de descuento del hogar.

\paragraph{Derivación rigurosa desde la regla de política y Fisher.}

El punto de partida es la regla sencilla de tasa de interés nominal utilizada en
la Sección 2.4.2:
\[
  i_t = \rho + \pi + \varphi_\pi(\pi_t - \pi) + v_t,
\]
donde:
\begin{itemize}
  \item $\rho + \pi$ es el componente de estado estacionario de la tasa nominal,
  \item $\varphi_\pi(\pi_t - \pi)$ es la respuesta sistemática del banco central
        a desviaciones de la inflación respecto a su meta,
  \item $v_t$ es el shock de política monetaria, que sigue el proceso AR(1) de
        la Ecuación (23).
\end{itemize}

Por otro lado, la ecuación de Fisher (Ecuación (22)) relaciona la tasa nominal,
la inflación esperada y la tasa real:
\[
  i_t = E_t\{\pi_{t+1}\} + r_t.
\]

Igualando ambas expresiones para $i_t$:
\[
  \rho + \pi + \varphi_\pi(\pi_t - \pi) + v_t
  = E_t\{\pi_{t+1}\} + r_t.
\]

Reordenando términos y usando la notación de desviaciones,
$\hat{\pi}_t = \pi_t - \pi$ y $\hat{r}_t = r_t - \rho$, se obtiene:
\begin{align*}
  \varphi_\pi \hat{\pi}_t
  &= E_t\{\pi_{t+1}\} - \pi + r_t - \rho - v_t \\
  &= E_t\{\hat{\pi}_{t+1}\} + \hat{r}_t - v_t,
\end{align*}
que es precisamente la Ecuación \eqref{eq:ecuacion24}.

\paragraph{Contenido económico de la Ecuación (24).}

La Ecuación (24) puede interpretarse como una ecuación de diferencias hacia
adelante para la inflación (en desviaciones), donde:
\begin{itemize}
  \item $E_t\{\hat{\pi}_{t+1}\}$ recoge las expectativas sobre la inflación
        futura,
  \item $\hat{r}_t$ está completamente determinada por los choques reales
        ($a_t$ y $z_t$) a través de la Ecuación (20),
  \item $v_t$ introduce la componente puramente monetaria (shock de política).
\end{itemize}

Es decir, una vez fijado el comportamiento de las variables reales, la dinámica
de la inflación queda gobernada por \eqref{eq:ecuacion24} en función de
$\varphi_\pi$, $\hat{r}_t$ y $v_t$.

\paragraph{El principio de Taylor y la unicidad del equilibrio.}

La Ecuación (24) es la base del resultado central sobre determinación del nivel
de precios:

\begin{itemize}
  \item\textbf{Caso $\boldsymbol{\varphi_\pi > 1}$ (principio de Taylor).}
        Si el banco central ajusta la tasa nominal más que uno a uno frente a
        cambios en la inflación (es decir, la respuesta de $i_t$ a $\pi_t$ es
        “agresiva”), la solución hacia adelante de \eqref{eq:ecuacion24} es
        convergente y se obtiene una trayectoria de inflación única y
        no explosiva. En este caso, inflación y nivel de precios quedan
        determinados de forma unívoca por los fundamentos reales y la regla de
        política monetaria.

  \item\textbf{Caso $\boldsymbol{\varphi_\pi < 1}$.}
        Si la respuesta de la política es débil, la solución hacia adelante no
        converge: aparecen múltiples trayectorias posibles para la inflación,
        y el nivel de precios se vuelve \emph{indeterminado}. En este entorno,
        shocks no fundamentales (\emph{sunspot shocks}) pueden generar
        fluctuaciones en precios e inflación sin cambios en los fundamentos
        reales.

  \item\textbf{Caso límite $\boldsymbol{\varphi_\pi = 1}$.}
        El sistema se encuentra en el borde entre determinación e
        indeterminación; pequeños cambios en la regla de política o en la
        estructura del modelo pueden inclinar el resultado hacia uno u otro
        lado.
\end{itemize}

La Ecuación (24) muestra que, una vez fijado el bloque real del
modelo, la agresividad de la respuesta de la tasa nominal a la inflación
($\varphi_\pi$) es el elemento decisivo que determina si la economía posee un
equilibrio nominal bien definido o sufre de indeterminación del nivel de
precios.

\subsubsection*{Ecuación (25): Solución de equilibrio única para la inflación}

La Ecuación (25) proporciona la solución de equilibrio para la desviación de la
inflación, $\hat{\pi}_t \equiv \pi_t - \pi$, cuando la política monetaria sigue
una regla de tasa de interés que satisface el principio de Taylor
($\varphi_\pi > 1$). A partir de la ecuación de diferencias para la inflación
(Ecuación (24)):

\begin{equation}
  \varphi_\pi \hat{\pi}_t = E_t\{\hat{\pi}_{t+1}\} + \hat{r}_t - v_t,
  \tag{24}\label{eq:ecu24_recall}
\end{equation}

la solución hacia adelante no explosiva viene dada por:
\begin{equation}
  \hat{\pi}_t
  = \sum_{k=0}^{\infty} \varphi_\pi^{-(k+1)} E_t\{\hat{r}_{t+k} - v_{t+k}\}.
  \tag{25}\label{eq:ecu25}
\end{equation}

\paragraph{Derivación hacia adelante de la Ecuación (24).}

Partimos de \eqref{eq:ecu24_recall} y resolvemos para $E_t\{\hat{\pi}_{t+1}\}$:
\[
  E_t\{\hat{\pi}_{t+1}\}
  = \varphi_\pi \hat{\pi}_t - (\hat{r}_t - v_t).
\]

Tomando expectativas condicionales un período adelante y usando la ley de
iteración de expectativas ($E_t[E_{t+1}(\cdot)] = E_t(\cdot)$), se obtiene:
\[
  E_t\{\hat{\pi}_{t+2}\}
  = \varphi_\pi E_t\{\hat{\pi}_{t+1}\} - E_t\{\hat{r}_{t+1} - v_{t+1}\}.
\]

Sustituyendo recursivamente $E_t\{\hat{\pi}_{t+1}\}$ en términos de
$\hat{\pi}_t$ y de las secuencias futuras de $(\hat{r}_s - v_s)$, se llega a la
expresión general para un horizonte finito $T$:
\[
  E_t\{\hat{\pi}_{t+T+1}\}
  = \varphi_\pi^{T+1}\hat{\pi}_t
    - \sum_{k=0}^{T} \varphi_\pi^{T-k} E_t\{\hat{r}_{t+k} - v_{t+k}\}.
\]

Reordenando:
\[
  \hat{\pi}_t
  = \varphi_\pi^{-(T+1)} E_t\{\hat{\pi}_{t+T+1}\}
    + \sum_{k=0}^{T} \varphi_\pi^{-(k+1)} E_t\{\hat{r}_{t+k} - v_{t+k}\}.
\]

Para obtener una solución bien comportada cuando $T \to \infty$, se impone la
condición de no explosión (condición de transversalidad en términos de
inflación):
\[
  \lim_{T \to \infty}
  \varphi_\pi^{-(T+1)} E_t\{\hat{\pi}_{t+T+1}\} = 0.
\]

Esta condición es válida si y solo si $\varphi_\pi > 1$, es decir, si la
respuesta de la tasa nominal a la inflación es lo suficientemente agresiva.
Bajo este supuesto, el primer término desaparece en el límite y obtenemos la
solución hacia adelante:

\[
  \hat{\pi}_t
  = \sum_{k=0}^{\infty} \varphi_\pi^{-(k+1)} E_t\{\hat{r}_{t+k} - v_{t+k}\},
\]
que coincide con la Ecuación \eqref{eq:ecu25}.

\paragraph{Condiciones para la existencia y unicidad de la solución.}

La Ecuación \eqref{eq:ecu25} es:
\begin{itemize}
  \item \emph{No explosiva}: las contribuciones de los términos lejanos en el
        tiempo se atenúan geométricamente gracias al factor
        $\varphi_\pi^{-(k+1)}$.
  \item \emph{Única}: una vez impuesta la condición de no explosión, no hay
        grados de libertad adicionales para elegir una senda alternativa de
        inflación; cualquier otra solución implicaría trayectorias explosivas.
\end{itemize}

La condición clave es:
\[
  \varphi_\pi > 1 \quad \Longleftrightarrow \quad
  \text{el banco central sube } i_t \text{ más que uno a uno ante cambios en }
  \pi_t.
\]
Esta condición es exactamente el \emph{principio de Taylor}.

\paragraph{Contenido económico de la Ecuación (25).}

La expresión \eqref{eq:ecu25} deja claras varias ideas:

\begin{itemize}
  \item \textbf{Determinación de la inflación por fuerzas reales y monetarias.}
        Dado que $\hat{r}_t$ está determinado exclusivamente por los choques
        reales (tecnología $a_t$ y preferencias $z_t$) a través de la
        Ecuación (20), la inflación queda anclada por:
        \begin{enumerate}
          \item la senda esperada de las variables reales
                ($\{\hat{r}_{t+k}\}_{k \ge 0}$), y
          \item la senda esperada de choques de política monetaria
                ($\{v_{t+k}\}_{k \ge 0}$).
        \end{enumerate}

  \item \textbf{Papel estabilizador de $\boldsymbol{\varphi_\pi}$.}
        Cuanto mayor es $\varphi_\pi$, más pequeño es el peso
        $\varphi_\pi^{-(k+1)}$ asociado a cada choque futuro. En términos
        didácticos: una respuesta más agresiva de la tasa de interés a la
        inflación reduce la sensibilidad de la inflación a cualquier secuencia
        de choques reales o monetarios.

  \item \textbf{Choques de política monetaria.}
        El término $-v_{t+k}$ muestra que los shocks de política introducen
        fluctuaciones adicionales e “innecesarias” en la inflación: incluso si
        los fundamentos reales no cambiaran, una secuencia de $v_t$ distintos
        de cero generaría variación en $\hat{\pi}_t$.

  \item \textbf{Determinación del nivel de precios.}
        Una vez determinada la senda de $\hat{\pi}_t$, la trayectoria del nivel
        de precios $p_t$ queda fijada (hasta una constante inicial) mediante la
        acumulación de la inflación. Bajo $\varphi_\pi > 1$, inflación y nivel
        de precios dejan de ser indeterminados.
\end{itemize}

La Ecuación (25) muestra que, en una economía donde el bloque real
ya está determinado, la elección de una regla de tasa de interés que satisfaga
el principio de Taylor es suficiente para anclar de manera única la senda de la
inflación y del nivel de precios.

\subsubsection*{Ecuación (26): Solución estacionaria e indeterminación del equilibrio nominal}

La Ecuación (26) describe la dinámica de la inflación cuando la regla de tasa
de interés del banco central \emph{no} satisface el principio de Taylor, es
decir, cuando el coeficiente de respuesta $\varphi_\pi$ es menor que uno
($\varphi_\pi < 1$). En este caso, la solución de la ecuación de diferencias
para la inflación es \emph{hacia atrás} y da lugar a indeterminación del nivel
de precios:

\begin{equation}
  \pi_t
  = (1 - \varphi_\pi)\,\pi
    + \varphi_\pi \pi_{t-1}
    - \hat{r}_{t-1}
    + v_{t-1}
    + \xi_t,
  \tag{26}\label{eq:ecu26}
\end{equation}

donde $\hat{r}_t \equiv r_t - \rho$ es la desviación de la tasa de interés real
respecto a su valor estacionario y $v_t$ es el choque de política monetaria
definido en la Sección 2.4.1.

\paragraph{I. Punto de partida: ecuación de diferencias de inflación.}

Recordemos la ecuación de diferencias para la desviación de la inflación
(Ecuación (24)):

\begin{equation}
  \varphi_\pi \hat{\pi}_t
  = E_t\{\hat{\pi}_{t+1}\} + \hat{r}_t - v_t,
  \tag{24}\label{eq:ecu24_again}
\end{equation}

donde $\hat{\pi}_t \equiv \pi_t - \pi$ y
$\hat{r}_t \equiv r_t - \rho$. Cuando $\varphi_\pi > 1$, esta ecuación admite
una solución \emph{hacia adelante} bien comportada (Ecuación (25)). En cambio,
si $\varphi_\pi < 1$, la solución hacia adelante no converge y debemos
considerar la solución \emph{hacia atrás}.

\paragraph{II. Solución hacia atrás cuando $\boldsymbol{\varphi_\pi < 1}$.}

Para ver la lógica, partimos de \eqref{eq:ecu24_again}, pero ahora tomamos
expectativas condicionales en $t-1$:

\[
  \varphi_\pi E_{t-1}\{\hat{\pi}_t\}
  = E_{t-1}\{\hat{\pi}_{t+1}\}
    + E_{t-1}\{\hat{r}_t - v_t\}.
\]

Una solución genérica para $\hat{\pi}_t$ que permite \emph{no unicidad} es:

\[
  \hat{\pi}_t
  = \varphi_\pi \hat{\pi}_{t-1}
    - \hat{r}_{t-1}
    + v_{t-1}
    + \xi_t,
\]

donde el término $\xi_t$ está construido de forma tal que:

\[
  E_{t-1}\{\xi_t\} = 0,
\]

y asegura que la ecuación en expectativas \eqref{eq:ecu24_again} se siga
cumpliendo. Volviendo a niveles ($\pi_t = \pi + \hat{\pi}_t$ y
$ \hat{r}_t = r_t - \rho$), se obtiene precisamente la forma en niveles de la
Ecuación (26):

\[
  \pi_t
  = (1 - \varphi_\pi)\,\pi
    + \varphi_\pi \pi_{t-1}
    - \hat{r}_{t-1}
    + v_{t-1}
    + \xi_t.
\]

\paragraph{III. El papel del término $\boldsymbol{\xi_t}$ y la indeterminación.}

El término $\xi_t$ es una secuencia de shocks que:
\begin{itemize}
  \item no está directamente vinculada a los fundamentos reales del modelo
        (productividad $a_t$, preferencias $z_t$), ni a la regla sistemática de
        política monetaria, y
  \item sólo está restringida por la condición
        $E_{t-1}\{\xi_t\} = 0$.
\end{itemize}

En la literatura, estos shocks se conocen como \emph{sunspot shocks} o choques
de “expectativas autocumplidas”:
\begin{itemize}
  \item Cualquier trayectoria de inflación $\{\pi_t\}$ que satisfaga
        \eqref{eq:ecu26} para alguna realización admisible de $\{\xi_t\}$ es
        consistente con equilibrio.
  \item Por tanto, la inflación (y el nivel de precios $p_t$) dejan de estar
        determinados de forma única por los fundamentos reales y monetarios.
\end{itemize}

\paragraph{IV. Interpretación económica.}

La Ecuación \eqref{eq:ecu26} permite extraer varias conclusiones:

\begin{itemize}
  \item \textbf{Respuesta insuficiente de la política monetaria.}
        Cuando $\varphi_\pi < 1$, el banco central ajusta la tasa nominal
        \emph{menos que uno a uno} frente a un aumento de la inflación. En
        términos de la Ecuación de Fisher, esto implica que la tasa real
        $r_t = i_t - E_t\{\pi_{t+1}\}$ no aumenta lo suficiente como para
        estabilizar las expectativas de inflación.
  \item \textbf{Indeterminación del nivel de precios.}
        A diferencia del caso $\varphi_\pi > 1$, donde existe una única
        solución hacia adelante para $\hat{\pi}_t$, aquí hay un conjunto
        continuo de soluciones. Distintas realizaciones de $\{\xi_t\}$ generan
        diferentes trayectorias de inflación y de precios, todas compatibles
        con equilibrio.
  \item \textbf{Amplificación de choques no fundamentales.}
        Puesto que $\xi_t$ no está anclado por los fundamentos, cambios puramente
        “psicológicos” o de expectativas pueden generar fluctuaciones en
        inflación y precios sin que haya variación en la productividad o en las
        preferencias. En este sentido, la política monetaria débil
        ($\varphi_\pi < 1$) abre la puerta a ciclos impulsados por sunspots.
\end{itemize}

La Ecuación (26) muestra que, si la regla de tasa de interés no
satisface el principio de Taylor, el modelo admite múltiples trayectorias de
inflación compatibles con equilibrio, y el nivel de precios se vuelve
indeterminado. La estabilidad nominal ya no está garantizada por la política
monetaria.



\section{Capítulo 3: El modelo Keynesiano básico}

%========================================
% Ecuación (1)
%========================================
\subsubsection*{Ecuación (1): Demanda individual de bienes diferenciados}

La Ecuación (1) describe la \emph{demanda óptima del hogar por cada variedad} de bien \(i \in [0,1]\) en el modelo básico nuevo keynesiano con bienes diferenciados y competencia monopolística en el mercado de bienes. Su forma es:
\begin{equation}
C_t(i) \;=\; \left( \frac{P_t(i)}{P_t} \right)^{-\epsilon} C_t
\tag{1}
\end{equation}

Cada productor fija el precio de su variedad \(i\), pero enfrenta esta curva de demanda con elasticidad constante \(\epsilon\). El hogar representativo elige, en cada periodo \(t\), un conjunto de cantidades \(\{C_t(i)\}_{i\in[0,1]}\) de bienes diferenciados, sujeto a una restricción presupuestaria agregada. El consumo total se resume mediante un índice CES de consumo \(C_t\), mientras que los precios \(\{P_t(i)\}\) se combinan en un índice de precios agregado \(P_t\).

La ecuación (1) relaciona el consumo del bien \(i\) con:
\begin{itemize}
  \item el nivel de consumo agregado \(C_t\), y
  \item el precio relativo del bien \(i\), \(\frac{P_t(i)}{P_t}\), elevado a la potencia \(-\epsilon\).
\end{itemize}

Así, la demanda por cada variedad es proporcional al consumo agregado y decreciente en su precio relativo, con elasticidad de sustitución constante \(\epsilon\).

\subsubsection*{Tabla de símbolos relevantes (Ecuación 1)}

\begin{table}[htbp]
  \centering
  \begin{tabular}{p{0.18\textwidth} p{0.42\textwidth} p{0.30\textwidth}}
    \hline
    \textbf{Término} & \textbf{Definición rigurosa} & \textbf{Rol económico} \\
    \hline
    \(C_t(i)\) 
      & Cantidad del bien \(i\) consumida en el periodo \(t\). 
      & Demanda individual por la variedad \(i\). \\
    \(C_t\) 
      & Índice de consumo agregado definido por un agregador CES. 
      & Nivel de consumo total (``cesta compuesta''). \\
    \(P_t(i)\) 
      & Precio nominal del bien \(i\) en el periodo \(t\). 
      & Precio fijado por la empresa \(i\). \\
    \(P_t\) 
      & Índice de precios agregado (``nivel general de precios''). 
      & Precio mínimo por unidad del índice \(C_t\). \\
    \(\epsilon > 1\) 
      & Elasticidad de sustitución entre variedades de bienes. 
      & Mide la sensibilidad de la demanda ante cambios en precios relativos. \\
    \hline
  \end{tabular}
  \caption{Símbolos usados en la Ecuación (1).}
\end{table}

\subsubsection*{Derivación matemática de la Ecuación (1)}

La Ecuación (1) se obtiene resolviendo el \emph{problema intraperiódico} del hogar: dado un nivel de gasto en consumo, el hogar elige la combinación de variedades que maximiza el índice de consumo \(C_t\).

\paragraph{1. Agregador CES de consumo}

El índice de consumo se define como:
\begin{equation}
C_t \;\equiv\; 
\left( \int_0^1 C_t(i)^{\,1 - \frac{1}{\epsilon}} \, di \right)^{\frac{\epsilon}{\epsilon - 1}}.
\tag{1.a}
\end{equation}
Este es un agregador de tipo CES (elasticidad de sustitución constante) con elasticidad de sustitución entre variedades igual a \(\epsilon\).

\paragraph{2. Gasto total en consumo}

Dado el vector de precios \(\{P_t(i)\}_{i\in[0,1]}\), el gasto total en consumo es:
\begin{equation}
X_t \;\equiv\; \int_0^1 P_t(i)\, C_t(i)\, di.
\tag{1.b}
\end{equation}
En el problema intraperiódico, \(X_t\) se toma como dado: el hogar ya decidió cuánto gastar en total en consumo y ahora decide cómo repartir ese gasto entre variedades.

\paragraph{3. Problema de maximización intraperiódico}

El hogar resuelve, para cada \(t\),
\begin{equation}
\max_{\{C_t(i)\}_{i\in[0,1]}} \; C_t
\quad\text{sujeto a}\quad
\int_0^1 P_t(i)\, C_t(i)\, di = X_t,
\tag{1.c}
\end{equation}
donde \(C_t\) está dado por (1.a). El lagrangiano es:
\begin{equation}
\mathcal{L} = 
\left( \int_0^1 C_t(i)^{\,1 - \frac{1}{\epsilon}} \, di \right)^{\frac{\epsilon}{\epsilon - 1}}
- \lambda_t \left( \int_0^1 P_t(i)\, C_t(i)\, di - X_t \right).
\tag{1.d}
\end{equation}

\paragraph{4. Condición de primer orden}

Para un \(i\) genérico, la condición de primer orden es:
\begin{equation}
\frac{\partial \mathcal{L}}{\partial C_t(i)} 
= \frac{\partial C_t}{\partial C_t(i)} - \lambda_t P_t(i) = 0.
\tag{1.e}
\end{equation}

Definamos
\[
S_t \;\equiv\; \int_0^1 C_t(j)^{\,1 - \frac{1}{\epsilon}} \, dj,
\quad\text{de modo que}\quad
C_t = S_t^{\frac{\epsilon}{\epsilon - 1}}.
\]
Aplicando la regla de la cadena, se obtiene:
\begin{align}
\frac{\partial C_t}{\partial C_t(i)}
&= \frac{\epsilon}{\epsilon - 1} S_t^{\frac{1}{\epsilon-1}} 
    \frac{\partial S_t}{\partial C_t(i)} \notag\\
&= \frac{\epsilon}{\epsilon - 1} S_t^{\frac{1}{\epsilon-1}}
    \left(1 - \frac{1}{\epsilon}\right) C_t(i)^{-\frac{1}{\epsilon}} \notag\\
&= S_t^{\frac{1}{\epsilon-1}} C_t(i)^{-\frac{1}{\epsilon}}.
\tag{1.f}
\end{align}

Como \(C_t = S_t^{\frac{\epsilon}{\epsilon - 1}}\), se verifica que
\[
C_t^{\frac{1}{\epsilon}} 
= \left(S_t^{\frac{\epsilon}{\epsilon - 1}}\right)^{\frac{1}{\epsilon}}
= S_t^{\frac{1}{\epsilon - 1}},
\]
y por tanto
\begin{equation}
\frac{\partial C_t}{\partial C_t(i)}
= C_t^{\frac{1}{\epsilon}}\, C_t(i)^{-\frac{1}{\epsilon}}.
\tag{1.g}
\end{equation}

Sustituyendo (1.g) en la condición de primer orden (1.e), obtenemos:
\begin{equation}
C_t^{\frac{1}{\epsilon}}\, C_t(i)^{-\frac{1}{\epsilon}} = \lambda_t P_t(i).
\tag{1.h}
\end{equation}

Elevando ambos lados de (1.h) a la potencia \(\epsilon\):
\begin{equation}
C_t\, C_t(i)^{-1} = \lambda_t^{\,\epsilon} P_t(i)^{\,\epsilon},
\tag{1.i}
\end{equation}
de donde:
\begin{equation}
C_t(i) = C_t \, \lambda_t^{-\epsilon} P_t(i)^{-\epsilon}.
\tag{1.j}
\end{equation}

Tomando dos variedades cualesquiera \(i\) y \(j\), el cociente de demandas es:
\begin{equation}
\frac{C_t(i)}{C_t(j)} 
= \left( \frac{P_t(i)}{P_t(j)} \right)^{-\epsilon},
\tag{1.k}
\end{equation}
lo que muestra directamente que \(\epsilon\) es la elasticidad de sustitución entre variedades.

\paragraph{5. Índice de precios \(P_t\) y condición de gasto mínimo}

El índice de precios ideal se define como:
\begin{equation}
P_t \equiv 
\left( \int_0^1 P_t(i)^{\,1 - \epsilon} \, di \right)^{\frac{1}{1 - \epsilon}},
\tag{1.l}
\end{equation}
y el gasto mínimo necesario para alcanzar un consumo \(C_t\) viene dado por:
\begin{equation}
X_t = P_t C_t.
\tag{1.m}
\end{equation}

Sustituyendo (1.j) en la definición de gasto (1.b):
\begin{align}
X_t &= \int_0^1 P_t(i)\, C_t(i)\, di \notag\\
    &= \int_0^1 P_t(i)\, C_t \lambda_t^{-\epsilon} P_t(i)^{-\epsilon} \, di \notag\\
    &= C_t \lambda_t^{-\epsilon} \int_0^1 P_t(i)^{\,1-\epsilon} \, di.
\tag{1.n}
\end{align}
Por la definición de \(P_t\) en (1.l),
\[
\int_0^1 P_t(i)^{\,1-\epsilon}\, di = P_t^{\,1-\epsilon},
\]
de modo que:
\begin{equation}
X_t = C_t \lambda_t^{-\epsilon} P_t^{\,1 - \epsilon}.
\tag{1.o}
\end{equation}

Imponiendo la condición de gasto mínimo (1.m), \(X_t = P_t C_t\), se obtiene:
\begin{equation}
P_t C_t = C_t \lambda_t^{-\epsilon} P_t^{\,1 - \epsilon}
\quad\Rightarrow\quad
P_t = \lambda_t^{-\epsilon} P_t^{\,1 - \epsilon}.
\tag{1.p}
\end{equation}
Dividiendo por \(P_t^{\,1-\epsilon}\) (suponiendo \(P_t>0\)):
\[
P_t^{\epsilon} = \lambda_t^{-\epsilon}
\quad\Rightarrow\quad
\lambda_t = P_t^{-1}.
\]

Sustituyendo \(\lambda_t = P_t^{-1}\) en (1.j):
\begin{align}
C_t(i) 
&= C_t \left(P_t^{-1}\right)^{-\epsilon} P_t(i)^{-\epsilon} \notag\\
&= C_t \left( \frac{P_t(i)}{P_t} \right)^{-\epsilon},
\tag{1.q}
\end{align}
con lo cual recuperamos exactamente la Ecuación (1).

\subsubsection*{Interpretación económica de la Ecuación (1)}

La ecuación (1) resume que:
\begin{itemize}
  \item La demanda por la variedad \(i\) es \emph{proporcional} al consumo agregado \(C_t\).
  \item Es \emph{inversamente proporcional} al precio relativo \(\frac{P_t(i)}{P_t}\), con elasticidad (en valor absoluto) \(\epsilon\).
\end{itemize}

Desde el punto de vista de la empresa \(i\), (1) es la \emph{curva de demanda} que enfrenta: dado \(C_t\) y \(P_t\), si fija un precio por encima del agregado, la cantidad demandada cae según una función de elasticidad constante.


%========================================
% Ecuación (2)
%========================================
\subsubsection*{Ecuación (2): Condición de oferta de trabajo log-linealizada}

La Ecuación (2) es la versión log-linealizada de la condición de optimalidad que determina la \emph{oferta de trabajo} del hogar representativo. En notación logarítmica, se escribe como:
\begin{equation}
w_t - p_t \;=\; \sigma\, c_t \;+\; \phi\, n_t
\tag{2}
\end{equation}
donde el lado izquierdo es el salario real en logaritmos y el lado derecho recoge cómo dicho salario real debe compensar, en equilibrio, la desutilidad marginal del trabajo y la utilidad marginal del consumo.



\subsubsection*{Tabla de símbología relevante (Ecuación 2)}
\begin{table}[htbp]
  \centering
  \begin{tabular}{p{0.16\textwidth} p{0.44\textwidth} p{0.30\textwidth}}
    \hline
    \textbf{Término} & \textbf{Definición rigurosa} & \textbf{Rol económico} \\
    \hline
    \(w_t\) 
      & Logaritmo del salario nominal \(W_t\). 
      & Contribuye al salario real \(w_t - p_t\). \\
    \(p_t\) 
      & Logaritmo del índice de precios agregado \(P_t\). 
      & Deflaciona el salario nominal para obtener el salario real. \\
    \(c_t\) 
      & Logaritmo del índice de consumo agregado \(C_t\). 
      & Afecta la utilidad marginal del consumo. \\
    \(n_t\) 
      & Logaritmo del empleo u horas trabajadas \(N_t\). 
      & Afecta la desutilidad marginal del trabajo. \\
    \(\sigma > 0\) 
      & Inverso de la elasticidad de sustitución intertemporal del consumo. 
      & Sensibilidad de la utilidad marginal del consumo al nivel de \(c_t\). \\
    \(\phi > 0\) 
      & Parámetro de curvatura de la desutilidad del trabajo. 
      & Controla la sensibilidad de la desutilidad marginal del trabajo a \(n_t\). \\
    \hline
  \end{tabular}
  \caption{Símbolos usados en la Ecuación (2).}
\end{table}






\subsubsection*{Derivación matemática de la Ecuación (2)}

\paragraph{1. Función de utilidad periódica}

El hogar representativo tiene una función de utilidad instantánea del tipo:
\begin{equation}
U(C_t, N_t; Z_t)
=
\frac{C_t^{1-\sigma} - 1}{1-\sigma}
\;-\;
\frac{N_t^{1+\phi}}{1+\phi}\, Z_t,
\qquad \sigma \neq 1.
\tag{2.a}
\end{equation}
Aquí \(Z_t\) es un factor que escala la desutilidad del trabajo (choque o preferencia sobre el trabajo).



\paragraph{2. Utilidades marginales}

Derivamos la utilidad con respecto a consumo y trabajo:
\begin{equation}
U_{c,t} 
\equiv \frac{\partial U}{\partial C_t}
= C_t^{-\sigma},
\tag{2.b}
\end{equation}
\begin{equation}
U_{n,t} 
\equiv \frac{\partial U}{\partial N_t}
= - Z_t\, N_t^{\phi}.
\tag{2.c}
\end{equation}

\paragraph{3. Condición de optimalidad intraperiódica}

La condición de primer orden respecto al trabajo iguala el salario real con la RMS entre trabajo y consumo:
\begin{equation}
-\,\frac{U_{n,t}}{U_{c,t}}
\;=\;
\frac{W_t}{P_t}.
\tag{2.d}
\end{equation}
En el lado derecho aparece el salario real \(\frac{W_t}{P_t}\). El lado izquierdo es la RMS entre trabajo y consumo.

Sustituyendo (2.b) y (2.c) en (2.d), se obtiene:
\begin{align}
-\,\frac{U_{n,t}}{U_{c,t}}
&= 
-\,\frac{- Z_t N_t^{\phi}}{C_t^{-\sigma}}
=
\frac{Z_t N_t^{\phi}}{C_t^{-\sigma}}
=
Z_t\, C_t^{\sigma} N_t^{\phi}.
\tag{2.e}
\end{align}

Bajo el supuesto estándar de que el término \(Z_t\) es constante en el entorno del estado estacionario (o que se absorbe en el nivel de los parámetros), la condición no lineal puede escribirse como:
\begin{equation}
\frac{W_t}{P_t}
=
C_t^{\sigma} N_t^{\phi}.
\tag{2.f}
\end{equation}

\paragraph{4. Paso a logaritmos}

Definimos los logaritmos de las variables reales y nominales:
\begin{equation}
w_t \equiv \log W_t,
\qquad
p_t \equiv \log P_t,
\qquad
c_t \equiv \log C_t,
\qquad
n_t \equiv \log N_t.
\tag{2.g}
\end{equation}

Tomando logaritmos en ambos lados de (2.f), se tiene:
\begin{align}
\log\!\left(\frac{W_t}{P_t}\right)
&= \log\!\left( C_t^{\sigma} N_t^{\phi} \right) \notag\\[0.15cm]
w_t - p_t
&= \sigma\, \log C_t \;+\; \phi\, \log N_t \notag\\[0.15cm]
w_t - p_t
&= \sigma c_t \;+\; \phi n_t.
\tag{2.h}
\end{align}

Esta relación es lineal en términos de logaritmos y puede interpretarse como la versión log-lineal de la condición de oferta de trabajo alrededor de un estado estacionario no estocástico.

Identificando (2.h) como la ecuación de oferta de trabajo log-linealizada, recuperamos precisamente la Ecuación (2):
\[
w_t - p_t
=
\sigma\, c_t \;+\; \phi\, n_t.
\]

\subsubsection*{Interpretación económica de la Ecuación (2)}

La Ecuación (2) proporciona una relación de equilibrio entre:

\begin{itemize}
  \item \textbf{Salario real (lado izquierdo):} \(w_t - p_t\) es el pago real por unidad de trabajo.
  \item \textbf{Condiciones de preferencia (lado derecho):} 
  \begin{itemize}
    \item \(\sigma c_t\): cuanto mayor es el consumo \(c_t\), menor es la utilidad marginal del consumo; para que el hogar esté dispuesto a trabajar, el salario real debe compensar esta menor utilidad marginal.
    \item \(\phi n_t\): cuanto mayor es el trabajo \(n_t\), mayor es la desutilidad marginal del trabajo, lo que exige un salario real más alto para inducir una mayor oferta laboral.
  \end{itemize}
\end{itemize}

En términos intuitivos:
\begin{enumerate}
  \item Un \textbf{aumento del salario real} incentiva una mayor oferta de trabajo \(n_t\), pues el beneficio de trabajar (medido en consumo adicional) crece respecto al costo en términos de ocio.
  \item Un \textbf{aumento del consumo} \(c_t\) reduce la utilidad marginal del consumo; para mantener la condición de optimalidad, el hogar requerirá un salario real más alto o ajustará su oferta laboral.
  \item El parámetro \(\phi\) controla la \emph{elasticidad de la oferta laboral}: cuanto más grande es \(\phi\), más rápido crece la desutilidad marginal del trabajo y menos elástica es la respuesta de \(n_t\) ante cambios en el salario real.
\end{enumerate}


%========================================
% Ecuación (3)
%========================================
\subsubsection*{Ecuación (3): Ecuación de Euler log-linealizada para el consumo}

La Ecuación (3) es la versión log-linealizada de la condición de optimalidad intertemporal del hogar (Ecuación de Euler), que determina la relación entre consumo presente, consumo futuro esperado y la tasa de interés real. En el modelo básico nuevo keynesiano se escribe como:
\begin{equation}
c_t 
= E_t\{c_{t+1}\}
- \frac{1}{\sigma}\bigl(i_t - E_t\{\pi_{t+1}\} - \rho\bigr)
+ \frac{1}{\sigma}(1 - \rho_z)\,z_t.
\tag{3}
\end{equation}

El lado izquierdo es el consumo presente (en logaritmos o desviaciones logarítmicas). El lado derecho combina:
\begin{itemize}
  \item el consumo futuro esperado \(E_t\{c_{t+1}\}\),
  \item la tasa de interés real esperada \(i_t - E_t\{\pi_{t+1}\}\), ajustada por la preferencia temporal \(\rho\),
  \item y un término que recoge los efectos del shock de preferencias \(z_t\).
\end{itemize}

\subsubsection*{Tabla de símbolos relevantes (Ecuación 3)}

\begin{table}[htbp]
  \centering
  \begin{tabular}{p{0.18\textwidth} p{0.42\textwidth} p{0.30\textwidth}}
    \hline
    \textbf{Término} & \textbf{Definición rigurosa} & \textbf{Rol económico} \\
    \hline
    \(c_t\) 
      & Logaritmo (o desviación logarítmica) del consumo agregado \(C_t\). 
      & Consumo presente. \\
    \(E_t\{c_{t+1}\}\) 
      & Expectativa condicional en \(t\) del logaritmo del consumo \(C_{t+1}\). 
      & Consumo futuro esperado. \\
    \(i_t\) 
      & Tasa de interés nominal a corto plazo. Definida por \(i_t \equiv -\log Q_t\). 
      & Entra en la tasa de interés real esperada. \\
    \(\pi_{t+1}\) 
      & Inflación entre \(t\) y \(t+1\): \(\pi_{t+1} \equiv p_{t+1} - p_t\). 
      & Corrige la tasa nominal para obtener la tasa real. \\
    \(E_t\{\pi_{t+1}\}\) 
      & Inflación esperada para el periodo \(t+1\). 
      & Determina la tasa de interés real esperada. \\
    \(\rho\) 
      & \(\rho \equiv -\log \beta\), donde \(\beta \in (0,1)\) es el factor de descuento subjetivo. 
      & Captura la impaciencia del hogar (tasa de descuento subjetiva). \\
    \(z_t\) 
      & Logaritmo del shock de preferencias \(Z_t\). Satisface \(z_t = \rho_z z_{t-1} + \varepsilon_t^z\). 
      & Shifter de preferencias intertemporales (choque de demanda). \\
    \(\rho_z\) 
      & Parámetro de persistencia del shock de preferencias, \(0 \leq \rho_z < 1\). 
      & Determina cuán persistente es \(z_t\). \\
    \(\sigma > 0\) 
      & Inverso de la elasticidad de sustitución intertemporal del consumo. 
      & Controla la sensibilidad del consumo a la tasa de interés real. \\
    \hline
  \end{tabular}
  \caption{Símbolos usados en la Ecuación (3).}
\end{table}

\subsubsection*{Derivación matemática de la Ecuación (3)}

\paragraph{1. Problema intertemporal y condición de Euler}

El hogar maximiza la utilidad esperada:
\[
E_0 \sum_{t=0}^{\infty} \beta^t \, U(C_t, N_t; Z_t),
\]
sujeto a una restricción presupuestaria intertemporal. Para la derivación de la Ecuación de Euler, nos centramos en la utilidad marginal del consumo. Suponemos que la parte relevante de la utilidad puede escribirse de forma que la utilidad marginal del consumo sea:
\begin{equation}
U_{c,t} \equiv \frac{\partial U}{\partial C_t}
= Z_t\, C_t^{-\sigma},
\tag{3.a}
\end{equation}
es decir, \(Z_t\) desplaza (escala) la utilidad marginal del consumo y \(\sigma\) gobierna su curvatura.

La condición de Euler para el bono nominal de un periodo, cuyo precio en términos nominales es \(Q_t\), es:
\begin{equation}
Q_t
=
\beta\, E_t\left\{
\frac{U_{c,t+1}}{U_{c,t}}\,
\frac{P_t}{P_{t+1}}
\right\}.
\tag{3.b}
\end{equation}
Aquí \(P_t\) es el nivel de precios en \(t\), de modo que \(P_t/P_{t+1}\) es el inverso de la inflación bruta.

\paragraph{2. Sustitución de la utilidad marginal del consumo}

Sustituyendo \(U_{c,t}\) de (3.a) en (3.b):
\begin{align}
Q_t
&=
\beta\, E_t\left\{
\frac{Z_{t+1} C_{t+1}^{-\sigma}}{Z_t C_t^{-\sigma}}\,
\frac{P_t}{P_{t+1}}
\right\} \notag\\[0.15cm]
&=
\beta\, E_t\left\{
\left(\frac{C_{t+1}}{C_t}\right)^{-\sigma}
\left(\frac{Z_{t+1}}{Z_t}\right)
\left(\frac{P_t}{P_{t+1}}\right)
\right\}.
\tag{3.c}
\end{align}

\paragraph{3. Definiciones logarítmicas y de tasas de interés}

Definimos logaritmos:
\begin{equation}
c_t \equiv \log C_t,
\quad
p_t \equiv \log P_t,
\quad
z_t \equiv \log Z_t.
\tag{3.d}
\end{equation}
La inflación se define como:
\begin{equation}
\pi_{t+1} \equiv p_{t+1} - p_t.
\tag{3.e}
\end{equation}
Definimos también el logaritmo del precio del bono:
\begin{equation}
q_t \equiv \log Q_t
\quad\Rightarrow\quad
i_t \equiv -q_t,
\tag{3.f}
\end{equation}
de forma que \(i_t\) representa (aproximadamente) la tasa de interés nominal de corto plazo.

Tomando logaritmos dentro de la esperanza en (3.c) y linealizando alrededor de un estado estacionario no estocástico, usamos las aproximaciones de primer orden:
\[
\log\left(\frac{C_{t+1}}{C_t}\right) \approx c_{t+1} - c_t,
\quad
\log\left(\frac{Z_{t+1}}{Z_t}\right) \approx z_{t+1} - z_t,
\quad
\log\left(\frac{P_t}{P_{t+1}}\right) \approx -\pi_{t+1}.
\]

La condición de Euler (3.c) implica, en forma log-lineal aproximada:
\begin{equation}
q_t
\approx
\log \beta
+
E_t\left[
-\sigma (c_{t+1}-c_t)
+ (z_{t+1}-z_t)
- \pi_{t+1}
\right].
\tag{3.g}
\end{equation}

\paragraph{4. Uso de \(\rho \equiv -\log \beta\) y tasa real}

Definimos:
\begin{equation}
\rho \equiv -\log \beta > 0,
\tag{3.h}
\end{equation}
de modo que \(\log \beta = -\rho\). Además, con \(i_t \equiv -q_t\), de (3.g) se obtiene:
\begin{align}
-i_t
&\approx
-\rho
+
E_t\left[
-\sigma (c_{t+1}-c_t)
+ (z_{t+1}-z_t)
- \pi_{t+1}
\right] \notag\\[0.15cm]
i_t
&\approx
\rho
-
E_t\left[
-\sigma (c_{t+1}-c_t)
+ (z_{t+1}-z_t)
- \pi_{t+1}
\right] \notag\\[0.15cm]
i_t
&\approx
\rho
+
\sigma E_t(c_{t+1}-c_t)
- E_t(z_{t+1}-z_t)
+ E_t(\pi_{t+1}).
\tag{3.i}
\end{align}

Reordenando para aislar \(c_t - E_t\{c_{t+1}\}\), tenemos:
\begin{align}
\sigma\bigl(c_t - E_t\{c_{t+1}\}\bigr)
&\approx
i_t - E_t\{\pi_{t+1}\} - \rho
- \bigl(E_t z_{t+1} - z_t\bigr).
\tag{3.j}
\end{align}
Dividiendo entre \(\sigma\):
\begin{equation}
c_t - E_t\{c_{t+1}\}
\approx
-\frac{1}{\sigma}\,\bigl(i_t - E_t\{\pi_{t+1}\} - \rho\bigr)
-\frac{1}{\sigma}\,\bigl(E_t z_{t+1} - z_t\bigr).
\tag{3.k}
\end{equation}

\paragraph{5. Proceso para el shock de preferencias \(z_t\)}

Suponemos que el shock de preferencias sigue un proceso AR(1):
\begin{equation}
z_t = \rho_z z_{t-1} + \varepsilon_t^z,
\qquad 0 \leq \rho_z < 1.
\tag{3.l}
\end{equation}
Entonces:
\[
E_t\{z_{t+1}\} = \rho_z z_t,
\quad\Rightarrow\quad
E_t z_{t+1} - z_t = (\rho_z - 1) z_t.
\]
Sustituyendo en (3.k):
\begin{align}
c_t - E_t\{c_{t+1}\}
&\approx
-\frac{1}{\sigma}\,\bigl(i_t - E_t\{\pi_{t+1}\} - \rho\bigr)
-\frac{1}{\sigma}(\rho_z - 1)\,z_t \notag\\[0.15cm]
&=
-\frac{1}{\sigma}\,\bigl(i_t - E_t\{\pi_{t+1}\} - \rho\bigr)
+\frac{1}{\sigma}(1 - \rho_z)\,z_t.
\tag{3.m}
\end{align}

Finalmente, reordenando para despejar \(c_t\):
\begin{equation}
c_t
=
E_t\{c_{t+1}\}
-\frac{1}{\sigma}\,\bigl(i_t - E_t\{\pi_{t+1}\} - \rho\bigr)
+\frac{1}{\sigma}(1 - \rho_z)\,z_t,
\end{equation}
que coincide exactamente con la Ecuación (3).

\subsubsection*{Interpretación económica de la Ecuación (3)}

La Ecuación (3) puede leerse como:
\[
c_t
=
E_t\{c_{t+1}\}
-\frac{1}{\sigma}\,\bigl(i_t - E_t\{\pi_{t+1}\} - \rho\bigr)
+\frac{1}{\sigma}(1 - \rho_z)\,z_t.
\]

\begin{itemize}
  \item \textbf{Consumo futuro esperado:}  
  El consumo presente \(c_t\) está positivamente relacionado con el consumo futuro esperado \(E_t\{c_{t+1}\}\). Si el hogar anticipa un nivel alto de consumo futuro, será óptimo suavizar el consumo y mantener un nivel relativamente alto también en \(t\).

  \item \textbf{Tasa de interés real esperada:}  
  El término \(i_t - E_t\{\pi_{t+1}\}\) es la tasa de interés real ex-ante. La ecuación muestra que:
  \[
  \frac{\partial c_t}{\partial (i_t - E_t\{\pi_{t+1}\})} = -\frac{1}{\sigma} < 0.
  \]
  Es decir, un aumento de la tasa de interés real (por encima de \(\rho\)) encarece el consumo presente respecto al consumo futuro, induciendo al hogar a reducir \(c_t\) y ahorrar más. La magnitud de esta respuesta está inversamente relacionada con \(\sigma\): una menor \(\sigma\) (mayor elasticidad de sustitución intertemporal) implica una respuesta más fuerte del consumo a la tasa real.

  \item \textbf{Choques de preferencias \(z_t\):}  
  El término \(\frac{1}{\sigma}(1 - \rho_z)\,z_t\) muestra cómo los shocks de preferencias afectan el consumo actual. Si \(\rho_z\) es bajo (shock poco persistente), el ajuste \((1-\rho_z)\) es grande y el shock afecta más intensamente al consumo de hoy; si \(\rho_z\) se acerca a 1 (shock muy persistente), el efecto se reparte a lo largo del tiempo.

  \item \textbf{Interpretación como DIS (Ecuación IS Dinámica):}  
  Al combinar esta ecuación con la condición de equilibrio del mercado de bienes (en el modelo básico, \(Y_t = C_t\)), puede reescribirse reemplazando \(c_t\) por \(y_t\), dando lugar a la llamada \emph{Ecuación IS dinámica}, que liga el nivel de actividad (producto) con la tasa de interés real esperada y con choques de demanda.
\end{itemize}

%========================================
% Ecuación (5)
%========================================
\subsubsection*{Ecuación (5): Función de producción de la empresa}

La Ecuación (5) introduce la tecnología de producción utilizada por cada empresa \(i\) en el modelo. Se asume que todas las empresas comparten una \emph{misma} función de producción, dada por:
\begin{equation}
Y_t(i) \;=\; A_t\, N_t(i)^{1-\alpha}
\tag{5}
\end{equation}
donde el único insumo variable es el trabajo \(N_t(i)\) y \(A_t\) representa el nivel agregado de tecnología o productividad total de los factores.

\subsubsection*{Tabla de símbolos relevantes (Ecuación 5)}

\begin{table}[htbp]
  \centering
  \begin{tabular}{p{0.18\textwidth} p{0.42\textwidth} p{0.30\textwidth}}
    \hline
    \textbf{Término} & \textbf{Definición rigurosa} & \textbf{Rol económico} \\
    \hline
    \(Y_t(i)\) 
      & Producción del bien diferenciado \(i\) en el periodo \(t\). 
      & Oferta individual de la variedad \(i\). \\
    \(A_t\) 
      & Nivel de tecnología común a todas las empresas. 
      & Factor de productividad total, exógeno. \\
    \(N_t(i)\) 
      & Empleo (u horas trabajadas) utilizadas por la empresa \(i\) en el periodo \(t\). 
      & Único insumo variable en la producción. \\
    \(\alpha \in [0,1)\) 
      & Parámetro de la función de producción. El exponente del trabajo es \(1-\alpha\). 
      & Determina la elasticidad del producto respecto al trabajo. \\
    \hline
  \end{tabular}
  \caption{Símbolos usados en la Ecuación (5).}
\end{table}

\subsubsection*{Derivación y propiedades básicas de la función de producción}

La Ecuación (5) se postula como una función de producción de tipo Cobb--Douglas con un único factor de producción explícito: el trabajo. Aquí no se modela el capital como input variable, por lo que la parte relevante de la tecnología se condensa en \(A_t\).

\paragraph{1. Forma funcional y rendimientos respecto al trabajo}

La función
\begin{equation}
Y_t(i) = A_t\, N_t(i)^{1-\alpha}
\tag{5.a}
\end{equation}
tiene las siguientes propiedades:

\begin{itemize}
  \item \textbf{Elasticidad del producto respecto al trabajo:}  
  La elasticidad del producto de la empresa \(i\) respecto a su insumo de trabajo \(N_t(i)\) es \(1-\alpha\). Es decir, un cambio proporcional en \(N_t(i)\) genera un cambio proporcional en \(Y_t(i)\) de magnitud \(1-\alpha\).

  \item \textbf{Rendimientos a escala en el trabajo:}  
  Dado que sólo varía \(N_t(i)\):
  \begin{itemize}
    \item Si \(\alpha = 0\), entonces \(Y_t(i) = A_t N_t(i)\): hay rendimientos constantes a escala en el trabajo (una duplicación de \(N_t(i)\) duplica la producción).
    \item Si \(\alpha > 0\), el exponente \(1-\alpha \in (0,1)\) implica rendimientos \emph{decrecientes} en el trabajo: duplicar \(N_t(i)\) aumenta \(Y_t(i)\) menos que proporcionalmente.
  \end{itemize}
\end{itemize}

\paragraph{2. Producto marginal del trabajo}

El \emph{producto marginal del trabajo} (PMN o MPN) de la empresa \(i\) se obtiene derivando \(Y_t(i)\) respecto a \(N_t(i)\):
\begin{align}
MPN_t(i)
&\equiv \frac{\partial Y_t(i)}{\partial N_t(i)} \notag\\[0.1cm]
&= A_t\, (1-\alpha)\, N_t(i)^{-\alpha}.
\tag{5.b}
\end{align}

Propiedades:
\begin{itemize}
  \item \(MPN_t(i)\) es proporcional a \(A_t\): un mayor nivel de tecnología aumenta el producto marginal del trabajo en la misma proporción.
  \item Si \(\alpha > 0\), el producto marginal del trabajo es \emph{decreciente} en \(N_t(i)\): al aumentar el empleo, el rendimiento adicional de una unidad extra de trabajo disminuye.
\end{itemize}

Esta expresión será central en la determinación del \emph{costo marginal} de la empresa, ya que el costo marginal real se obtiene como salario real dividido entre el producto marginal del trabajo:
\begin{equation}
MC_t(i) \propto \frac{W_t / P_t}{MPN_t(i)}.
\tag{5.c}
\end{equation}
En el modelo, con empresas idénticas y competencia monopolística, el costo marginal será un determinante clave del precio óptimo.

\paragraph{3. Versión en logaritmos}

Definimos:
\begin{equation}
y_t(i) \equiv \log Y_t(i),
\quad
a_t \equiv \log A_t,
\quad
n_t(i) \equiv \log N_t(i).
\tag{5.d}
\end{equation}

Tomando logaritmos en (5.a):
\begin{align}
\log Y_t(i) 
&= \log A_t + (1-\alpha)\, \log N_t(i) \notag\\[0.1cm]
y_t(i)
&= a_t + (1-\alpha)\, n_t(i).
\tag{5.e}
\end{align}

Esta relación lineal en logaritmos es la base para la posterior \emph{log-linealización agregada}. Más adelante, cuando se agreguen las decisiones de todas las empresas (bajo simetría en \(A_t\) y en ciertos casos en \(N_t(i)\)), se obtendrá una relación agregada entre el producto total, el empleo total y la tecnología.

\subsubsection*{Interpretación económica de la Ecuación (5)}

La Ecuación (5) cumple varios roles clave en el modelo:

\begin{itemize}
  \item \textbf{Tecnología común y choques de productividad:}  
  El término \(A_t\) representa el nivel de productividad total de la economía. Un aumento de \(A_t\) (choque tecnológico positivo) desplaza hacia arriba la función de producción de todas las empresas: para un mismo nivel de trabajo \(N_t(i)\), la producción es mayor. Estos choques son responsables de variaciones en el producto natural y en el equilibrio de largo plazo del modelo.

  \item \textbf{Demanda de trabajo de la empresa:}  
  Dado un salario real \(W_t/P_t\) y un precio de venta \(P_t(i)\), cada empresa elige \(N_t(i)\) para maximizar beneficios. La condición de primer orden iguala el valor del producto marginal del trabajo con el salario real. Por lo tanto, el \(MPN_t(i)\) derivado en (5.b) entra directamente en la ecuación de demanda de trabajo de la empresa.

  \item \textbf{Costo marginal y fijación de precios:}  
  En un mercado de bienes con competencia monopolística, cada empresa fija su precio como un margen sobre el costo marginal:
  \[
  P_t(i) \propto \mu \times MC_t(i),
  \]
  donde \(\mu\) es el mark-up. Dado que \(MC_t(i)\) depende del \(MPN_t(i)\) y por tanto de \(A_t\) y \(N_t(i)\), la tecnología de producción condiciona el comportamiento de precios y, en última instancia, la inflación.

  \item \textbf{Relación agregada producto-trabajo-tecnología:}  
  Bajo aproximaciones de primer orden y simetría entre empresas, la versión agregada de (5.e) implica una relación del tipo:
  \[
  y_t \approx a_t + (1-\alpha)\, n_t,
  \]
  que será usada más adelante para conectar empleo agregado, producto agregado y choques tecnológicos, así como para definir el nivel de producto natural.
\end{itemize}

%========================================
% Ecuación (6)
%========================================
\subsubsection*{Ecuación (6): Proceso AR(1) para la tecnología}

La Ecuación (6) describe la dinámica estocástica del nivel de tecnología agregada en el modelo. En términos logarítmicos, se supone que la tecnología sigue un proceso autorregresivo de primer orden:
\begin{equation}
a_t = \rho_a\, a_{t-1} + \varepsilon_{a,t}
\tag{6}
\end{equation}
donde \(a_t\) es el logaritmo del nivel de tecnología \(A_t\), \(\rho_a\) mide la persistencia del proceso y \(\varepsilon_{a,t}\) es un shock puramente exógeno.

\subsubsection*{Tabla de símbolos relevantes (Ecuación 6)}

\begin{table}[htbp]
  \centering
  \begin{tabular}{p{0.18\textwidth} p{0.42\textwidth} p{0.30\textwidth}}
    \hline
    \textbf{Término} & \textbf{Definición rigurosa} & \textbf{Rol económico} \\
    \hline
    \(A_t\) 
      & Nivel de tecnología agregada en el periodo \(t\). 
      & Multiplicador de productividad en la función de producción. \\
    \(a_t\) 
      & Logaritmo de \(A_t\), es decir \(a_t \equiv \log A_t\). 
      & Representa la tecnología en unidades logarítmicas. \\
    \(\rho_a\) 
      & Coeficiente autorregresivo del proceso de \(a_t\), con \(0 \le \rho_a < 1\). 
      & Mide la persistencia del shock tecnológico. \\
    \(\varepsilon_{a,t}\) 
      & Shock de tecnología, ruido blanco con media cero y varianza constante. 
      & Fuente exógena de fluctuaciones en la productividad. \\
    \hline
  \end{tabular}
  \caption{Símbolos usados en la Ecuación (6).}
\end{table}

\subsubsection*{Origen y formulación del proceso tecnológico}

\paragraph{1. Tecnología en la función de producción}

En la Ecuación (5) se postuló la función de producción de cada empresa:
\begin{equation}
Y_t(i) = A_t\, N_t(i)^{1-\alpha},
\tag{6.a}
\end{equation}
donde \(A_t\) es un factor de productividad común a todas las firmas. Para analizar la dinámica del modelo, es necesario especificar cómo evoluciona \(A_t\) a lo largo del tiempo.

\paragraph{2. Paso a logaritmos}

Definimos:
\begin{equation}
a_t \equiv \log A_t.
\tag{6.b}
\end{equation}
En estos términos, las variaciones en \(a_t\) miden cambios porcentuales (aproximados) en el nivel de tecnología \(A_t\).

\paragraph{3. Suposición AR(1) sobre la tecnología}

Se postula que \(a_t\) sigue un proceso autorregresivo de primer orden:
\begin{equation}
a_t = \rho_a\, a_{t-1} + \varepsilon_{a,t},
\tag{6.c}
\end{equation}
donde:
\begin{itemize}
  \item \(|\rho_a| < 1\): garantiza que el proceso sea estacionario en torno a una media (que tomamos como cero en desviaciones).
  \item \(\varepsilon_{a,t}\) es un shock de ruido blanco: \(E(\varepsilon_{a,t})=0\), \(E(\varepsilon_{a,t}\varepsilon_{a,s})=0\) para \(t \neq s\).
\end{itemize}

Este tipo de proceso es estándar en macroeconomía para modelar variables exógenas que muestran persistencia en el tiempo, como la productividad agregada.

\paragraph{4. Propiedades básicas del proceso}

Repitiendo (6.c) hacia atrás recursivamente, se obtiene:
\begin{align}
a_t
&= \rho_a a_{t-1} + \varepsilon_{a,t} \notag\\
&= \rho_a (\rho_a a_{t-2} + \varepsilon_{a,t-1}) + \varepsilon_{a,t} \notag\\
&= \rho_a^2 a_{t-2} + \rho_a \varepsilon_{a,t-1} + \varepsilon_{a,t} \notag\\
&\;\;\vdots \notag\\
&= \rho_a^k a_{t-k} + \sum_{j=0}^{k-1} \rho_a^j \varepsilon_{a,t-j}.
\tag{6.d}
\end{align}
Para \(k\) grande y \(|\rho_a|<1\), el término \(\rho_a^k a_{t-k}\) se vuelve despreciable, y la dinámica actual de \(a_t\) está dominada por el historial descontado de shocks pasados:
\begin{equation}
a_t \approx \sum_{j=0}^{\infty} \rho_a^j \varepsilon_{a,t-j}.
\tag{6.e}
\end{equation}
Esto muestra que un shock \(\varepsilon_{a,t}\) tiene efectos persistentes, pero decrecientes, sobre la trayectoria futura de la productividad.

\subsubsection*{Interpretación económica de la Ecuación (6)}

La Ecuación (6) cumple el papel de \emph{proceso generador de shocks de oferta}:

\begin{itemize}
  \item \textbf{Persistencia de los shocks tecnológicos:}  
  Si \(\rho_a\) está cerca de 1, un aumento de \(a_t\) (choque tecnológico positivo) afecta la productividad durante muchos periodos, aproximándose a un shock casi permanente. Si \(\rho_a\) es pequeño, el impacto se disipa rápidamente.

  \item \textbf{Impacto sobre la oferta agregada y el producto natural:}  
  Dado que la tecnología entra multiplicativamente en la función de producción, un shock positivo \(\varepsilon_{a,t}>0\) eleva \(A_t\), desplaza hacia arriba la productividad de todas las empresas, reduce el costo marginal y tiende a aumentar el nivel de producción que prevalecería con precios flexibles (producto natural).

  \item \textbf{Relación con la tasa natural de interés:}  
  En el equilibrio con precios flexibles, un aumento persistente de la productividad modifica el perfil óptimo de consumo intertemporal, afectando la tasa de interés real que iguala ahorro e inversión (tasa natural). Por ello, \(a_t\) aparece como determinante de \(y_{n,t}\) y de \(r_{n,t}\) en las ecuaciones posteriores del modelo.

  \item \textbf{Generador de ciclos económicos:}  
  En el modelo, las innovaciones \(\varepsilon_{a,t}\) constituyen una de las fuentes fundamentales de fluctuaciones macroeconómicas: al alterar la productividad, cambian el producto, el empleo y, a través del costo marginal, la inflación.
\end{itemize}

%========================================
% Ecuación (7)
%========================================
\subsubsection*{Ecuación (7): Ley de movimiento del nivel de precios bajo Calvo}

La Ecuación (7) describe la dinámica del índice de precios agregado en presencia de rigideces nominales a la Calvo. Bajo el supuesto de que en cada periodo sólo una fracción \(1-\theta\) de las empresas puede reajustar su precio, la evolución del nivel de precios viene dada por:
\begin{equation}
\Pi_t^{\,1-\epsilon}
\;=\;
\theta
\;+\;
(1-\theta)\left(\frac{P_t^*}{P_{t-1}}\right)^{1-\epsilon},
\tag{7}
\end{equation}
donde \(\Pi_t \equiv \dfrac{P_t}{P_{t-1}}\) es la inflación bruta entre \(t-1\) y \(t\), \(P_t\) es el nivel agregado de precios y \(P_t^*\) es el precio fijado por las empresas que reajustan en el periodo \(t\).

\subsubsection*{Tabla de símbolos relevantes (Ecuación 7)}

\begin{table}[htbp]
  \centering
  \begin{tabular}{p{0.18\textwidth} p{0.42\textwidth} p{0.30\textwidth}}
    \hline
    \textbf{Término} & \textbf{Definición rigurosa} & \textbf{Rol económico} \\
    \hline
    \(P_t\) 
      & Índice de precios agregado en \(t\):
        \(
        P_t 
        \equiv 
        \Bigl(\int_0^1 P_t(i)^{1-\epsilon} di \Bigr)^{\frac{1}{1-\epsilon}}.
        \)
      & Nivel general de precios de la economía. \\
    \(P_{t-1}\) 
      & Índice de precios agregado en el periodo \(t-1\). 
      & Referencia para medir la inflación entre \(t-1\) y \(t\). \\
    \(\Pi_t\) 
      & Inflación bruta entre \(t-1\) y \(t\): \(\Pi_t \equiv P_t/P_{t-1}\). 
      & Tasa de crecimiento del nivel de precios. \\
    \(P_t^*\) 
      & Precio fijado en \(t\) por las empresas que pueden reajustar. 
      & Precio óptimo común de la cohorte que reoptimiza. \\
    \(\epsilon > 1\) 
      & Elasticidad de sustitución entre variedades de bienes. 
      & Determina el grado de competencia monopolística. \\
    \(\theta \in [0,1)\) 
      & Probabilidad (fraccción) de \emph{no} reajustar el precio en un periodo. 
      & Índice de rigidez nominal; \(1/(1-\theta)\) es la duración media de un precio. \\
    \hline
  \end{tabular}
  \caption{Símbolos usados en la Ecuación (7).}
\end{table}

\subsubsection*{Derivación de la ley de movimiento del nivel de precios}

\paragraph{1. Índice de precios CES}

El índice de precios agregado se define a partir de la estructura CES de preferencias sobre las variedades \(i \in [0,1]\):
\begin{equation}
P_t 
\equiv 
\left(
  \int_0^1 P_t(i)^{1-\epsilon}\, di
\right)^{\frac{1}{1-\epsilon}},
\tag{7.a}
\end{equation}
lo que implica, elevando a la potencia \(1-\epsilon\):
\begin{equation}
P_t^{\,1-\epsilon}
=
\int_0^1 P_t(i)^{1-\epsilon}\, di.
\tag{7.b}
\end{equation}

\paragraph{2. Estructura de fijación de precios de Calvo}

Bajo el esquema de Calvo:
\begin{itemize}
  \item En cada periodo \(t\), una fracción \(1-\theta\) de empresas puede reajustar su precio y elige un mismo precio óptimo \(P_t^*\).
  \item Una fracción \(\theta\) de empresas \emph{no} reajusta su precio y mantiene el precio fijado en algún periodo anterior. En particular, las que no reajustan en \(t\) conservan el precio que tenían en \(t-1\).
\end{itemize}

Sea \(\mathcal{N}_t\) el conjunto de empresas que no reajustan en \(t\) y \(\mathcal{R}_t\) el conjunto de empresas que sí reajustan. La medida (masa) de cada conjunto es:
\[
\text{medida}(\mathcal{N}_t) = \theta,
\qquad
\text{medida}(\mathcal{R}_t) = 1-\theta.
\]

\paragraph{3. Descomposición del índice de precios}

Usando (7.b) y descomponiendo la integral en los dos subconjuntos:
\begin{align}
P_t^{\,1-\epsilon}
&=
\int_{\mathcal{N}_t} P_t(i)^{1-\epsilon} di
\;+\;
\int_{\mathcal{R}_t} P_t(i)^{1-\epsilon} di.
\tag{7.c}
\end{align}

\emph{(i) Empresas que no reajustan:}  
Si una empresa no reajusta en \(t\), su precio permanece igual al que tenía en \(t-1\):
\[
P_t(i) = P_{t-1}(i), \quad \forall i \in \mathcal{N}_t.
\]
Bajo el supuesto de que la distribución de precios entre estas empresas es la misma que la distribución general de precios en el periodo anterior, la integral sobre \(\mathcal{N}_t\) se puede escribir como:
\begin{equation}
\int_{\mathcal{N}_t} P_t(i)^{1-\epsilon} di
=
\theta \int_0^1 P_{t-1}(i)^{1-\epsilon} di
=
\theta\, P_{t-1}^{\,1-\epsilon},
\tag{7.d}
\end{equation}
donde en el último paso hemos usado la definición (análoga a (7.b)) para \(P_{t-1}\).

\emph{(ii) Empresas que sí reajustan:}  
Las firmas en \(\mathcal{R}_t\) fijan todas el mismo precio optimizado \(P_t^*\). Por lo tanto:
\[
P_t(i) = P_t^*, \quad \forall i \in \mathcal{R}_t,
\]
y la integral sobre \(\mathcal{R}_t\) es:
\begin{equation}
\int_{\mathcal{R}_t} P_t(i)^{1-\epsilon} di
=
(1-\theta)\, (P_t^*)^{1-\epsilon}.
\tag{7.e}
\end{equation}

\paragraph{4. Combinación y normalización por \(P_{t-1}\)}

Sustituyendo (7.d) y (7.e) en (7.c), obtenemos:
\begin{equation}
P_t^{\,1-\epsilon}
=
\theta\, P_{t-1}^{\,1-\epsilon}
+
(1-\theta)\,(P_t^*)^{1-\epsilon}.
\tag{7.f}
\end{equation}

Dividimos ambos lados de (7.f) por \(P_{t-1}^{\,1-\epsilon}\):
\begin{align}
\left(\frac{P_t}{P_{t-1}}\right)^{1-\epsilon}
&=
\theta
+
(1-\theta)\left(\frac{P_t^*}{P_{t-1}}\right)^{1-\epsilon}.
\tag{7.g}
\end{align}

Finalmente, al definir la inflación bruta como
\begin{equation}
\Pi_t \equiv \frac{P_t}{P_{t-1}},
\tag{7.h}
\end{equation}
podemos reescribir (7.g) como:
\begin{equation}
\Pi_t^{\,1-\epsilon}
=
\theta
+
(1-\theta)\left(\frac{P_t^*}{P_{t-1}}\right)^{1-\epsilon},
\end{equation}
que coincide exactamente con la Ecuación (7).

\subsubsection*{Interpretación económica de la Ecuación (7)}

La Ecuación (7) es la \emph{ley de movimiento del nivel de precios agregado} bajo rigideces de Calvo:

\[
\Pi_t^{\,1-\epsilon}
=
\theta
+
(1-\theta)\left(\frac{P_t^*}{P_{t-1}}\right)^{1-\epsilon}.
\]

\begin{itemize}
  \item \textbf{Peso de los precios viejos vs precios nuevos:}  
  El lado derecho es una combinación de:
  \begin{itemize}
    \item un peso \(\theta\) asociado a los precios que permanecen sin cambio y reflejan la estructura de \(P_{t-1}\);
    \item un peso \(1-\theta\) asociado a los precios nuevos \(P_t^*\) fijados por las firmas que ajustan.
  \end{itemize}
  Cuanto mayor es \(\theta\), más inercia presenta el índice de precios agregado, pues una mayor fracción de empresas mantiene su precio anterior.

  \item \textbf{Inflación como resultado del precio óptimo \(P_t^*\):}  
  Si el precio óptimo de las firmas que ajustan está por encima del nivel general de precios anterior (\(P_t^* > P_{t-1}\)), la nueva cohorte de precios tiende a empujar hacia arriba el nivel agregado \(P_t\), generando inflación (\(\Pi_t > 1\)).  
  Si \(P_t^* = P_{t-1}\), entonces los precios nuevos no introducen presiones adicionales y se obtiene \(\Pi_t = 1\) (inflación cero).

  \item \textbf{Duración media de los precios y rigideces nominales:}  
  El parámetro \(\theta\) puede interpretarse como la probabilidad de no reajuste en un periodo; la duración media de un precio es \(1/(1-\theta)\). Valores altos de \(\theta\) implican precios muy rígidos y, por tanto, una respuesta gradual del nivel de precios agregado a shocks.

  \item \textbf{Punto de partida para la Curva de Phillips Nuevo Keynesiana:}  
  La Ecuación (7) es el paso previo a la derivación de la Curva de Phillips Nuevo Keynesiana (NKPC).  
  La log-linealización de (7) alrededor de un estado estacionario con inflación constante (a menudo \(\Pi=1\)) conduce a una relación aproximada entre la inflación \(\pi_t\) y el precio óptimo en desviaciones \((p_t^* - p_{t-1})\), que luego se combina con la condición de fijación óptima de precios para obtener la NKPC en su forma estándar.
\end{itemize}

\subsubsection*{Sugerencia de gráfico y código: inflación bruta vs precio óptimo relativo}

Para un valor dado de \(\theta\) y \(\epsilon\), la Ecuación (7) establece una relación entre la inflación bruta \(\Pi_t\) y el precio relativo \(P_t^*/P_{t-1}\). Fijando \(\theta\) y \(\epsilon\), podemos resolver numéricamente \(\Pi_t\) a partir de:
\[
\Pi_t^{\,1-\epsilon}
=
\theta
+
(1-\theta)\left(\frac{P_t^*}{P_{t-1}}\right)^{1-\epsilon}.
\]

Un pequeño script en \texttt{Python} + \texttt{matplotlib} para ilustrar esta relación:

\begin{figure}[htbp]
  \centering
  \includegraphics[width=0.7\textwidth]{inflacion_vs_precio_optimo_eq7.png}
  \caption{Relación entre la inflación bruta y el precio óptimo relativo según la Ecuación (7).}
  \label{fig:inflacion_vs_precio_optimo_eq7}
\end{figure}

%========================================
% Ecuación (8)
%========================================
\subsubsection*{Ecuación (8): Log-linealización de la ley de movimiento de precios}

La Ecuación (8) es la versión log-linealizada de la ley de movimiento del nivel de precios bajo rigideces de Calvo (Ecuación (7)). En términos de inflación (en logaritmos) y del precio óptimo fijado por las empresas que reajustan, se escribe como:
\begin{equation}
\pi_t 
= (1 - \theta)\,\bigl(p_t^* - p_{t-1}\bigr),
\tag{8}
\end{equation}
donde \(\pi_t \equiv p_t - p_{t-1}\) es la inflación (en desviaciones logarítmicas), \(p_t^*\) es el logaritmo del precio óptimo fijado en \(t\) por las empresas que ajustan, y \(\theta\) es la probabilidad de \emph{no} poder ajustar el precio en un periodo dado.

\subsubsection*{Tabla de símbolos relevantes (Ecuación 8)}

\begin{table}[htbp]
  \centering
  \begin{tabular}{p{0.18\textwidth} p{0.42\textwidth} p{0.30\textwidth}}
    \hline
    \textbf{Término} & \textbf{Definición rigurosa} & \textbf{Rol económico} \\
    \hline
    \(p_t\) 
      & Logaritmo del nivel de precios agregado \(P_t\). 
      & Nivel de precios vigente en el periodo \(t\). \\
    \(p_{t-1}\) 
      & Logaritmo del nivel de precios agregado en \(t-1\). 
      & Nivel de precios de referencia (periodo anterior). \\
    \(\pi_t\) 
      & Inflación (aproximada) entre \(t-1\) y \(t\): \(\pi_t \equiv p_t - p_{t-1}\). 
      & Variación logarítmica del nivel de precios. \\
    \(p_t^*\) 
      & Logaritmo del precio óptimo \(P_t^*\) fijado por las empresas que reajustan en \(t\). 
      & Precio de la nueva cohorte de empresas que reoptimiza. \\
    \(\theta \in [0,1)\) 
      & Probabilidad de \emph{no} reajuste del precio en cada periodo. 
      & Índice de rigidez nominal; \(1-\theta\) es la fracción que ajusta en \(t\). \\
    \hline
  \end{tabular}
  \caption{Símbolos usados en la Ecuación (8).}
\end{table}

\subsubsection*{Derivación de la Ecuación (8) a partir de la Ecuación (7)}

Partimos de la ley exacta de movimiento del nivel de precios bajo Calvo, expresada en términos de inflación bruta \(\Pi_t \equiv P_t/P_{t-1}\):
\begin{equation}
\Pi_t^{\,1-\epsilon}
=
\theta
+
(1-\theta)\left(\frac{P_t^*}{P_{t-1}}\right)^{1-\epsilon},
\tag{7}
\end{equation}
donde \(\epsilon\) es la elasticidad de sustitución entre variedades.

\paragraph{1. Definiciones en logaritmos}

Definimos:
\begin{equation}
p_t \equiv \log P_t,
\quad
p_{t-1} \equiv \log P_{t-1},
\quad
p_t^* \equiv \log P_t^*.
\tag{8.a}
\end{equation}
La inflación logarítmica (aproximada) es:
\begin{equation}
\pi_t \equiv p_t - p_{t-1}.
\tag{8.b}
\end{equation}

Además, observamos que:
\[
\Pi_t = \frac{P_t}{P_{t-1}}
\quad\Rightarrow\quad
\log \Pi_t = \log P_t - \log P_{t-1} = p_t - p_{t-1} = \pi_t.
\]

\paragraph{2. Expresión de los términos en desviaciones alrededor del estado estacionario}

Consideramos un estado estacionario con inflación constante \(\bar{\Pi} = 1\), de modo que \(\bar{P}_t = \bar{P}_{t-1} = \bar{P}\). Para log-linealizar, trabajamos con pequeñas desviaciones alrededor de este estado estacionario. Definimos:
\[
\tilde{\Pi}_t \equiv \Pi_t - 1,
\quad
\tilde{x} \approx \log(1 + \tilde{x}) \;\; \text{para }|\tilde{x}|\text{ pequeño}.
\]

Similarmente, para el precio relativo de las empresas que ajustan:
\begin{equation}
\frac{P_t^*}{P_{t-1}}
=
\frac{P_t^*/\bar{P}}{P_{t-1}/\bar{P}}
\approx
\exp\bigl(p_t^* - p_{t-1}\bigr),
\tag{8.c}
\end{equation}
de modo que, para desviaciones pequeñas:
\[
\log\left(\frac{P_t^*}{P_{t-1}}\right) \approx p_t^* - p_{t-1}.
\]

\paragraph{3. Aproximación de primer orden de los términos \(\Pi_t^{1-\epsilon}\) y \(\bigl(P_t^*/P_{t-1}\bigr)^{1-\epsilon}\)}

Para valores pequeños de \(\pi_t\), tenemos:
\begin{align}
\Pi_t^{\,1-\epsilon}
&= \exp\bigl((1-\epsilon)\log \Pi_t\bigr)
= \exp\bigl((1-\epsilon)\pi_t\bigr)
\approx
1 + (1-\epsilon)\,\pi_t.
\tag{8.d}
\end{align}
De forma análoga, para el precio relativo de las empresas que ajustan:
\begin{align}
\left(\frac{P_t^*}{P_{t-1}}\right)^{1-\epsilon}
&= \exp\bigl((1-\epsilon)\log (P_t^*/P_{t-1})\bigr)
\approx
1 + (1-\epsilon)\,(p_t^* - p_{t-1}).
\tag{8.e}
\end{align}

\paragraph{4. Sustitución en la Ecuación (7) y simplificación}

Sustituimos las aproximaciones (8.d) y (8.e) en (7):
\begin{align}
1 + (1-\epsilon)\,\pi_t
&\approx
\theta
+
(1-\theta)\left[\,1 + (1-\epsilon)\,(p_t^* - p_{t-1})\right].
\tag{8.f}
\end{align}

Desarrollando el lado derecho:
\begin{align}
\theta
+
(1-\theta)\left[1 + (1-\epsilon)\,(p_t^* - p_{t-1})\right]
&=
\theta + (1-\theta) + (1-\theta)(1-\epsilon)\,(p_t^* - p_{t-1}) \notag\\[0.1cm]
&=
1 + (1-\theta)(1-\epsilon)\,(p_t^* - p_{t-1}).
\tag{8.g}
\end{align}

Igualando ambos lados de (8.f) y (8.g), los términos constantes \(1\) se cancelan y obtenemos:
\begin{equation}
(1-\epsilon)\,\pi_t
\approx
(1-\theta)(1-\epsilon)\,(p_t^* - p_{t-1}).
\tag{8.h}
\end{equation}

Suponiendo \(\epsilon \neq 1\), podemos dividir ambos lados por \((1-\epsilon)\) y, por lo tanto, la relación se simplifica a:
\begin{equation}
\pi_t 
=
(1-\theta)\,(p_t^* - p_{t-1}),
\end{equation}
que es exactamente la Ecuación (8).

\subsubsection*{Interpretación económica de la Ecuación (8)}

La Ecuación (8),
\[
\pi_t = (1-\theta)\,(p_t^* - p_{t-1}),
\]
tiene una interpretación directa y muy útil:

\begin{itemize}
  \item \textbf{Inflación como promedio ponderado de cambios de precios:}  
  La inflación logarítmica \(\pi_t\) es proporcional al diferencial entre:
  \begin{itemize}
    \item el logaritmo del precio óptimo fijado por las empresas que reajustan en \(t\), \(p_t^*\),
    \item y el logaritmo del nivel de precios agregado heredado del periodo anterior, \(p_{t-1}\).
  \end{itemize}
  Si \(p_t^* > p_{t-1}\), las empresas que reajustan tienden a aumentar el nivel de precios agregado, generando inflación positiva.

  \item \textbf{Rol de la fracción que ajusta \((1-\theta)\):}  
  El factor \((1-\theta)\) refleja que sólo una fracción de empresas actualiza su precio en cada periodo.  
  \begin{itemize}
    \item Si \(\theta\) es grande (precios muy rígidos), la inflación responde \emph{poco} a una dada diferencia \(p_t^* - p_{t-1}\); los cambios de precios se difunden lentamente.
    \item Si \(\theta\) es pequeña (precios muy flexibles), la inflación se ajusta rápidamente al “gap” entre el precio óptimo y el nivel de precios pasado.
  \end{itemize}

  \item \textbf{Puente hacia la Curva de Phillips Nuevo Keynesiana (NKPC):}  
  La Ecuación (8) conecta directamente la inflación con el precio óptimo \(p_t^*\).  
  Una vez que se derive la decisión óptima de fijación de precios (donde \(p_t^*\) se expresa como función del costo marginal presente y futuro), sustituir esa expresión en (8) permite obtener la NKPC en su forma habitual:
  \[
  \pi_t = \beta E_t\{\pi_{t+1}\} + \kappa\, \hat{mc}_t,
  \]
  donde \(\hat{mc}_t\) es el costo marginal real en desviaciones, y \(\kappa\) es un coeficiente que depende de \(\theta\), \(\beta\) y \(\epsilon\).
\end{itemize}

\subsubsection*{Sugerencia de gráfico y código: inflación vs gap de precios \((p_t^* - p_{t-1})\)}

La Ecuación (8) establece una relación lineal muy sencilla:
\[
\pi_t = (1-\theta)\,(p_t^* - p_{t-1}).
\]
Para un \(\theta\) dado, la inflación es una recta con pendiente \(1-\theta\) en el plano \(\bigl(p_t^* - p_{t-1},\, \pi_t\bigr)\).

\begin{figure}[htbp]
  \centering
  \includegraphics[width=0.7\textwidth]{inflacion_vs_gap_precios_eq8.png}
  \caption{Relación entre la inflación y el gap entre el precio óptimo y el nivel de precios pasado, según la Ecuación (8).}
  \label{fig:inflacion_vs_gap_precios_eq8}
\end{figure}


%========================================
% Ecuación (9)
%========================================
\subsubsection*{Ecuación (9): Restricción de demanda de la empresa que reajusta precios}

La Ecuación (9) recoge la secuencia de restricciones de demanda que enfrenta una empresa que reoptimiza su precio en el periodo \(t\). Dado el precio que fija en \(t\), \(P_t^*\), la cantidad demandada de su bien en cada periodo futuro \(t+k\) viene determinada por:
\begin{equation}
Y_{t+k\mid t}
=
\left(\frac{P_t^*}{P_{t+k}}\right)^{-\epsilon} C_{t+k},
\qquad k = 0,1,2,\dots
\tag{9}
\end{equation}
donde \(Y_{t+k\mid t}\) denota la producción (o ventas) en el periodo \(t+k\) de una empresa cuyo último reajuste de precios tuvo lugar en \(t\).

\subsubsection*{Tabla de símbolos relevantes (Ecuación 9)}

\begin{table}[htbp]
  \centering
  \begin{tabular}{p{0.19\textwidth} p{0.41\textwidth} p{0.30\textwidth}}
    \hline
    \textbf{Término} & \textbf{Definición rigurosa} & \textbf{Rol económico} \\
    \hline
    \(Y_{t+k\mid t}\) 
      & Producción (o ventas) en el periodo \(t+k\) de una empresa que fijó su precio por última vez en \(t\). 
      & Cantidad demandada del bien de esa empresa; coincide con su oferta efectiva. \\
    \(P_t^*\) 
      & Precio nominal fijado por la empresa en el periodo \(t\) cuando tiene la oportunidad de reajustar. 
      & Permanece fijo mientras la empresa no vuelva a tener oportunidad de reajustar. \\
    \(P_{t+k}\) 
      & Índice de precios agregado de la economía en el periodo \(t+k\). 
      & Referencia para el precio relativo del bien de la empresa. \\
    \(C_{t+k}\) 
      & Índice de consumo agregado en el periodo \(t+k\). 
      & Mide la demanda agregada de bienes en la economía. \\
    \(\epsilon > 1\) 
      & Elasticidad de sustitución entre variedades de bienes diferenciados. 
      & Determina la sensibilidad de la demanda al precio relativo. \\
    \hline
  \end{tabular}
  \caption{Símbolos usados en la Ecuación (9).}
\end{table}

\subsubsection*{Derivación de la restricción de demanda a partir de la Ecuación (1)}

El punto de partida es la función de demanda individual para cada variedad \(i\) que se obtuvo del problema de maximización del hogar (Ecuación (1)):
\begin{equation}
C_t(i)
=
\left(\frac{P_t(i)}{P_t}\right)^{-\epsilon} C_t,
\tag{1}
\end{equation}
donde \(C_t(i)\) es la cantidad del bien \(i\) demandada en el periodo \(t\), \(P_t(i)\) es el precio de la variedad \(i\) y \(P_t\) es el índice agregado de precios.

En el contexto de Calvo, consideramos una empresa genérica que reajusta su precio en \(t\) y fija un precio \(P_t^*\). A partir de ahí:

\begin{itemize}
  \item En el periodo \(t\), esta empresa cobra \(P_t(i) = P_t^*\).
  \item En el periodo \(t+1\), si no ha vuelto a reajustar, sigue cobrando \(P_{t+1}(i) = P_t^*\).
  \item Más generalmente, mientras no vuelva a reajustar, en \(t+k\) su precio sigue siendo:
  \begin{equation}
  P_{t+k}(i) = P_t^*, \qquad k = 0,1,2,\dots
  \tag{9.a}
  \end{equation}
\end{itemize}

\paragraph{1. Demanda de la variedad \(i\) en el periodo \(t+k\)}

Aplicando la fórmula de demanda (1) al periodo \(t+k\):
\begin{equation}
C_{t+k}(i)
=
\left(\frac{P_{t+k}(i)}{P_{t+k}}\right)^{-\epsilon} C_{t+k}.
\tag{9.b}
\end{equation}
Para una empresa que fijó su precio por última vez en \(t\), usamos \(P_{t+k}(i) = P_t^*\) (según (9.a)), de modo que:
\begin{equation}
C_{t+k}(i)
=
\left(\frac{P_t^*}{P_{t+k}}\right)^{-\epsilon} C_{t+k}.
\tag{9.c}
\end{equation}

\paragraph{2. Identificación con la producción \(Y_{t+k\mid t}\)}

Se supone que la empresa siempre ajusta su oferta para satisfacer la demanda al precio vigente, esto es:
\begin{equation}
Y_{t+k\mid t} = C_{t+k}(i),
\tag{9.d}
\end{equation}
para la empresa que fijó su precio en \(t\). Sustituyendo (9.c) en (9.d) obtenemos:
\begin{equation}
Y_{t+k\mid t}
=
\left(\frac{P_t^*}{P_{t+k}}\right)^{-\epsilon} C_{t+k},
\qquad k = 0,1,2,\dots,
\end{equation}
que es exactamente la Ecuación (9).

\subsubsection*{Interpretación económica de la Ecuación (9)}

La Ecuación (9) es la \emph{restricción de demanda} que enfrenta cada empresa que reajusta su precio. Sus elementos clave son:

\begin{itemize}
  \item \textbf{Demanda condicionada al precio relativo:}  
  La cantidad que vende la empresa en cada periodo futuro \((t+k)\) depende de:
  \begin{itemize}
    \item su precio fijo \(P_t^*\),
    \item el índice de precios agregado futuro \(P_{t+k}\),
    \item y el nivel de demanda agregada \(C_{t+k}\).
  \end{itemize}
  Todo esto entra a través del precio relativo \(\dfrac{P_t^*}{P_{t+k}}\): si la empresa fija un precio por encima del nivel general de precios futuro, \(\frac{P_t^*}{P_{t+k}}\) es alto, y la demanda de su producto se reduce, ya que el exponente \(-\epsilon < 0\) hace decreciente la demanda en el precio relativo.

  \item \textbf{Elasticidad de sustitución \(\epsilon\):}  
  El parámetro \(\epsilon\) determina cuán sensible es la demanda a cambios en el precio relativo.  
  \begin{itemize}
    \item Un \(\epsilon\) grande implica que los bienes son fácilmente sustituibles, por lo que pequeñas desviaciones del precio relativo con respecto al promedio tienen un gran efecto sobre la cantidad demandada.
    \item Un \(\epsilon\) cercano a 1 implica una sustitución limitada; la demanda es menos sensible al precio relativo.
  \end{itemize}

  \item \textbf{Conexión con beneficios y fijación óptima de precios:}  
  Dado que los beneficios de la empresa en cada periodo futuro dependen de:
  \[
  \Pi_{t+k\mid t} 
  = P_t^* Y_{t+k\mid t} - TC_{t+k\mid t},
  \]
  y \(Y_{t+k\mid t}\) viene determinado por (9), la empresa debe escoger \(P_t^*\) tomando en cuenta cómo este precio condiciona toda la senda de demanda futura (y por tanto de costos y beneficios). Esta consideración intertemporal es central para derivar la condición de primer orden de fijación de precios (la “NKPC no lineal”).

  \item \textbf{Hipótesis de demanda satisfecha:}  
  Implícitamente se asume que la empresa \emph{no raciona} la demanda: siempre produce \(Y_{t+k\mid t}\) para igualar la demanda al precio vigente. Esto es coherente con la idea de competencia monopolística con fijación de precios, donde la cantidad se ajusta para satisfacer la demanda dada la decisión de precios.
\end{itemize}

\subsubsection*{Sugerencia de gráfico y código: demanda de la empresa como función del precio relativo}

Para un periodo futuro \(t+k\) dado, y tomando \(P_{t+k}\) y \(C_{t+k}\) como dados, la Ecuación (9) implica una curva de demanda decreciente en el precio relativo \(\frac{P_t^*}{P_{t+k}}\):
\[
Y_{t+k\mid t}
=
\left(\frac{P_t^*}{P_{t+k}}\right)^{-\epsilon} C_{t+k}.
\]

Podemos fijar \(P_{t+k}=1\) sin pérdida de generalidad (normalización) y representar \(Y\) en función de \(P_t^*\). 

%========================================
% Ecuación (10)
%========================================
\subsubsection*{Ecuación (10): Condición de primer orden para el precio óptimo bajo Calvo}

La Ecuación (10) es la condición de primer orden que determina el precio óptimo \(P_t^*\) fijado por las empresas que tienen la oportunidad de reajustar en el periodo \(t\). Bajo rigideces de precios a la Calvo, este precio se elige tomando en cuenta que, con probabilidad \(\theta^k\), seguirá vigente en los periodos futuros \(t+k\). La condición de optimalidad se escribe como:
\begin{equation}
\sum_{k=0}^{\infty} \theta^k
E_t\!\left\{
  \Lambda_{t,t+k}\,
  Y_{t+k\mid t}\,
  \frac{1}{P_{t+k}}\,
  \bigl(P_t^* - M_{t+k\mid t}\bigr)
\right\}
= 0.
\tag{10}
\end{equation}

\subsubsection*{Tabla de símbolos relevantes (Ecuación 10)}

\begin{table}[htbp]
  \centering
  \begin{tabular}{p{0.19\textwidth} p{0.41\textwidth} p{0.30\textwidth}}
    \hline
    \textbf{Término} & \textbf{Definición rigurosa} & \textbf{Rol económico} \\
    \hline
    \(P_t^*\) 
      & Precio nominal fijado en \(t\) por las empresas que reoptimizan. 
      & Variable de elección de la empresa; se mantiene fijo mientras no pueda reajustar. \\
    \(Y_{t+k\mid t}\) 
      & Producción (ventas) en el periodo \(t+k\) de una empresa cuyo último reajuste se realizó en \(t\). 
      & Cantidad demandada de su bien al precio \(P_t^*\). \\
    \(M_{t+k\mid t}\) 
      & Costo marginal nominal de la empresa en el periodo \(t+k\), dado que su último reajuste fue en \(t\). 
      & Costo de producir una unidad adicional de output. \\
    \(P_{t+k}\) 
      & Índice de precios agregado de la economía en el periodo \(t+k\). 
      & Deflactor de magnitudes nominales a reales. \\
    \(\Lambda_{t,t+k}\) 
      & Factor estocástico de descuento: \(\Lambda_{t,t+k} \equiv \beta^k \frac{U_{c,t+k}}{U_{c,t}}\). 
      & Convierte magnitudes reales futuras en valor presente desde el punto de vista del hogar. \\
    \(\theta \in [0,1)\) 
      & Probabilidad de \emph{no} poder reajustar el precio en un periodo. 
      & Determina la duración esperada del precio \(P_t^*\): \(1/(1-\theta)\). \\
    \(M \equiv \dfrac{\epsilon}{\epsilon - 1}\) 
      & Markup deseado bajo competencia monopolística y precios flexibles. 
      & Relaciona el precio óptimo con el costo marginal en el caso sin rigideces. \\
    \hline
  \end{tabular}
  \caption{Símbolos usados en la Ecuación (10).}
\end{table}

\subsubsection*{Problema de la empresa y construcción de la condición de primer orden}

\paragraph{1. Flujo de beneficios y rigidez de Calvo}

Considere una empresa representativa que en el periodo \(t\) tiene la oportunidad de reajustar su precio. Si decide fijar \(P_t^*\), este precio seguirá vigente en los periodos futuros mientras la empresa no vuelva a tener oportunidad de reajustar. Bajo Calvo:

\begin{itemize}
  \item Con probabilidad \(1-\theta\), en el siguiente periodo podrá reajustar de nuevo.
  \item Con probabilidad \(\theta\), el precio \(P_t^*\) seguirá vigente.
\end{itemize}

La probabilidad de que el precio fijado en \(t\) continúe vigente en el periodo \(t+k\) es \(\theta^k\). Dado esto, el valor presente (desde \(t\)) de los beneficios reales esperados generados por el precio \(P_t^*\) es:
\begin{equation}
\max_{P_t^*}
\;
\sum_{k=0}^{\infty} \theta^k\,
E_t\!\left\{
  \Lambda_{t,t+k}\,
  \left[
    \frac{P_t^*}{P_{t+k}}\,Y_{t+k\mid t}
    \;-\;
    \frac{M_{t+k\mid t}}{P_{t+k}}\,Y_{t+k\mid t}
  \right]
\right\}.
\tag{10.a}
\end{equation}

Dentro de corchetes se encuentra el beneficio real en \(t+k\):
\[
\underbrace{\frac{P_t^*}{P_{t+k}} Y_{t+k\mid t}}_{\text{ingreso real}}
-
\underbrace{\frac{M_{t+k\mid t}}{P_{t+k}} Y_{t+k\mid t}}_{\text{costo real}}.
\]

\paragraph{2. Derivada respecto a \(P_t^*\): intuición}

Para obtener la condición de primer orden, derivamos el valor presente de beneficios reales con respecto a \(P_t^*\). Nótese que \(P_t^*\) afecta los beneficios de dos maneras:

\begin{itemize}
  \item Directamente, a través del término de ingreso \(\frac{P_t^*}{P_{t+k}} Y_{t+k\mid t}\).
  \item Indirectamente, a través de la cantidad demandada \(Y_{t+k\mid t}\), que depende de \(P_t^*\) mediante la restricción de demanda (Ecuación (9)):
  \[
  Y_{t+k\mid t}
  =
  \left(\frac{P_t^*}{P_{t+k}}\right)^{-\epsilon} C_{t+k}.
  \]
\end{itemize}

La derivada total de los beneficios reales de la empresa en \(t+k\) con respecto a \(P_t^*\) es, esquemáticamente:
\begin{align}
\frac{\partial}{\partial P_t^*}
\left[
  \left(\frac{P_t^*}{P_{t+k}}
         - \frac{M_{t+k\mid t}}{P_{t+k}}\right) Y_{t+k\mid t}
\right]
&=
\underbrace{\frac{1}{P_{t+k}} Y_{t+k\mid t}}_{\text{efecto directo}}
+
\underbrace{\left(\frac{P_t^*}{P_{t+k}}
              - \frac{M_{t+k\mid t}}{P_{t+k}}\right)
            \frac{\partial Y_{t+k\mid t}}{\partial P_t^*}}_{\text{efecto vía demanda}}.
\tag{10.b}
\end{align}

La parte clave es el efecto vía demanda, pues \(Y_{t+k\mid t}\) responde al cambio en el precio a través de la elasticidad de sustitución \(\epsilon\). Sustituyendo la forma de la demanda, se obtiene:
\[
\frac{\partial Y_{t+k\mid t}}{\partial P_t^*}
=
-\epsilon
\left(\frac{P_t^*}{P_{t+k}}\right)^{-\epsilon-1}
\frac{1}{P_{t+k}} C_{t+k}
=
-\epsilon
\frac{1}{P_t^*}\,
Y_{t+k\mid t}.
\]

Reemplazando en (10.b), y reordenando factores comunes, se puede escribir la derivada de los beneficios en una forma proporcional a:
\[
\frac{Y_{t+k\mid t}}{P_{t+k}}
\bigl(P_t^* - M_{t+k\mid t}\bigr),
\]
donde el coeficiente de proporcionalidad (que depende de \(\epsilon\)) no altera el hecho de que la condición de primer orden establezca la suma igual a cero.

\paragraph{3. Condición de primer orden agregada}

Al imponer que la derivada del valor presente (10.a) respecto a \(P_t^*\) sea igual a cero, obtenemos:
\begin{align}
0
&=
\sum_{k=0}^{\infty}
\theta^k\,
E_t\!\left\{
  \Lambda_{t,t+k}\,
  Y_{t+k\mid t}\,
  \frac{1}{P_{t+k}}\,
  \bigl(P_t^* - M_{t+k\mid t}\bigr)
\right\},
\tag{10.c}
\end{align}
que coincide con la Ecuación (10). Esta condición puede interpretarse como la igualdad (en promedio descontado) entre el \emph{beneficio marginal} y el \emph{costo marginal} de aumentar el precio \(P_t^*\).

\subsubsection*{Interpretación económica de la Ecuación (10)}

La Ecuación (10) resume el comportamiento de fijación de precios bajo rigideces a la Calvo:

\[
\sum_{k=0}^{\infty} \theta^k
E_t\!\left\{
  \Lambda_{t,t+k}\,
  Y_{t+k\mid t}\,
  \frac{1}{P_{t+k}}\,
  \bigl(P_t^* - M_{t+k\mid t}\bigr)
\right\}
= 0.
\]

\begin{itemize}
  \item \textbf{Carácter prospectivo (forward-looking):}  
  El precio \(P_t^*\) no se fija mirando sólo al coste actual, sino a la \emph{trayectoria esperada} de costos marginales nominales \(M_{t+k\mid t}\), cantidades \(Y_{t+k\mid t}\) y nivel de precios \(P_{t+k}\), ponderados por el factor de descuento del hogar \(\Lambda_{t,t+k}\) y por la probabilidad \(\theta^k\) de que el precio siga vigente.

  \item \textbf{Peso de cada periodo futuro:}  
  El término \(\theta^k \Lambda_{t,t+k} Y_{t+k\mid t} / P_{t+k}\) actúa como un \emph{peso} que indica cuán importante es el periodo \(t+k\) en la decisión de precios de hoy.  
  \begin{itemize}
    \item \(\theta^k\) captura la probabilidad de que el precio fijado en \(t\) permanezca hasta \(t+k\).
    \item \(\Lambda_{t,t+k}\) refleja el descuento intertemporal desde el punto de vista del hogar accionista.
    \item \(Y_{t+k\mid t} / P_{t+k}\) da el tamaño del mercado (en unidades reales) relevante para la empresa en ese periodo.
  \end{itemize}

  \item \textbf{Interpretación del término \(\bigl(P_t^* - M_{t+k\mid t}\bigr)\):}  
  Este término mide el \emph{margen nominal} entre el precio fijado y el costo marginal nominal en \(t+k\).  
  La Ecuación (10) exige que el promedio (descontado y ponderado) de ese margen, multiplicado por el volumen de ventas, sea nulo. En equilibrio, la empresa elige \(P_t^*\) de modo que esta condición se satisfaga, lo que equivale a fijar un precio que internaliza su efecto sobre la cantidad demandada y los beneficios futuros.

  \item \textbf{Caso límite: precios flexibles (\(\theta = 0\)):}  
  Si no hubiera rigideces de Calvo (\(\theta = 0\)), sólo el término \(k=0\) aparece en la suma:
  \[
  E_t\!\left\{
    \Lambda_{t,t}\,
    Y_{t\mid t}\,
    \frac{1}{P_t}\,
    \bigl(P_t^* - M_{t\mid t}\bigr)
  \right\}
  = 0.
  \]
  Como \(\Lambda_{t,t} = 1\) y los demás factores son positivos, esto implica:
  \[
  P_t^* = M_{t\mid t},
  \]
  es decir, el precio óptimo iguala al costo marginal nominal (o, si se incluye explícitamente el markup deseado \(M = \epsilon/(\epsilon - 1)\), el precio se fija como un markup constante sobre el costo marginal).  
  Bajo precios flexibles, la fijación de precios es “estática” y no mira hacia el futuro, en contraste con el caso \(\theta>0\).

  \item \textbf{Punto de partida de la Curva de Phillips Nuevo Keynesiana:}  
  Al log-linealizar la Ecuación (10) y combinarla con la Ecuación (8), se obtiene una relación entre inflación y costo marginal real (o brecha del producto), que es la Curva de Phillips Nuevo Keynesiana. En esa forma, el precio óptimo \(p_t^*\) aparece como una combinación ponderada de los costos marginales esperados, y su efecto sobre la inflación se canaliza a través de la rigidez de precios (\(\theta\)).
\end{itemize}

\subsubsection*{Sugerencia de gráfico y código: precio óptimo como promedio ponderado de costos marginales esperados}

Si resolvemos algebraicamente la Ecuación (10) en su versión log-lineal (que el texto desarrolla en ecuaciones posteriores), el precio óptimo en términos reales puede interpretarse como una media ponderada de los costos marginales reales esperados. Para ilustrar esta idea de forma simple, podemos suponer una trayectoria dada de costos marginales reales \(\{mc_{t+k}\}\) y mostrar cómo el precio real óptimo \(p_t^*\) se aproxima a un promedio ponderado de esos \(mc_{t+k}\).

%========================================
% Ecuación (11)
%========================================
\subsubsection*{Ecuación (11): Regla log-linealizada de fijación de precios óptimos}

La Ecuación (11) es la versión log-linealizada de la condición de primer orden de fijación de precios (Ecuación (10)). Expresa el \emph{precio óptimo en logaritmos} que fija una empresa que reajusta en \(t\), \(p_t^*\), como la suma de:
\begin{itemize}
  \item el logaritmo del markup deseado \(\mu\), y
  \item un promedio ponderado de los costos marginales (en log) esperados, \(\psi_{t+k\mid t}\).
\end{itemize}

Formalmente:
\begin{equation}
p_t^*
=
\mu
+
(1 - \beta\theta)
\sum_{k=0}^{\infty}
(\beta\theta)^k
E_t\{\psi_{t+k\mid t}\}.
\tag{11}
\end{equation}

\subsubsection*{Tabla de símbolos relevantes (Ecuación 11)}

\begin{table}[htbp]
  \centering
  \begin{tabular}{p{0.19\textwidth} p{0.41\textwidth} p{0.30\textwidth}}
    \hline
    \textbf{Término} & \textbf{Definición rigurosa} & \textbf{Rol económico} \\
    \hline
    \(p_t^*\)
      & Logaritmo del precio óptimo \(P_t^*\) fijado por las empresas que reajustan en \(t\). 
      & Variable de elección en el problema de fijación de precios. \\
    \(\mu\)
      & Logaritmo del markup deseado o “natural”; si \(M \equiv \dfrac{\epsilon}{\epsilon-1}\), entonces \(\mu \equiv \log M\). 
      & Nivel (en log) del margen de beneficio al que la empresa aspira en el estado estacionario. \\
    \(\psi_{t+k\mid t}\)
      & Logaritmo del costo marginal nominal \(M_{t+k\mid t}\) para una empresa que reajustó por última vez en \(t\). Puede interpretarse como el costo marginal (en log) relevante para fijar precios. 
      & Resume la información sobre los costos futuros que la empresa intenta trasladar a sus precios. \\
    \(\beta \in (0,1)\)
      & Factor de descuento intertemporal del hogar (y, por tanto, de los accionistas de la empresa). 
      & Descuenta el valor de los beneficios futuros. \\
    \(\theta \in [0,1)\)
      & Probabilidad de \emph{no} poder reajustar el precio en un periodo. 
      & Determina la duración esperada de un precio y el peso de los periodos futuros en la decisión actual. \\
    \((\beta\theta)^k\)
      & Producto del descuento intertemporal \(\beta^k\) y de la probabilidad de que el precio siga vigente \(\theta^k\). 
      & Peso relativo de los costos marginales en el periodo \(t+k\) en la decisión de \(p_t^*\). \\
    \(1 - \beta\theta\)
      & Factor de normalización. 
      & Garantiza que los pesos \((1-\beta\theta)(\beta\theta)^k\) sumen a uno. \\
    \hline
  \end{tabular}
  \caption{Símbolos usados en la Ecuación (11).}
\end{table}

\subsubsection*{De la condición de primer orden no lineal a la regla log-lineal de precios}

\paragraph{1. Recordatorio de la condición de primer orden (Ecuación 10)}

La Ecuación (10) impone que la suma descontada y ponderada de los márgenes nominales esperados sea igual a cero:
\begin{equation}
\sum_{k=0}^{\infty} \theta^k
E_t\!\left\{
  \Lambda_{t,t+k}\,
  Y_{t+k\mid t}\,
  \frac{1}{P_{t+k}}\,
  \bigl(P_t^* - M_{t+k\mid t}\bigr)
\right\}
= 0.
\tag{10}
\end{equation}
Aquí, \(\Lambda_{t,t+k}\) es el factor estocástico de descuento, \(Y_{t+k\mid t}\) las ventas futuras cuando el último reajuste fue en \(t\), \(P_{t+k}\) el nivel de precios agregado y \(M_{t+k\mid t}\) el costo marginal nominal.

\paragraph{2. Estado estacionario y expansión de primer orden}

La log-linealización se efectúa alrededor de un estado estacionario con:
\begin{itemize}
  \item inflación cero (o constante), de modo que \(P_t = \bar{P}\), \(P_{t+k} = \bar{P}\),
  \item factor de descuento determinista: \(\Lambda_{t,t+k} = \beta^k\),
  \item costo marginal nominal constante \(\bar{M}\) y precio \(\bar{P}\) que satisfacen la condición de markup deseado:
  \[
  \bar{P} = M \cdot \bar{MC}, 
  \quad M \equiv \frac{\epsilon}{\epsilon-1},
  \]
  y en logaritmos:
  \[
  \bar{p} = \mu + \bar{\psi},
  \]
  donde \(\bar{p} \equiv \log \bar{P}\) y \(\bar{\psi} \equiv \log \bar{MC}\).
\end{itemize}

To\-mamos desviaciones (en log) respecto a ese estado estacionario. Denotamos:
\[
p_t^* - \bar{p}, 
\quad
\psi_{t+k\mid t} - \bar{\psi},
\]
como las desviaciones de precio óptimo y de costo marginal.

\paragraph{3. Linealización de la condición de primer orden}

Usando que, cerca del estado estacionario, factores como
\[
\frac{\Lambda_{t,t+k}Y_{t+k\mid t}}{P_{t+k}}
\]
pueden aproximarse por su valor estacionario \(\beta^k \bar{Y} / \bar{P}\) más términos de orden superior, la condición (10) puede escribirse, a primer orden, como:
\begin{equation}
\sum_{k=0}^{\infty}
\theta^k \beta^k
E_t\!\left\{
  \bigl(p_t^* - \psi_{t+k\mid t} - \mu\bigr)
\right\}
= 0,
\tag{11.a}
\end{equation}
donde:
\begin{itemize}
  \item hemos pasado de niveles nominales \((P_t^*, M_{t+k\mid t})\) a logaritmos \((p_t^*, \psi_{t+k\mid t})\),
  \item \(\mu\) recoge el término constante asociado al markup deseado.
\end{itemize}

La interpretación de (11.a) es: el promedio (ponderado por \((\beta\theta)^k\)) de la diferencia entre el precio óptimo en log \(p_t^*\) y el “nivel deseado” \(\mu + \psi_{t+k\mid t}\) debe ser cero.

\paragraph{4. Resolución para \(p_t^*\) como promedio ponderado}

Reescribimos (11.a) separando \(p_t^*\) de los costos:
\begin{align}
0
&=
\sum_{k=0}^{\infty}
(\beta\theta)^k
E_t\!\left\{
  p_t^* - \mu - \psi_{t+k\mid t}
\right\} \notag\\[0.2cm]
&=
\left[
  p_t^* - \mu
\right]
\sum_{k=0}^{\infty} (\beta\theta)^k
-
\sum_{k=0}^{\infty}
(\beta\theta)^k
E_t\{\psi_{t+k\mid t}\}.
\tag{11.b}
\end{align}
El primer término usa que \(p_t^* - \mu\) es común para todos los \(k\). La suma geométrica es:
\[
\sum_{k=0}^{\infty} (\beta\theta)^k
=
\frac{1}{1-\beta\theta},
\quad \text{para } |\beta\theta|<1.
\]

Sustituyendo:
\begin{equation}
\left(p_t^* - \mu\right)\frac{1}{1-\beta\theta}
=
\sum_{k=0}^{\infty}
(\beta\theta)^k
E_t\{\psi_{t+k\mid t}\}.
\tag{11.c}
\end{equation}
Multiplicando ambos lados por \(1-\beta\theta\), obtenemos:
\begin{equation}
p_t^*
=
\mu
+
(1-\beta\theta)
\sum_{k=0}^{\infty}
(\beta\theta)^k
E_t\{\psi_{t+k\mid t}\},
\end{equation}
que es precisamente la Ecuación (11).

Obsérvese que los pesos:
\[
\omega_k \equiv (1-\beta\theta)(\beta\theta)^k,
\]
satisfacen:
\[
\sum_{k=0}^{\infty} \omega_k
=
(1-\beta\theta)\sum_{k=0}^{\infty}(\beta\theta)^k
=
(1-\beta\theta)\frac{1}{1-\beta\theta}
=
1,
\]
lo que confirma que la suma es un \emph{promedio ponderado} de los costos marginales esperados.

\subsubsection*{Interpretación económica de la Ecuación (11)}

La Ecuación (11):
\[
p_t^*
=
\mu
+
(1-\beta\theta)
\sum_{k=0}^{\infty}
(\beta\theta)^k
E_t\{\psi_{t+k\mid t}\},
\]
resume en forma compacta la lógica de fijación de precios bajo rigideces:

\begin{itemize}
  \item \textbf{Precio óptimo como markup sobre costos esperados:}  
  El precio en log fijado por la empresa es el markup deseado \(\mu\) más un promedio ponderado de los costos marginales (en log) que se esperan a lo largo de la vida útil del precio (mientras no se pueda reajustar).

  \item \textbf{Prospectivo (“forward-looking”):}  
  La empresa no sólo considera el costo actual, sino toda la trayectoria esperada \(\{\psi_{t+k\mid t}\}_{k\ge 0}\), con pesos decrecientes \((\beta\theta)^k\).  
  \begin{itemize}
    \item \(\theta^k\) refleja la probabilidad de que el precio siga vigente en \(t+k\).
    \item \(\beta^k\) incorpora el descuento intertemporal.
  \end{itemize}

  \item \textbf{Importancia de la rigidez \(\theta\):}  
  Cuando \(\theta\) es grande (precios muy rígidos), los periodos futuros reciben mucho peso, y el precio actual se fija de forma más “promediada” frente a shocks transitorios de costos.  
  Si \(\theta\to 0\) (precios casi flexibles), los pesos se concentran en \(k=0\) y:
  \[
  p_t^* \approx \mu + \psi_{t\mid t},
  \]
  es decir, el precio se ajusta casi exclusivamente al costo marginal actual más el markup.

  \item \textbf{Vínculo directo con la Curva de Phillips NK:}  
  La Ecuación (11), combinada con la Ecuación (8),
  \[
  \pi_t = (1-\theta)\,(p_t^* - p_{t-1}),
  \]
  y con las relaciones que conectan \(\psi_{t+k\mid t}\) con el costo marginal real y la brecha del producto, permite derivar la Curva de Phillips Nuevo Keynesiana:
  \[
  \pi_t = \beta E_t\{\pi_{t+1}\} + \kappa \,\hat{mc}_t,
  \]
  donde \(\hat{mc}_t\) es el costo marginal real (o una función de la brecha del producto), y \(\kappa\) depende de \(\theta\), \(\beta\) y \(\epsilon\). En este sentido, la Ecuación (11) es el “eslabón microeconómico” que transforma información sobre costos en dinámica de inflación.
\end{itemize}

\subsubsection*{Sugerencia de gráfico y código: pesos de \((\beta\theta)^k\) en el promedio de costos}

Una forma de visualizar la Ecuación (11) es representar los pesos:
\[
\omega_k = (1-\beta\theta)(\beta\theta)^k
\]
como función del horizonte \(k\). Esto muestra qué tan importante es cada periodo futuro en la fijación del precio presente.

%========================================
% Ecuación (12)
%========================================
\subsubsection*{Ecuación (12): Equilibrio en el mercado de bienes}

La Ecuación (12) impone la condición de equilibrio (\emph{market clearing}) en el mercado agregado de bienes. En el modelo básico nuevo keynesiano considerado, el \emph{único} componente de la demanda de bienes es el consumo del hogar representativo, por lo que la producción agregada debe igualar al consumo agregado:

\begin{equation}
Y_t = C_t.
\tag{12}
\end{equation}

\subsubsection*{Tabla de símbolos relevantes (Ecuación 12)}

\begin{table}[htbp]
  \centering
  \begin{tabular}{p{0.19\textwidth} p{0.41\textwidth} p{0.30\textwidth}}
    \hline
    \textbf{Término} & \textbf{Definición rigurosa} & \textbf{Rol económico} \\
    \hline
    \(Y_t\)
      & Índice de producto agregado en el periodo \(t\). Se obtiene agregando la producción de todos los bienes diferenciados \(i \in [0,1]\) mediante un índice de tipo CES. 
      & Representa la oferta total de bienes finales disponible en la economía en el periodo \(t\). \\[0.1cm]
    \(C_t\)
      & Índice de consumo agregado del hogar representativo en el periodo \(t\), definido también a partir de un índice CES sobre el vector de consumos \(\{C_t(i)\}_{i\in[0,1]}\). 
      & Representa la demanda total de bienes finales; en este modelo coincide con el uso total de la producción. \\
    \hline
  \end{tabular}
  \caption{Símbolos usados en la Ecuación (12).}
\end{table}

\subsubsection*{Origen microeconómico de la condición \(Y_t = C_t\)}

\paragraph{1. Equilibrio bien por bien}

Para cada bien diferenciado \(i\), la condición de equilibrio en el mercado requiere que:
\begin{equation}
Y_t(i) = C_t(i)
\quad \text{para todo } i \in [0,1], \; \text{y para todo } t.
\tag{12.a}
\end{equation}
Es decir, la cantidad producida del bien \(i\) debe igualar la cantidad consumida de ese mismo bien. No se consideran ni inversión, ni gasto público, ni exportaciones netas: toda la producción se destina al consumo privado.

\paragraph{2. Agregación mediante el índice CES}

El consumo agregado se define a partir de un índice de elasticidad de sustitución constante (CES). Por ejemplo, el índice de consumo agregado es:
\begin{equation}
C_t
=
\left(
\int_0^1
C_t(i)^{\frac{\epsilon-1}{\epsilon}}\, \mathrm{d}i
\right)^{\!\frac{\epsilon}{\epsilon-1}},
\tag{12.b}
\end{equation}
y, de modo análogo, el producto agregado puede definirse (dado el equilibrio bien por bien) como:
\begin{equation}
Y_t
=
\left(
\int_0^1
Y_t(i)^{\frac{\epsilon-1}{\epsilon}}\, \mathrm{d}i
\right)^{\!\frac{\epsilon}{\epsilon-1}}.
\tag{12.c}
\end{equation}

Sustituyendo la condición de equilibrio individual \(Y_t(i) = C_t(i)\) en \eqref{12.c}, se obtiene:
\begin{align}
Y_t
&=
\left(
\int_0^1
\bigl(Y_t(i)\bigr)^{\frac{\epsilon-1}{\epsilon}}\, \mathrm{d}i
\right)^{\!\frac{\epsilon}{\epsilon-1}}
=
\left(
\int_0^1
\bigl(C_t(i)\bigr)^{\frac{\epsilon-1}{\epsilon}}\, \mathrm{d}i
\right)^{\!\frac{\epsilon}{\epsilon-1}}
=
C_t.
\tag{12.d}
\end{align}
Por lo tanto, el índice de producto agregado coincide exactamente con el índice de consumo agregado:
\[
Y_t = C_t,
\]
que es la Ecuación (12).

\subsubsection*{Interpretación económica de la Ecuación (12)}

La condición
\[
Y_t = C_t
\]
tiene varias implicaciones económicas importantes en el modelo:

\begin{itemize}
  \item \textbf{Equilibrio del mercado de bienes:}  
  La producción total de la economía se destina completamente al consumo del hogar representativo. No hay otros usos de la producción (inversión, gasto público, exportaciones netas), de modo que el mercado de bienes se limpia exclusivamente a través del consumo.

  \item \textbf{Identidad oferta–demanda:}  
  En cada periodo, la oferta agregada de bienes \(Y_t\) debe igualar la demanda agregada \(C_t\). Esto implica que cualquier cambio en la demanda de consumo se refleja uno a uno en la producción, sujeto a las restricciones tecnológicas y a las rigideces nominales del modelo.

  \item \textbf{Puente entre decisiones intertemporales y producto:}  
  La Ecuación (3), que proviene de la condición de Euler del hogar, estaba formulada en términos de consumo:
  \[
  c_t
  =
  E_t\{c_{t+1}\}
  - \frac{1}{\sigma}
    \bigl(i_t - E_t\{\pi_{t+1}\} - \rho\bigr)
  + \frac{1}{\sigma}(1-\rho_z)z_t.
  \]
  Al tomar logaritmos de \(Y_t = C_t\) y definir \(y_t \equiv \log Y_t\), \(c_t \equiv \log C_t\), se tiene:
  \[
  y_t = c_t.
  \]
  De este modo, la ecuación de Euler puede reescribirse directamente en términos del producto:
  \begin{equation}
  y_t
  =
  E_t\{y_{t+1}\}
  - \frac{1}{\sigma}
    \bigl(i_t - E_t\{\pi_{t+1}\} - \rho\bigr)
  + \frac{1}{\sigma}(1-\rho_z)z_t,
  \tag{13}
  \end{equation}
  que es la \emph{Ecuación IS Dinámica} (DIS). La Ecuación (12) es, por tanto, la pieza que permite traducir la decisión intertemporal de consumo del hogar en una relación entre producto actual, expectativas de producto futuro y tasa de interés real.

  \item \textbf{Simplicidad contable del modelo básico:}  
  Esta especificación de equilibrio (sin inversión ni gasto público) hace que la demanda agregada coincida exactamente con el consumo privado, lo que simplifica la estructura del modelo. En extensiones más ricas (con inversión, gasto público o sector externo), la condición de equilibrio se generaliza a:
  \[
  Y_t = C_t + I_t + G_t + NX_t,
  \]
  pero en el modelo básico nuevo keynesiano utilizado aquí se abstrae de esos componentes para focalizarse en el vínculo entre política monetaria, demanda agregada y precios.
\end{itemize}

%========================================
% Ecuación (13)
%========================================
\subsubsection*{Ecuación (13): Ecuación IS Dinámica (DIS) en términos de producto}

La Ecuación (13) constituye la versión log-lineal de la \emph{Ecuación IS Dinámica} (DIS) del modelo nuevo keynesiano básico. Relaciona el nivel actual de producto con:
\begin{itemize}
  \item las expectativas de producto futuro,
  \item la tasa de interés real esperada, y
  \item los shocks de preferencias intertemporales.
\end{itemize}

En su forma estándar:
\begin{equation}
y_t
=
E_t\{y_{t+1}\}
-
\frac{1}{\sigma}
\bigl(
  i_t - E_t\{\pi_{t+1}\} - \rho
\bigr)
+
\frac{1}{\sigma}(1 - \rho_z)z_t.
\tag{13}
\label{eq:DIS_13}
\end{equation}

\subsubsection*{Tabla de símbolos relevantes (Ecuación 13)}

\begin{table}[htbp]
  \centering
  \begin{tabular}{p{0.19\textwidth} p{0.41\textwidth} p{0.30\textwidth}}
    \hline
    \textbf{Término} & \textbf{Definición rigurosa} & \textbf{Rol económico} \\
    \hline
    \(y_t\)
      & Logaritmo del producto agregado \(Y_t\). 
      & Nivel de demanda/oferta agregada actual de la economía. \\
    \(E_t\{y_{t+1}\}\)
      & Valor esperado, dado \(t\), del logaritmo del producto agregado en \(t+1\). 
      & Producción (demanda agregada) esperada para el periodo siguiente; ancla las decisiones actuales. \\
    \(i_t\)
      & Tasa de interés nominal a corto plazo (en desviaciones logarítmicas respecto al estado estacionario). 
      & Componente nominal de la tasa de interés real relevante para las decisiones intertemporales. \\
    \(E_t\{\pi_{t+1}\}\)
      & Inflación esperada para el periodo \(t+1\), \(\pi_{t+1} \equiv p_{t+1} - p_t\). 
      & Ajusta la tasa nominal para obtener la tasa de interés real esperada. \\
    \(\rho\)
      & Tasa de descuento subjetiva, \(\rho \equiv -\log \beta\). 
      & Nivel de referencia de la tasa de interés real consistente con preferencias y estado estacionario. \\
    \(z_t\)
      & Shock (en log) a las preferencias intertemporales, asociado al shifter \(Z_t\). Sigue un proceso AR(1): \(z_t = \rho_z z_{t-1} + \varepsilon_{z,t}\). 
      & Shock de demanda agregada vía preferencias por el consumo presente vs. futuro. \\
    \(\rho_z\)
      & Coeficiente autorregresivo del proceso para \(z_t\). 
      & Mide la persistencia de los shocks de preferencias. \\
    \(\sigma > 0\)
      & Parámetro que es el inverso de la elasticidad de sustitución intertemporal del consumo. 
      & Determina la sensibilidad de \(y_t\) ante variaciones en la tasa de interés real. \\
    \hline
  \end{tabular}
  \caption{Símbolos usados en la Ecuación (13).}
\end{table}

\subsubsection*{Derivación de la Ecuación IS Dinámica a partir de las ecuaciones (3) y (12)}

La Ecuación (13) se obtiene combinando:
\begin{itemize}
  \item la condición de Euler del hogar (Ecuación (3)), formulada en términos de consumo,
  \item y la condición de equilibrio en el mercado de bienes (Ecuación (12)), que iguala producto y consumo agregados.
\end{itemize}

\paragraph{1. Condición de Euler en términos de consumo (Ecuación 3)}

La Ecuación (3) expresa la decisión óptima intertemporal del hogar en términos de consumo:
\begin{equation}
c_t
=
E_t\{c_{t+1}\}
-
\frac{1}{\sigma}
\bigl(
  i_t - E_t\{\pi_{t+1}\} - \rho
\bigr)
+
\frac{1}{\sigma}(1 - \rho_z)z_t,
\tag{3}
\label{eq:Euler_consumo}
\end{equation}
donde \(c_t \equiv \log C_t\) es el logaritmo del consumo agregado.

\paragraph{2. Equilibrio en el mercado de bienes (Ecuación 12) y paso a logaritmos}

La condición de equilibrio agregada es:
\begin{equation}
Y_t = C_t.
\tag{12}
\label{eq:market_clearing}
\end{equation}
Tomando logaritmos y trabajando en torno a un estado estacionario común, se tiene:
\begin{equation}
y_t = c_t,
\quad
E_t\{y_{t+1}\} = E_t\{c_{t+1}\},
\tag{12'}
\end{equation}
donde \(y_t \equiv \log Y_t\).

\paragraph{3. Sustitución de consumo por producto}

Sustituimos \(c_t = y_t\) y \(E_t\{c_{t+1}\} = E_t\{y_{t+1}\}\) en la Ecuación de Euler \eqref{eq:Euler_consumo}. Obtenemos:
\begin{align}
y_t
&=
E_t\{y_{t+1}\}
-
\frac{1}{\sigma}
\bigl(
  i_t - E_t\{\pi_{t+1}\} - \rho
\bigr)
+
\frac{1}{\sigma}(1 - \rho_z)z_t.
\tag{13}
\end{align}
Esta es exactamente la Ecuación IS Dinámica (13).

\subsubsection*{Interpretación económica de la Ecuación IS Dinámica}

La Ecuación \eqref{eq:DIS_13} puede leerse de manera intuitiva como:
\[
y_t
=
E_t\{y_{t+1}\}
-
\frac{1}{\sigma}
\Bigl(r_t - \rho\Bigr)
+
\frac{1}{\sigma}(1 - \rho_z)z_t,
\]
donde
\[
r_t \equiv i_t - E_t\{\pi_{t+1}\}
\]
es la tasa de interés real esperada (para el periodo \(t+1\)). Así, la ecuación se interpreta como:

\begin{itemize}
  \item \textbf{Carácter prospectivo (\emph{forward-looking}):}  
  La producción actual \(y_t\) está positivamente relacionada con la producción esperada futura \(E_t\{y_{t+1}\}\).  
  En equilibrio, los hogares eligen una senda de consumo (y por la Ecuación (12), de producto) que equilibra el costo intertemporal de consumir hoy vs. mañana. Si esperan que el producto futuro sea elevado, tienden a consumir más hoy, elevando el producto actual.

  \item \textbf{Rol de la tasa de interés real:}  
  El término
  \[
  - \frac{1}{\sigma}
  \bigl(
    i_t - E_t\{\pi_{t+1}\} - \rho
  \bigr)
  =
  -\frac{1}{\sigma}(r_t - \rho)
  \]
  recoge el efecto de la tasa de interés real esperada \(r_t\) relativa a la tasa de descuento \(\rho\).  
  \begin{itemize}
    \item Si \(r_t > \rho\), el costo de oportunidad de consumir hoy (en términos de consumo futuro) es alto. Los hogares prefieren posponer consumo, lo que reduce \(y_t\) dado \(E_t\{y_{t+1}\}\).
    \item Si \(r_t < \rho\), el consumo presente es relativamente barato en términos intertemporales, lo que eleva el producto actual.
    \item El parámetro \(\sigma\) (inverso de la elasticidad de sustitución intertemporal) controla \emph{cuánto} reacciona el producto a una variación dada en la tasa real: cuanto mayor es \(\sigma\), menor es la sensibilidad de \(y_t\) ante cambios en \(r_t\).
  \end{itemize}

  \item \textbf{Impacto de shocks de preferencias \(z_t\):}  
  El término
  \[
  \frac{1}{\sigma}(1 - \rho_z)z_t
  \]
  refleja el efecto de un shock de preferencias que altera el valor relativo que los hogares asignan al consumo en el tiempo.  
  \begin{itemize}
    \item Si \(z_t\) aumenta (por ejemplo, un mayor “apetito” por el consumo presente), el consumo y, por la Ecuación (12), el producto \(y_t\) se elevan.
    \item El factor \((1-\rho_z)\) indica que sólo la parte \emph{no} puramente persistente del shock tiene efecto inmediato sobre la dinámica de consumo/producto en esta representación log-lineal.
  \end{itemize}

  \item \textbf{Conexión con la política monetaria:}  
  La autoridad monetaria controla (directa o indirectamente) la tasa nominal \(i_t\). Dado \(E_t\{\pi_{t+1}\}\) y \(\rho\), la ecuación IS muestra el canal a través del cual la política monetaria, al modificar \(i_t\), altera la tasa real \(r_t\) y, por esa vía, la demanda agregada \(y_t\).

  \item \textbf{Punto de partida para la versión en brechas:}  
  Más adelante, al introducir el producto natural \(y_{n,t}\) y definir la brecha del producto \(\tilde{y}_t \equiv y_t - y_{n,t}\), la Ecuación (13) se reescribe en términos de \(\tilde{y}_t\) y de la \emph{tasa natural de interés} \(r_{n,t}\). Esto conduce a la forma canónica de la IS dinámica en brechas:
  \[
  \tilde{y}_t
  =
  E_t\{\tilde{y}_{t+1}\}
  -
  \frac{1}{\sigma}
  \bigl(
    i_t - E_t\{\pi_{t+1}\} - r_{n,t}
  \bigr),
  \]
  donde la política monetaria es neutral sólo si alinea la tasa real ex post con la tasa natural \(r_{n,t}\).
\end{itemize}


%========================================
% Ecuación (14)
%========================================
\subsubsection*{Ecuación (14): Relación producto–empleo bajo tecnología agregada}

La Ecuación (14) proporciona una relación aproximada de equilibrio entre el empleo agregado, el producto agregado y el nivel de tecnología. Es una restricción tecnológica que se obtiene a partir de la función de producción y de la agregación de las decisiones de las empresas:

\begin{equation}
n_t
=
\frac{1}{1 - \alpha}\,\bigl(y_t - a_t\bigr).
\tag{14}
\label{eq:relacion_n_y_a_14}
\end{equation}

\subsubsection*{Tabla de símbolos relevantes (Ecuación 14)}

\begin{table}[htbp]
  \centering
  \begin{tabular}{p{0.19\textwidth} p{0.41\textwidth} p{0.30\textwidth}}
    \hline
    \textbf{Término} & \textbf{Definición rigurosa} & \textbf{Rol económico} \\
    \hline
    \(n_t\)
      & Logaritmo del empleo agregado \(N_t\), entendido como horas trabajadas totales en la economía. 
      & Insumo total de trabajo utilizado en la producción del bien final. \\[0.1cm]
    \(y_t\)
      & Logaritmo del producto agregado \(Y_t\). 
      & Resultado de la producción agregada; coincide con la demanda agregada en equilibrio. \\[0.1cm]
    \(a_t\)
      & Logaritmo del nivel de tecnología agregada \(A_t\). 
      & Captura variaciones exógenas en la productividad total de los factores. \\[0.1cm]
    \(\alpha\)
      & Parámetro tecnológico tal que \(1-\alpha\) es la elasticidad del producto con respecto al trabajo en la función de producción. 
      & Determina en qué medida cambios en el trabajo afectan el producto, y viceversa. \\[0.1cm]
    \(d_t\)
      & (Sólo en la derivación) Término que recoge la dispersión de precios relativa entre firmas. 
      & Mide ineficiencias debidas a la heterogeneidad de precios; es de orden superior y se descarta en la aproximación de primer orden. \\
    \hline
  \end{tabular}
  \caption{Símbolos usados en la Ecuación (14) y en su derivación.}
\end{table}

\subsubsection*{Derivación a partir de la función de producción y la agregación}

La Ecuación (14) se obtiene a partir de tres elementos:
\begin{enumerate}
  \item La función de producción a nivel de empresa.
  \item La forma en que se agrega la producción de las distintas empresas.
  \item Una aproximación de primer orden que ignora la dispersión de precios.
\end{enumerate}

\paragraph{1. Función de producción individual (Ecuación (5))}

Cada empresa \(i\) produce un bien diferenciado utilizando trabajo como único factor variable:
\begin{equation}
Y_t(i) = A_t \, N_t(i)^{1-\alpha},
\tag{5}
\end{equation}
donde \(A_t\) es el nivel de tecnología común y \(N_t(i)\) es el empleo de la empresa \(i\).

Tomando logaritmos:
\begin{equation}
y_t(i)
=
\log Y_t(i)
=
a_t + (1-\alpha)\,n_t(i),
\tag{14.a}
\label{eq:log_prod_individual}
\end{equation}
con \(a_t \equiv \log A_t\) y \(n_t(i) \equiv \log N_t(i)\).

\paragraph{2. Agregación del producto y aparición de la dispersión de precios}

El producto agregado se construye a partir del agregado CES de los bienes diferenciados. Bajo competencia monopolística y demanda CES, puede mostrarse que el producto agregado se puede escribir como:
\begin{equation}
Y_t
=
A_t \, N_t^{1-\alpha} \, \Delta_t,
\tag{14.b}
\label{eq:Yt_con_dispersion}
\end{equation}
donde:
\begin{itemize}
  \item \(N_t\) es el empleo agregado, que recoge la utilización total de trabajo,
  \item \(\Delta_t\) es un término de \emph{dispersión de precios}, definido como un funcional de los precios relativos \(\{P_t(i)/P_t\}\).
\end{itemize}

Definiendo \(d_t \equiv \log \Delta_t\), tomar logaritmos en \eqref{eq:Yt_con_dispersion} da:
\begin{equation}
y_t
=
a_t + (1-\alpha)n_t + d_t.
\tag{14.c}
\label{eq:relacion_y_a_n_d}
\end{equation}

Reordenando:
\begin{equation}
(1-\alpha)n_t
=
y_t - a_t - d_t.
\tag{14.d}
\end{equation}

\paragraph{3. Aproximación de primer orden: ignorando \(d_t\)}

El término \(d_t\) mide la ineficiencia asociada a la dispersión de precios derivada de las rigideces nominales (distintas empresas con precios distintos, alejados del óptimo común). Sin embargo, alrededor de un estado estacionario con inflación cero y precios relativamente homogéneos, dicho término es de \emph{orden superior} en la aproximación de primer orden.

Por tanto, en el análisis linealizado se impone:
\begin{equation}
d_t \approx 0
\quad \Rightarrow \quad
(1-\alpha)n_t \approx y_t - a_t.
\tag{14.e}
\end{equation}

Despejando \(n_t\):
\begin{equation}
n_t
\approx
\frac{1}{1-\alpha}\bigl(y_t - a_t\bigr),
\tag{14.f}
\end{equation}
que, al adoptar la igualdad en la notación estándar de la solución linealizada, se escribe como:
\begin{equation}
n_t
=
\frac{1}{1-\alpha}\bigl(y_t - a_t\bigr),
\tag{14}
\end{equation}
que es precisamente la Ecuación (14).

\subsubsection*{Interpretación económica de la Ecuación (14)}

La Ecuación \eqref{eq:relacion_n_y_a_14} ofrece una lectura directa en términos de tecnología y utilización del trabajo:

\begin{itemize}
  \item \textbf{Relación producto–empleo corregida por tecnología:}  
  El término \(y_t - a_t\) recoge el componente del producto que no se explica por la tecnología (es decir, el “esfuerzo” en términos de insumo trabajo). La ecuación indica que el empleo (en logaritmos) es proporcional a esta “parte no explicada por la productividad”, con factor de proporcionalidad \(1/(1-\alpha)\).

  \item \textbf{Productividad y necesidad de empleo:}  
  Para un nivel dado de producto \(y_t\), un aumento en el nivel tecnológico \(a_t\) reduce la cantidad de trabajo necesaria:
  \[
  \frac{\partial n_t}{\partial a_t}
  =
  -\frac{1}{1-\alpha} < 0.
  \]
  Un shock tecnológico positivo permite sostener el mismo nivel de output con menos empleo (o un mayor output con el mismo empleo).

  \item \textbf{Elasticidad del producto respecto al trabajo:}  
  El parámetro \(1-\alpha\) es la elasticidad del producto respecto al trabajo en la función de producción Cobb–Douglas. Invirtiendo esta relación, la Ecuación (14) expresa la elasticidad del empleo con respecto al producto:
  \[
  \frac{\partial n_t}{\partial y_t}
  =
  \frac{1}{1-\alpha} > 1
  \quad \text{si } 0 < \alpha < 1.
  \]
  Esto significa que para mantener una variación dada en \(y_t\), el empleo debe ajustarse más que proporcionalmente cuando la participación del trabajo en la producción (representada por \(1-\alpha\)) es menor.

  \item \textbf{Papel en la determinación del costo marginal:}  
  La relación entre \(y_t\), \(n_t\) y \(a_t\) es crucial para derivar el \emph{costo marginal real}. Combinando la Ecuación (14) con la condición de oferta de trabajo del hogar (Ecuación (2)),
  \[
  w_t - p_t = \sigma c_t + \phi n_t,
  \]
  se obtiene una expresión que vincula el salario real (por tanto, el costo marginal) con el producto y la tecnología. Esta relación es un insumo central para la Curva de Phillips Nuevo Keynesiana, en la que la inflación depende del costo marginal real y de sus expectativas futuras.

  \item \textbf{Dispersión de precios e ineficiencia:}  
  Aunque \(d_t\) se ignora en la aproximación de primer orden, su presencia en \eqref{eq:relacion_y_a_n_d} recuerda que, con precios rígidos y heterogéneos, la producción no se asigna eficientemente entre las empresas. En un análisis de segundo orden (bienestar), \(d_t\) deja de ser despreciable y las rigideces de precios se traducen en pérdidas de eficiencia reales.
\end{itemize}

%========================================
% Ecuación (15)
%========================================
\subsubsection*{Ecuación (15): Costo marginal individual vs. costo marginal promedio}

La Ecuación (15) describe cómo se relaciona el costo marginal nominal (en logaritmos) de una firma que fijó su precio por última vez en \(t\) con:
\begin{itemize}
  \item el costo marginal nominal promedio de la economía en \(t+k\), y
  \item la posición relativa de su precio con respecto al nivel de precios agregado.
\end{itemize}

En forma compacta:
\begin{equation}
\psi_{t+k\mid t}
=
\psi_{t+k}
+
\alpha\bigl(n_{t+k\mid t} - n_{t+k}\bigr)
=
\psi_{t+k}
+
\frac{\alpha}{1-\alpha}\bigl(y_{t+k\mid t} - y_{t+k}\bigr)
=
\psi_{t+k}
-
\frac{\alpha\epsilon}{1-\alpha}\bigl(p^*_t - p_{t+k}\bigr).
\tag{15}
\label{eq:psi_15}
\end{equation}

\subsubsection*{Tabla de símbolos relevantes (Ecuación 15)}

\begin{table}[htbp]
  \centering
  \begin{tabular}{p{0.19\textwidth} p{0.41\textwidth} p{0.30\textwidth}}
    \hline
    \textbf{Término} & \textbf{Definición rigurosa} & \textbf{Rol económico} \\
    \hline
    \(\psi_{t+k\mid t}\)
      & Logaritmo del costo marginal nominal de una firma que fijó su último precio en \(t\), evaluado en el periodo \(t+k\). 
      & Costo marginal relevante para esa firma al producir con el precio fijo \(P^*_t\). \\[0.1cm]
    \(\psi_{t+k}\)
      & Logaritmo del costo marginal nominal promedio de la economía en el periodo \(t+k\). 
      & Referencia agregada de costos marginales; sirve de “benchmark” para comparar a las firmas individuales. \\[0.1cm]
    \(n_{t+k\mid t}\)
      & Logaritmo del empleo de la firma que fijó su precio en \(t\), en el periodo \(t+k\). 
      & Insumo trabajo utilizado por esa firma para atender su demanda futura. \\[0.1cm]
    \(n_{t+k}\)
      & Logaritmo del empleo agregado en el periodo \(t+k\). 
      & Nivel de empleo promedio en la economía. \\[0.1cm]
    \(y_{t+k\mid t}\)
      & Logaritmo del output de la firma que fijó su precio en \(t\), en el periodo \(t+k\). 
      & Producción individual resultante de la demanda a su precio relativo \(P^*_t\). \\[0.1cm]
    \(y_{t+k}\)
      & Logaritmo del output agregado en el periodo \(t+k\). 
      & Nivel de producto total de la economía. \\[0.1cm]
    \(p^*_t\)
      & Logaritmo del precio óptimo fijado por la firma (o cohorte de firmas) que reajusta en \(t\). 
      & Precio nominal individual que permanece fijo mientras no haya nueva oportunidad de ajuste. \\[0.1cm]
    \(p_{t+k}\)
      & Logaritmo del índice de precios agregado en \(t+k\). 
      & Nivel promedio de precios de la economía en ese periodo. \\[0.1cm]
    \(\alpha\)
      & Parámetro tecnológico tal que \(1-\alpha\) es la elasticidad del producto respecto al trabajo en la función de producción. 
      & Determina cómo se traduce una variación en el empleo en el costo marginal. \\[0.1cm]
    \(\epsilon\)
      & Elasticidad de sustitución entre los distintos bienes diferenciados. 
      & Mide la sensibilidad de la demanda de un bien ante cambios en su precio relativo. \\
    \hline
  \end{tabular}
  \caption{Símbolos usados en la Ecuación (15) y su derivación.}
\end{table}

\subsubsection*{Derivación paso a paso de la Ecuación (15)}

La Ecuación (15) se construye a partir de tres piezas:
\begin{enumerate}
  \item la relación entre el costo marginal y el empleo de la firma,
  \item la relación entre empleo y producto (individual y agregado),
  \item y la restricción de demanda que vincula el producto relativo con el precio relativo.
\end{enumerate}

\paragraph{1. Costo marginal individual vs. promedio (primer tramo de (15))}

De la teoría de la firma con función de producción Cobb--Douglas en trabajo,
\[
Y_t(i) = A_t N_t(i)^{1-\alpha},
\]
se obtiene que el costo marginal real (y, por tanto, el nominal en logaritmos) depende positivamente del salario real y negativamente del producto marginal del trabajo. Al log-linearizar alrededor de un estado estacionario común se obtiene una relación de la forma:
\begin{equation}
\psi_{t+k\mid t}
=
\psi_{t+k}
+
\alpha\bigl(n_{t+k\mid t} - n_{t+k}\bigr),
\tag{15.a}
\label{eq:psi_vs_n}
\end{equation}
es decir, el costo marginal logarítmico de la firma que fijó precio en \(t\) difiere del costo marginal promedio en una magnitud proporcional a la desviación de su empleo respecto al empleo agregado. Cuanto mayor empleo relativo utiliza la firma, mayor será su costo marginal relativo.

\paragraph{2. Uso de la relación producto–empleo (segunda igualdad de (15))}

Recordemos la relación aproximada producto–empleo–tecnología (Ecuación (14)), aplicada tanto al nivel individual como al agregado:
\begin{align}
n_{t+k\mid t} &= \frac{1}{1-\alpha}\bigl(y_{t+k\mid t} - a_{t+k}\bigr), \tag{15.b}\\
n_{t+k}      &= \frac{1}{1-\alpha}\bigl(y_{t+k}      - a_{t+k}\bigr). \tag{15.c}
\end{align}
Restando \((15.c)\) de \((15.b)\):
\begin{equation}
n_{t+k\mid t} - n_{t+k}
=
\frac{1}{1-\alpha}\bigl(y_{t+k\mid t} - y_{t+k}\bigr),
\tag{15.d}
\label{eq:diff_n_y}
\end{equation}
donde la tecnología \(a_{t+k}\) se cancela al ser común a todas las firmas.

Sustituyendo \eqref{eq:diff_n_y} en \eqref{eq:psi_vs_n}:
\begin{equation}
\psi_{t+k\mid t}
=
\psi_{t+k}
+
\alpha\,\frac{1}{1-\alpha}\bigl(y_{t+k\mid t} - y_{t+k}\bigr)
=
\psi_{t+k}
+
\frac{\alpha}{1-\alpha}\bigl(y_{t+k\mid t} - y_{t+k}\bigr).
\tag{15.e}
\label{eq:psi_vs_y}
\end{equation}
Esta es la segunda igualdad de la Ecuación (15).

\paragraph{3. Restricción de demanda y precios relativos (tercer tramo de (15))}

La restricción de demanda que enfrenta la firma que fijó su precio en \(t\) viene dada (en niveles) por la Ecuación (9):
\begin{equation}
Y_{t+k\mid t}
=
\left(\frac{P^*_t}{P_{t+k}}\right)^{-\epsilon} C_{t+k},
\tag{9}
\label{eq:demand_9}
\end{equation}
donde \(C_{t+k}\) es el consumo agregado en \(t+k\).

Tomando logaritmos,
\begin{equation}
y_{t+k\mid t}
=
-\epsilon\bigl(p^*_t - p_{t+k}\bigr) + c_{t+k},
\tag{15.f}
\end{equation}
mientras que, por equilibrio en el mercado de bienes,
\begin{equation}
y_{t+k} = c_{t+k}.
\tag{15.g}
\end{equation}

Restando \((15.g)\) de \((15.f)\):
\begin{equation}
y_{t+k\mid t} - y_{t+k}
=
-\epsilon\bigl(p^*_t - p_{t+k}\bigr).
\tag{15.h}
\label{eq:diff_y_p}
\end{equation}

Sustituyendo \eqref{eq:diff_y_p} en \eqref{eq:psi_vs_y}:
\begin{align}
\psi_{t+k\mid t}
&=
\psi_{t+k}
+
\frac{\alpha}{1-\alpha}\bigl(y_{t+k\mid t} - y_{t+k}\bigr) \\
&=
\psi_{t+k}
+
\frac{\alpha}{1-\alpha}\bigl[-\epsilon\bigl(p^*_t - p_{t+k}\bigr)\bigr] \\
&=
\psi_{t+k}
-
\frac{\alpha\epsilon}{1-\alpha}\bigl(p^*_t - p_{t+k}\bigr),
\tag{15}
\end{align}
que es la tercera igualdad de la Ecuación (15).

\subsubsection*{Interpretación económica de la Ecuación (15)}

La Ecuación \eqref{eq:psi_15} ofrece una interpretación clara del vínculo entre precios relativos, demanda y costos marginales:

\begin{itemize}
  \item \textbf{Costo marginal individual vs. promedio:}  
  El costo marginal nominal (en logaritmos) de una firma que fijó su precio en \(t\) (\(\psi_{t+k\mid t}\)) es igual al costo marginal promedio \(\psi_{t+k}\) más (o menos) un ajuste que recoge cómo difiere su producción de la producción agregada, o, equivalentemente, cómo difiere su precio relativo del nivel de precios agregado.

  \item \textbf{Efecto del precio relativo sobre la demanda y el costo:}  
  Si una firma fija un precio por encima del promedio futuro esperado, \(p^*_t - p_{t+k} > 0\), entonces:
  \[
  y_{t+k\mid t} - y_{t+k}
  =
  -\epsilon\bigl(p^*_t - p_{t+k}\bigr) < 0.
  \]
  Es decir, su producción (y ventas) será menor que la producción agregada. Dado que la función de producción presenta rendimientos decrecientes en el trabajo, un menor nivel de producción se asocia con un menor empleo relativo y, por tanto, con un \emph{menor} costo marginal relativo. Esto queda reflejado en el signo negativo del término:
  \[
  -\frac{\alpha\epsilon}{1-\alpha}\bigl(p^*_t - p_{t+k}\bigr).
  \]

  \item \textbf{Papel de \(\alpha\): rendimientos a escala en el trabajo}  
  El parámetro \(\alpha\) controla la intensidad con la que una desviación de producción se traduce en una desviación del costo marginal:
  \begin{itemize}
    \item Si \(\alpha = 0\) (rendimientos constantes a escala en el trabajo), se tiene:
      \[
      \psi_{t+k\mid t} = \psi_{t+k},
      \]
      es decir, el costo marginal es el mismo para todas las firmas, independientemente de su nivel de producción o de su precio relativo.
    \item Si \(0 < \alpha < 1\), el costo marginal individual se reduce cuando la firma produce menos (por un precio relativamente alto) y aumenta cuando produce más (por un precio relativamente bajo).
  \end{itemize}

  \item \textbf{Papel de \(\epsilon\): sensibilidad de la demanda a los precios relativos}  
  La elasticidad de sustitución \(\epsilon\) amplifica o atenúa el efecto de un cambio en el precio relativo sobre la demanda de la firma. Un \(\epsilon\) alto implica que una pequeña diferencia de precio relativo induce una gran variación en la cantidad demandada y, por ende, en el costo marginal relativo.

  \item \textbf{Relevancia para la NKPC:}  
  La Ecuación (15) es crucial para sustituir \(\psi_{t+k\mid t}\) en la regla de fijación de precios (Ecuación (11)). De este modo, la condición de optimalidad de precios, que originalmente está formulada en términos de costos marginales individuales, puede reescribirse en términos de:
  \begin{itemize}
    \item el costo marginal promedio \(\psi_{t+k}\), y
    \item la diferencia \(p^*_t - p_{t+k}\), que a su vez se relaciona con la inflación vía la ley de movimiento del nivel de precios.
  \end{itemize}
  Esta sustitución es un paso esencial en la derivación de la Curva de Phillips Nuevo Keynesiana (Ecuación (22)), donde la inflación se expresa como función del costo marginal real y de sus expectativas futuras.
\end{itemize}

%========================================
% Ecuación (16)
%========================================
\subsubsection*{Ecuación (16): Regla recursiva de fijación de precios óptimos}

La Ecuación (16) es la forma \emph{recursiva} y log-linealizada de la condición de fijación de precios óptimos por parte de las empresas que pueden reajustar su precio en el periodo \(t\). Resume la decisión de fijar el precio óptimo \(p^*_t\) como una combinación de:
\begin{itemize}
  \item el precio óptimo esperado para el periodo siguiente, \(E_t\{p^*_{t+1}\}\), y
  \item el nivel de precios actual \(p_t\) ajustado por la desviación del \emph{markup} promedio respecto a su nivel deseado, \(\hat{\mu}_t\).
\end{itemize}

En forma compacta:
\begin{equation}
p^*_t
=
\beta\theta\,E_t\{p^*_{t+1}\}
+
(1-\beta\theta)\bigl(p_t - \epsilon\,\hat{\mu}_t\bigr),
\tag{16}
\label{eq:pstar_16}
\end{equation}
donde \(\beta\theta \in (0,1)\) es el factor de descuento efectivo que combina el descuento intertemporal \(\beta\) y la rigidez de precios \(\theta\).

\subsubsection*{Tabla de símbolos relevantes (Ecuación 16)}

\begin{table}[htbp]
  \centering
  \begin{tabular}{p{0.19\textwidth} p{0.41\textwidth} p{0.30\textwidth}}
    \hline
    \textbf{Término} & \textbf{Definición rigurosa} & \textbf{Rol económico} \\
    \hline
    \(p^*_t\)
      & Logaritmo del precio óptimo fijado en el periodo \(t\) por la cohorte de empresas que puede reajustar precios. 
      & Variable de elección de las firmas; precio que maximiza el valor presente de beneficios mientras el precio permanezca fijo. \\[0.1cm]
    \(E_t\{p^*_{t+1}\}\)
      & Expectativa condicional (formada en \(t\)) del logaritmo del precio óptimo que fijarán las empresas que ajusten en \(t+1\). 
      & Componente \emph{prospectivo}: introduce la naturaleza hacia adelante de la fijación de precios. \\[0.1cm]
    \(p_t\)
      & Logaritmo del nivel de precios agregado \(P_t\). 
      & Referencia para el precio promedio de la economía en el periodo actual. \\[0.1cm]
    \(\hat{\mu}_t\)
      & Desviación del \emph{markup} promedio respecto a su nivel deseado: \(\hat{\mu}_t \equiv \mu_t - \mu\), donde \(\mu_t = p_t - \psi_t\) y \(\mu\) es el \emph{markup} de estado estacionario. 
      & Mide cuánto difiere el margen actual precio–costo del margen “natural” o sin fricciones; es la señal de presión de costos que impulsa los ajustes de precios. \\[0.1cm]
    \(\psi_t\)
      & Logaritmo del costo marginal nominal promedio en \(t\). 
      & Junto con \(p_t\), determina el \emph{markup} promedio \(\mu_t\). \\[0.1cm]
    \(\beta\)
      & Factor de descuento intertemporal del hogar/empresa, con \(0<\beta<1\). 
      & Descuenta los beneficios futuros; reduce el peso de costos y precios lejanos en el tiempo. \\[0.1cm]
    \(\theta\)
      & Fracción de empresas que \emph{no} reajusta su precio en un periodo dado. 
      & Parámetro de rigidez nominal: cuanto mayor es \(\theta\), más persistentes son los precios individuales. \\[0.1cm]
    \(\epsilon\)
      & Coeficiente compuesto (distinto de la elasticidad de sustitución entre variedades), igual a
        \(\displaystyle \epsilon = \frac{1-\alpha}{1-\alpha+\alpha\varepsilon}\),
        donde \(\varepsilon\) es la elasticidad de sustitución entre bienes diferenciados. 
      & Traduce la brecha de \emph{markup} \(\hat{\mu}_t\) en una corrección sobre el precio óptimo: mide cuán sensible es el precio deseado a la desviación del \emph{markup} respecto a su nivel natural. \\
    \hline
  \end{tabular}
  \caption{Símbolos usados en la Ecuación (16) y su interpretación económica.}
\end{table}

\subsubsection*{Derivación paso a paso de la Ecuación (16)}

La Ecuación (16) se obtiene a partir de la versión en suma infinita de la regla de fijación de precios log-linealizada. El punto de partida es la expresión:
\begin{equation}
p^*_t
=
(1-\beta\theta)\sum_{k=0}^{\infty}(\beta\theta)^k
E_t\bigl\{\,p_{t+k} - \epsilon\,\hat{\mu}_{t+k}\bigr\},
\tag{21}
\label{eq:pstar_21}
\end{equation}
que puede interpretarse como el \emph{promedio ponderado (descontado)} de los precios agregados futuros, ajustados por la brecha de \emph{markup} futura \(\hat{\mu}_{t+k}\). Los pesos \((1-\beta\theta)(\beta\theta)^k\) suman uno y combinan:
\begin{itemize}
  \item la probabilidad de que el precio fijado hoy siga vigente en \(t+k\) (\(\theta^k\)),
  \item y el descuento temporal de los beneficios futuros (\(\beta^k\)).
\end{itemize}

\paragraph{1. Desplazar un periodo hacia adelante la ecuación de suma infinita}

Consideremos la misma relación, pero ahora vista desde el periodo \(t+1\). Análogamente a \eqref{eq:pstar_21}:
\begin{equation}
p^*_{t+1}
=
(1-\beta\theta)\sum_{j=0}^{\infty}(\beta\theta)^j
E_{t+1}\bigl\{\,p_{t+1+j} - \epsilon\,\hat{\mu}_{t+1+j}\bigr\}.
\tag{16.a}
\label{eq:pstar_tplus1}
\end{equation}
Tomando expectativas condicionales en \(t\) de \eqref{eq:pstar_tplus1} y utilizando la propiedad de iteración de expectativas (\(E_t E_{t+1} = E_t\)):
\begin{align}
E_t\{p^*_{t+1}\}
&=
(1-\beta\theta)\sum_{j=0}^{\infty}(\beta\theta)^j
E_t\bigl\{\,p_{t+1+j} - \epsilon\,\hat{\mu}_{t+1+j}\bigr\}.
\tag{16.b}
\label{eq:Et_pstar_tplus1}
\end{align}

\paragraph{2. Reescritura de la suma de \(p^*_t\) separando el primer término}

Volvemos a \eqref{eq:pstar_21} y separamos explícitamente el término para \(k=0\) del resto de la suma:
\begin{align}
p^*_t
&=
(1-\beta\theta)\Bigl[
  E_t\{p_t - \epsilon\,\hat{\mu}_t\}
  +
  \sum_{k=1}^{\infty}(\beta\theta)^k
  E_t\bigl\{\,p_{t+k} - \epsilon\,\hat{\mu}_{t+k}\bigr\}
  \Bigr].
\tag{16.c}
\label{eq:pstar_split}
\end{align}
Obsérvese que el término para \(k=0\) es simplemente \(E_t\{p_t - \epsilon\,\hat{\mu}_t\} = p_t - \epsilon\,\hat{\mu}_t\), ya que \(p_t\) y \(\hat{\mu}_t\) son conocidos en \(t\).

En la parte de la suma para \(k\geq 1\), hacemos el cambio de índice \(j = k-1\). Entonces:
\[
\sum_{k=1}^{\infty}(\beta\theta)^k
E_t\bigl\{\,p_{t+k} - \epsilon\,\hat{\mu}_{t+k}\bigr\}
=
\beta\theta
\sum_{j=0}^{\infty}(\beta\theta)^j
E_t\bigl\{\,p_{t+1+j} - \epsilon\,\hat{\mu}_{t+1+j}\bigr\}.
\]
Sustituyendo en \eqref{eq:pstar_split}:
\begin{align}
p^*_t
&=
(1-\beta\theta)\bigl(p_t - \epsilon\,\hat{\mu}_t\bigr)
+
(1-\beta\theta)\,\beta\theta
\sum_{j=0}^{\infty}(\beta\theta)^j
E_t\bigl\{\,p_{t+1+j} - \epsilon\,\hat{\mu}_{t+1+j}\bigr\}.
\tag{16.d}
\label{eq:pstar_split2}
\end{align}

\paragraph{3. Identificación de la suma con \(E_t\{p^*_{t+1}\}\)}

Compare ahora \eqref{eq:Et_pstar_tplus1} y la suma que aparece en \eqref{eq:pstar_split2}. Por \eqref{eq:Et_pstar_tplus1}, tenemos:
\[
E_t\{p^*_{t+1}\}
=
(1-\beta\theta)
\sum_{j=0}^{\infty}(\beta\theta)^j
E_t\bigl\{\,p_{t+1+j} - \epsilon\,\hat{\mu}_{t+1+j}\bigr\}.
\]
Sustituyendo esta expresión en \eqref{eq:pstar_split2}:
\begin{align}
p^*_t
&=
(1-\beta\theta)\bigl(p_t - \epsilon\,\hat{\mu}_t\bigr)
+
\beta\theta\,E_t\{p^*_{t+1}\}.
\tag{16.e}
\end{align}
Reordenando términos se obtiene exactamente la Ecuación (16):
\begin{equation}
\boxed{
p^*_t
=
\beta\theta\,E_t\{p^*_{t+1}\}
+
(1-\beta\theta)\bigl(p_t - \epsilon\,\hat{\mu}_t\bigr)
}
\tag{16}
\end{equation}
que es la forma recursiva de la regla de fijación de precios óptimos.

\subsubsection*{Interpretación económica de la Ecuación (16)}

La Ecuación \eqref{eq:pstar_16} encapsula de manera muy clara la lógica de fijación de precios en el modelo Nuevo Keynesiano con precios a la Calvo:

\begin{itemize}
  \item \textbf{Naturaleza hacia adelante (\emph{forward-looking}):}  
  El término \(\beta\theta\,E_t\{p^*_{t+1}\}\) indica que el precio que las firmas fijan hoy está fuertemente influido por el precio óptimo que anticipan fijar en el futuro.  
  \begin{itemize}
    \item Si \(\theta\) es grande (precios muy rígidos), es muy probable que el precio fijado hoy permanezca vigente en periodos futuros. En consecuencia, las empresas otorgan un peso importante al precio óptimo futuro, ya que desean que el precio de hoy sea coherente con las condiciones futuras.
    \item Si \(\theta\) es pequeño (precios muy flexibles), el término \(\beta\theta\) es pequeño y la empresa se centra principalmente en las condiciones actuales.
  \end{itemize}

  \item \textbf{Corrección en función del \emph{markup} actual:}  
  El componente \((1-\beta\theta)\bigl(p_t - \epsilon\,\hat{\mu}_t\bigr)\) refleja el ajuste de precios motivado por:
  \begin{itemize}
    \item el nivel actual de precios agregados \(p_t\), y
    \item la desviación del \emph{markup} promedio \(\hat{\mu}_t\).
  \end{itemize}
  Si el \emph{markup} promedio actual es inferior al deseado (\(\hat{\mu}_t < 0\)), esto significa que los costos marginales están relativamente altos en comparación con los precios vigentes. Para restablecer el margen de beneficio, las firmas desearán fijar un precio óptimo \(p^*_t\) relativamente mayor que \(p_t\), lo que está recogido en el término \(-\epsilon\,\hat{\mu}_t\).

  \item \textbf{Peso relativo de presente vs. futuro:}  
  Los coeficientes \(\beta\theta\) y \((1-\beta\theta)\) determinan cómo se ponderan:
  \begin{itemize}
    \item las condiciones futuras (vía \(E_t\{p^*_{t+1}\}\)), y
    \item las condiciones actuales (vía \(p_t\) y \(\hat{\mu}_t\)).
  \end{itemize}
  Cuanto mayor sea \(\theta\), mayor será el peso de las expectativas futuras en la fijación de precios de hoy; cuanto menor sea \(\theta\), más “miope” parecerá el comportamiento de precios de las empresas.

  \item \textbf{Puente hacia la Curva de Phillips Nuevo Keynesiana (NKPC):}  
  La Ecuación (16) es esencial porque permite, junto con la ley de movimiento del nivel de precios (Ecuación (8), \(\pi_t = (1-\theta)(p^*_t - p_{t-1})\)), eliminar la variable \(p^*_t\) y obtener una relación que vincula directamente la inflación con:
  \begin{itemize}
    \item sus expectativas futuras (\(\beta E_t\{\pi_{t+1}\}\)), y
    \item la brecha de \emph{markup} \(\hat{\mu}_t\), proporcional al costo marginal real.
  \end{itemize}
  Ese resultado es la versión canónica de la Curva de Phillips Nuevo Keynesiana:
  \[
  \pi_t = \beta E_t\{\pi_{t+1}\} - \lambda\,\hat{\mu}_t,
  \]
  donde \(\lambda\) es una función de \(\theta\), \(\beta\) y los parámetros tecnológicos. Así, la Ecuación (16) constituye el eslabón microfundado que conecta las decisiones de fijación de precios de las firmas con la dinámica de la inflación agregada.
\end{itemize}


%========================================
% Ecuación (17)
%========================================
\subsubsection*{Ecuación (17): Curva de Phillips Nuevo Keynesiana (CPNK) en términos del \emph{markup}}

La Ecuación (17) resume la dinámica de la inflación en el Modelo Básico Nuevo Keynesiano cuando se combinan:
\begin{itemize}
  \item la ley de movimiento del nivel de precios bajo precios a la Calvo, y
  \item la regla óptima de fijación de precios de las empresas.
\end{itemize}
El resultado es la denominada \emph{Curva de Phillips Nuevo Keynesiana} (CPNK) en su versión expresada en función de la brecha del \emph{markup}.

En forma compacta:
\begin{equation}
  \pi_t = \beta E_t\{\pi_{t+1}\} - \lambda\,\hat{\mu}_t,
  \tag{17}
  \label{eq:NKPC_17}
\end{equation}
donde \(\lambda>0\) es la pendiente de la Curva de Phillips.

\subsubsection*{Tabla de símbolos relevantes (Ecuación 17)}

\begin{table}[htbp]
  \centering
  \begin{tabular}{p{0.19\textwidth} p{0.41\textwidth} p{0.30\textwidth}}
    \hline
    \textbf{Término} & \textbf{Definición rigurosa} & \textbf{Rol económico} \\
    \hline
    \(\pi_t\)
      & Tasa de inflación log-linealizada en el periodo \(t\):
        \(\pi_t \equiv p_t - p_{t-1}\), donde \(p_t = \log P_t\). 
      & Mide el cambio porcentual (aproximado) en el nivel de precios agregado entre \(t-1\) y \(t\). \\[0.1cm]
    \(E_t\{\pi_{t+1}\}\)
      & Expectativa condicional, formada en \(t\), de la inflación en \(t+1\). 
      & Componente \emph{prospectivo}: recoge el carácter hacia adelante de la inflación en el modelo NK. \\[0.1cm]
    \(\mu_t\)
      & \emph{Markup} promedio en \(t\):
        \(\mu_t \equiv p_t - \psi_t\), donde \(\psi_t\) es el logaritmo del costo marginal nominal promedio. 
      & Mide el margen entre el precio agregado y el costo marginal nominal promedio. \\[0.1cm]
    \(\mu\)
      & \emph{Markup} deseado o “natural” de estado estacionario:
        \(\mu \equiv \log M\), con \(M = \frac{\varepsilon}{\varepsilon-1}\), donde \(\varepsilon\) es la elasticidad de sustitución entre variedades. 
      & Nivel de referencia del margen precio–costo en ausencia de fricciones nominales. \\[0.1cm]
    \(\hat{\mu}_t\)
      & Brecha del \emph{markup}:
        \(\hat{\mu}_t \equiv \mu_t - \mu\). 
      & Variable de “presión de costos”: si \(\hat{\mu}_t<0\), los costos reales son altos relativamente a los precios, empujando al alza los precios óptimos. \\[0.1cm]
    \(\beta\)
      & Factor de descuento intertemporal del hogar/empresa, con \(0<\beta<1\). 
      & Descuenta la inflación futura en la determinación de la inflación presente. \\[0.1cm]
    \(\theta\)
      & Fracción de empresas que \emph{no} reajusta su precio en un periodo dado (parámetro de Calvo). 
      & Índice de rigidez nominal: cuanto mayor es \(\theta\), más persistentes son los precios individuales. \\[0.1cm]
    \(\epsilon\)
      & Parámetro compuesto:
        \[
          \epsilon
          \equiv
          \frac{1-\alpha}{\,1-\alpha + \alpha\varepsilon\,},
        \]
        donde \(\alpha\) proviene de la tecnología \(Y_t(i)=A_tN_t(i)^{1-\alpha}\) y \(\varepsilon\) es la elasticidad de sustitución entre variedades. 
      & Vincula el \emph{markup} promedio con el costo marginal real y la respuesta de precios a choques de costos. \\[0.1cm]
    \(\lambda\)
      & Pendiente de la CPNK:
        \[
          \lambda
          \equiv
          \frac{(1-\theta)(1-\beta\theta)}{\theta}\,\epsilon.
        \]
      & Mide cuán sensible es la inflación a la brecha del \emph{markup}. Disminuye cuando la rigidez de precios \(\theta\) aumenta. \\
    \hline
  \end{tabular}
  \caption{Símbolos usados en la Ecuación (17) y su interpretación económica.}
\end{table}

\subsubsection*{Derivación paso a paso de la Ecuación (17)}

El punto de partida son dos relaciones clave del bloque de precios:

\begin{enumerate}
  \item \textbf{Ley de movimiento del nivel de precios (Ecuación 8)}  
  La dinámica del índice de precios bajo precios a la Calvo se resume en:
  \begin{equation}
    \pi_t = (1-\theta)\bigl(p^*_t - p_{t-1}\bigr),
    \tag{8}
    \label{eq:pi_8}
  \end{equation}
  donde \(p^*_t\) es el logaritmo del precio óptimo fijado por las empresas que pueden reajustar en \(t\).

  \item \textbf{Regla recursiva de fijación de precios óptimos (Ecuación 16)}  
  La decisión óptima de precios satisface:
  \begin{equation}
    p^*_t
    =
    \beta\theta\,E_t\{p^*_{t+1}\}
    +
    (1-\beta\theta)\bigl(p_t - \epsilon\,\hat{\mu}_t\bigr).
    \tag{16}
    \label{eq:pstar_16_again}
  \end{equation}
\end{enumerate}

El objetivo es eliminar \(p^*_t\) y \(E_t\{p^*_{t+1}\}\) de \eqref{eq:pi_8} y \eqref{eq:pstar_16_again} para obtener una relación directa entre la inflación y la brecha de \emph{markup}.

\paragraph{1. Expresar \(p^*_t - p_{t-1}\) en términos de inflación y \(p^*_t - p_t\)}

Partimos de la identidad:
\[
p^*_t - p_{t-1}
=
(p^*_t - p_t) + (p_t - p_{t-1})
=
(p^*_t - p_t) + \pi_t.
\]
Sustituyendo en \eqref{eq:pi_8}:
\begin{align}
\pi_t
&=
(1-\theta)\,\bigl[(p^*_t - p_t) + \pi_t\bigr] \notag \\
&=
(1-\theta)(p^*_t - p_t) + (1-\theta)\pi_t.
\end{align}
Reordenando para despejar \(\pi_t\):
\begin{align}
\pi_t - (1-\theta)\pi_t
&=
(1-\theta)(p^*_t - p_t), \notag\\[0.1cm]
\theta\,\pi_t
&=
(1-\theta)(p^*_t - p_t).
\end{align}
Por tanto:
\begin{equation}
  \pi_t
  =
  \frac{1-\theta}{\theta}\,\bigl(p^*_t - p_t\bigr).
  \label{eq:pi_pstar_gap}
\end{equation}

\paragraph{2. Usar la Ecuación (16) para reescribir \((p^*_t - p_t)\)}

De \eqref{eq:pstar_16_again}:
\[
p^*_t - p_t
=
\beta\theta\,E_t\{p^*_{t+1}\}
+
(1-\beta\theta)\bigl(p_t - \epsilon\,\hat{\mu}_t\bigr)
- p_t.
\]
Agrupando términos en \(p_t\):
\begin{align}
p^*_t - p_t
&=
\beta\theta\,E_t\{p^*_{t+1}\}
+
(1-\beta\theta)p_t
- (1-\beta\theta)\epsilon\,\hat{\mu}_t
- p_t \notag\\[0.1cm]
&=
\beta\theta\,E_t\{p^*_{t+1}\}
-
\beta\theta\,p_t
-
(1-\beta\theta)\epsilon\,\hat{\mu}_t.
\end{align}
Es decir:
\begin{equation}
  p^*_t - p_t
  =
  \beta\theta\bigl(E_t\{p^*_{t+1}\} - p_t\bigr)
  -
  (1-\beta\theta)\epsilon\,\hat{\mu}_t.
  \label{eq:pstar_minus_p}
\end{equation}

\paragraph{3. Sustituir \eqref{eq:pstar_minus_p} en \eqref{eq:pi_pstar_gap}}

Insertando \eqref{eq:pstar_minus_p} en \eqref{eq:pi_pstar_gap}:
\begin{align}
\pi_t
&=
\frac{1-\theta}{\theta}
\left[
  \beta\theta\bigl(E_t\{p^*_{t+1}\} - p_t\bigr)
  -
  (1-\beta\theta)\epsilon\,\hat{\mu}_t
\right] \notag\\[0.1cm]
&=
(1-\theta)\beta\,\bigl(E_t\{p^*_{t+1}\} - p_t\bigr)
-
\frac{(1-\theta)(1-\beta\theta)}{\theta}\,\epsilon\,\hat{\mu}_t.
\label{eq:pi_with_pstar_next}
\end{align}

\paragraph{4. Relacionar \(E_t\{p^*_{t+1}\} - p_t\) con la inflación esperada}

Aplicamos el mismo argumento un periodo hacia adelante. A partir de \eqref{eq:pi_pstar_gap} desplazada a \(t+1\):
\[
\pi_{t+1}
=
\frac{1-\theta}{\theta}\,\bigl(p^*_{t+1} - p_t\bigr)
\quad\Rightarrow\quad
p^*_{t+1} - p_t
=
\frac{\theta}{1-\theta}\,\pi_{t+1}.
\]
Tomando expectativas condicionales en \(t\):
\[
E_t\{p^*_{t+1} - p_t\}
=
\frac{\theta}{1-\theta}\,E_t\{\pi_{t+1}\}.
\]
Sustituyendo en \eqref{eq:pi_with_pstar_next}:
\begin{align}
\pi_t
&=
(1-\theta)\beta\,
\frac{\theta}{1-\theta}E_t\{\pi_{t+1}\}
-
\frac{(1-\theta)(1-\beta\theta)}{\theta}\,\epsilon\,\hat{\mu}_t \notag\\[0.1cm]
&=
\beta\theta\,E_t\{\pi_{t+1}\}
-
\frac{(1-\theta)(1-\beta\theta)}{\theta}\,\epsilon\,\hat{\mu}_t.
\end{align}
Definiendo:
\begin{equation}
  \lambda
  \equiv
  \frac{(1-\theta)(1-\beta\theta)}{\theta}\,\epsilon,
  \label{eq:lambda_def}
\end{equation}
obtenemos finalmente la Ecuación (17):
\begin{equation}
  \boxed{
    \pi_t
    =
    \beta\,E_t\{\pi_{t+1}\}
    -
    \lambda\,\hat{\mu}_t
  }.
\end{equation}

\subsubsection*{Forma alternativa en términos de costo marginal real}

Dado que el \emph{markup} promedio está definido como \(\mu_t = p_t - \psi_t\), el costo marginal real logarítmico promedio es:
\[
m c_t \equiv \psi_t - p_t = -\mu_t.
\]
Si \(\bar{m}c\) denota el valor de estado estacionario del costo marginal real, se tiene:
\[
\hat{m}c_t \equiv m c_t - \bar{m}c = -(\mu_t - \mu) = -\hat{\mu}_t.
\]
Sustituyendo \(\hat{\mu}_t = -\hat{m}c_t\) en \eqref{eq:NKPC_17}:
\begin{equation}
  \pi_t
  =
  \beta E_t\{\pi_{t+1}\}
  +
  \lambda\,\hat{m}c_t,
  \tag{17'}
  \label{eq:NKPC_mc}
\end{equation}
que es la versión estándar de la CPNK escrita en función del costo marginal real.

\subsubsection*{Interpretación económica de la Ecuación (17)}

La Ecuación \eqref{eq:NKPC_17} concentra varios elementos clave del enfoque Nuevo Keynesiano:

\begin{itemize}
  \item \textbf{Inflación hacia adelante (\emph{forward-looking}):}  
  El término \(\beta E_t\{\pi_{t+1}\}\) muestra que la inflación actual depende de la inflación que las empresas esperan para el futuro. Como los precios se fijan bajo rigidez de Calvo y pueden permanecer vigentes varios periodos, las firmas internalizan la inflación futura al fijar sus precios hoy.

  \item \textbf{La brecha del \emph{markup} como forzador de la inflación:}  
  La variable \(\hat{\mu}_t\) resume la discrepancia entre:
  \[
  \text{\emph{markup} promedio actual} \quad \mu_t
  \quad \text{y el \emph{markup} deseado} \quad \mu.
  \]
  \begin{itemize}
    \item Si \(\hat{\mu}_t<0\), los costos marginales reales están elevados relativamente a los precios: las firmas desean subir sus precios para restaurar el margen, lo que genera presiones inflacionarias (\(\pi_t\) aumenta).
    \item Si \(\hat{\mu}_t>0\), el \emph{markup} es “demasiado alto” y, en ausencia de otros factores, tendería a generarse presión desinflacionaria.
  \end{itemize}

  \item \textbf{La pendiente de la CPNK (\(\lambda\)):}  
  El parámetro \(\lambda\) determina la sensibilidad de la inflación a la brecha de \emph{markup}. De su definición \eqref{eq:lambda_def} se desprende que:
  \begin{itemize}
    \item \(\lambda\) es \emph{decreciente} en la rigidez de precios \(\theta\):  
    cuanto más rígidos son los precios (mayor \(\theta\)), más lenta es la respuesta de la inflación a un mismo choque de costos.
    \item \(\lambda\) aumenta cuando la respuesta de precios a los costos (capturada por \(\epsilon\)) es mayor, o cuando los precios se ajustan con mayor frecuencia (menor \(\theta\)).
  \end{itemize}

  \item \textbf{Versión en términos de costo marginal real:}  
  Al reescribir la ecuación como en \eqref{eq:NKPC_mc}, la inflación responde positivamente a la desviación del costo marginal real respecto a su estado estacionario:
  \[
    \hat{m}c_t \uparrow
    \quad\Rightarrow\quad
    \pi_t \uparrow.
  \]
  Esta forma es especialmente útil para conectar la CPNK con la brecha del producto, ya que el costo marginal real está íntimamente ligado al nivel de actividad económica vía condiciones de oferta laboral y tecnología.
\end{itemize}

% --------------------------------------------------
% Ecuación (18): Markup promedio en función de y_t y a_t
% --------------------------------------------------

\subsection*{Ecuación (18): Markup promedio en función del producto y la tecnología}

La Ecuación (18) expresa el \emph{logaritmo del markup promedio} de la economía, $\mu_t$,
como una combinación lineal del producto agregado $y_t$ y del nivel de tecnología $a_t$.
Esta ecuación resume cómo las condiciones reales (demanda y productividad) se traducen en
presión sobre los márgenes de ganancia de las firmas.

\begin{equation}
  \mu_t
  \;=\;
  -\left( \sigma + \phi + \frac{\alpha}{1-\alpha} \right) y_t
  \;+\;
  \left( \frac{1+\phi}{1-\alpha} \right) a_t
  \;+\;
  \log(1-\alpha)
  \tag{18}
\end{equation}

donde, por definición,
\[
\mu_t \;\equiv\; p_t - \psi_t
\]
es el logaritmo del markup promedio, i.e. la diferencia entre el logaritmo del nivel
de precios agregado $p_t$ y el logaritmo del costo marginal nominal promedio $\psi_t$.

%---------------------------------------------
% Tabla de símbolos relevantes (Ecuación 18)
%---------------------------------------------

\begin{table}[h!]
\centering
\begin{tabular}{p{0.18\textwidth} p{0.18\textwidth} p{0.55\textwidth}}
\hline
\textbf{Símbolo} & \textbf{Tipo} & \textbf{Descripción} \\
\hline
$\mu_t$  & Variable (log) &
Markup promedio en logaritmos: $\mu_t \equiv p_t - \psi_t$. Mide el margen de ganancia
promedio de las empresas sobre su costo marginal nominal. \\[4pt]

$y_t$    & Variable (log) &
Logaritmo del producto agregado ($Y_t$). Refleja el nivel de actividad económica y
la demanda de trabajo. \\[4pt]

$a_t$    & Variable (log) &
Logaritmo del nivel de tecnología ($A_t$). Captura el estado de la productividad
total de los factores. \\[4pt]

$\sigma$ & Parámetro &
Inverso de la elasticidad de sustitución intertemporal del consumo. Influye en la
sensibilidad del salario real (y por tanto del costo marginal) al nivel de producto. \\[4pt]

$\phi$   & Parámetro &
Parámetro de desutilidad del trabajo (relacionado con la elasticidad de la oferta
de trabajo). Determina cómo responde el salario real al empleo. \\[4pt]

$\alpha$ & Parámetro &
Parámetro asociado a los rendimientos decrecientes del trabajo en la función de
producción. Se asume que $1-\alpha$ es la elasticidad del producto respecto al trabajo. \\[4pt]
\hline
\end{tabular}
\end{table}

%---------------------------------------------
% Origen y derivación (boceto)
%---------------------------------------------

\subsubsection*{Origen y derivación de la Ecuación (18)}

La lógica de la Ecuación (18) es vincular el markup promedio $\mu_t$ con las variables reales
agregadas del modelo. El punto de partida es:

\begin{itemize}
  \item La definición del \emph{markup promedio} como diferencia entre el logaritmo del
  nivel de precios y el logaritmo del costo marginal nominal promedio:
  \[
     \mu_t \;=\; p_t - \psi_t.
  \]

  \item La expresión del costo marginal real (y, por tanto, del markup) en términos del
  salario real y de la productividad, que puede escribirse (en forma log-lineal) como:
  \[
     \mu_t
     \;=\;
     - (w_t - p_t)
     \;+\;
     \bigl(a_t - \alpha n_t + \log(1-\alpha)\bigr),
  \]
  donde $w_t - p_t$ es el salario real y $n_t$ el logaritmo del empleo agregado.

  \item La condición de oferta de trabajo del hogar (Ecuación (2)), que en logaritmos y
  usando el equilibrio en el mercado de bienes ($c_t = y_t$) se escribe como:
  \[
     w_t - p_t = \sigma y_t + \phi n_t.
  \]

  \item La relación tecnológica entre producto, empleo y tecnología (Ecuación (14)):
  \[
     n_t = \frac{1}{1-\alpha}(y_t - a_t).
  \]
\end{itemize}

Sustituyendo primero la ecuación de oferta de trabajo en la expresión de $\mu_t$ se elimina
el salario real $w_t - p_t$:

\begin{equation*}
  \mu_t
  \;=\;
  -(\sigma y_t + \phi n_t)
  \;+\;
  \bigl(a_t - \alpha n_t + \log(1-\alpha)\bigr).
\end{equation*}

Luego, sustituyendo la expresión de $n_t$ en términos de $y_t$ y $a_t$ (Ecuación (14)) y
reagrupando términos en $y_t$ y $a_t$, se obtiene, tras álgebra sencilla, que el markup
promedio puede escribirse únicamente en función de $y_t$ y $a_t$:

\begin{equation*}
  \mu_t
  \;=\;
  -\left( \sigma + \phi + \frac{\alpha}{1-\alpha} \right) y_t
  \;+\;
  \left( \frac{1+\phi}{1-\alpha} \right) a_t
  \;+\;
  \log(1-\alpha),
\end{equation*}

lo cual coincide con la Ecuación (18).

%---------------------------------------------
% Interpretación económica
%---------------------------------------------

\subsubsection*{Interpretación económica}

La Ecuación (18) es crucial porque conecta el \emph{markup promedio} ---la variable que impulsa
la inflación en la Curva de Phillips Nuevo Keynesiana (Ecuación (17))--- con el lado real
de la economía:

\begin{itemize}
  \item \textbf{Relación negativa con el producto $y_t$:}  
  El coeficiente que multiplica $y_t$ es negativo:
  \[
    -\left(\sigma + \phi + \frac{\alpha}{1-\alpha}\right) < 0.
  \]
  Cuando el producto agregado $y_t$ aumenta, la demanda de trabajo se incrementa, lo que
  eleva el salario real y, en consecuencia, el costo marginal real de las empresas.
  Dado que los precios no se ajustan de manera totalmente flexible (rigidez de Calvo),
  el aumento del costo marginal se traduce en una reducción del markup promedio $\mu_t$.

  \item \textbf{Relación positiva con la tecnología $a_t$:}  
  El coeficiente que acompaña a $a_t$ es positivo:
  \[
    \frac{1+\phi}{1-\alpha} > 0.
  \]
  Una mejora en la tecnología aumenta la productividad del trabajo. Para un nivel dado de
  producto, se requiere menos empleo; ello reduce el costo marginal y permite a las firmas
  sostener un mayor markup promedio, ceteris paribus.

  \item \textbf{Papel de los parámetros estructurales:}  
  Los parámetros $\sigma$, $\phi$ y $\alpha$ controlan la magnitud de la respuesta del
  markup a cambios en $y_t$ y $a_t$:
  \begin{itemize}
    \item Un $\sigma$ grande implica que el salario real es muy sensible al nivel de
    producto (por el lado de la utilidad marginal del consumo), intensificando la caída
    del markup cuando $y_t$ aumenta.
    \item Un $\phi$ grande refleja una oferta de trabajo menos elástica: el salario real
    reacciona con fuerza a cambios en el empleo, elevando el costo marginal y reduciendo
    el markup cuando la economía está más activa.
    \item Un $\alpha$ mayor acentúa los rendimientos decrecientes del trabajo, haciendo
    que aumentos en $y_t$ requieran incrementos relativamente mayores en $n_t$ y, por
    tanto, en el costo marginal.
  \end{itemize}
\end{itemize}

Esta caracterización del markup promedio es el insumo esencial para construir la relación
entre la \emph{brecha del producto} y la inflación. Al evaluar la Ecuación (18) en el equilibrio
de precios flexibles se obtiene el producto natural $y_{n,t}$ (Ecuación (19)), y la diferencia
entre $\mu_t$ y el markup deseado $\mu$ conduce a la expresión:

\[
  \hat{\mu}_t
  \;=\;
  -\left( \sigma + \phi + \frac{\alpha}{1-\alpha} \right) \tilde{y}_t,
\]

donde $\tilde{y}_t \equiv y_t - y_{n,t}$ es la brecha del producto. Esta última relación se
sustituye en la Curva de Phillips Nuevo Keynesiana en términos del markup (Ecuación (17)) para
obtener su forma estándar en función de la brecha del producto (Ecuación (22)).

% --------------------------------------------------
% Ecuación (19): Nivel natural de producto
% --------------------------------------------------

\subsection*{Ecuación (19): Nivel Natural de Output}

La Ecuación (19) define el \emph{nivel natural de producto} $y_{n,t}$ como el nivel de
producción que prevalecería en la economía si los precios fueran completamente flexibles
y el markup promedio fuera igual a su valor deseado (sin fricciones).

Partiendo de la relación general entre el markup promedio y el producto (Ecuación (18)),
al imponer precios flexibles se obtiene:

\begin{equation}
  \mu
  \;=\;
  -\left( \sigma + \phi + \frac{\alpha}{1-\alpha} \right) y_{n,t}
  \;+\;
  \left( \frac{1+\phi}{1-\alpha} \right) a_t
  \;+\;
  \log(1-\alpha)
  \tag{19}
\end{equation}

donde $\mu$ es el logaritmo del markup deseado (constante) bajo precios flexibles.

%---------------------------------------------
% Tabla de símbolos relevantes (Ecuación 19)
%---------------------------------------------

\begin{table}[h!]
\centering
\begin{tabular}{p{0.18\textwidth} p{0.20\textwidth} p{0.52\textwidth}}
\hline
\textbf{Símbolo} & \textbf{Tipo} & \textbf{Descripción} \\
\hline
$\mu$        & Parámetro (log) &
Logaritmo del markup deseado o "sin fricciones". Bajo precios flexibles el markup
promedio es constante e igual a $\mu$. \\[4pt]

$y_{n,t}$    & Variable (log) &
Nivel natural de producto: logaritmo de la producción de equilibrio que prevalecería
con precios completamente flexibles en el período $t$. \\[4pt]

$a_t$        & Variable (log) &
Logaritmo del nivel de tecnología agregada ($A_t$). Es el shock real que desplaza el
producto natural. \\[4pt]

$\sigma$     & Parámetro &
Inverso de la elasticidad de sustitución intertemporal del consumo. Afecta cómo el
salario real responde al nivel de producto. \\[4pt]

$\phi$       & Parámetro &
Parámetro asociado a la desutilidad del trabajo (relacionado con la elasticidad de la
oferta laboral). Controla la respuesta del salario real al empleo. \\[4pt]

$\alpha$     & Parámetro &
Parámetro de rendimientos decrecientes del trabajo en la función de producción. Se
asume que $1-\alpha$ es la elasticidad del producto respecto al trabajo. \\[4pt]
\hline
\end{tabular}
\end{table}

%---------------------------------------------
% Origen de la Ecuación (19)
%---------------------------------------------

\subsubsection*{Origen de la Ecuación (19)}

Recordemos la relación general entre el markup promedio y el producto (Ecuación (18)):

\begin{equation*}
  \mu_t
  \;=\;
  -\left( \sigma + \phi + \frac{\alpha}{1-\alpha} \right) y_t
  \;+\;
  \left( \frac{1+\phi}{1-\alpha} \right) a_t
  \;+\;
  \log(1-\alpha).
\end{equation*}

Bajo \textbf{precios flexibles} se cumple:

\begin{itemize}
  \item El markup promedio es constante e igual al markup deseado:
  \[
    \mu_t = \mu.
  \]
  \item El nivel de producto de equilibrio se denomina \emph{nivel natural de output}:
  \[
    y_t = y_{n,t}.
  \]
\end{itemize}

Sustituyendo estas igualdades en la Ecuación (18) se obtiene directamente la Ecuación (19):

\begin{equation*}
  \mu
  \;=\;
  -\left( \sigma + \phi + \frac{\alpha}{1-\alpha} \right) y_{n,t}
  \;+\;
  \left( \frac{1+\phi}{1-\alpha} \right) a_t
  \;+\;
  \log(1-\alpha).
\end{equation*}

Es decir, el nivel natural de producto $y_{n,t}$ queda determinado exclusivamente por el
estado de la tecnología $a_t$ y por los parámetros estructurales del modelo.

%---------------------------------------------
% Interpretación económica
%---------------------------------------------

\subsubsection*{Interpretación económica}

La Ecuación (19) cumple tres funciones clave en el modelo:

\begin{itemize}
  \item \textbf{Define el punto de referencia real de la economía.}  
  $y_{n,t}$ es el nivel de output que prevalecería en ausencia de rigideces nominales.
  Todas las brechas que importan para la dinámica de la inflación y la política monetaria
  se miden en torno a este nivel.

  \item \textbf{Es independiente de la política monetaria y de shocks de preferencias.}  
  Bajo precios flexibles, la política monetaria no puede afectar de manera persistente
  las variables reales; el nivel natural de output está completamente determinado por
  la tecnología $a_t$ y los parámetros reales. Los shocks de preferencias $z_t$ no
  influyen en $y_{n,t}$.

  \item \textbf{Es la base para definir la brecha del producto.}  
  Una vez que se conoce $y_{n,t}$, se define la brecha del producto como:
  \[
    \tilde{y}_t \equiv y_t - y_{n,t},
  \]
  que es la variable que aparece en la Curva de Phillips Nuevo Keynesiana en su forma
  más familiar. Restando la Ecuación (19) de la Ecuación (18) se obtiene la relación
  entre el \emph{markup gap} y la brecha del producto (Ecuación (21)), paso crucial
  para reescribir la NKPC en términos de $\tilde{y}_t$.
\end{itemize}

% --------------------------------------------------
% Ecuación (20): Nivel natural de producto (forma reducida)
% --------------------------------------------------

\subsection*{Ecuación (20): Nivel Natural de Producto en forma reducida}

La Ecuación (20) presenta el \emph{nivel natural de producto} $y_{n,t}$ como una función
lineal del shock tecnológico $a_t$, al despejar $y_{n,t}$ de la Ecuación (19). Esta es la
forma analítica compacta del nivel natural de output:

\begin{equation}
  y_{n,t} \;=\; \psi_{ya}\, a_t \;+\; \psi_y
  \tag{20}
\end{equation}

donde los coeficientes $\psi_{ya}$ y $\psi_y$ son combinaciones de parámetros
estructurales del modelo.

%---------------------------------------------
% Tabla de símbolos relevantes (Ecuación 20)
%---------------------------------------------

\begin{table}[h!]
\centering
\begin{tabular}{p{0.18\textwidth} p{0.20\textwidth} p{0.52\textwidth}}
\hline
\textbf{Símbolo} & \textbf{Tipo} & \textbf{Descripción} \\
\hline
$y_{n,t}$   & Variable (log) &
Nivel natural de output: logaritmo de la producción de equilibrio que prevalecería
si los precios fueran flexibles en el período $t$. \\[4pt]

$a_t$       & Variable (log) &
Logaritmo del nivel de tecnología agregada. Es el shock real que desplaza el nivel
natural de producto. \\[4pt]

$\psi_{ya}$ & Parámetro &
Coeficiente que mide la sensibilidad del nivel natural de output $y_{n,t}$ frente
a los shocks tecnológicos $a_t$. \\[4pt]

$\psi_y$    & Parámetro &
Término constante (intercepto) del nivel natural de output. Resume el efecto
combinado de los parámetros estructurales y del markup deseado. \\[4pt]
\hline
\end{tabular}
\end{table}

%---------------------------------------------
% Definición de los coeficientes \psi_{ya} y \psi_y
%---------------------------------------------

\subsubsection*{Definición de los coeficientes}

Los coeficientes $\psi_{ya}$ y $\psi_y$ se definen como:

\begin{align*}
  \psi_{ya} &\equiv
  \frac{1+\phi}{\sigma (1-\alpha) + \phi + \alpha}, \\[6pt]
  \psi_y &\equiv
  -\,\frac{(1-\alpha)\bigl(\mu - \log(1-\alpha)\bigr)}
          {\sigma (1-\alpha) + \phi + \alpha},
\end{align*}

donde:

\begin{itemize}
  \item $\sigma$ es el inverso de la elasticidad de sustitución intertemporal del consumo.
  \item $\phi$ es el inverso de la elasticidad de la oferta de trabajo.
  \item $\alpha$ es el parámetro asociado a los rendimientos decrecientes del trabajo en la función de producción.
  \item $\mu$ es el logaritmo del markup deseado (sin fricciones) bajo precios flexibles.
\end{itemize}

Por construcción, se cumple $\psi_{ya} > 0$, de modo que los aumentos en $a_t$
incrementan el nivel natural de output.

%---------------------------------------------
% Origen de la Ecuación (20)
%---------------------------------------------

\subsubsection*{Origen de la Ecuación (20)}

La Ecuación (20) se obtiene directamente al despejar $y_{n,t}$ en la Ecuación (19):

\begin{equation*}
  \mu
  \;=\;
  -\left( \sigma + \phi + \frac{\alpha}{1-\alpha} \right) y_{n,t}
  \;+\;
  \left( \frac{1+\phi}{1-\alpha} \right) a_t
  \;+\;
  \log(1-\alpha),
\end{equation*}

que puede reordenarse como:

\begin{equation*}
  \left( \sigma (1-\alpha) + \phi + \alpha \right) y_{n,t}
  \;=\;
  (1+\phi)\, a_t
  \;-\;
  (1-\alpha)\bigl(\mu - \log(1-\alpha)\bigr).
\end{equation*}

Dividiendo ambos lados por $\sigma (1-\alpha) + \phi + \alpha$ se obtiene:

\begin{equation*}
  y_{n,t}
  \;=\;
  \underbrace{\frac{1+\phi}{\sigma (1-\alpha) + \phi + \alpha}}_{\psi_{ya}} a_t
  \;+\;
  \underbrace{
  -\,\frac{(1-\alpha)\bigl(\mu - \log(1-\alpha)\bigr)}
          {\sigma (1-\alpha) + \phi + \alpha}
  }_{\psi_y},
\end{equation*}

lo que coincide con la Ecuación (20):
\[
  y_{n,t} = \psi_{ya} a_t + \psi_y.
\]

%---------------------------------------------
% Interpretación económica
%---------------------------------------------

\subsubsection*{Interpretación económica}

La Ecuación (20) tiene varias implicaciones centrales:

\begin{itemize}
  \item \textbf{Neutralidad monetaria en el equilibrio flexible.}  
  El nivel natural de output depende únicamente del shock tecnológico $a_t$ y de los
  parámetros reales del modelo; la política monetaria y los shocks de preferencias
  $z_t$ no afectan $y_{n,t}$.

  \item \textbf{Impacto de la tecnología.}  
  Dado que $\psi_{ya} > 0$, un aumento en $a_t$ (mejora de la productividad agregada)
  incrementa el nivel natural de output. El coeficiente $\psi_{ya}$ gobierna el tamaño
  de esta respuesta.

  \item \textbf{Rol del markup deseado $\mu$.}  
  El término constante $\psi_y$ incorpora el efecto del poder de mercado. Un markup
  más alto ($\mu$ mayor) reduce el nivel natural de output (vía un $\psi_y$ más
  negativo), aun cuando no altera la sensibilidad de $y_{n,t}$ a la tecnología
  ($\psi_{ya}$).

  \item \textbf{Base para la brecha del producto.}  
  Una vez que $y_{n,t}$ está expresado en función de $a_t$, puede definirse la
  \emph{brecha del producto} como $\tilde{y}_t \equiv y_t - y_{n,t}$. Al combinar
  esta definición con la relación entre el markup y el producto (Ecuación (18)),
  se obtiene que la brecha del markup $\hat{\mu}_t$ es proporcional y opuesta a
  $\tilde{y}_t$ (Ecuación (21)), paso clave para reescribir la CPNK en su forma
  estándar.
\end{itemize}

% --------------------------------------------------
% Ecuación (21): Brecha de markup y brecha del producto
% --------------------------------------------------

\subsection*{Ecuación (21): Brecha del \textit{markup} y brecha del producto}

La Ecuación (21) vincula de forma directa la \emph{brecha del markup} con la
\emph{brecha del producto}. Es el paso final que permite pasar de una Curva de
Phillips formulada en términos de márgenes a una formulada en términos de
output gap.

\begin{equation}
  \hat{\mu}_t
  \;=\;
  -\left(
    \sigma + \phi + \frac{\alpha}{1-\alpha}
  \right)
  (y_t - y_{n,t})
  \;=\;
  -\left(
    \sigma + \phi + \frac{\alpha}{1-\alpha}
  \right)\tilde{y}_t
  \tag{21}
\end{equation}

donde hemos definido la \emph{brecha del producto} como
$\tilde{y}_t \equiv y_t - y_{n,t}$.

%---------------------------------------------
% Tabla de símbolos relevantes (Ecuación 21)
%---------------------------------------------

\begin{table}[h!]
\centering
\begin{tabular}{p{0.18\textwidth} p{0.20\textwidth} p{0.52\textwidth}}
\hline
\textbf{Símbolo} & \textbf{Tipo} & \textbf{Descripción} \\
\hline
$\hat{\mu}_t$ & Variable (log) &
Brecha del \textit{markup}: diferencia entre el markup promedio actual 
y el markup deseado (sin fricciones), $\hat{\mu}_t \equiv \mu_t - \mu$. \\[4pt]

$y_t$ & Variable (log) &
Producto agregado actual (logaritmo del output). \\[4pt]

$y_{n,t}$ & Variable (log) &
Nivel natural de producto: producción de equilibrio bajo precios flexibles. \\[4pt]

$\tilde{y}_t$ & Variable (log) &
Brecha del producto: $\tilde{y}_t \equiv y_t - y_{n,t}$. Mide el exceso de demanda
o capacidad ociosa. \\[4pt]

$\sigma$ & Parámetro &
Inverso de la elasticidad de sustitución intertemporal del consumo. \\[4pt]

$\phi$ & Parámetro &
Inverso de la elasticidad de la oferta de trabajo. \\[4pt]

$\alpha$ & Parámetro &
Parámetro asociado a los rendimientos decrecientes del trabajo en la función
de producción. \\[4pt]
\hline
\end{tabular}
\end{table}

%---------------------------------------------
% Origen de la Ecuación (21)
%---------------------------------------------

\subsubsection*{Origen de la Ecuación (21)}

La Ecuación (21) se obtiene restando la relación del markup bajo precios
flexibles (Ecuación (19)) de la relación general del markup (Ecuación (18)):

\begin{align*}
  \mu_t
  &= -\Bigl(\sigma + \phi + \frac{\alpha}{1-\alpha}\Bigr) y_t
     + \Bigl(\frac{1+\phi}{1-\alpha}\Bigr)a_t
     + \log(1-\alpha), \tag{18}\\[4pt]
  \mu
  &= -\Bigl(\sigma + \phi + \frac{\alpha}{1-\alpha}\Bigr) y_{n,t}
     + \Bigl(\frac{1+\phi}{1-\alpha}\Bigr)a_t
     + \log(1-\alpha). \tag{19}
\end{align*}

Restando (19) de (18) miembro a miembro:

\begin{equation*}
  \mu_t - \mu
  \;=\;
  -\Bigl(\sigma + \phi + \frac{\alpha}{1-\alpha}\Bigr)
   (y_t - y_{n,t}),
\end{equation*}

es decir,

\begin{equation*}
  \hat{\mu}_t
  \;=\;
  -\Bigl(\sigma + \phi + \frac{\alpha}{1-\alpha}\Bigr)\tilde{y}_t,
\end{equation*}

lo cual coincide con la Ecuación (21).

El coeficiente
\(
  \sigma + \phi + \frac{\alpha}{1-\alpha}
\)
es estrictamente positivo, por lo que la relación entre $\hat{\mu}_t$ y
$\tilde{y}_t$ es negativa y proporcional.

%---------------------------------------------
% Interpretación económica
%---------------------------------------------

\subsubsection*{Interpretación económica}

\begin{itemize}
  \item \textbf{Relación inversa entre brecha del producto y margen.}  
  Cuando el producto actual $y_t$ se sitúa por encima de su nivel natural
  $y_{n,t}$ (esto es, $\tilde{y}_t > 0$), la demanda de trabajo aumenta, lo que
  eleva el salario real y el costo marginal. Dado que los precios no se ajustan
  perfectamente, el markup promedio cae por debajo de su nivel deseado:
  $\hat{\mu}_t < 0$.

  \item \textbf{Presión de costos como origen de la inflación.}  
  En la Curva de Phillips Nuevo Keynesiana formulada en términos de markup
  (Ecuación (17)),
  \[
    \pi_t = \beta E_t\{\pi_{t+1}\} - \lambda \hat{\mu}_t,
  \]
  la inflación es impulsada por $-\lambda \hat{\mu}_t$. Usando (21),
  una brecha positiva del producto ($\tilde{y}_t>0$) implica un
  $\hat{\mu}_t<0$, lo que genera un efecto positivo sobre la inflación, dando
  lugar a la forma estándar:

  \[
    \pi_t = \beta E_t\{\pi_{t+1}\} + \kappa \tilde{y}_t,
  \]

  donde
  \(
    \kappa \equiv \lambda\left( \sigma + \phi + \frac{\alpha}{1-\alpha} \right) > 0.
  \)

  \item \textbf{Cierre del lado de la oferta.}  
  La Ecuación (21) cierra el bloque de oferta del modelo: conecta el margen
  de beneficio con la brecha del producto, permitiendo expresar la Curva
  de Phillips únicamente en términos de variables agregadas observables
  (inflación y brecha del producto) y parámetros estructurales.
\end{itemize}


% --------------------------------------------------
% Ecuación (22): Curva de Phillips Nuevo Keynesiana (forma canónica)
% --------------------------------------------------

\subsection*{Ecuación (22): Curva de Phillips Nuevo Keynesiana (NKPC)}

La Ecuación (22) recoge la versión canónica de la Curva de Phillips
Nuevo Keynesiana (CPNK), que relaciona la inflación actual con la
inflación esperada futura y la brecha del producto:

\begin{equation}
  \pi_t
  \;=\;
  \beta E_t\{\pi_{t+1}\}
  +
  \kappa \tilde{y}_t
  \tag{22}
\end{equation}

donde $\tilde{y}_t \equiv y_t - y_{n,t}$ es la brecha del producto.

%---------------------------------------------
% Tabla de símbolos relevantes (Ecuación 22)
%---------------------------------------------

\begin{table}[h!]
\centering
\begin{tabular}{p{0.18\textwidth} p{0.20\textwidth} p{0.52\textwidth}}
\hline
\textbf{Símbolo} & \textbf{Tipo} & \textbf{Descripción} \\
\hline
$\pi_t$ & Variable &
Inflación actual: $\pi_t \equiv p_t - p_{t-1}$. \\[4pt]

$E_t\{\pi_{t+1}\}$ & Variable (expectativa) &
Inflación esperada para el período $t+1$, condicional en la información
disponible en $t$. \\[4pt]

$\tilde{y}_t$ & Variable (log) &
Brecha del producto: $\tilde{y}_t \equiv y_t - y_{n,t}$, diferencia entre
el producto efectivo y el nivel natural (bajo precios flexibles). \\[4pt]

$\beta$ & Parámetro &
Factor de descuento intertemporal de los hogares, $0<\beta<1$. \\[4pt]

$\kappa$ & Parámetro &
Pendiente de la Curva de Phillips Nuevo Keynesiana: mide la sensibilidad
de la inflación a la brecha del producto. \\[4pt]
\hline
\end{tabular}
\end{table}

El parámetro $\kappa$ recoge tanto la rigidez de precios como la estructura
real del modelo. Se define como:

\begin{equation}
  \kappa
  \;\equiv\;
  \lambda\left(
    \sigma + \phi + \frac{\alpha}{1-\alpha}
  \right),
  \qquad
  \lambda \equiv
  \frac{(1-\theta)(1-\beta\theta)}{\theta}\,\epsilon
\end{equation}

donde $\theta$ es el parámetro de Calvo (probabilidad de no reajustar
precios), $\epsilon$ la elasticidad de sustitución entre variedades,
$\sigma$ el inverso de la elasticidad de sustitución intertemporal,
$\phi$ el inverso de la elasticidad de oferta de trabajo y $\alpha$
el parámetro de rendimientos decrecientes del trabajo.

%---------------------------------------------
% Origen de la Ecuación (22)
%---------------------------------------------

\subsubsection*{Origen de la Ecuación (22)}

La Ecuación (22) resulta de combinar:

\begin{itemize}
  \item La Curva de Phillips en términos de brecha de \textit{markup}
  (Ecuación (17)):
  \[
    \pi_t
    =
    \beta E_t\{\pi_{t+1}\}
    -
    \lambda \hat{\mu}_t,
    \tag{17}
  \]
  donde $\hat{\mu}_t \equiv \mu_t - \mu$.

  \item La relación entre la brecha de \textit{markup} y la brecha del
  producto (Ecuación (21)):
  \[
    \hat{\mu}_t
    =
    -\left(
      \sigma + \phi + \frac{\alpha}{1-\alpha}
    \right)\tilde{y}_t.
    \tag{21}
  \]
\end{itemize}

Sustituyendo (21) en (17):

\begin{align*}
  \pi_t
  &= \beta E_t\{\pi_{t+1}\}
     - \lambda \hat{\mu}_t \\[4pt]
  &= \beta E_t\{\pi_{t+1}\}
     - \lambda
     \Bigl[
       -\Bigl(
         \sigma + \phi + \frac{\alpha}{1-\alpha}
       \Bigr)\tilde{y}_t
     \Bigr] \\[4pt]
  &= \beta E_t\{\pi_{t+1}\}
     +
     \underbrace{
       \lambda\left(
         \sigma + \phi + \frac{\alpha}{1-\alpha}
       \right)
     }_{\kappa}
     \tilde{y}_t,
\end{align*}

lo cual coincide con la Ecuación (22).

%---------------------------------------------
% Interpretación económica
%---------------------------------------------

\subsubsection*{Interpretación económica}

\begin{itemize}
  \item \textbf{Carácter prospectivo de la inflación.}  
  La inflación actual depende positivamente de la inflación esperada
  futura, ponderada por el factor de descuento $\beta$. Esto refleja que
  la fijación de precios en el esquema de Calvo es \emph{forward-looking}:
  las empresas eligen su precio óptimo anticipando las condiciones futuras,
  porque su precio puede permanecer fijo durante varios períodos.

  \item \textbf{Inflación impulsada por la brecha del producto.}  
  La brecha del producto $\tilde{y}_t$ actúa como el \emph{forzador real}
  de la inflación. Cuando $\tilde{y}_t>0$ (demanda agregada por encima del
  nivel natural de producción), la mayor utilización de factores empuja
  al alza los costos marginales, reduce los markups y lleva a las empresas
  que pueden ajustar precios a aumentarlos, generando inflación positiva.

  \item \textbf{Pendiente $\kappa$ y rigidez nominal.}  
  El parámetro $\kappa$ aumenta cuando los precios son más flexibles
  (menor $\theta$), cuando la oferta laboral es más sensible (menor $\phi$),
  cuando el consumo reacciona más al tipo de interés real (menor $\sigma$) o
  cuando la demanda es más elástica ($\epsilon$ mayor). Una mayor $\kappa$
  implica que pequeños cambios en la brecha del producto generan
  variaciones relativamente grandes en la inflación.

  \item \textbf{Bloque de oferta en el modelo NK.}  
  Junto con la Ecuación IS Dinámica (DIS), la NKPC de (22) constituye
  el corazón del modelo básico Nuevo Keynesiano: la DIS describe el
  comportamiento de la demanda agregada; la Ecuación (22) describe cómo
  la inflación responde a la actividad real y a las expectativas
  futuras de inflación.
\end{itemize}

% --------------------------------------------------
% Ecuación (22): Curva de Phillips Nuevo Keynesiana (forma canónica)
% --------------------------------------------------

\subsection*{Ecuación (22): Curva de Phillips Nuevo Keynesiana (NKPC)}

La Ecuación (22) recoge la versión canónica de la Curva de Phillips
Nuevo Keynesiana (CPNK), que relaciona la inflación actual con la
inflación esperada futura y la brecha del producto:

\begin{equation}
  \pi_t
  \;=\;
  \beta E_t\{\pi_{t+1}\}
  +
  \kappa \tilde{y}_t
  \tag{22}
\end{equation}

donde $\tilde{y}_t \equiv y_t - y_{n,t}$ es la brecha del producto.

%---------------------------------------------
% Tabla de símbolos relevantes (Ecuación 22)
%---------------------------------------------

\begin{table}[h!]
\centering
\begin{tabular}{p{0.18\textwidth} p{0.20\textwidth} p{0.52\textwidth}}
\hline
\textbf{Símbolo} & \textbf{Tipo} & \textbf{Descripción} \\
\hline
$\pi_t$ & Variable &
Inflación actual: $\pi_t \equiv p_t - p_{t-1}$. \\[4pt]

$E_t\{\pi_{t+1}\}$ & Variable (expectativa) &
Inflación esperada para el período $t+1$, condicional en la información
disponible en $t$. \\[4pt]

$\tilde{y}_t$ & Variable (log) &
Brecha del producto: $\tilde{y}_t \equiv y_t - y_{n,t}$, diferencia entre
el producto efectivo y el nivel natural (bajo precios flexibles). \\[4pt]

$\beta$ & Parámetro &
Factor de descuento intertemporal de los hogares, $0<\beta<1$. \\[4pt]

$\kappa$ & Parámetro &
Pendiente de la Curva de Phillips Nuevo Keynesiana: mide la sensibilidad
de la inflación a la brecha del producto. \\[4pt]
\hline
\end{tabular}
\end{table}

El parámetro $\kappa$ recoge tanto la rigidez de precios como la estructura
real del modelo. Se define como:

\begin{equation}
  \kappa
  \;\equiv\;
  \lambda\left(
    \sigma + \phi + \frac{\alpha}{1-\alpha}
  \right),
  \qquad
  \lambda \equiv
  \frac{(1-\theta)(1-\beta\theta)}{\theta}\,\epsilon
\end{equation}

donde $\theta$ es el parámetro de Calvo (probabilidad de no reajustar
precios), $\epsilon$ la elasticidad de sustitución entre variedades,
$\sigma$ el inverso de la elasticidad de sustitución intertemporal,
$\phi$ el inverso de la elasticidad de oferta de trabajo y $\alpha$
el parámetro de rendimientos decrecientes del trabajo.

%---------------------------------------------
% Origen de la Ecuación (22)
%---------------------------------------------

\subsubsection*{Origen de la Ecuación (22)}

La Ecuación (22) resulta de combinar:

\begin{itemize}
  \item La Curva de Phillips en términos de brecha de \textit{markup}
  (Ecuación (17)):
  \[
    \pi_t
    =
    \beta E_t\{\pi_{t+1}\}
    -
    \lambda \hat{\mu}_t,
    \tag{17}
  \]
  donde $\hat{\mu}_t \equiv \mu_t - \mu$.

  \item La relación entre la brecha de \textit{markup} y la brecha del
  producto (Ecuación (21)):
  \[
    \hat{\mu}_t
    =
    -\left(
      \sigma + \phi + \frac{\alpha}{1-\alpha}
    \right)\tilde{y}_t.
    \tag{21}
  \]
\end{itemize}

Sustituyendo (21) en (17):

\begin{align*}
  \pi_t
  &= \beta E_t\{\pi_{t+1}\}
     - \lambda \hat{\mu}_t \\[4pt]
  &= \beta E_t\{\pi_{t+1}\}
     - \lambda
     \Bigl[
       -\Bigl(
         \sigma + \phi + \frac{\alpha}{1-\alpha}
       \Bigr)\tilde{y}_t
     \Bigr] \\[4pt]
  &= \beta E_t\{\pi_{t+1}\}
     +
     \underbrace{
       \lambda\left(
         \sigma + \phi + \frac{\alpha}{1-\alpha}
       \right)
     }_{\kappa}
     \tilde{y}_t,
\end{align*}

lo cual coincide con la Ecuación (22).

%---------------------------------------------
% Interpretación económica
%---------------------------------------------

\subsubsection*{Interpretación económica}

\begin{itemize}
  \item \textbf{Carácter prospectivo de la inflación.}  
  La inflación actual depende positivamente de la inflación esperada
  futura, ponderada por el factor de descuento $\beta$. Esto refleja que
  la fijación de precios en el esquema de Calvo es \emph{forward-looking}:
  las empresas eligen su precio óptimo anticipando las condiciones futuras,
  porque su precio puede permanecer fijo durante varios períodos.

  \item \textbf{Inflación impulsada por la brecha del producto.}  
  La brecha del producto $\tilde{y}_t$ actúa como el \emph{forzador real}
  de la inflación. Cuando $\tilde{y}_t>0$ (demanda agregada por encima del
  nivel natural de producción), la mayor utilización de factores empuja
  al alza los costos marginales, reduce los markups y lleva a las empresas
  que pueden ajustar precios a aumentarlos, generando inflación positiva.

  \item \textbf{Pendiente $\kappa$ y rigidez nominal.}  
  El parámetro $\kappa$ aumenta cuando los precios son más flexibles
  (menor $\theta$), cuando la oferta laboral es más sensible (menor $\phi$),
  cuando el consumo reacciona más al tipo de interés real (menor $\sigma$) o
  cuando la demanda es más elástica ($\epsilon$ mayor). Una mayor $\kappa$
  implica que pequeños cambios en la brecha del producto generan
  variaciones relativamente grandes en la inflación.

  \item \textbf{Bloque de oferta en el modelo NK.}  
  Junto con la Ecuación IS Dinámica (DIS), la NKPC de (22) constituye
  el corazón del modelo básico Nuevo Keynesiano: la DIS describe el
  comportamiento de la demanda agregada; la Ecuación (22) describe cómo
  la inflación responde a la actividad real y a las expectativas
  futuras de inflación.
\end{itemize}

% --------------------------------------------------
% Ecuación (23): Ecuación IS Dinámica en términos de la brecha del producto
% --------------------------------------------------

\subsection*{Ecuación (23): Ecuación IS Dinámica (DIS) en brecha del producto}

La Ecuación (23) expresa la Ecuación IS Dinámica en términos de la
\emph{brecha del producto} $\tilde{y}_t \equiv y_t - y_{n,t}$. Describe
la dinámica de la demanda agregada relativa a su nivel natural:

\begin{equation}
  \tilde{y}_t
  \;=\;
  -\frac{1}{\sigma}
  \bigl(
    i_t - E_t\{\pi_{t+1}\} - r_{n,t}
  \bigr)
  \;+\;
  E_t\{\tilde{y}_{t+1}\}
  \tag{23}
\end{equation}

%---------------------------------------------
% Tabla de símbolos relevantes (Ecuación 23)
%---------------------------------------------

\begin{table}[h!]
\centering
\begin{tabular}{p{0.19\textwidth} p{0.20\textwidth} p{0.51\textwidth}}
\hline
\textbf{Símbolo} & \textbf{Tipo} & \textbf{Descripción} \\
\hline
$\tilde{y}_t$ & Variable (log) &
Brecha del producto: $\tilde{y}_t \equiv y_t - y_{n,t}$, diferencia entre
el producto efectivo y el nivel natural bajo precios flexibles. \\[4pt]

$E_t\{\tilde{y}_{t+1}\}$ & Variable (expectativa) &
Brecha del producto esperada para el período $t+1$, condicional en la
información disponible en $t$. \\[4pt]

$i_t$ & Variable &
Tasa de interés nominal de corto plazo fijada por la autoridad
monetaria. \\[4pt]

$E_t\{\pi_{t+1}\}$ & Variable (expectativa) &
Inflación esperada para el período $t+1$; define junto con $i_t$ la
tasa de interés real esperada. \\[4pt]

$r_{n,t}$ & Variable &
\emph{Tasa natural de interés} (real): tasa que prevalecería en el
equilibrio con precios flexibles, dada la trayectoria de los shocks
reales. \\[4pt]

$\sigma$ & Parámetro &
Inverso de la elasticidad de sustitución intertemporal del consumo;
controla la sensibilidad de la brecha del producto a desviaciones de
la tasa real respecto a su nivel natural. \\[4pt]
\hline
\end{tabular}
\end{table}

La tasa de interés real \emph{ex ante} se define como
$r_t \equiv i_t - E_t\{\pi_{t+1}\}$. La Ecuación (23) puede escribirse
equivalentemente como:
\[
  \tilde{y}_t
  \;=\;
  -\frac{1}{\sigma}\bigl(r_t - r_{n,t}\bigr)
  +
  E_t\{\tilde{y}_{t+1}\}.
\]

%---------------------------------------------
% Origen de la Ecuación (23)
%---------------------------------------------

\subsubsection*{Origen de la Ecuación (23)}

La Ecuación (23) se obtiene a partir de la Ecuación IS Dinámica en
niveles (Ecuación (13)) y de la definición del Nivel Natural de
Output $y_{n,t}$ y de la Tasa Natural de Interés $r_{n,t}$.

\begin{itemize}
  \item \textbf{Ecuación IS Dinámica en niveles (Ecuación (13)):}
  \[
    y_t
    =
    E_t\{y_{t+1}\}
    -
    \frac{1}{\sigma}
    \bigl(
      i_t - E_t\{\pi_{t+1}\} - \rho
    \bigr)
    +
    \frac{1}{\sigma}(1-\rho_z)z_t.
    \tag{13}
  \]

  \item \textbf{Nivel Natural de Output y tasa natural de interés.}  
  El Nivel Natural de Output $y_{n,t}$ se determina por el equilibrio
  de precios flexibles (Ecs. (19)–(20)). La dinámica asociada define
  una \emph{tasa natural de interés} $r_{n,t}$ que resume los efectos de
  los shocks reales (preferencias $z_t$ y tecnología $a_t$):
  \[
    r_{n,t}
    =
    \rho
    -
    \sigma (1-\rho_a)\psi_{ya}a_t
    +
    (1-\rho_z)z_t.
    \tag{24}
  \]

  \item \textbf{Paso a brechas.}  
  Al restar de (13) la expresión análoga para el Nivel Natural de
  Output (obtenida imponiendo precios flexibles y sustituyendo $r_{n,t}$),
  y usando $\tilde{y}_t \equiv y_t - y_{n,t}$, se obtiene:
  \[
    \tilde{y}_t
    =
    -\frac{1}{\sigma}\bigl(r_t - r_{n,t}\bigr)
    +
    E_t\{\tilde{y}_{t+1}\},
  \]
  que coincide con la Ecuación (23).
\end{itemize}

%---------------------------------------------
% Interpretación económica
%---------------------------------------------

\subsubsection*{Interpretación económica}

\begin{itemize}
  \item \textbf{Demanda agregada en términos relativos.}  
  La Ecuación (23) describe la dinámica de la demanda agregada medida en
  relación con su potencial. La brecha del producto actual
  $\tilde{y}_t$ depende de la brecha esperada futura
  $E_t\{\tilde{y}_{t+1}\}$ y de la posición de la tasa real
  $r_t$ en relación con la tasa natural $r_{n,t}$.

  \item \textbf{Papel de la política monetaria.}  
  La autoridad monetaria influye sobre la demanda agregada
  exclusivamente a través del \emph{desalineamiento} entre la tasa
  real que induce $r_t = i_t - E_t\{\pi_{t+1}\}$ y la tasa natural
  $r_{n,t}$:
  \begin{itemize}
    \item Si $r_t > r_{n,t}$, el término
    $-\frac{1}{\sigma}(r_t - r_{n,t})$ es negativo y la brecha del
    producto se contrae ($\tilde{y}_t < 0$).
    \item Si $r_t < r_{n,t}$, la política monetaria es expansiva
    en términos relativos y la brecha del producto tiende a ser positiva.
  \end{itemize}

  \item \textbf{Carácter prospectivo de la demanda.}  
  La presencia de $E_t\{\tilde{y}_{t+1}\}$ enfatiza que las decisiones
  de gasto de los hogares son \emph{forward-looking}: el nivel actual
  de demanda no solo depende de la tasa real de hoy, sino también de las
  expectativas sobre la actividad futura.

  \item \textbf{Solución hacia adelante.}  
  Iterando la Ecuación (23) hacia adelante y suponiendo que
  $\lim_{T\to\infty}E_t\{\tilde{y}_{t+T}\}=0$, se obtiene:
  \begin{equation}
    \tilde{y}_t
    =
    -\frac{1}{\sigma}
    \sum_{k=0}^{\infty}
    E_t\{r_{t+k} - r_{n,t+k}\},
    \tag{25}
  \end{equation}
  lo cual muestra que la brecha del producto actual es proporcional
  (en sentido inverso) a la suma descontada de las desviaciones
  presentes y futuras entre la tasa real y su nivel natural.
\end{itemize}

% --------------------------------------------------
% Ecuación (24): Tasa Natural de Interés
% --------------------------------------------------

\subsection*{Ecuación (24): Tasa Natural de Interés $r_{n,t}$}

La Ecuación (24) define la \emph{tasa natural de interés real}
$r_{n,t}$, es decir, la tasa que prevalecería en la economía si los
precios fueran completamente flexibles y el producto se ubicara en su
nivel natural $y_{n,t}$:

\begin{equation}
  r_{n,t}
  \;=\;
  \rho
  \;-\;
  \sigma (1 - \rho_a)\,\psi_{ya}\,a_t
  \;+\;
  (1 - \rho_z)\,z_t
  \tag{24}
\end{equation}

%---------------------------------------------
% Tabla de símbolos relevantes (Ecuación 24)
%---------------------------------------------

\begin{table}[h!]
\centering
\begin{tabular}{p{0.19\textwidth} p{0.20\textwidth} p{0.51\textwidth}}
\hline
\textbf{Símbolo} & \textbf{Tipo} & \textbf{Descripción} \\
\hline
$r_{n,t}$ & Variable &
Tasa natural de interés real en $t$: tasa real consistente con
$\tilde{y}_t = 0$ (brecha del producto nula) bajo precios flexibles. \\[4pt]

$\rho$ & Parámetro &
Tasa de descuento de los hogares ($\rho \equiv -\log\beta$); componente
constante de la tasa natural. \\[4pt]

$a_t$ & Variable (log) &
Shock tecnológico: logaritmo del nivel de productividad agregada. \\[4pt]

$z_t$ & Variable (log) &
Shock de preferencias (o de impaciencia): logaritmo del proceso que
mueve la utilidad marginal intertemporal. \\[4pt]

$\sigma$ & Parámetro &
Inverso de la elasticidad de sustitución intertemporal del consumo. \\[4pt]

$\rho_a$ & Parámetro &
Coeficiente AR(1) del proceso tecnológico
($a_{t+1} = \rho_a a_t + \varepsilon_{a,t+1}$). \\[4pt]

$\rho_z$ & Parámetro &
Coeficiente AR(1) del shock de preferencias
($z_{t+1} = \rho_z z_t + \varepsilon_{z,t+1}$). \\[4pt]

$\psi_{ya}$ & Parámetro &
Sensibilidad del Nivel Natural de Output al shock tecnológico:
$y_{n,t} = \psi_{ya} a_t + \psi_y$. \\[4pt]
\hline
\end{tabular}
\end{table}

%---------------------------------------------
% Origen de la Ecuación (24)
%---------------------------------------------

\subsubsection*{Origen de la Ecuación (24)}

El punto de partida es la Ecuación IS Dinámica en niveles (Ecuación (13)):

\begin{equation}
  y_t
  =
  E_t\{y_{t+1}\}
  -
  \frac{1}{\sigma}
  \bigl(
    i_t - E_t\{\pi_{t+1}\} - \rho
  \bigr)
  +
  \frac{1}{\sigma}(1 - \rho_z)z_t.
  \tag{13}
\end{equation}

\begin{itemize}
  \item \textbf{Equilibrio con precios flexibles.}  
  Bajo precios flexibles el producto se iguala a su nivel natural,
  $y_t = y_{n,t}$, y la tasa de interés real ex ante
  $r_t \equiv i_t - E_t\{\pi_{t+1}\}$ coincide con la tasa natural
  $r_{n,t}$. Sustituyendo en (13):

  \[
    y_{n,t}
    =
    E_t\{y_{n,t+1}\}
    -
    \frac{1}{\sigma}
    \bigl(r_{n,t} - \rho\bigr)
    +
    \frac{1}{\sigma}(1 - \rho_z)z_t.
  \]

  Despejando $r_{n,t}$:

  \[
    r_{n,t}
    =
    \rho
    +
    \sigma\,E_t\{y_{n,t+1} - y_{n,t}\}
    +
    (1 - \rho_z)z_t.
  \]

  \item \textbf{Dinámica del Output Natural.}  
  El Nivel Natural de Output viene dado por (Ecuación (20)):

  \[
    y_{n,t} = \psi_{ya} a_t + \psi_y,
    \tag{20}
  \]

  donde $a_t$ sigue un proceso AR(1):

  \[
    a_{t+1} = \rho_a a_t + \varepsilon_{a,t+1}.
    \tag{6}
  \]

  Tomando expectativas condicionales:

  \[
    E_t\{a_{t+1}\} = \rho_a a_t
    \quad\Rightarrow\quad
    E_t\{y_{n,t+1}\} - y_{n,t}
    =
    \psi_{ya}(\rho_a a_t - a_t)
    =
    \psi_{ya}(\rho_a - 1)a_t.
  \]

  \item \textbf{Sustitución final.}  
  Sustituyendo en la expresión de $r_{n,t}$:

  \[
    r_{n,t}
    =
    \rho
    +
    \sigma\,\psi_{ya}(\rho_a - 1)a_t
    +
    (1 - \rho_z)z_t
    =
    \rho
    -
    \sigma (1 - \rho_a)\psi_{ya}a_t
    +
    (1 - \rho_z)z_t,
  \]

  que coincide con la Ecuación (24).
\end{itemize}

%---------------------------------------------
% Interpretación económica
%---------------------------------------------

\subsubsection*{Interpretación económica}

\begin{itemize}
  \item \textbf{Tasa de referencia real de la economía.}  
  $r_{n,t}$ es la tasa real consistente con una brecha del producto
  nula ($\tilde{y}_t = 0$) cuando los precios son flexibles. Es, por
  tanto, la tasa de referencia frente a la cual se evalúa la postura
  de la política monetaria.

  \item \textbf{Sensibilidad a shocks de preferencias ($z_t$).}  
  Un aumento en $z_t$ (mayor impaciencia o preferencia por el consumo
  presente) presiona al alza la demanda actual. Para mantener
  $\tilde{y}_t = 0$, la tasa natural debe subir: el término
  $(1 - \rho_z)z_t$ incrementa $r_{n,t}$.

  \item \textbf{Sensibilidad a shocks tecnológicos ($a_t$).}  
  Una mejora tecnológica $a_t$ eleva el nivel futuro de producción
  natural. Esto induce a los hogares a querer trasladar consumo hacia
  el futuro. Para compatibilizar decisiones óptimas de consumo y
  producción, la tasa natural debe caer. El término
  $-\sigma (1 - \rho_a)\psi_{ya}a_t$ refleja este efecto (coeficiente
  negativo ante un aumento de $a_t$).

  \item \textbf{Papel en la Ecuación IS en brechas.}  
  En la Ecuación IS en términos de brecha del producto (Ecuación (23)):

  \[
    \tilde{y}_t
    =
    -\frac{1}{\sigma}\bigl(r_t - r_{n,t}\bigr)
    +
    E_t\{\tilde{y}_{t+1}\},
    \tag{23}
  \]

  la brecha del producto se abre únicamente si la tasa real inducida
  por la política monetaria $r_t = i_t - E_t\{\pi_{t+1}\}$ se desvía
  de la tasa natural $r_{n,t}$. Si $r_t = r_{n,t}$, la política
  monetaria es neutral respecto a la brecha del producto.

  \item \textbf{Desafío para el Banco Central.}  
  Dado que $r_{n,t}$ depende de shocks reales no observables
  directamente ($a_t$, $z_t$), el banco central enfrenta el problema
  de estimar esta tasa natural. Errores persistentes en esa estimación
  pueden generar brechas del producto sistemáticas, con consecuencias
  para la inflación a través de la Curva de Phillips Nuevo Keynesiana.
\end{itemize}

% --------------------------------------------------
% Ecuación (24): Tasa Natural de Interés
% --------------------------------------------------

\subsection*{Ecuación (24): Tasa Natural de Interés $r_{n,t}$}

La Ecuación (24) define la \emph{tasa natural de interés real}
$r_{n,t}$, es decir, la tasa que prevalecería en la economía si los
precios fueran completamente flexibles y el producto se ubicara en su
nivel natural $y_{n,t}$:

\begin{equation}
  r_{n,t}
  \;=\;
  \rho
  \;-\;
  \sigma (1 - \rho_a)\,\psi_{ya}\,a_t
  \;+\;
  (1 - \rho_z)\,z_t
  \tag{24}
\end{equation}

%---------------------------------------------
% Tabla de símbolos relevantes (Ecuación 24)
%---------------------------------------------

\begin{table}[h!]
\centering
\begin{tabular}{p{0.19\textwidth} p{0.20\textwidth} p{0.51\textwidth}}
\hline
\textbf{Símbolo} & \textbf{Tipo} & \textbf{Descripción} \\
\hline
$r_{n,t}$ & Variable &
Tasa natural de interés real en $t$: tasa real consistente con
$\tilde{y}_t = 0$ (brecha del producto nula) bajo precios flexibles. \\[4pt]

$\rho$ & Parámetro &
Tasa de descuento de los hogares ($\rho \equiv -\log\beta$); componente
constante de la tasa natural. \\[4pt]

$a_t$ & Variable (log) &
Shock tecnológico: logaritmo del nivel de productividad agregada. \\[4pt]

$z_t$ & Variable (log) &
Shock de preferencias (o de impaciencia): logaritmo del proceso que
mueve la utilidad marginal intertemporal. \\[4pt]

$\sigma$ & Parámetro &
Inverso de la elasticidad de sustitución intertemporal del consumo. \\[4pt]

$\rho_a$ & Parámetro &
Coeficiente AR(1) del proceso tecnológico
($a_{t+1} = \rho_a a_t + \varepsilon_{a,t+1}$). \\[4pt]

$\rho_z$ & Parámetro &
Coeficiente AR(1) del shock de preferencias
($z_{t+1} = \rho_z z_t + \varepsilon_{z,t+1}$). \\[4pt]

$\psi_{ya}$ & Parámetro &
Sensibilidad del Nivel Natural de Output al shock tecnológico:
$y_{n,t} = \psi_{ya} a_t + \psi_y$. \\[4pt]
\hline
\end{tabular}
\end{table}

%---------------------------------------------
% Origen de la Ecuación (24)
%---------------------------------------------

\subsubsection*{Origen de la Ecuación (24)}

El punto de partida es la Ecuación IS Dinámica en niveles (Ecuación (13)):

\begin{equation}
  y_t
  =
  E_t\{y_{t+1}\}
  -
  \frac{1}{\sigma}
  \bigl(
    i_t - E_t\{\pi_{t+1}\} - \rho
  \bigr)
  +
  \frac{1}{\sigma}(1 - \rho_z)z_t.
  \tag{13}
\end{equation}

\begin{itemize}
  \item \textbf{Equilibrio con precios flexibles.}  
  Bajo precios flexibles el producto se iguala a su nivel natural,
  $y_t = y_{n,t}$, y la tasa de interés real ex ante
  $r_t \equiv i_t - E_t\{\pi_{t+1}\}$ coincide con la tasa natural
  $r_{n,t}$. Sustituyendo en (13):

  \[
    y_{n,t}
    =
    E_t\{y_{n,t+1}\}
    -
    \frac{1}{\sigma}
    \bigl(r_{n,t} - \rho\bigr)
    +
    \frac{1}{\sigma}(1 - \rho_z)z_t.
  \]

  Despejando $r_{n,t}$:

  \[
    r_{n,t}
    =
    \rho
    +
    \sigma\,E_t\{y_{n,t+1} - y_{n,t}\}
    +
    (1 - \rho_z)z_t.
  \]

  \item \textbf{Dinámica del Output Natural.}  
  El Nivel Natural de Output viene dado por (Ecuación (20)):

  \[
    y_{n,t} = \psi_{ya} a_t + \psi_y,
    \tag{20}
  \]

  donde $a_t$ sigue un proceso AR(1):

  \[
    a_{t+1} = \rho_a a_t + \varepsilon_{a,t+1}.
    \tag{6}
  \]

  Tomando expectativas condicionales:

  \[
    E_t\{a_{t+1}\} = \rho_a a_t
    \quad\Rightarrow\quad
    E_t\{y_{n,t+1}\} - y_{n,t}
    =
    \psi_{ya}(\rho_a a_t - a_t)
    =
    \psi_{ya}(\rho_a - 1)a_t.
  \]

  \item \textbf{Sustitución final.}  
  Sustituyendo en la expresión de $r_{n,t}$:

  \[
    r_{n,t}
    =
    \rho
    +
    \sigma\,\psi_{ya}(\rho_a - 1)a_t
    +
    (1 - \rho_z)z_t
    =
    \rho
    -
    \sigma (1 - \rho_a)\psi_{ya}a_t
    +
    (1 - \rho_z)z_t,
  \]

  que coincide con la Ecuación (24).
\end{itemize}

%---------------------------------------------
% Interpretación económica
%---------------------------------------------

\subsubsection*{Interpretación económica}

\begin{itemize}
  \item \textbf{Tasa de referencia real de la economía.}  
  $r_{n,t}$ es la tasa real consistente con una brecha del producto
  nula ($\tilde{y}_t = 0$) cuando los precios son flexibles. Es, por
  tanto, la tasa de referencia frente a la cual se evalúa la postura
  de la política monetaria.

  \item \textbf{Sensibilidad a shocks de preferencias ($z_t$).}  
  Un aumento en $z_t$ (mayor impaciencia o preferencia por el consumo
  presente) presiona al alza la demanda actual. Para mantener
  $\tilde{y}_t = 0$, la tasa natural debe subir: el término
  $(1 - \rho_z)z_t$ incrementa $r_{n,t}$.

  \item \textbf{Sensibilidad a shocks tecnológicos ($a_t$).}  
  Una mejora tecnológica $a_t$ eleva el nivel futuro de producción
  natural. Esto induce a los hogares a querer trasladar consumo hacia
  el futuro. Para compatibilizar decisiones óptimas de consumo y
  producción, la tasa natural debe caer. El término
  $-\sigma (1 - \rho_a)\psi_{ya}a_t$ refleja este efecto (coeficiente
  negativo ante un aumento de $a_t$).

  \item \textbf{Papel en la Ecuación IS en brechas.}  
  En la Ecuación IS en términos de brecha del producto (Ecuación (23)):

  \[
    \tilde{y}_t
    =
    -\frac{1}{\sigma}\bigl(r_t - r_{n,t}\bigr)
    +
    E_t\{\tilde{y}_{t+1}\},
    \tag{23}
  \]

  la brecha del producto se abre únicamente si la tasa real inducida
  por la política monetaria $r_t = i_t - E_t\{\pi_{t+1}\}$ se desvía
  de la tasa natural $r_{n,t}$. Si $r_t = r_{n,t}$, la política
  monetaria es neutral respecto a la brecha del producto.

  \item \textbf{Desafío para el Banco Central.}  
  Dado que $r_{n,t}$ depende de shocks reales no observables
  directamente ($a_t$, $z_t$), el banco central enfrenta el problema
  de estimar esta tasa natural. Errores persistentes en esa estimación
  pueden generar brechas del producto sistemáticas, con consecuencias
  para la inflación a través de la Curva de Phillips Nuevo Keynesiana.
\end{itemize}

% --------------------------------------------------
% Ecuación (25): Solución hacia adelante de la DIS
% --------------------------------------------------

\subsection*{Ecuación (25): Solución hacia adelante de la Ecuación IS Dinámica}

La Ecuación (25) es la solución con visión de futuro
(\textit{forward-looking}) de la Ecuación IS Dinámica en brechas
(Ecuación (23)). Expresa la brecha del producto actual $\tilde{y}_t$
como la suma descontada de las desviaciones esperadas entre la tasa de
interés real y su tasa natural:

\begin{equation}
  \tilde{y}_t
  \;=\;
  - \frac{1}{\sigma}
  \sum_{k=0}^{\infty}
  E_t\!\left\{ r_{t+k} - r_{n,t+k} \right\}
  \tag{25}
\end{equation}

%---------------------------------------------
% Tabla de símbolos relevantes (Ecuación 25)
%---------------------------------------------

\begin{table}[h!]
\centering
\begin{tabular}{p{0.19\textwidth} p{0.20\textwidth} p{0.51\textwidth}}
\hline
\textbf{Símbolo} & \textbf{Tipo} & \textbf{Descripción} \\
\hline
$\tilde{y}_t$ & Variable (log) &
Brecha del producto: $\tilde{y}_t \equiv y_t - y_{n,t}$.
Mide la desviación del producto efectivo respecto al nivel natural. \\[4pt]

$r_{t+k}$ & Variable (log) &
Tasa de interés real en $t+k$:
$r_t \equiv i_t - E_t\{\pi_{t+1}\}$. \\[4pt]

$r_{n,t+k}$ & Variable (log) &
Tasa natural de interés en $t+k$, consistente con
$\tilde{y}_{t+k} = 0$ bajo precios flexibles. \\[4pt]

$\sigma$ & Parámetro &
Inverso de la elasticidad de sustitución intertemporal del consumo;
controla la sensibilidad de la demanda a la tasa real. \\[4pt]

$E_t\{\cdot\}$ & Operador &
Valor esperado condicional a la información disponible en $t$. \\[4pt]
\hline
\end{tabular}
\end{table}

%---------------------------------------------
% Origen de la Ecuación (25)
%---------------------------------------------

\subsubsection*{Origen de la Ecuación (25)}

La Ecuación (25) se deriva resolviendo hacia adelante la Ecuación IS
Dinámica en términos de brecha del producto (Ecuación (23)).

\paragraph{1. Punto de partida: Ecuación IS Dinámica en brechas.}

Partimos de la Ecuación (23):

\begin{equation}
  \tilde{y}_t
  =
  E_t\{\tilde{y}_{t+1}\}
  -
  \frac{1}{\sigma}
  \bigl( r_t - r_{n,t} \bigr),
  \qquad
  r_t \equiv i_t - E_t\{\pi_{t+1}\}.
  \tag{23}
\end{equation}

\paragraph{2. Sustitución hacia adelante.}

Aplicando recursivamente (23):

\[
  \tilde{y}_{t+1}
  =
  E_{t+1}\{\tilde{y}_{t+2}\}
  -
  \frac{1}{\sigma}
  \bigl( r_{t+1} - r_{n,t+1} \bigr),
\]

y tomando expectativas condicionales en $t$:

\[
  E_t\{\tilde{y}_{t+1}\}
  =
  E_t\{\tilde{y}_{t+2}\}
  -
  \frac{1}{\sigma}
  E_t\bigl\{ r_{t+1} - r_{n,t+1} \bigr\}.
\]

Sustituyendo de nuevo en (23):

\[
  \tilde{y}_t
  =
  E_t\{\tilde{y}_{t+2}\}
  -
  \frac{1}{\sigma}
  \bigl( r_t - r_{n,t} \bigr)
  -
  \frac{1}{\sigma}
  E_t\bigl\{ r_{t+1} - r_{n,t+1} \bigr\}.
\]

Repitiendo este procedimiento $T$ veces, obtenemos:

\[
  \tilde{y}_t
  =
  E_t\{\tilde{y}_{t+T}\}
  -
  \frac{1}{\sigma}
  \sum_{k=0}^{T-1}
  E_t\bigl\{ r_{t+k} - r_{n,t+k} \bigr\}.
\]

\paragraph{3. Condición terminal.}

Suponiendo que los efectos de las fricciones nominales se desvanecen en
el largo plazo o, equivalentemente,

\[
  \lim_{T \to \infty} E_t\{\tilde{y}_{t+T}\} = 0,
\]

entonces, tomando el límite cuando $T \to \infty$:

\[
  \tilde{y}_t
  =
  - \frac{1}{\sigma}
  \sum_{k=0}^{\infty}
  E_t\bigl\{ r_{t+k} - r_{n,t+k} \bigr\},
\]

que es precisamente la Ecuación (25).

%---------------------------------------------
% Interpretación económica
%---------------------------------------------

\subsubsection*{Interpretación económica}

\begin{itemize}
  \item \textbf{Brecha del producto como suma de “desviaciones de postura”.}  
  La Ecuación (25) muestra que la brecha del producto actual
  $\tilde{y}_t$ es proporcional (con signo negativo) a la suma de todas
  las desviaciones esperadas futuras entre la tasa real de la economía
  $r_{t+k}$ y la tasa natural $r_{n,t+k}$. Lo relevante no es el nivel
  de $r_t$ en sí, sino su \emph{distancia} respecto a $r_{n,t}$ a lo
  largo del tiempo.

  \item \textbf{Rol de la política monetaria.}  
  Dado que $r_t = i_t - E_t\{\pi_{t+1}\}$ está influido por la política
  monetaria (vía $i_t$), la brecha del producto depende de la senda
  esperada de la política monetaria comparada con la senda esperada de
  la tasa natural. Una política sistemáticamente más contractiva que el
  nivel natural ($r_{t+k} > r_{n,t+k}$ en promedio) genera
  $\tilde{y}_t < 0$ hoy mismo.

  \item \textbf{Carácter prospectivo de la demanda agregada.}  
  La demanda actual no solo responde a la postura presente de la tasa
  de interés real, sino también a cómo los agentes anticipan que será la
  brecha $r_{t+k} - r_{n,t+k}$ en el futuro. Un anuncio creíble de
  tasas reales altas persistentes (en relación con la tasa natural)
  puede contraer la demanda actual, incluso si la tasa actual todavía
  no ha subido completamente.

  \item \textbf{Sensibilidad intertemporal ($\sigma$).}  
  El parámetro $\sigma$ escala la respuesta de la brecha del producto:
  cuanto mayor es $\sigma$ (menor elasticidad intertemporal), menor es
  la reacción de $\tilde{y}_t$ ante una misma secuencia de desviaciones
  $r_{t+k} - r_{n,t+k}$; es decir, los hogares son menos sensibles a
  los incentivos intertemporales.

  \item \textbf{Inserción en el bloque no político.}  
  Combinada con la NKPC (Ecuación (22)) y la tasa natural de interés
  (Ecuación (24)), la Ecuación (25) completa la descripción de cómo las
  fricciones nominales, las expectativas y la política monetaria
  determinan conjuntamente la trayectoria de la brecha del producto y
  de la inflación en el Modelo Básico Nuevo Keynesiano.
\end{itemize}


% --------------------------------------------------
% Ecuación (26): Regla simple de tasa de interés
% --------------------------------------------------

\subsection*{Ecuación (26): Regla simple de tasa de interés nominal}

La Ecuación (26) introduce la política monetaria en el Modelo Básico
Nuevo Keynesiano. Es una \emph{regla simple de tasa de interés} que
describe cómo el banco central fija la tasa nominal de corto plazo
$i_t$ en respuesta a la inflación y al nivel de actividad económica:

\begin{equation}
  i_t \;=\; \rho \;+\; \varphi_{\pi}\,\pi_t \;+\; \varphi_y\,\hat{y}_t \;+\; v_t
  \tag{26}
\end{equation}

donde $\hat{y}_t \equiv y_t - y$ denota la desviación del producto
respecto a su valor de estado estacionario de largo plazo $y$.

%---------------------------------------------
% Tabla de símbolos relevantes (Ecuación 26)
%---------------------------------------------

\begin{table}[h!]
\centering
\begin{tabular}{p{0.19\textwidth} p{0.20\textwidth} p{0.51\textwidth}}
\hline
\textbf{Símbolo} & \textbf{Tipo} & \textbf{Descripción} \\
\hline
$i_t$ & Variable (log) &
Tasa de interés nominal a corto plazo. Es el instrumento de política
monetaria fijado por el banco central. \\[4pt]

$\pi_t$ & Variable (log) &
Tasa de inflación actual. Una de las variables a las que reacciona la
política monetaria. \\[4pt]

$\hat{y}_t$ & Variable (log) &
Desviación del producto respecto a su valor de estado estacionario:
$\hat{y}_t \equiv y_t - y$. Mide el nivel de actividad en relación al
promedio de largo plazo. \\[4pt]

$v_t$ & Variable (log) &
Shock exógeno de política monetaria, típicamente modelado como un
proceso AR(1): $v_t = \rho_v v_{t-1} + \varepsilon_{v,t}$. Captura
movimientos discrecionales o no sistemáticos en la tasa de interés. \\[4pt]

$\rho$ & Parámetro &
Tasa de descuento de los hogares, definida como $\rho \equiv -\log\beta$.
Funciona como intercepto de la regla y se elige de modo consistente con
un estado estacionario de inflación cero. \\[4pt]

$\varphi_{\pi}$ & Parámetro &
Intensidad de la respuesta de la tasa de interés nominal frente a la
inflación. Se asume $\varphi_{\pi} \ge 0$. \\[4pt]

$\varphi_y$ & Parámetro &
Intensidad de la respuesta de la tasa de interés nominal frente a la
actividad económica (desviación del producto). Se asume
$\varphi_y \ge 0$. \\[4pt]
\hline
\end{tabular}
\end{table}

%---------------------------------------------
% Origen y función de la Ecuación (26)
%---------------------------------------------

\subsubsection*{Origen y función de la Ecuación (26)}

\begin{itemize}
  \item \textbf{Cierre del modelo.}  
  Antes de especificar (26), el modelo está descrito por:
  \begin{itemize}
    \item La Curva de Phillips Nuevo Keynesiana (NKPC, Ecuación (22)):
      $\pi_t = \beta E_t\{\pi_{t+1}\} + \kappa \tilde{y}_t$.
    \item La Ecuación IS Dinámica en brechas (Ecuación (23)):
      $\tilde{y}_t = E_t\{\tilde{y}_{t+1}\}
      - \frac{1}{\sigma}(r_t - r_{n,t})$,
      con $r_t \equiv i_t - E_t\{\pi_{t+1}\}$.
  \end{itemize}
  Estas ecuaciones describen el \emph{bloque no político} del modelo:
  determinan cómo se comportan inflación y brecha del producto dadas las
  trayectorias de $r_t$ y $r_{n,t}$. La regla (26) añade el vínculo
  faltante entre el instrumento de política ($i_t$) y las variables
  endógenas (inflación y actividad), cerrando el modelo.

  \item \textbf{Interpretación como regla de Taylor.}  
  La forma de (26) es consistente con la \emph{regla de Taylor} propuesta
  originalmente para describir la conducta de la Reserva Federal:
  \begin{itemize}
    \item La tasa nominal $i_t$ aumenta cuando la inflación $\pi_t$
    supera su objetivo implícito.
    \item La tasa nominal $i_t$ también reacciona positivamente cuando
    el producto está por encima de su valor de referencia de largo
    plazo ($\hat{y}_t > 0$).
  \end{itemize}

  \item \textbf{Intercepto y estado estacionario.}  
  El término constante $\rho$ se elige para que, en ausencia de shocks
  ($v_t = 0$) y con inflación y producto en su estado estacionario
  ($\pi_t = 0$, $\hat{y}_t = 0$), la tasa nominal de equilibrio sea
  compatible con la tasa real de estado estacionario y la paridad
  de Fisher:
  \[
    i = \rho \quad\Rightarrow\quad r = \rho \quad\text{cuando}\quad \pi = 0.
  \]
\end{itemize}

%---------------------------------------------
% Interpretación económica de la regla (26)
%---------------------------------------------

\subsubsection*{Interpretación económica}

\begin{itemize}
  \item \textbf{Respuestas sistemáticas.}  
  Los coeficientes $\varphi_{\pi}$ y $\varphi_y$ describen la respuesta
  sistemática de la política:
  \begin{itemize}
    \item $\varphi_{\pi}$ mide cuánto sube (o baja) la tasa nominal
    cuando la inflación aumenta (o disminuye).
    \item $\varphi_y$ mide cuánto ajusta la tasa nominal al observar
    un diferencial de actividad respecto al nivel de largo plazo.
  \end{itemize}
  Estos parámetros determinan la \emph{agresividad} de la política
  monetaria frente a desviaciones de inflación y producto.

  \item \textbf{Componente discrecional: el shock $v_t$.}  
  El término $v_t$ captura perturbaciones de política monetaria que no
  responden a la regla sistemática (por ejemplo, errores de medición,
  cambios discrecionales, sorpresas de corto plazo). Típicamente se
  modela como:
  \[
    v_t = \rho_v v_{t-1} + \varepsilon_{v,t},
  \]
  donde una innovación positiva $\varepsilon_{v,t}$ se interpreta como
  un \emph{shock contractivo} (eleva $i_t$ para dada $\pi_t$ y $\hat{y}_t$).

  \item \textbf{Relación con la brecha del producto.}  
  Aunque (26) se escribe en términos de $\hat{y}_t \equiv y_t - y$, es
  útil reescribirla en términos de la \emph{brecha del producto}
  $\tilde{y}_t \equiv y_t - y_{n,t}$ para combinarla con (22) y (23):
  \[
    \hat{y}_t
    =
    (y_t - y_{n,t}) + (y_{n,t} - y)
    =
    \tilde{y}_t + \hat{y}^n_t,
  \]
  donde $\hat{y}^n_t \equiv y_{n,t} - y$. Entonces:
  \begin{equation*}
    i_t
    =
    \rho
    +
    \varphi_{\pi}\pi_t
    +
    \varphi_y \tilde{y}_t
    +
    \varphi_y \hat{y}^n_t
    +
    v_t,
  \end{equation*}
  lo que separa la reacción a la brecha del producto ($\tilde{y}_t$) de
  la reacción automática a cambios en el producto natural
  ($\hat{y}^n_t$).

\end{itemize}

%---------------------------------------------
% Condición de unicidad del equilibrio
%---------------------------------------------

\subsubsection*{Condición de unicidad y Principio de Taylor}

Al combinar la NKPC (Ecuación (22)), la DIS en brechas (Ecuación (23))
y la regla de tasa de interés (Ecuación (26)), se obtiene un sistema
lineal en diferencias para $(\tilde{y}_t, \pi_t)$ cuya solución es
localmente única si y sólo si se cumple la condición:

\begin{equation}
  \kappa(\varphi_{\pi} - 1)
  +
  (1 - \beta)\,\varphi_y
  \;>\; 0.
  \tag{28}
\end{equation}

\begin{itemize}
  \item Esta desigualdad recoge una versión log-linealizada del
  \emph{Principio de Taylor}: la política monetaria debe reaccionar lo
  suficientemente fuerte a la inflación (y, en menor medida, al nivel
  de actividad) como para garantizar que el equilibrio sea único y
  estable.
  \item En términos intuitivos, cuando la inflación sube, la regla debe
  implicar un aumento de la tasa nominal $i_t$ que genere un aumento
  más que proporcional de la tasa real $r_t$ (descontando expectativas
  de inflación), de modo que la demanda se contraiga y la inflación
  vuelva a su objetivo.
\end{itemize}

% --------------------------------------------------
% Ecuación (27): Sistema dinámico en (\tilde{y}_t, \pi_t)
% --------------------------------------------------

\subsection*{Ecuación (27): Dinámica conjunta de la brecha del producto y la inflación}

La Ecuación (27) no es una ecuación aislada, sino el \emph{sistema de
ecuaciones en diferencias} que resulta de combinar los tres bloques del
Modelo Básico Nuevo Keynesiano bajo una regla de política monetaria
(Ecuación (26)). Este sistema describe la dinámica de equilibrio de la
brecha del producto y la inflación:

\begin{equation}
  \begin{bmatrix}
    \tilde{y}_t \\
    \pi_t
  \end{bmatrix}
  =
  A_T
  \begin{bmatrix}
    E_t\{\tilde{y}_{t+1}\} \\
    E_t\{\pi_{t+1}\}
  \end{bmatrix}
  +
  B_T\,u_t
  \tag{27}
\end{equation}

donde $\tilde{y}_t \equiv y_t - y_{n,t}$ es la brecha del producto.

%---------------------------------------------
% Tabla de símbolos relevantes (Ecuación 27)
%---------------------------------------------

\begin{table}[h!]
\centering
\begin{tabular}{p{0.18\textwidth} p{0.22\textwidth} p{0.50\textwidth}}
\hline
\textbf{Símbolo} & \textbf{Tipo} & \textbf{Descripción} \\
\hline
$\tilde{y}_t$ & Variable (log) &
Brecha del producto: $\tilde{y}_t \equiv y_t - y_{n,t}$. Mide la
desviación de la actividad respecto a su nivel natural. \\[4pt]

$\pi_t$ & Variable (log) &
Inflación actual. \\[4pt]

$A_T$ & Matriz $2\times 2$ &
Matriz de transición que combina parámetros estructurales y de política
monetaria. Gobernará la dinámica prospectiva de
$\tilde{y}_t$ y $\pi_t$. \\[4pt]

$B_T$ & Vector $2\times 1$ &
Vector de coeficientes que mide cómo los shocks exógenos afectan
simultáneamente a $\tilde{y}_t$ y $\pi_t$. \\[4pt]

$u_t$ & Escalar (shocks agregados) &
Shock agregado que combina shocks reales (tecnología, preferencias) y
shocks de política monetaria. \\[4pt]

$\Delta$ & Escalar &
Factor de normalización: $\Delta \equiv \dfrac{1}{\sigma + \varphi_y + \kappa\varphi_{\pi}}$. \\[4pt]
\hline
\end{tabular}
\end{table}

%---------------------------------------------
% Definición de matrices y del shock agregado
%---------------------------------------------

\subsubsection*{Definición de $A_T$, $B_T$ y $u_t$}

La fuente define las matrices y el shock agregado como:

\begin{align*}
  \Delta
  &\equiv
  \frac{1}{\sigma + \varphi_y + \kappa\varphi_{\pi}},
  \\[6pt]
  A_T
  &\equiv
  \Delta
  \begin{bmatrix}
    \sigma
    &
    1 - \beta\varphi_{\pi}
    \\[4pt]
    \sigma\kappa
    &
    \kappa + \beta(\sigma + \varphi_y)
  \end{bmatrix},
  \\[6pt]
  B_T
  &\equiv
  \Delta
  \begin{bmatrix}
    1 \\
    \kappa
  \end{bmatrix},
  \\[6pt]
  u_t
  &\equiv
  \hat{r}_{n,t} - \varphi_y \hat{y}^n_t - v_t,
\end{align*}

donde:

\begin{itemize}
  \item $\hat{r}_{n,t} \equiv r_{n,t} - r_n$ es la desviación de la
  tasa natural de interés respecto a su valor de estado estacionario.
  \item $\hat{y}^n_t \equiv y_{n,t} - y$ es la desviación del producto
  natural respecto a su valor de estado estacionario.
  \item $v_t$ es el shock de política monetaria en la regla de tasa de
  interés (Ecuación (26)).
\end{itemize}

En términos de los shocks fundamentales de tecnología ($a_t$),
preferencias ($z_t$) y política monetaria ($v_t$), el término $u_t$ se
puede escribir como:

\begin{equation*}
  u_t
  =
  -\psi_{ya}\bigl(\varphi_y + \sigma(1-\rho_a)\bigr)a_t
  +
  (1-\rho_z)z_t
  -
  v_t,
\end{equation*}

donde $\psi_{ya}$ recoge la sensibilidad del output natural a la
tecnología (Ecuación (20)), y $\rho_a$, $\rho_z$ son los parámetros
AR(1) de los procesos de $a_t$ y $z_t$ respectivamente.

%---------------------------------------------
% Origen del sistema (27)
%---------------------------------------------

\subsubsection*{Origen: combinación de NKPC, DIS y regla de tasa de interés}

El sistema (27) es el resultado de:

\begin{enumerate}
  \item Tomar la \textbf{Curva de Phillips Nuevo Keynesiana} (NKPC,
  Ecuación (22)):
  \[
    \pi_t = \beta E_t\{\pi_{t+1}\} + \kappa \tilde{y}_t.
  \]

  \item Tomar la \textbf{Ecuación IS Dinámica en brechas} (Ecuación (23)):
  \[
    \tilde{y}_t
    =
    E_t\{\tilde{y}_{t+1}\}
    -
    \frac{1}{\sigma}\bigl(r_t - r_{n,t}\bigr),
    \quad
    r_t \equiv i_t - E_t\{\pi_{t+1}\},
  \]
  y sustituir la \textbf{regla de tasa de interés nominal} (Ecuación (26)):
  \[
    i_t = \rho + \varphi_{\pi}\pi_t + \varphi_y \hat{y}_t + v_t.
  \]

  \item Reexpresar todo en términos de $\tilde{y}_t$ y $\pi_t$, usando
  las definiciones de $y_{n,t}$, $\hat{y}_t$ y $r_{n,t}$, para eliminar
  $i_t$ y obtener un sistema sólo en las variables no predeterminadas
  $(\tilde{y}_t, \pi_t)$.
\end{enumerate}

El resultado algebraico se compacta en la forma matricial de la
Ecuación (27).

%---------------------------------------------
% Implicación económica: unicidad y estabilidad
%---------------------------------------------

\subsubsection*{Implicación económica: dinámica, unicidad y Principio de Taylor}

\begin{itemize}
  \item \textbf{Variables no predeterminadas y expectativas.}  
  Tanto la brecha del producto $\tilde{y}_t$ como la inflación $\pi_t$
  son variables \emph{no predeterminadas}: dependen de sus valores
  futuros esperados. El sistema (27) recoge explícitamente esta
  naturaleza \emph{forward-looking} del modelo.

  \item \textbf{Rol de $A_T$.}  
  La matriz $A_T$ condensa la interacción entre:
  \begin{itemize}
    \item la rigidez nominal (vía $\kappa$ y $\beta$),
    \item la sensibilidad intertemporal del consumo ($\sigma$),
    \item y la agresividad de la política monetaria
    ($\varphi_{\pi}, \varphi_y$).
  \end{itemize}
  Sus valores propios determinan si el equilibrio es único o si hay
  múltiples trayectorias compatibles con las expectativas.

  \item \textbf{Condición de unicidad.}  
  Bajo el supuesto estándar de $\varphi_{\pi} \ge 0$,
  $\varphi_y \ge 0$, la condición necesaria y suficiente para que el
  sistema (27) tenga una solución estacionaria y localmente única es:
  \begin{equation}
    \kappa(\varphi_{\pi} - 1)
    +
    (1 - \beta)\,\varphi_y
    \;>\; 0,
    \tag{28}
  \end{equation}
  lo que constituye una versión formal del \emph{Principio de Taylor}:
  la política monetaria debe reaccionar de manera suficientemente fuerte
  a la inflación (y, en menor medida, a la actividad) para anclar
  expectativas y descartar equilibrios indeterminados.

  \item \textbf{Shock agregado $u_t$.}  
  El término $u_t$ resume cómo los shocks fundamentales (tecnología,
  preferencias y política monetaria) se transmiten a la brecha del
  producto y a la inflación. Dado que $B_T$ tiene dimensión $2\times1$,
  el sistema deja claro que distintas combinaciones de shocks reales y
  de política afectan simultáneamente a ambas variables.
\end{itemize}

% --------------------------------------------------
% Ecuación (28): Condición de unicidad y estabilidad
% --------------------------------------------------

\subsection*{Ecuación (28): Condición de unicidad del equilibrio}

La Ecuación (28) no describe una relación económica directa, sino una
\emph{condición de estabilidad y unicidad} que deben cumplir los
parámetros estructurales del modelo y los coeficientes de la regla de
política monetaria (Ecuación (26)) para que el sistema dinámico de
equilibrio (Ecuación (27)) tenga una solución estacionaria y localmente
única.

\begin{equation}
  \kappa(\varphi_{\pi} - 1) + (1 - \beta)\,\varphi_y > 0
  \tag{28}
\end{equation}

%---------------------------------------------
% Tabla de símbolos relevantes (Ecuación 28)
%---------------------------------------------

\begin{table}[h!]
\centering
\begin{tabular}{p{0.18\textwidth} p{0.22\textwidth} p{0.50\textwidth}}
\hline
\textbf{Símbolo} & \textbf{Tipo} & \textbf{Descripción} \\
\hline
$\kappa$ & Parámetro &
Pendiente de la Curva de Phillips Nuevo Keynesiana (Ecuación (22));
mide la sensibilidad de la inflación a la brecha del producto. \\[4pt]

$\varphi_{\pi}$ & Parámetro &
Coeficiente de respuesta de la tasa de interés nominal a la inflación en
la regla de política monetaria (Ecuación (26)). \\[4pt]

$\varphi_y$ & Parámetro &
Coeficiente de respuesta de la tasa de interés nominal al producto (o a
la brecha del producto) en la regla de política. \\[4pt]

$\beta$ & Parámetro &
Factor de descuento intertemporal; $(1-\beta)$ está relacionado con la
tasa de descuento $\rho \equiv -\log\beta$. \\[4pt]
\hline
\end{tabular}
\end{table}

%---------------------------------------------
% Origen de la condición (28)
%---------------------------------------------

\subsubsection*{Origen matemático}

La condición (28) se obtiene al analizar las propiedades del sistema
dinámico representado por la Ecuación (27):

\begin{equation*}
  \begin{bmatrix} \tilde{y}_t \\ \pi_t \end{bmatrix}
  =
  A_T
  \begin{bmatrix}
    E_t\{\tilde{y}_{t+1}\} \\
    E_t\{\pi_{t+1}\}
  \end{bmatrix}
  +
  B_T u_t,
\end{equation*}

donde tanto la brecha del producto $\tilde{y}_t$ como la inflación
$\pi_t$ son variables \emph{no predeterminadas}. La unicidad de la
solución estacionaria requiere que los dos valores propios de la matriz
$A_T$ se encuentren dentro del círculo unitario. Bajo el supuesto
natural de que $\varphi_{\pi} \ge 0$ y $\varphi_y \ge 0$, esta condición
de valores propios es equivalente a la desigualdad:

\[
  \kappa(\varphi_{\pi} - 1) + (1 - \beta)\,\varphi_y > 0.
\]

%---------------------------------------------
% Interpretación económica
%---------------------------------------------

\subsubsection*{Implicación económica: Principio de Taylor y estabilidad}

\begin{itemize}
  \item \textbf{Regla de política suficientemente agresiva.}  
  La condición (28) exige que la política monetaria reaccione de forma
  suficientemente fuerte a las desviaciones de la inflación y, en menor
  medida, del producto. De lo contrario, el sistema presenta múltiples
  trayectorias compatibles con las expectativas (indeterminación).

  \item \textbf{Estabilización de expectativas.}  
  Para que el equilibrio sea único, cuando la inflación esperada aumenta,
  el Banco Central debe elevar la tasa nominal $i_t$ de forma tal que la
  \emph{tasa de interés real esperada} $r_t \equiv i_t - E_t\{\pi_{t+1}\}$
  también aumente lo suficiente. El aumento de $r_t$ contrae la demanda
  (Ecuación (23)) y reduce la presión inflacionaria (Ecuación (22)),
  estabilizando así las expectativas.

  \item \textbf{Principio de Taylor (caso simplificado).}  
  En el caso simplificado donde se ignora la respuesta al producto
  ($\varphi_y = 0$) y se toma $\beta \approx 1$, la condición (28) se
  reduce aproximadamente a:
  \[
    \varphi_{\pi} > 1,
  \]
  lo que corresponde al conocido \emph{Principio de Taylor}: la tasa de
  interés nominal debe aumentar más que proporcionalmente ante un aumento
  de la inflación, de modo que la tasa real también suba y discipline a
  la demanda agregada.

  \item \textbf{Fallo de política y equilibrios de \emph{sunspots}.}  
  Si la condición (28) no se cumple, el modelo admite equilibrios
  indeterminados en los que la dinámica de $\tilde{y}_t$ y $\pi_t$ puede
  estar impulsada por cambios puramente expectacionales (equilibrios de
  \emph{sunspots}), aun en ausencia de shocks fundamentales.
\end{itemize}


% -----------------------------------------------
% Ecuación (29): Respuesta de la brecha del producto al shock agregado
% -----------------------------------------------

\subsection*{Ecuación (29): Coeficiente de respuesta de la brecha del producto}

La Ecuación (29) forma parte de la solución estacionaria del sistema
Nuevo Keynesiano cerrado con una regla tipo Taylor. Especifica el
coeficiente $\psi_y$ que vincula la brecha del producto con el shock
agregado $u_t$ en una solución lineal del tipo:
\[
  \tilde{y}_t = \psi_y\,u_t.
\]

\begin{equation}
  \psi_y = (1 - \beta \rho_u)\,\Delta_u
  \tag{29}
\end{equation}

donde el término $\Delta_u$ recoge la interacción entre los parámetros
estructurales del modelo y los coeficientes de política monetaria.

%---------------------------------------------
% Definición de símbolos (Ecuación 29)
%---------------------------------------------

\begin{table}[h!]
\centering
\begin{tabular}{p{0.18\textwidth} p{0.22\textwidth} p{0.50\textwidth}}
\hline
\textbf{Símbolo} & \textbf{Tipo} & \textbf{Descripción} \\
\hline
$\tilde{y}_t$ & Variable &
Brecha del producto: $\tilde{y}_t \equiv y_t - y_{n,t}$. \\[4pt]

$u_t$ & Variable &
Shock agregado que resume shocks reales (tasa natural, producto
natural) y de política monetaria en el sistema reducido. \\[4pt]

$\psi_y$ & Coeficiente &
Coeficiente de la solución estacionaria lineal: mide la sensibilidad de
la brecha del producto al shock $u_t$. \\[4pt]

$\beta$ & Parámetro &
Factor de descuento intertemporal, $0 < \beta < 1$. \\[4pt]

$\rho_u$ & Parámetro &
Coeficiente autorregresivo del shock agregado $u_t$ (procesos AR(1)). \\[4pt]

$\Delta_u$ & Parámetro &
Coeficiente compuesto positivo que agrupa parámetros estructurales del
modelo y coeficientes de política monetaria. \\[4pt]
\hline
\end{tabular}
\end{table}

La expresión exacta de $\Delta_u$ es:

\begin{equation}
  \Delta_u \equiv 
  \frac{1}{
    (1-\beta\rho_u)\,\big[\sigma (1-\rho_u)+\varphi_y\big]
    +\kappa(\varphi_{\pi}-\rho_u)
  }
  \;>\; 0.
\end{equation}

%---------------------------------------------
% Origen y método de obtención
%---------------------------------------------

\subsubsection*{Origen: método de coeficientes indeterminados}

\begin{itemize}
  \item Partimos del sistema reducido de equilibrio para
  $(\tilde{y}_t,\pi_t)$ (Ecuación (27)), obtenido al combinar:
  \begin{enumerate}
    \item La Curva de Phillips Nuevo Keynesiana (Ecuación (22)).
    \item La Ecuación IS Dinámica en términos de brecha del producto
    (Ecuación (23)).
    \item La Regla de tasa de interés nominal (Ecuación (26)).
  \end{enumerate}

  \item Se supone que el shock agregado $u_t$ sigue un proceso AR(1):
  \[
    u_t = \rho_u u_{t-1} + \varepsilon_{u,t},
  \qquad |\rho_u| < 1.
  \]

  \item Se conjetura una solución estacionaria lineal de la forma:
  \[
    \tilde{y}_t = \psi_y u_t, 
    \qquad
    \pi_t = \psi_{\pi} u_t.
  \]

  \item Al sustituir estas conjeturas en el sistema (27) y comparar
  coeficientes en $u_t$, se obtiene un sistema algebraico en los
  parámetros desconocidos $(\psi_y, \psi_{\pi})$. La solución de ese
  sistema da lugar a:
  \[
    \psi_y = (1 - \beta\rho_u)\,\Delta_u,
    \qquad
    \psi_{\pi} = \kappa\,\Delta_u,
  \]
  donde $\Delta_u$ viene dado por la expresión anterior.
\end{itemize}

%---------------------------------------------
% Interpretación económica
%---------------------------------------------

\subsubsection*{Implicación económica: sensibilidad del output gap al shock agregado}

\begin{itemize}
  \item \textbf{Co-movimiento con el shock.}  
  Dado que $\Delta_u>0$ y, en equilibrio razonable, $0<\beta<1$ y
  $|\rho_u|<1$, el término $(1-\beta\rho_u)$ es típicamente positivo.
  Entonces, $\psi_y>0$, de modo que la brecha del producto
  $\tilde{y}_t$ tiende a moverse en la misma dirección que el shock
  agregado $u_t$:
  \[
    u_t \uparrow \;\Rightarrow\; \tilde{y}_t \uparrow \quad
    (\text{para } \psi_y>0).
  \]

  \item \textbf{Persistencia del shock y descuento intertemporal.}  
  El factor $(1-\beta\rho_u)$ muestra cómo la combinación entre la
  persistencia del shock ($\rho_u$) y el grado de paciencia de los
  agentes ($\beta$) modula el tamaño de la respuesta de la brecha del
  producto:
  \begin{itemize}
    \item Si $\rho_u$ es muy alto (shocks muy persistentes), el término
    $(1-\beta\rho_u)$ se hace más pequeño, reduciendo $\psi_y$.
    \item Si los agentes son muy pacientes ($\beta$ cercana a 1), también
    se atenúa la respuesta contemporánea, porque el choque se reparte a
    lo largo del tiempo.
  \end{itemize}

  \item \textbf{Papel de la política monetaria.}  
  Toda la información sobre la agresividad de la política monetaria
  respecto a inflación y producto $(\varphi_{\pi},\varphi_y)$, así como
  sobre las rigideces nominales (a través de $\kappa$), se concentra en
  el denominador de $\Delta_u$. Una política más agresiva (por ejemplo,
  mayores valores de $\varphi_{\pi}$ y/o $\varphi_y$) tiende a:
  \begin{itemize}
    \item Aumentar el denominador de $\Delta_u$,
    \item Reducir el valor de $\Delta_u$,
    \item Y, por tanto, disminuir $\psi_y$.
  \end{itemize}
  Es decir, una política monetaria más reactiva amortigua la respuesta
  del \emph{output gap} frente a un mismo shock $u_t$.

  \item \textbf{Shocks de política contraccionista.}  
  Recordando que el shock de política monetaria $v_t$ entra en $u_t$
  con signo negativo, un incremento de $v_t$ (endurecimiento de la
  política) reduce $u_t$, lo que a través de
  $\tilde{y}_t = \psi_y u_t$ implica una caída de la brecha del producto
  ($\tilde{y}_t<0$). En la solución conjunta del modelo, esto se
  acompaña de una reducción de la inflación (vía el coeficiente
  $\psi_{\pi}$).
\end{itemize}

% -----------------------------------------------
% Ecuación (30): Respuesta de la inflación al shock agregado
% -----------------------------------------------

\subsection*{Ecuación (30): Coeficiente de respuesta de la inflación}

La Ecuación (30) completa, junto con la Ecuación (29), la solución
estacionaria del modelo Nuevo Keynesiano básico cerrado con una regla
tipo Taylor. Especifica el coeficiente $\psi_{\pi}$ que vincula la
inflación con el shock agregado $u_t$ en una solución lineal del tipo:
\[
  \pi_t = \psi_{\pi}\,u_t.
\]

\begin{equation}
  \psi_{\pi} = \kappa\,\Delta_u
  \tag{30}
\end{equation}

donde $\kappa$ es la pendiente de la Curva de Phillips Nuevo Keynesiana
(NKPC) y $\Delta_u$ es un término positivo que resume la interacción
entre parámetros estructurales y de política monetaria.

%---------------------------------------------
% Definición de símbolos (Ecuación 30)
%---------------------------------------------

\begin{table}[h!]
\centering
\begin{tabular}{p{0.18\textwidth} p{0.22\textwidth} p{0.50\textwidth}}
\hline
\textbf{Símbolo} & \textbf{Tipo} & \textbf{Descripción} \\
\hline
$\pi_t$ & Variable &
Tasa de inflación actual en el período $t$. \\[4pt]

$u_t$ & Variable &
Shock agregado exógeno que combina shocks reales (tasa natural,
output natural) y de política monetaria en el sistema reducido. \\[4pt]

$\psi_{\pi}$ & Coeficiente &
Coeficiente de la solución estacionaria lineal:
$\pi_t = \psi_{\pi} u_t$; mide la sensibilidad de la inflación al shock
$u_t$. \\[4pt]

$\kappa$ & Parámetro &
Pendiente de la NKPC (Ecuación (22)); mide la sensibilidad de la
inflación a la brecha del producto $\tilde{y}_t$. \\[4pt]

$\Delta_u$ & Parámetro &
Coeficiente compuesto positivo que recoge parámetros estructurales
($\sigma,\beta,\kappa$) y de política monetaria ($\varphi_{\pi},
\varphi_y$). \\[4pt]
\hline
\end{tabular}
\end{table}

La expresión exacta de $\Delta_u$ es:
\begin{equation}
  \Delta_u \equiv 
  \frac{1}{
    (1-\beta\rho_u)\,\big[\sigma (1-\rho_u)+\varphi_y\big]
    +\kappa(\varphi_{\pi}-\rho_u)
  }
  \;>\; 0,
\end{equation}
donde $\rho_u$ es el coeficiente autorregresivo del shock agregado
$u_t$.

%---------------------------------------------
% Origen y método de obtención
%---------------------------------------------

\subsubsection*{Origen: solución estacionaria del sistema reducido}

\begin{itemize}
  \item El punto de partida es el sistema reducido de equilibrio para
  $(\tilde{y}_t,\pi_t)$ (Ecuación (27)), obtenido al combinar:
  \begin{enumerate}
    \item La Curva de Phillips Nuevo Keynesiana:
    \[
      \pi_t = \beta E_t\{\pi_{t+1}\} + \kappa \tilde{y}_t \quad (22),
    \]
    \item La Ecuación IS Dinámica en términos de brecha del producto:
    \[
      \tilde{y}_t = E_t\{\tilde{y}_{t+1}\}
      -\frac{1}{\sigma}(r_t - r_{n,t}) \quad (23),
    \]
    \item La Regla de tasa de interés nominal:
    \[
      i_t = \rho + \varphi_{\pi}\pi_t + \varphi_y \hat{y}_t + v_t
      \quad (26).
    \]
  \end{enumerate}

  \item Estos elementos se combinan para escribir un sistema matricial
  del tipo:
  \[
    \begin{bmatrix} \tilde{y}_t \\ \pi_t \end{bmatrix}
    =
    A_T
    \begin{bmatrix} E_t\{\tilde{y}_{t+1}\} \\ E_t\{\pi_{t+1}\}
    \end{bmatrix}
    +
    B_T u_t,
  \]
  donde $u_t$ es el shock agregado y $A_T$, $B_T$ dependen de
  $(\sigma,\beta,\kappa,\varphi_{\pi},\varphi_y)$.

  \item Se supone que $u_t$ sigue un proceso AR(1):
  \[
    u_t = \rho_u u_{t-1} + \varepsilon_{u,t}, \quad |\rho_u|<1,
  \]
  y se conjetura una solución estacionaria lineal:
  \[
    \tilde{y}_t = \psi_y u_t, 
    \qquad
    \pi_t = \psi_{\pi} u_t.
  \]

  \item Al sustituir estas conjeturas en el sistema (27), se obtiene un
  sistema algebraico en los dos coeficientes desconocidos
  $(\psi_y,\psi_{\pi})$. La resolución de ese sistema produce:
  \[
    \psi_y = (1 - \beta\rho_u)\,\Delta_u \quad \text{(Ecuación 29)},
    \qquad
    \psi_{\pi} = \kappa\,\Delta_u \quad \text{(Ecuación 30)}.
  \]
\end{itemize}

%---------------------------------------------
% Interpretación económica
%---------------------------------------------

\subsubsection*{Implicación económica: sensibilidad de la inflación a los shocks}

\begin{itemize}
  \item \textbf{Relación con la NKPC.}  
  La presencia de $\kappa$ en
  \[
    \psi_{\pi} = \kappa\,\Delta_u
  \]
  refleja que la inflación responde a los shocks únicamente en la medida
  en que estos generen una brecha del producto $\tilde{y}_t$, y que la
  intensidad de esa respuesta está gobernada por la pendiente de la
  Curva de Phillips. Cuanto mayor es $\kappa$, más sensible es la
  inflación a un mismo movimiento de la brecha del producto.

  \item \textbf{Signo de la respuesta.}  
  Dado que $\kappa>0$ y $\Delta_u>0$, se tiene $\psi_{\pi}>0$. Por lo
  tanto, en la solución estacionaria:
  \[
    u_t \uparrow \;\Rightarrow\; \pi_t \uparrow,
  \]
  es decir, la inflación se mueve en la misma dirección que el shock
  agregado.

  \item \textbf{Shocks de política monetaria contraccionista.}  
  En el caso de un shock de política monetaria $v_t$:
  \begin{itemize}
    \item $u_t$ incorpora $-v_t$, de modo que un shock contraccionista
    (un aumento en $v_t$) reduce $u_t$.
    \item Con $\psi_{\pi}>0$, la Ecuación (30) implica una caída de la
    inflación ($\pi_t<0$) ante un shock contraccionista.
    \item Este resultado es coherente con el mecanismo del modelo:
    el aumento en la tasa nominal eleva la tasa real, contrae la brecha
    del producto (Ecuación (29)), y la menor brecha del producto reduce
    la inflación vía la NKPC (Ecuación (22)).
  \end{itemize}

  \item \textbf{Papel de la política monetaria en la estabilidad.}  
  La magnitud de $\Delta_u$ depende de los parámetros de política
  $(\varphi_{\pi},\varphi_y)$ y de los parámetros estructurales
  $(\sigma,\beta,\kappa,\rho_u)$. Una política monetaria más agresiva
  frente a la inflación y al producto:
  \begin{itemize}
    \item tiende a aumentar el denominador de $\Delta_u$,
    \item reduce el valor de $\Delta_u$,
    \item y, en consecuencia, reduce tanto $\psi_y$ (Ecuación (29)) como
    $\psi_{\pi}$ (Ecuación (30)).
  \end{itemize}
  Es decir, una política más reactiva amortigua la respuesta tanto de la
  brecha del producto como de la inflación frente a los shocks exógenos,
  contribuyendo a la estabilidad macroeconómica.
\end{itemize}

\begin{thebibliography}{99}
\bibitem{Gali2015}
Galí, J. (2015).
\textit{Monetary Policy, Inflation, and the Business Cycle:
An Introduction to the New Keynesian Framework and Its Applications}.
2nd ed.
Princeton: Princeton University Press.

\end{thebibliography}

\end{document}