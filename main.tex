\documentclass[12pt]{article}

\usepackage[margin=1in]{geometry}

% Español y codificación
\usepackage[spanish]{babel}
\usepackage[utf8]{inputenc}
\usepackage[T1]{fontenc}

% Matemáticas
\usepackage{amsmath}
\usepackage{amssymb}
\usepackage{mathtools}   % <-- ya no dará error cuando esté instalado

% Gráficos
%\usepackage{float}       % para [H]
%\usepackage{tikz}
%\usepackage{pgfplots}
%5\pgfplotsset{compat=1.18}



\title{Title}
\author{Gabriel Tehozol}

\begin{document}

\maketitle
%========================================
% 2.1 HOGARES: Problema de optimización
%========================================
\subsection*{Hogares: definición del problema de optimización dinámica}
\label{subsec:hogares}

En este apartado describimos qué hace el hogar en el modelo.
La idea básica es:

\begin{itemize}
  \item El hogar vive muchos periodos (t, t+1, t+2, \dots).
  \item En cada periodo decide cuánto consumir ($C_t$) y cuánto trabajar ($N_t$).
  \item También puede ahorrar o endeudarse con bonos ($B_t$).
  \item Quiere que, en promedio, su ``nivel de felicidad'' (utilidad) a lo largo del tiempo
        sea lo más alto posible, pero está limitado por el dinero que tiene
        y el que puede conseguir.
\end{itemize}

A esto le llamamos un \emph{problema de optimización dinámica}: el hogar
toma decisiones hoy pensando en sus consecuencias mañana y en el futuro.

%--------------------------------------------------
\subsubsection*{Ecuación (1): función objetivo (utilidad esperada descontada)}
\label{subsubsec:ecuacion1}

El objetivo del hogar se resume en la siguiente expresión:

\begin{equation}
E_0 \sum_{t=0}^{\infty} \beta^t U(C_t, N_t; Z_t).
\tag{1}
\end{equation}

Esta ecuación se puede leer \emph{de izquierda a derecha} como una frase:

\begin{center}
\emph{``El hogar quiere maximizar el valor esperado, visto desde hoy,
de la suma de su utilidad en cada periodo, ponderada por cuánto le
importa el futuro.''}
\end{center}

\paragraph{Paso a paso.}

\begin{enumerate}
  \item $U(C_t,N_t;Z_t)$ es la \textbf{utilidad} del hogar en el periodo $t$.
  Depende de:
  \begin{itemize}
    \item $C_t$: consumo del bien (comida, servicios, etc.).
    \item $N_t$: horas de trabajo (más trabajo suele dar más ingreso, pero menos ocio).
    \item $Z_t$: shock o desplazador de preferencias, que representa cambios
          en gustos, hábitos, etc.
  \end{itemize}

  \item La suma $\displaystyle \sum_{t=0}^{\infty}$ indica que el hogar
  se preocupa por \textbf{todos} los periodos: $t=0,1,2,\dots$.

  \item El factor $\beta^t$, donde $0<\beta<1$, \emph{descuenta} la importancia
  de la utilidad de cada periodo. Cuanto mayor es $t$, menor es el peso $\beta^t$.
  \begin{itemize}
    \item Si $\beta$ es cercano a 1, el hogar es paciente (le importa mucho el futuro).
    \item Si $\beta$ es pequeño, el hogar es impaciente (valora más el presente).
  \end{itemize}

  \item $E_0[\cdot]$ es el \textbf{operador de expectativas}. Simplemente
  significa: ``el valor promedio que el hogar espera hoy ($t=0$) que tendrá
  esa suma, tomando en cuenta que el futuro es incierto''.
\end{enumerate}

\paragraph{Supuestos básicos sobre la utilidad.}

Para que el problema tenga sentido económico, se imponen algunos supuestos
estándar sobre $U(C_t,N_t;Z_t)$:

\begin{itemize}
  \item La utilidad aumenta con el consumo:
  \[
    U_{c,t} \equiv \frac{\partial U}{\partial C_t} > 0,
  \]
  y la utilidad marginal del consumo es decreciente:
  \[
    U_{cc,t} \equiv \frac{\partial^2 U}{\partial C_t^2} \le 0.
  \]
  Es decir, consumir más siempre gusta, pero cada unidad extra aporta un
  poco menos que la anterior.

  \item El trabajo genera desutilidad:
  \[
    U_{n,t} \equiv \frac{\partial U}{\partial N_t} \le 0,
  \]
  de modo que $-U_{n,t} > 0$ es la \emph{desutilidad marginal del trabajo}
  (trabajar más cansa).

  \item El shock $Z_t$ desplaza las preferencias. Supondremos que un aumento
  en $Z_t$ eleva la utilidad marginal del consumo:
  \[
    U_{cz,t} \equiv \frac{\partial^2 U}{\partial C_t\,\partial Z_t} > 0.
  \]
\end{itemize}

\noindent
La ecuación (1) dice:
el hogar quiere elegir sus secuencias de $C_t$ y $N_t$ para que, en promedio,
la suma de sus felicidades presentes y futuras sea lo más alta posible.

%--------------------------------------------------
\subsubsection*{Ecuación (2): restricción presupuestaria de flujo}
\label{subsubsec:ecuacion2}

El hogar no puede elegir cualquier combinación de consumo y trabajo:
cada periodo está limitado por sus ingresos y por lo que puede ahorrar
o endeudarse. Esta idea se resume en la \textbf{restricción de flujo}:

\begin{equation}
P_t C_t + Q_t B_t \le B_{t-1} + W_t N_t + D_t.
\tag{2}
\end{equation}

Podemos leer esta ecuación como:

\begin{center}
\emph{``Uso de recursos en el periodo $t$ $\le$ Fuentes de recursos en el periodo $t$.''}
\end{center}

\paragraph{Lado izquierdo: usos (en qué se gasta).}

\begin{itemize}
  \item $P_t C_t$: gasto en consumo.  
        $P_t$ es el precio nominal del bien; $C_t$ es la cantidad consumida.
  \item $Q_t B_t$: gasto en compra de bonos.  
        $Q_t$ es el precio hoy de un bono que paga 1 unidad de dinero
        en $t+1$; $B_t$ es el número de bonos que compra el hogar.
\end{itemize}

\paragraph{Lado derecho: fuentes (de dónde vienen los recursos).}

\begin{itemize}
  \item $B_{t-1}$: bonos que se compraron en $t-1$ y pagan hoy (ingreso financiero).
  \item $W_t N_t$: ingreso laboral (salario nominal $W_t$ por horas trabajadas $N_t$).
  \item $D_t$: dividendos que el hogar recibe como dueño de las empresas.
\end{itemize}

En equilibrio competitivo es natural que la desigualdad se cumpla
con igualdad (el hogar no deja dinero sin usar), es decir:
\[
  P_t C_t + Q_t B_t = B_{t-1} + W_t N_t + D_t.
\]

\noindent
Cada periodo, lo que el hogar gasta en consumo y bonos no puede superar 
lo que entra por salario, dividendos y bonos que vencen.

%--------------------------------------------------
\subsubsection*{Ecuación (3): restricción de solvencia (no-Ponzi)}
\label{subsubsec:ecuacion3}

Hasta ahora, la restricción de flujo controla qué pasa \emph{dentro} de cada
periodo. Pero como el horizonte es infinito, en principio el hogar podría
intentar endeudarse cada vez más, sin pagar nunca. Para evitar ese tipo
de planes no realistas, se impone una \textbf{restricción de solvencia}
o \textbf{restricción de no-Ponzi}:

\begin{equation}
\lim_{T\to\infty} E_t \left\{ \varXi_{t,T} \frac{B_T}{P_T} \right\} \ge 0.
\tag{3}
\end{equation}

\paragraph{¿Qué significa esta condición?}

\begin{itemize}
  \item $B_T/P_T$ es la \emph{riqueza real en bonos} en el periodo $T$.
  \item $\varXi_{t,T}$ es un \textbf{factor de descuento estocástico},
  definido como:
  \[
    \varXi_{t,T} \equiv \beta^{T-t} \frac{U_{c,T}}{U_{c,t}}.
  \]
  Este factor combina:
  \begin{itemize}
    \item el descuento puro del tiempo ($\beta^{T-t}$),
    \item y el cambio en la utilidad marginal del consumo
          ($U_{c,T}/U_{c,t}$).
  \end{itemize}
\end{itemize}

La expresión $E_t\{\varXi_{t,T} B_T/P_T\}$ se puede interpretar como el
\emph{valor presente esperado} de la riqueza real en bonos en el horizonte $T$,
medido desde el punto de vista del periodo $t$.

La condición
\[
  \lim_{T\to\infty} E_t \left\{ \varXi_{t,T} \frac{B_T}{P_T} \right\} \ge 0
\]
dice que, en el límite, ese valor presente esperado no puede ser negativo.
Intuitivamente:

\begin{center}
\emph{``El hogar no puede sostener un esquema en el que su deuda crezca
tanto que, incluso descontada, termine siendo impagable.''}
\end{center}

\noindent
La ecuación (3) evita el endeudamiento explosivo.
Es una condición técnica, pero muy importante, para que el problema infinito
esté bien definido.

%--------------------------------------------------
\subsubsection*{Ecuación (4): condición de optimalidad intratemporal}
\label{subsubsec:ecuacion4}

Ahora sí, pasamos a una de las \textbf{condiciones de primer orden}
más importantes: la que relaciona consumo, trabajo y salarios \emph{dentro}
de un mismo periodo. Esta condición se escribe como:

\begin{equation}
-\frac{U_{n,t}}{U_{c,t}} = \frac{W_t}{P_t}.
\tag{4}
\end{equation}

Esta ecuación se llama:
\begin{itemize}
  \item \textbf{Condición de optimalidad intratemporal}, porque compara
        decisiones dentro del periodo $t$.
  \item \textbf{Condición de oferta de trabajo}, porque nos dice cómo el hogar
        decide cuántas horas trabajar dado el salario real.
\end{itemize}

\paragraph{Interpretación de los términos.}

\begin{itemize}
  \item $U_{c,t}$: utilidad marginal del consumo (cuánto aumenta la utilidad
        si el hogar consume un poco más).
  \item $U_{n,t}$: utilidad marginal del trabajo (cuánto cambia la utilidad
        si el hogar trabaja un poco más). Suele ser negativa: trabajar cansa.
  \item $-U_{n,t}/U_{c,t}$: \textbf{tasa marginal de sustitución} (TMS)
        entre ocio y consumo.  
        Dice cuántas unidades extra de consumo necesita el hogar para
        aceptar trabajar una unidad extra.
  \item $W_t/P_t$: \textbf{salario real}, es decir, cuántas unidades de bien
        de consumo recibe el hogar por cada unidad de trabajo.
\end{itemize}

La ecuación (4) dice:

\begin{center}
\emph{``El hogar elige sus horas de trabajo de modo que su TMS subjetiva
entre ocio y consumo sea igual al salario real que ofrece el mercado.''}
\end{center}

\paragraph{Derivación paso a paso (argumento variacional).}

La idea es imaginar un pequeño cambio en las decisiones del hogar en el
periodo $t$, y pedir que ese cambio \emph{no mejore} la situación si ya
estamos en el óptimo.

\begin{enumerate}
  \item \textbf{Paso 1: pequeña desviación en consumo y trabajo.}  
  Partimos de un plan óptimo $(C_t,N_t)$ y consideramos una pequeña desviación
  $(dC_t,dN_t)$ en el periodo $t$, manteniendo todo lo demás igual.
  
  El cambio en la utilidad instantánea es:
  \[
    dU_t = U_{c,t}\, dC_t + U_{n,t}\, dN_t.
  \]

  Si el plan es óptimo, cualquier desviación que respete el presupuesto
  no puede mejorar la utilidad. En el margen, esto implica que el cambio
  en utilidad asociado a una desviación factible debe ser cero:
  \begin{equation}
    U_{c,t}\, dC_t + U_{n,t}\, dN_t = 0.
    \label{eq:variacion_utilidad}
  \end{equation}

  \item \textbf{Paso 2: pequeña desviación en la restricción de flujo.}  
  Ahora miramos cómo se ve esa misma desviación en la restricción presupuestaria
  del periodo $t$:
  \[
    P_t C_t + Q_t B_t = B_{t-1} + W_t N_t + D_t.
  \]
  Si mantenemos fijo $B_t$ (no tocamos el ahorro) y sólo cambiamos $C_t$ y $N_t$,
  la variación de esta igualdad es:
  \begin{equation}
    P_t\, dC_t = W_t\, dN_t.
    \label{eq:variacion_presupuesto}
  \end{equation}
  Es decir, el gasto adicional en consumo tiene que ser financiado por un
  ingreso adicional de trabajo.

  \item \textbf{Paso 3: expresar $dC_t$ en función de $dN_t$.}  
  De \eqref{eq:variacion_presupuesto} despejamos:
  \begin{equation}
    dC_t = \frac{W_t}{P_t} dN_t.
    \label{eq:dC_en_funcion_dN}
  \end{equation}

  \item \textbf{Paso 4: sustituir en la variación de utilidad.}  
  Sustituimos \eqref{eq:dC_en_funcion_dN} en \eqref{eq:variacion_utilidad}:
  \[
    U_{c,t} \left( \frac{W_t}{P_t} dN_t \right) + U_{n,t} dN_t = 0.
  \]
  Factorizamos $dN_t$:
  \[
    \left[
      U_{c,t} \frac{W_t}{P_t} + U_{n,t}
    \right] dN_t = 0.
  \]

  Como $dN_t$ representa una desviación arbitraria (pequeña pero no necesariamente
  cero), para que el producto sea cero, el término entre corchetes debe ser:
  \[
    U_{c,t} \frac{W_t}{P_t} + U_{n,t} = 0.
  \]

  \item \textbf{Paso 5: aislar la TMS.}  
  Reordenando:
  \[
    U_{c,t} \frac{W_t}{P_t} = -U_{n,t},
  \]
  y dividiendo entre $U_{c,t} > 0$:
  \[
    -\frac{U_{n,t}}{U_{c,t}} = \frac{W_t}{P_t},
  \]
  que es exactamente la ecuación (4).
\end{enumerate}

\paragraph{Entonces tenemos...}

\begin{itemize}
  \item La variación en la utilidad nos dice cómo cambian las preferencias
        del hogar ante pequeños ajustes en $C_t$ y $N_t$.
  \item La variación en el presupuesto nos dice qué combinaciones de
        $dC_t$ y $dN_t$ son factibles (se pueden pagar).
  \item Al combinar ambas y exigir que no haya manera de mejorar la utilidad
        con una desviación factible, obtenemos la condición de primer orden.
  \item El resultado final iguala:
        \begin{itemize}
          \item el \emph{costo subjetivo} de trabajar más (perder ocio),
          \item con el \emph{beneficio objetivo} de mercado (el salario real).
        \end{itemize}
\end{itemize}

%--------------------------------------------------
\subsubsection*{Ecuación (5): condición de optimalidad intertemporal (Ecuación de Euler)}
\label{subsubsec:ecuacion5}

Además de decidir cuánto trabajar dentro de cada periodo (condición intratemporal),
el hogar debe decidir \emph{cuánto consumir hoy y cuánto dejar para consumir mañana}.
Esta decisión se resume en la \textbf{condición de optimalidad intertemporal}
o \textbf{Ecuación de Euler}:

\begin{equation}
Q_t = \beta E_t \left\{ \frac{U_{c,t+1}}{U_{c,t}} \frac{P_t}{P_{t+1}} \right\}.
\tag{5}
\end{equation}

\paragraph{Lectura intuitiva.}

La ecuación (5) se puede leer como:

\begin{center}
\emph{``El precio del bono hoy ($Q_t$) debe ser igual al valor presente esperado
de la tasa a la que el hogar está dispuesto a intercambiar consumo de hoy por
consumo de mañana, medido en unidades de bien de consumo y multiplicado por
el factor de descuento $\beta$.''}
\end{center}

\noindent
Donde:
\begin{itemize}
  \item $Q_t$ es el \textbf{precio del bono} nominal libre de riesgo.
  \item $\displaystyle \frac{U_{c,t+1}}{U_{c,t}}$ mide cómo cambia la utilidad marginal
        del consumo entre $t$ y $t+1$.
  \item $\displaystyle \frac{P_t}{P_{t+1}}$ ajusta por la inflación esperada
        (pasa de unidades nominales a reales y viceversa).
  \item $\beta$ descuenta el futuro (impaciencia).
\end{itemize}

\paragraph{Desglose de los términos de (5).}

\begin{itemize}
  \item \textbf{Precio del bono:} $Q_t$ es el precio en $t$ de un bono que paga
  1 unidad de dinero en $t+1$. El rendimiento nominal bruto del bono es
  $R_t^n = 1/Q_t$.

  \item \textbf{Utilidad marginal del consumo:} 
  $U_{c,t} \equiv \partial U(C_t,N_t;Z_t)/\partial C_t$ es la utilidad marginal
  en el periodo $t$. La razón $\displaystyle U_{c,t+1}/U_{c,t}$ captura cómo el
  hogar valora una unidad extra de consumo mañana en comparación con hoy.

  \item \textbf{Relación de precios:} $\displaystyle P_t/P_{t+1}$ es el inverso
  de la inflación bruta esperada entre $t$ y $t+1$. Si los precios suben mucho,
  una misma cantidad de dinero rinde menos en términos de consumo futuro.

  \item \textbf{Operador de expectativas:} $E_t\{\cdot\}$ promedia sobre todos
  los escenarios posibles del futuro, dados los shocks, usando la información
  que el hogar tiene en $t$.
\end{itemize}

\paragraph{Derivación paso a paso}

La ecuación (5) se obtiene al analizar una pequeña reubicación de consumo
entre los periodos $t$ y $t+1$. La idea es:

\begin{center}
\emph{``¿Qué pasa si reduzco mi consumo hoy y uso ese ahorro para
aumentar mi consumo mañana, sin violar el presupuesto? En el óptimo, ese
cambio no debe mejorar mi utilidad esperada.''}
\end{center}

\begin{enumerate}
  \item \textbf{Paso 1: variación en la utilidad intertemporal.}  

  Consideremos un plan óptimo y una pequeña desviación $(dC_t, dC_{t+1})$
  que sólo afecta al consumo en $t$ y $t+1$. El cambio en la utilidad total
  (medida en $t$ y descontando el futuro) es aproximadamente:
  \begin{equation}
    U_{c,t}\, dC_t + \beta E_t\{U_{c,t+1}\, dC_{t+1}\} = 0.
    \tag{A}
  \end{equation}
  Explicación de cada término:
  \begin{itemize}
    \item $U_{c,t} dC_t$: cambio en la utilidad del \emph{periodo actual}.
    \item $\beta E_t\{U_{c,t+1} dC_{t+1}\}$: cambio esperado en la utilidad
          del periodo siguiente, descontado a valor presente con $\beta$.
  \end{itemize}
  En un óptimo, cualquier desviación factible en la que sólo movemos consumo
  entre $t$ y $t+1$ no debe aumentar la utilidad, por lo que el cambio marginal
  debe ser cero.

  \item \textbf{Paso 2: variación en el presupuesto intertemporal.}  

  Ahora vemos cómo se traduce esa misma desviación en la restricción
  presupuestaria. La idea es:

  \begin{itemize}
    \item Si reducimos el consumo hoy en una cantidad $dC_t<0$, liberamos
          recursos por un monto nominal $P_t dC_t$.
    \item Esos recursos se usan para comprar más bonos: el número adicional
          de bonos es
          \[
            dB_t = -\,\frac{P_t}{Q_t}\, dC_t,
          \]
          porque cada bono cuesta $Q_t$ unidades de dinero.
    \item Mañana, esos bonos pagan 1 unidad de dinero cada uno, de modo que
          el ingreso adicional nominal en $t+1$ es $(1)\cdot dB_t$.
  \end{itemize}

  El cambio en la restricción de flujo en $t+1$ implica que ese ingreso
  adicional se destina a mayor consumo $dC_{t+1}$:
  \[
    P_{t+1} dC_{t+1} = dB_t
    = -\,\frac{P_t}{Q_t} dC_t.
  \]
  Es decir:
  \begin{equation}
    P_{t+1} dC_{t+1} = -\,\frac{P_t}{Q_t} dC_t.
    \tag{B}
  \end{equation}

  \item \textbf{Paso 3: expresar $dC_{t+1}$ en función de $dC_t$.}  

  De (B) despejamos:
  \begin{equation}
    dC_{t+1} = -\,\frac{P_t}{Q_t P_{t+1}}\, dC_t.
    \tag{C}
  \end{equation}

  \item \textbf{Paso 4: sustituir en la variación de utilidad.}  

  Sustituimos (C) en (A):
  \[
    U_{c,t} dC_t 
    + \beta E_t\left\{
      U_{c,t+1}
      \left(-\,\frac{P_t}{Q_t P_{t+1}} dC_t\right)
    \right\} = 0.
  \]
  Factorizamos $dC_t$:
  \[
    \left[
      U_{c,t}
      - \beta E_t\left\{
        U_{c,t+1}\frac{P_t}{Q_t P_{t+1}}
      \right\}
    \right] dC_t = 0.
  \]
  Como $dC_t$ representa una desviación arbitraria (pequeña, pero distinta de cero),
  el término entre corchetes debe ser nulo:
  \begin{equation}
    U_{c,t} 
    = \beta E_t\left\{
      U_{c,t+1}\frac{P_t}{Q_t P_{t+1}}
    \right\}.
    \label{eq:euler_bruta}
  \end{equation}

  \item \textbf{Paso 5: aislar $Q_t$ y obtener la Ecuación de Euler.}  

  De \eqref{eq:euler_bruta}, multiplicamos ambos lados por
  $Q_t/U_{c,t}$:
  \[
    Q_t = \beta E_t\left\{
      \frac{U_{c,t+1}}{U_{c,t}}
      \frac{P_t}{P_{t+1}}
    \right\},
  \]
  que es exactamente la ecuación (5).
\end{enumerate}

\paragraph{Entonces tenemos...}

\begin{itemize}
  \item El hogar compara el ``costo'' de sacrificar consumo hoy con el
        ``beneficio'' de tener más consumo mañana.
  \item El costo se mide por la utilidad marginal actual $U_{c,t}$.
  \item El beneficio se mide por la utilidad marginal futura $U_{c,t+1}$,
        ajustada por:
        \begin{itemize}
          \item el factor de descuento $\beta$,
          \item la inflación esperada $P_t/P_{t+1}$,
          \item y el precio del bono $Q_t$.
        \end{itemize}
  \item En el óptimo, no hay reubicación de consumo (entre hoy y mañana)
        que pueda mejorar la utilidad: eso es lo que recoge la Ecuación de Euler.
\end{itemize}

%--------------------------------------------------
\subsubsection*{Ecuación (6): condición de transversalidad}
\label{subsubsec:ecuacion6}

La última pieza de las condiciones de optimalidad del hogar es la llamada
\textbf{condición de transversalidad}. Esta condición está estrechamente
ligada a la restricción de solvencia (3), pero ahora se impone como
\emph{igualdad} en el óptimo:

\begin{equation}
\lim_{T\to\infty} E_t \left\{ \varXi_{t,T} \frac{B_T}{P_T} \right\} = 0.
\tag{6}
\end{equation}

Recordemos que:
\[
  \varXi_{t,T} \equiv \beta^{T-t} \frac{U_{c,T}}{U_{c,t}}
\]
es el \textbf{factor de descuento estocástico} que mide cuánto vale, en términos
de utilidad marginal en $t$, una unidad de consumo (o riqueza real) en $T$.

\paragraph{Relación con la restricción de solvencia.}

La restricción de solvencia (3) decía:
\[
  \lim_{T\to\infty} E_t \left\{ \varXi_{t,T} \frac{B_T}{P_T} \right\} \ge 0.
  \tag{3}
\]

Es decir:
\begin{itemize}
  \item En el límite, el \emph{valor presente esperado} de la riqueza real
        en bonos no puede ser negativo.
  \item Esto impide que el hogar aplique esquemas de endeudamiento tipo Ponzi.
\end{itemize}

La \textbf{condición de transversalidad} (6) dice algo más fuerte:
\[
  \lim_{T\to\infty} E_t \left\{ \varXi_{t,T} \frac{B_T}{P_T} \right\} = 0.
\]

\paragraph{¿Por qué debe valer la igualdad en el óptimo?.}

Supongamos, para ver el argumento, que en el óptimo se cumpliera:
\[
  \lim_{T\to\infty} E_t \left\{ \varXi_{t,T} \frac{B_T}{P_T} \right\} = L > 0.
\]

\begin{itemize}
  \item Entonces, incluso después de descontar por $\beta^{T-t}$ y por
        el cociente $U_{c,T}/U_{c,t}$, el hogar estaría \emph{dejando}
        una cantidad positiva de riqueza real en el infinito.
  \item Dado que la utilidad marginal del consumo es positiva
        ($U_{c,t} > 0$ para todo $t$), el hogar podría mejorar su plan:
        reducir ligeramente $B_T$ (o, en la práctica, consumir un poco más
        en algún periodo finito) sin violar la restricción de solvencia,
        usando parte de ese ``excedente'' $L$.
  \item Al hacer esto, aumentaría su consumo en algún periodo sin disminuir
        el consumo en otros lo suficiente como para compensar, por lo que
        la utilidad total aumentaría.
\end{itemize}

Pero esto contradice la suposición de que el plan original era óptimo.
Por tanto, en un óptimo:
\[
  \lim_{T\to\infty} E_t \left\{ \varXi_{t,T} \frac{B_T}{P_T} \right\} = 0,
\]
que es precisamente la ecuación (6).

\paragraph{En términos económicos...}

\begin{itemize}
  \item La ecuación (3) asegura que la deuda no crece de manera explosiva.
  \item La ecuación (6) asegura que el hogar \emph{no deja recursos sin usar}
        en el límite: toda la riqueza potencial que pueda aumentar la utilidad
        se habrá utilizado en algún momento.
  \item En muchas aplicaciones, con un hogar representativo y oferta neta
        de deuda cero, se cumple en equilibrio que $B_t = 0$ para todo $t$,
        lo que hace que (6) sea automáticamente cierta. Aun así, es importante
        tenerla explícitamente como parte de las condiciones de optimalidad.
\end{itemize}

\bigskip

\noindent
Con las ecuaciones (4), (5) y (6) tenemos ahora el conjunto completo de
\textbf{condiciones de optimalidad del hogar}:
\begin{itemize}
  \item (4): condición intratemporal (oferta de trabajo).
  \item (5): condición intertemporal (Ecuación de Euler para consumo/ahorro).
  \item (6): condición de transversalidad (uso eficiente de la riqueza en el tiempo).
\end{itemize}

En el siguiente paso, estas condiciones se combinarán con el comportamiento
de las empresas y las identidades de equilibrio para construir el modelo
macroeconómico completo.

%--------------------------------------------------
\subsubsection*{Ecuación (7): especificación funcional de la utilidad
y oferta de trabajo competitiva}
\label{subsubsec:ecuacion7}

En esta sección, seguimos el capítulo y adoptamos una \textbf{forma funcional
particular} para la utilidad periódica:

\begin{equation}
U(C_t, N_t; Z_t) =
\left(
\frac{C_t^{1-\sigma} - 1}{1-\sigma}
- \frac{N_t^{1+\phi}}{1+\phi}
\right) Z_t,
\qquad \sigma \neq 1,
\tag{11}
\end{equation}
donde $\sigma \ge 0$ y $\phi \ge 0$ son parámetros.

\paragraph{Parámetros:}

Para mayor claridad, presentamos los parámetros que aparecen en la ecuación (11):

\begin{itemize}
  \item $\sigma$ (\emph{sigma}): parámetro de curvatura de la utilidad del consumo.
  En esta clase de modelos suele interpretarse como:
  \begin{itemize}
    \item medida de \textbf{aversión relativa al riesgo} del hogar, y
    \item (en muchas especificaciones) inverso de la \textbf{elasticidad intertemporal
          de sustitución del consumo}.
  \end{itemize}
  Valores más altos de $\sigma$ implican que la utilidad marginal del consumo
  cae más rápido cuando aumenta $C_t$.

  \item $\phi$ (\emph{phi}): parámetro de curvatura de la desutilidad del trabajo.
  Controla qué tan rápido aumenta la desutilidad marginal de trabajar más horas.
  Suele interpretarse como el \textbf{inverso de la elasticidad de la oferta
  de trabajo}: valores altos de $\phi$ indican que la oferta de trabajo es menos
  sensible a cambios en el salario real.
\end{itemize}

Nuestro objetivo ahora es:  
(1) calcular $U_{c,t}$ y $U_{n,t}$ a partir de (11), y  
(2) sustituirlos en (4) para obtener la versión \emph{específica} de la
oferta de trabajo, la \textbf{Ecuación (7)}.

%-----------------------
\paragraph{Paso 1: utilidad marginal del consumo $U_{c,t}$.}

Partimos de:
\[
  U(C_t, N_t; Z_t) =
  \left(
    \frac{C_t^{1-\sigma} - 1}{1-\sigma}
    - \frac{N_t^{1+\phi}}{1+\phi}
  \right) Z_t.
\]

Tomamos la derivada parcial respecto a $C_t$:
\begin{equation}
  U_{c,t} \equiv \frac{\partial U_t}{\partial C_t}
  = \frac{\partial}{\partial C_t}
    \left[
      \left(
        \frac{C_t^{1-\sigma} - 1}{1-\sigma}
        - \frac{N_t^{1+\phi}}{1+\phi}
      \right) Z_t
    \right].
\end{equation}

Como $Z_t$ está multiplicando todo, podemos sacarlo como constante:
\[
  U_{c,t}
  = Z_t \, \frac{\partial}{\partial C_t}
    \left(
      \frac{C_t^{1-\sigma} - 1}{1-\sigma}
      - \frac{N_t^{1+\phi}}{1+\phi}
    \right).
\]

El segundo término no depende de $C_t$, así que su derivada es cero.
Nos queda:
\[
  U_{c,t}
  = Z_t \, \frac{\partial}{\partial C_t}
    \left(
      \frac{C_t^{1-\sigma} - 1}{1-\sigma}
    \right).
\]

Ahora derivamos usando la regla de la potencia:
\[
  \frac{\partial}{\partial C_t}
  \left(
    \frac{C_t^{1-\sigma} - 1}{1-\sigma}
  \right)
  = \frac{1}{1-\sigma} \cdot (1-\sigma) C_t^{(1-\sigma)-1}
  = C_t^{-\sigma}.
\]

Por lo tanto,
\begin{equation}
  U_{c,t} = C_t^{-\sigma} Z_t.
\end{equation}

\noindent
\textbf{Tenemos...}
\begin{itemize}
  \item Si $\sigma > 0$, entonces $U_{c,t} > 0$ y decrece cuando $C_t$ aumenta
        (porque $C_t^{-\sigma}$ disminuye con $C_t$).  
        Esto refleja la idea de \emph{utilidad marginal decreciente del consumo}.
  \item El factor $Z_t$ amplifica o reduce la utilidad marginal del consumo
        según el estado de las preferencias.
\end{itemize}

%-----------------------
\paragraph{Paso 2: utilidad marginal del trabajo $U_{n,t}$.}

Volvemos a la función de utilidad:
\[
  U(C_t, N_t; Z_t) =
  \left(
    \frac{C_t^{1-\sigma} - 1}{1-\sigma}
    - \frac{N_t^{1+\phi}}{1+\phi}
  \right) Z_t.
\]

Tomamos ahora la derivada parcial respecto a $N_t$:
\begin{equation}
  U_{n,t} \equiv \frac{\partial U_t}{\partial N_t}
  = \frac{\partial}{\partial N_t}
    \left[
      \left(
        \frac{C_t^{1-\sigma} - 1}{1-\sigma}
        - \frac{N_t^{1+\phi}}{1+\phi}
      \right) Z_t
    \right].
\end{equation}

De nuevo, sacamos $Z_t$ como constante:
\[
  U_{n,t}
  = Z_t \, \frac{\partial}{\partial N_t}
    \left(
      \frac{C_t^{1-\sigma} - 1}{1-\sigma}
      - \frac{N_t^{1+\phi}}{1+\phi}
    \right).
\]

El primer término no depende de $N_t$, por lo que su derivada es cero.
Nos queda:
\[
  U_{n,t}
  = Z_t \, \frac{\partial}{\partial N_t}
    \left(
      - \frac{N_t^{1+\phi}}{1+\phi}
    \right).
\]

Aplicando la regla de la potencia:
\[
  \frac{\partial}{\partial N_t}
  \left(
    - \frac{N_t^{1+\phi}}{1+\phi}
  \right)
  = - \frac{1}{1+\phi} (1+\phi) N_t^{(1+\phi)-1}
  = - N_t^{\phi}.
\]

Por lo tanto,
\begin{equation}
  U_{n,t} = - N_t^{\phi} Z_t.
\end{equation}

\noindent
\textbf{Tenmos...}
\begin{itemize}
  \item Para $N_t > 0$ y $Z_t > 0$, se cumple $U_{n,t} < 0$: trabajar más
        \emph{reduce} la utilidad (desutilidad del esfuerzo).
  \item El término $-U_{n,t} = N_t^{\phi} Z_t$ es la \emph{desutilidad marginal
        del trabajo}, creciente en $N_t$ si $\phi > 0$.
\end{itemize}

%-----------------------
\paragraph{Paso 3: sustitución en la condición intratemporal (4).}

La condición general de optimalidad intratemporal es:
\[
  -\frac{U_{n,t}}{U_{c,t}} = \frac{W_t}{P_t}.
\]

Sustituimos las expresiones obtenidas:
\[
  -\frac{U_{n,t}}{U_{c,t}}
  = -\frac{-N_t^{\phi} Z_t}{C_t^{-\sigma} Z_t}
  = \frac{N_t^{\phi} Z_t}{C_t^{-\sigma} Z_t}.
\]

Observamos que $Z_t$ aparece como factor multiplicando tanto en el numerador
como en el denominador, por lo que se cancela:
\[
  -\frac{U_{n,t}}{U_{c,t}}
  = \frac{N_t^{\phi}}{C_t^{-\sigma}}.
\]

Usando que $1/C_t^{-\sigma} = C_t^{\sigma}$, obtenemos:
\[
  \frac{N_t^{\phi}}{C_t^{-\sigma}}
  = N_t^{\phi} C_t^{\sigma}.
\]

Por lo tanto,
\begin{equation}
  -\frac{U_{n,t}}{U_{c,t}}
  = C_t^{\sigma} N_t^{\phi}.
\end{equation}

Igualando esto al salario real, según (4), llegamos a:
\begin{equation}
  \frac{W_t}{P_t} = C_t^{\sigma} N_t^{\phi},
\tag{7}
\end{equation}
que es la \textbf{Ecuación (7)}.

%-----------------------
\paragraph{Interpretación económica de la Ecuación (7).}

La ecuación
\[
  \frac{W_t}{P_t} = C_t^{\sigma} N_t^{\phi}
\]
puede leerse como:

\begin{center}
\emph{``El salario real debe igualar el costo subjetivo marginal de trabajar
una unidad adicional, medido en unidades de consumo, dado el nivel de consumo
$C_t$ y trabajo $N_t$.''}
\end{center}

Más concretamente:

\begin{itemize}
  \item Si $N_t$ aumenta (el hogar trabaja más), el término $N_t^{\phi}$
        aumenta (para $\phi > 0$), por lo que el lado derecho crece.  
        \textbf{Es decir:} para convencer al hogar de trabajar más,
        el salario real $W_t/P_t$ debe ser mayor.

  \item Si $C_t$ aumenta, el término $C_t^{\sigma}$ también aumenta
        (para $\sigma > 0$). Esto refleja que, cuando el hogar ya consume mucho,
        la utilidad marginal del consumo es baja; por tanto, el hogar necesita
        un salario real más alto para renunciar a ocio (seguir trabajando)
        y mantener el equilibrio óptimo. Esto está relacionado con el
        \emph{efecto ingreso} sobre la oferta de trabajo.

  \item El producto $C_t^{\sigma} N_t^{\phi}$ es, en este contexto,
        la \textbf{tasa marginal de sustitución} entre ocio y consumo bajo
        la forma funcional elegida.
\end{itemize}

%-----------------------
\paragraph{Separabilidad y el papel de $Z_t$.}

Un detalle importante del resultado es que $Z_t$ \emph{desaparece}
de la condición intratemporal:

\begin{itemize}
  \item El choque de preferencias $Z_t$ multiplica toda la utilidad, por lo que
        aparece tanto en $U_{c,t}$ como en $U_{n,t}$, y se cancela en el cociente
        $-U_{n,t}/U_{c,t}$.
  \item Esto implica que la \textbf{oferta de trabajo intratemporal} (ecuación 7)
        no se ve afectada directamente por $Z_t$.
  \item En cambio, $Z_t$ sí influye en las decisiones \emph{intertemporales} a través
        de la Ecuación de Euler (5), donde entra la razón $U_{c,t+1}/U_{c,t}$.
\end{itemize}

La Ecuación (7) es la versión concreta, para la utilidad (11),
de la condición de optimalidad intratemporal del hogar. Representa la
\textbf{curva de oferta de trabajo competitiva}: para cada nivel de consumo
$C_t$, indica qué combinación de salario real $W_t/P_t$ y horas trabajadas
$N_t$ es consistente con el óptimo del hogar.
%--------------------------------------------------
\subsubsection*{Ecuación (8): Ecuación de Euler intertemporal específica}
\label{subsubsec:ecuacion8}

En la sección anterior obtuvimos la \emph{Ecuación de Euler general} para el
consumo y el ahorro del hogar:
\begin{equation}
Q_t = \beta E_t \left\{
  \frac{U_{c,t+1}}{U_{c,t}} \frac{P_t}{P_{t+1}}
\right\},
\tag{5}
\end{equation}
donde $U_{c,t}$ es la utilidad marginal del consumo en el periodo $t$.

Ahora queremos ver \emph{cómo se ve} esta condición cuando usamos la
función de utilidad específica:
\begin{equation}
U(C_t, N_t; Z_t) =
\left(
\frac{C_t^{1-\sigma} - 1}{1-\sigma}
- \frac{N_t^{1+\phi}}{1+\phi}
\right) Z_t,
\qquad \sigma \neq 1,
\tag{11}
\end{equation}
que introdujimos al derivar la ecuación (7).

%-----------------------
\paragraph{Parámetros relevantes}

En esta ecuación de Euler aparecen de nuevo dos parámetros importantes:

\begin{itemize}
  \item $\sigma$ (\emph{sigma}): parámetro de curvatura de la utilidad del consumo.
  Como ya se discutió, se interpreta habitualmente como una medida de
  \textbf{aversión relativa al riesgo} y está estrechamente relacionada con la
  \textbf{elasticidad intertemporal de sustitución del consumo}.

  \item $\beta$ (\emph{beta}): \textbf{factor de descuento intertemporal} del hogar,
  con $0 < \beta < 1$. Controla qué tanto valora el hogar la utilidad futura
  respecto a la presente; valores altos de $\beta$ indican mayor paciencia.
\end{itemize}

Nuestro objetivo es expresar la ecuación (5) en términos de tasas de
crecimiento de consumo y del factor de preferencias $Z_t$.

%-----------------------
\paragraph{Paso 1: utilidades marginales del consumo.}

De la derivación previa (sección de la ecuación 7), recordamos que:
\[
  U_{c,t} = C_t^{-\sigma} Z_t,
  \qquad
  U_{c,t+1} = C_{t+1}^{-\sigma} Z_{t+1}.
\]

Estas expresiones provienen de derivar (11) respecto a $C_t$ y $C_{t+1}$:

\[
  U_{c,t} \equiv \frac{\partial U(C_t,N_t;Z_t)}{\partial C_t}
  = C_t^{-\sigma} Z_t,
\]
\[
  U_{c,t+1} \equiv \frac{\partial U(C_{t+1},N_{t+1};Z_{t+1})}{\partial C_{t+1}}
  = C_{t+1}^{-\sigma} Z_{t+1}.
\]

%-----------------------
\paragraph{Paso 2: razón de utilidades marginales $U_{c,t+1}/U_{c,t}$.}

Formamos la razón que aparece en la Ecuación de Euler:
\begin{equation}
\frac{U_{c,t+1}}{U_{c,t}}
= \frac{C_{t+1}^{-\sigma} Z_{t+1}}{C_t^{-\sigma} Z_t}.
\end{equation}

Agrupamos términos de consumo y de preferencias por separado:
\[
  \frac{U_{c,t+1}}{U_{c,t}}
  = \left(
      \frac{C_{t+1}^{-\sigma}}{C_t^{-\sigma}}
    \right)
    \left(
      \frac{Z_{t+1}}{Z_t}
    \right).
\]

En el primer cociente aplicamos la propiedad de potencias:
\[
  \frac{C_{t+1}^{-\sigma}}{C_t^{-\sigma}}
  = \left(
      \frac{C_{t+1}}{C_t}
    \right)^{-\sigma}.
\]

Por lo tanto:
\begin{equation}
\frac{U_{c,t+1}}{U_{c,t}}
= \left(
    \frac{C_{t+1}}{C_t}
  \right)^{-\sigma}
  \left(
    \frac{Z_{t+1}}{Z_t}
  \right).
\label{eq:razon_Uc}
\end{equation}

\noindent
\textbf{Podemos leer...}
\begin{itemize}
  \item El término $\displaystyle (C_{t+1}/C_t)^{-\sigma}$ recoge cómo el
        crecimiento del consumo afecta la utilidad marginal relativa entre
        $t$ y $t+1$.
  \item El término $\displaystyle Z_{t+1}/Z_t$ refleja cómo cambian las
        preferencias (el ``peso'' que el hogar asigna a la utilidad del consumo)
        entre ambos periodos.
\end{itemize}

%-----------------------
\paragraph{Paso 3: sustitución en la Ecuación de Euler general.}

Tomamos la Ecuación de Euler:
\[
  Q_t = \beta E_t \left\{
    \frac{U_{c,t+1}}{U_{c,t}} \frac{P_t}{P_{t+1}}
  \right\},
\]
y sustituimos la expresión de \eqref{eq:razon_Uc}:

\[
  Q_t = \beta E_t \left\{
    \left(
      \frac{C_{t+1}}{C_t}
    \right)^{-\sigma}
    \left(
      \frac{Z_{t+1}}{Z_t}
    \right)
    \frac{P_t}{P_{t+1}}
  \right\}.
\]

Esto nos da la \textbf{Ecuación (8)}:

\begin{equation}
  Q_t = \beta E_t \left\{
    \left(
      \frac{C_{t+1}}{C_t}
    \right)^{-\sigma}
    \left(
      \frac{Z_{t+1}}{Z_t}
    \right)
    \left(
      \frac{P_t}{P_{t+1}}
    \right)
  \right\}.
\tag{8}
\end{equation}

%-----------------------
\paragraph{Interpretación de la Ecuación (8).}

La ecuación (8) es la versión \emph{específica} de la condición de Euler del
hogar bajo la utilidad (11). Podemos leerla así:

\begin{center}
\emph{``El precio del bono hoy ($Q_t$) es igual al valor esperado, descontado,
de lo que vale una unidad de consumo futuro en términos de utilidad marginal
de hoy, ajustado por inflación y por cambios en preferencias.''}
\end{center}

Más en detalle:

\begin{itemize}
  \item $Q_t$: precio actual de una promesa de 1 unidad de dinero en $t+1$.
        Como antes, el rendimiento nominal bruto del bono es $R_t^n = 1/Q_t$.

  \item $\beta$: factor de descuento (\emph{beta}). Captura la impaciencia
        pura del hogar: cuanto menor es $\beta$, más ``caro'' le resulta
        postergar consumo.

  \item $\displaystyle \left(\frac{C_{t+1}}{C_t}\right)^{-\sigma}$:
        este término recoge la \textbf{sustitución intertemporal del consumo}.
        \begin{itemize}
          \item Si se espera que $C_{t+1}$ sea bajo respecto a $C_t$, la
                utilidad marginal futura será alta, y el hogar valora mucho
                poder aumentar $C_{t+1}$: esto tiende a elevar el valor del
                bono (sube $Q_t$).
          \item El parámetro $\sigma$ (\emph{sigma}) gobierna qué tan sensible
                es esta valoración a cambios en la tasa de crecimiento del
                consumo.
        \end{itemize}

  \item $\displaystyle \frac{Z_{t+1}}{Z_t}$:
        refleja la \textbf{tasa de crecimiento del factor de preferencias}.
        \begin{itemize}
          \item Si $Z_{t+1}/Z_t$ es grande, el consumo futuro ``pesa más''
                en la función de utilidad; el hogar está dispuesto a pagar
                más hoy por una unidad de consumo mañana (aumenta $Q_t$).
          \item Dicho de otra forma, $Z_t$ actúa como un \emph{choque al
                factor de descuento efectivo}.
        \end{itemize}

  \item $\displaystyle \frac{P_t}{P_{t+1}}$:
        es el \textbf{inverso de la inflación bruta esperada}. Si se espera
        alta inflación, una unidad nominal de mañana vale menos en términos
        de consumo, lo que tiende a reducir $Q_t$ (para un mismo nivel de
        utilidad marginal futura).
\end{itemize}

\noindent
\textbf{Tenemos...}

La Ecuación (8) combina tres elementos fundamentales:

\begin{enumerate}
  \item \textbf{Tiempo e impaciencia} ($\beta$): preferencia por el presente.
  \item \textbf{Riesgo y crecimiento del consumo} 
        ($\left(C_{t+1}/C_t\right)^{-\sigma}$ y $\sigma$): cómo valora el hogar
        el consumo en estados futuros donde $C_{t+1}$ puede ser alto o bajo.
  \item \textbf{Choques de preferencias e inflación}
        ($Z_{t+1}/Z_t$ y $P_t/P_{t+1}$): cómo cambian la ``importancia''
        del consumo futuro y el poder adquisitivo de los pagos nominales.
\end{enumerate}

En conjunto, estos factores determinan el \emph{precio justo} del bono
$Q_t$ en equilibrio.
%--------------------------------------------------
















\section{Conclusion}

\begin{thebibliography}{99}
\bibitem{} Reference here.
\end{thebibliography}

\end{document}