\documentclass[12pt]{article}
\usepackage[margin=1in]{geometry}
\usepackage{amsmath}
\usepackage{amssymb}

\title{Title}
\author{Gabriel Tehozol}

\begin{document}

\maketitle
%========================================
% 2.1 HOGARES: Problema de Optimización
%========================================
\subsection{Hogares: definición del problema de optimización dinámica}
\label{subsec:hogares}

El modelo asume un \emph{hogar representativo} que resuelve un problema de
\emph{optimización dinámica}. El objetivo es maximizar la utilidad a lo largo
de un horizonte infinito, sujeta a restricciones presupuestarias de flujo y a
una restricción de solvencia que descarta esquemas de tipo Ponzi.

%--------------------------------------------------
\subsubsection{Ecuación (1): función objetivo (utilidad esperada descontada)}
\label{subsubsec:hogares_objetivo}

El hogar busca maximizar la utilidad intertemporal dada por:
\begin{equation}
E_0 \sum_{t=0}^{\infty} \beta^t U(C_t, N_t; Z_t),
\tag{1}
\end{equation}
donde $C_t$ es el consumo, $N_t$ las horas de trabajo y $Z_t$ un
desplazador exógeno de preferencias. El parámetro $\beta \in (0,1)$ es
el factor de descuento intertemporal, y $E_0\{\cdot\}$ denota el
operador de expectativas condicional a la información disponible en $t=0$.

\paragraph{Definiciones de términos en (1).}

\begin{description}
  \item[$E_0\{\cdot\}$] Operador de expectativas. Indica que el hogar
  maximiza el valor esperado de la utilidad futura, condicional a la
  información disponible en $t=0$.

  \item[$\beta^t$] Factor de descuento (\emph{discount factor}),
  con $\beta \in (0,1)$. Descuenta la utilidad de los periodos futuros
  a su valor presente.

  \item[$U(\cdot)$] Función de utilidad periódica, continua y dos veces
  diferenciable. Mide la satisfacción instantánea del hogar.

  \item[$C_t$] Cantidad consumida del único bien disponible en la economía.
  La utilidad marginal del consumo es
  \[
    U_{c,t} \equiv \frac{\partial U(C_t,N_t;Z_t)}{\partial C_t} > 0,
  \]
  y es no creciente ($U_{cc,t} \le 0$).

  \item[$N_t$] Horas de trabajo u oferta de empleo. La utilidad marginal
  del trabajo es
  \[
    U_{n,t} \equiv \frac{\partial U(C_t,N_t;Z_t)}{\partial N_t} \le 0,
  \]
  es decir, el trabajo genera desutilidad; $-U_{n,t}$ es positiva
  y no decreciente.

  \item[$Z_t$] \emph{Factor exógeno de desplazamiento de preferencias}
  (\emph{exogenous preference shifter}). Un aumento en $Z_t$ eleva la
  utilidad marginal del consumo:
  \[
    U_{cz,t} \equiv 
    \frac{\partial^2 U(C_t,N_t;Z_t)}{\partial C_t\,\partial Z_t} > 0.
  \]
  Dada la forma funcional específica que se adopta en el texto principal,
  el efecto de $Z_t$ se restringe a las decisiones intertemporales (a través
  de la razón $U_{c,t+k}/U_{c,t}$), sin afectar directamente las decisiones
  intratemporales (la razón $U_{n,t}/U_{c,t}$).
\end{description}

\paragraph{Integración.}

La ecuación (1) define el \emph{objetivo} del hogar: elegir secuencias
$\{C_t,N_t\}_{t=0}^{\infty}$ que maximizan la utilidad esperada
descontada. Sin embargo, estas elecciones están restringidas por la
capacidad del hogar para generar ingreso y endeudarse/ahorrar, lo que
da lugar a la restricción presupuestaria de flujo.

%--------------------------------------------------
\subsubsection{Ecuación (2): restricción presupuestaria de flujo}
\label{subsubsec:hogares_flujo}

En cada periodo $t$, el gasto del hogar y sus decisiones de ahorro deben
estar limitados por su ingreso corriente:

\begin{equation}
P_t C_t + Q_t B_t \le B_{t-1} + W_t N_t + D_t,
\tag{2}
\end{equation}
donde el lado izquierdo recoge el gasto en consumo y en bonos, y el lado
derecho el ingreso disponible.

\paragraph{Definiciones de términos en (2).}

\begin{description}
  \item[$P_t$] Precio nominal del bien de consumo.
  \item[$P_t C_t$] Gasto nominal en consumo.
  \item[$Q_t$] Precio del bono nominal libre de riesgo a un periodo;
  pagar una unidad de dinero en $t+1$ cuesta $Q_t$ unidades de dinero
  en el periodo $t$.
  \item[$B_t$] Cantidad de bonos nominales libres de riesgo comprados
  en el periodo $t$. El término $Q_t B_t$ es el gasto en ahorro financiero.
  \item[$B_{t-1}$] Valor nominal de los bonos que maduran y se pagan
  al hogar en el periodo $t$.
  \item[$W_t$] Salario nominal (por hora o por trabajador).
  \item[$W_t N_t$] Ingreso nominal del trabajo.
  \item[$D_t$] Dividendos que recibe el hogar como propietario de las empresas.
\end{description}

\paragraph{Integración.}

La secuencia de restricciones como (2) determina el \emph{conjunto factible}
de planes $(C_t,N_t,B_t)$ para el hogar. El problema del hogar consiste
en elegir la secuencia $\{C_t,N_t,B_t\}_{t=0}^{\infty}$ que maximiza (1)
sujeta a (2), dado el vector de precios $(P_t,Q_t,W_t)$ y los dividendos $(D_t)$.

%--------------------------------------------------
\subsubsection{Ecuación (3): restricción de solvencia (no-Ponzi)}
\label{subsubsec:hogares_solvencia}

Además de la restricción de flujo, el hogar enfrenta una \emph{restricción
de solvencia} que impide esquemas de endeudamiento tipo Ponzi, garantizando
que su deuda pueda ser pagada en el largo plazo:

\begin{equation}
\lim_{T\to\infty} E_t \left\{ \varXi_{t,T} \frac{B_T}{P_T} \right\} \ge 0,
\tag{3}
\end{equation}
donde $\varXi_{t,T}$ es el factor de descuento estocástico asociado a la
utilidad marginal del consumo.

\paragraph{Definiciones de términos en (3).}

\begin{description}
  \item[$\displaystyle \lim_{T\to\infty} E_t\{\cdot\}$]
  Límite del valor esperado condicional en $t$ a medida que el horizonte
  $T$ tiende a infinito.

  \item[$B_T/P_T$] Riqueza real terminal asociada a los bonos en el horizonte $T$.

  \item[$\varXi_{t,T}$] \emph{Factor de descuento estocástico}, definido como
  \[
    \varXi_{t,T} \equiv \beta^{T-t} \frac{U_{c,T}}{U_{c,t}},
  \]
  que descuenta los flujos reales futuros a su valor presente, incorporando
  tanto la impaciencia ($\beta$) como la evolución de la utilidad marginal del
  consumo.
\end{description}

\paragraph{Integración.}

La ecuación (3) asegura que el plan dinámico del hogar es sostenible a
larga plazo. En equilibrio, con un hogar representativo y oferta neta de
deuda igual a cero, esta condición suele cumplirse con igualdad (lo que
se expresará más adelante en forma de condición de transversalidad).

%--------------------------------------------------
\subsubsection{Ecuación (4): condición de optimalidad intratemporal (oferta de trabajo)}
\label{subsubsec:hogares_intratemporal}

Una de las condiciones de primer orden resultantes de la maximización de (1)
sujeta a (2) y (3) es la \emph{condición de optimalidad intratemporal}, que
rige la elección óptima entre consumo y trabajo dentro de un mismo periodo $t$:

\begin{equation}
-\frac{U_{n,t}}{U_{c,t}} = \frac{W_t}{P_t}.
\tag{4}
\end{equation}

\paragraph{Definiciones de términos en (4).}

\begin{description}
  \item[$U_{n,t}$] Utilidad marginal del trabajo,
  $U_{n,t} \equiv \partial U(C_t,N_t;Z_t)/\partial N_t \le 0$.
  El trabajo genera desutilidad, por lo que $-U_{n,t} > 0$.

  \item[$U_{c,t}$] Utilidad marginal del consumo,
  $U_{c,t} \equiv \partial U(C_t,N_t;Z_t)/\partial C_t > 0$.

  \item[$-\dfrac{U_{n,t}}{U_{c,t}}$] \emph{Tasa marginal de sustitución}
  (TMS) entre ocio y consumo: mide cuántas unidades adicionales de
  consumo son necesarias para compensar la desutilidad de trabajar
  una unidad extra.

  \item[$W_t/P_t$] \emph{Salario real}, es decir, el precio del trabajo
  en términos del bien de consumo.
\end{description}

\paragraph{Derivación (argumento variacional).}

La ecuación (4) puede obtenerse a partir de un argumento de variación
en el periodo $t$:

\begin{enumerate}
  \item \textbf{Variación en la utilidad.} Considérese una pequeña desviación
  $(dC_t,dN_t)$ respecto al plan óptimo en el periodo $t$. La variación de la
  utilidad instantánea es:
  \[
    dU_t = U_{c,t}\,dC_t + U_{n,t}\,dN_t.
  \]
  En un óptimo, cualquier desviación factible que mantenga constantes las
  demás variables no debe elevar la utilidad, por lo que un cambio marginalmente
  factible debe cumplir $dU_t = 0$.

  \item \textbf{Variación en la restricción presupuestaria.} Si solo se ajustan
  $C_t$ y $N_t$ en el periodo $t$, manteniendo fijo $B_t$, la variación de la
  restricción de flujo (2) es:
  \[
    P_t\,dC_t = W_t\,dN_t,
  \]
  es decir, el gasto adicional en consumo debe igualar el ingreso adicional
  por trabajar más.

  \item \textbf{Combinación de ambas condiciones.} De la variación presupuestaria
  se obtiene:
  \[
    dC_t = \frac{W_t}{P_t} dN_t.
  \]
  Sustituyendo en la variación de utilidad:
  \[
    U_{c,t}\left(\frac{W_t}{P_t} dN_t\right) + U_{n,t} dN_t = 0.
  \]
  Dividiendo por $dN_t \neq 0$ y reordenando términos se llega a:
  \[
    -\frac{U_{n,t}}{U_{c,t}} = \frac{W_t}{P_t},
  \]
  que coincide con la ecuación (4).
\end{enumerate}

\paragraph{Interpretación.}

La ecuación (4) es la condición de \emph{oferta de trabajo} del hogar:
en el óptimo, el hogar iguala su tasa marginal de sustitución entre
ocio y consumo (costo de oportunidad de trabajar) con el salario real
ofrecido en el mercado.


Content here.

\section{Conclusion}

\begin{thebibliography}{99}
\bibitem{} Reference here.
\end{thebibliography}

\end{document}